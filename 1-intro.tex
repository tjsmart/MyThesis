% //////////////////////////////////////////////////////////|
% ++++++++++++++++++++++++++++++++++++++++++++++++++++++++++|
% %%%%%%%%%%%%%%%%%%%%%%%%%%%%%%%%%%%%%%%%%%%%%%%%%%%%%%%%%%|
%    CHAPTER                                                |
% %%%%%%%%%%%%%%%%%%%%%%%%%%%%%%%%%%%%%%%%%%%%%%%%%%%%%%%%%%|
% ++++++++++++++++++++++++++++++++++++++++++++++++++++++++++|
% \\\\\\\\\\\\\\\\\\\\\\\\\\\\\\\\\\\\\\\\\\\\\\\\\\\\\\\\\\|
\chapter{Introduction}

% general "big picture" motivation, will introduce briefly both renewable energy and quantum information science motivations
Bad science
Cons of fracking: ground water contamination, air contamination, animal death & disease, earthquakes
Pros: no waste produced, no carbon footprint at all
Cons: ok it may have explosive issues but we are working out the kinks
Pictures: hydrogen fuel car (Toyota), hydrogen fuel bus (Mercedes-benz), hydrogen fuel train (Germany)
IBM quantum computer – superconducting qubits


% 
The rational design of materials by accurate yet feasible computational approach can offer researchers the ability to discover and describe otherwise unattainable phenomena. Specifically in the era of supercomputers, large-scale calculations have reached the stage of being more obtainable than ever before, heightening the interest in the computational approach to science. Here, large-scale may refer to pushing the limits of computational resources in terms of processing power, computer memory, or simply time to complete the task. Specifically the past 10 years, the scale of these computational resources has grown exponentially. And perhaps to no surprise, so has humanities interest in utilizing these resources to achieve tasks once never believed possible.
% <May put here a graph of supercomputer resources etc.>


The task which is relevant to this dissertation is that of the multi-electron problem. In the interrelated field of physics, chemistry, and material science,  the multi-electron problem essentially refers to any problem beyond involving interactions between more than one electron, often in an external field.


% %%%%%%%%%%%%%%%%%%%%%%%%%%%%%%%%%%%%%%%%%%%%%%%%%%%%%%%%%%
%    section
% %%%%%%%%%%%%%%%%%%%%%%%%%%%%%%%%%%%%%%%%%%%%%%%%%%%%%%%%%%
\section{The Multi-Electron Problem}

\subsection{Hohenberg-Kohn-Sham Theory}

The many-electron problem refers to any 

In this thesis, state-of-the-art first-principles calculations are implemented and utilized in order to design material properties which can in some way improve their applications. These applications, and therefore the research, have two primary focuses. The first is in the 