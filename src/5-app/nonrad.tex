\def\vhq{\frac{\partial h}{\partial Q}}

\chapter{Formalism of Nonradiative Recombination}

\section{Static Coupling}

\begin{figure}[H]
    \centering
    \includegraphics[width=0.41\textwidth]{5-app/figures/nonrad-diag.png}
    \caption{Diagram of the transition between two vibronic states. The two quadratic curves represent the electronic states $(i,f)$ and the vibrational modes present at each state are superimposed with states $(n,m)$.}
    \label{app:fig:nonrad}
\end{figure}


Here we consider a system which is initially in a vibronic state $\ket{\Psi_{in}(r,R)}$ and transitions to a final vibronic state $\ket{\Psi_{fm}(r,R)}$ with $i\neq f$. Here the indices $(i,f)$ denote the electronic state, while $(n,m)$ denote the phonon state as shown schematically in Figure~\ref{app:fig:nonrad}. Within the Born-Oppenheimer approximation these states are a direct product of the electronic states $\ket{\psi_{(i,f)}(r,R)}$ and phonon states $\ket{\phi_{(n,m)}(R)}$.
\begin{align}
    \ket{\Psi_{(in,fm)}(r,R)}=\ket{\psi_{(i,f)}(r,R)\phi_{(n,m)}(R)}   \label{app:nonrad:eq:vib}
\end{align}
The coordinate $r$ denotes the spatial dependence of the electronic wavefunction $\psi_{(i,f)}$ and $R$ is the configuration of atomic positions.

The probability of transitioning between vibronic state $\ket{\Psi_{i,n}(r,R)}$ to the state $\ket{\Psi_{f,m}(r,R)}$ is given by Fermi's Golden Rule:
\begin{align}
    \Gamma_{in\rightarrow fm} = \frac{2\pi}{\hbar} f(i,n) |V_{in,fm}|^2 \delta(E_{in}-E_{fm})
    \label{app:nonrad:eq:g1}
\end{align}
Here $f(i,n)$ is the probability of occupying phonon state $n$ when in the electronic state $i$, which follows a thermal Maxwell-Boltzmann distribution. The dirac delta function ensures the conservation of energy between vibronic states $E_{in}$ and $E_{fm}$. And finally $V_{in,fm}$ is the electron phonon coupling matrix, discussed in more detail later.

If we wish to compute the collective transition rate between electronic states $i$ and $f$ this follows readily from Eq. (\ref{app:nonrad:eq:g1}).
\begin{align}
    \Gamma_{i \rightarrow f} &=\sum_{n,m} \Gamma_{in\rightarrow fm} \nonumber \\
    \Gamma_{i \rightarrow f}
    &=\frac{2\pi}{\hbar}\sum_{n,m}f(i,n)|V_{in,fm}|^2\delta(E_{in}-E_{fm})  \label{app:nonrad:eq:g2}
\end{align}
Here we have now summed over all possible initial and final phonon states ($n$ and $m$) to give the full probability of transitioning from electronic state $i$ to state $f$.

A term of particular interest in Eq. (\ref{app:nonrad:eq:g2}) is the electron-phonon coupling matrix $V_{in,fm}$. Within the static coupling approximation we approximate the electron-phonon coupling to first order in $R$. First consider the total Hamiltonian $H_{tot}$ to first order in $R$ about some position $R_{0}$ as:
\begin{align}
    H_{tot}(r,R) = H(r,R_0) + \sum_R \frac{\partial H}{\partial R} (R-R_0) \label{app:nonrad:eq:fo1}
\end{align}
where $H$ is the electron Hamiltonian, and the partial of $H$ with respect to $R$ is for every atomic position in 3D space. Meanwhile the electronic wavefunction is $\psi_{i}(r,R)$ to first order in $R$ is given by
\begin{align}
    \ket{\psi_i(r,R)} = \ket{\psi_i(r,R_0)} + \sum_R (R-R_0) \ket{\frac{\partial \psi_{i}}{\partial R}}\label{app:nonrad:eq:fo2}
\end{align}
where in Eq. (\ref{app:nonrad:eq:fo1}-\ref{app:nonrad:eq:fo2}) the evaluation of the derivative with respect to $R$ at $R_{0}$ is implicit.


Using the approximations of Eq. (\ref{app:nonrad:eq:fo1}-\ref{app:nonrad:eq:fo2}), the electron-phonon coupling matrix is given by:
\begin{align}
    V_{in,fm}
    &= \braket{\Psi_{f,m}(r,R)|
    H_{tot}(r,R)
    |\Psi_{i,n}(r,R)} \nonumber \\
    &= \braket{\psi_f(r,R_0)\phi_{f,m}(R)|
    H_0
    |\psi_i(r,R_0)\phi_{i,n}(R)} \nonumber \\
    & \hspace{1cm} + \sum_R \bigg( \braket{\psi_f(r,R_0)\phi_{f,m}(R)|
    \frac{\partial H}{\partial R}(R-R_0)
    |\psi_i(r,R_0)\phi_{i,n}(R)} \nonumber \\
    & \hspace{1cm} + \braket{\psi_f(r,R_0)\phi_{f,m}(R)|
    H_0 (R-R_0)
    |\frac{\partial \psi_i}{\partial R}\phi_{i,n}(R)} \nonumber \\
    & \hspace{1cm} + \braket{\frac{\partial \psi_f}{\partial R}\phi_{f,m}(R)|
    (R-R_0)H_0
    |\psi_i(r,R_0)\phi_{i,n}(R)} \bigg) \nonumber \\
    & \hspace{1cm} + \mathcal{O}(R^2) \nonumber \\
    &= \sum_R \braket{\phi_{f,m}(R)|(R-R_0)
    |\phi_{i,n}(R)}
    \bigg[ \braket{\psi_f(r,R_0)|
    \frac{\partial H}{\partial R}
    |\psi_i(r,R_0)} \nonumber \\
    & \hspace{2cm} + \braket{\psi_f(r,R_0)|
    H_0
    |\frac{\partial \psi_i}{\partial R}}
    + \braket{\frac{\partial \psi_f}{\partial R}|
    H_0
    |\psi_i(r,R_0)} \bigg] \nonumber \\
    &= \sum_R
    \braket{\phi_{f,m}(R)|(R-R_0)|\phi_{i,n}(R)}
    \braket{\psi_f(r,R_0)|\frac{\partial H}{\partial R}|\psi_i(r,R_0)}
\end{align}
Where in the first step we have we have removed any terms of order $R^2$ (denoted with $\mathcal{O}(R^2)$). In the second step, the first term is removed do to orthogonality $\braket{\psi_f|\psi_i}=0$, and the latter part is rewritten with the factorization of the electronic and phonon parts due to there independence on $R$ and $H$, respectively. In the final step, we are left with only one term as the last two terms cancel (see Appendix). Thus, the static coupling approximation gives an electron-phonon coupling matrix of the form:
\begin{align}
    V_{in,fm}= \sum_R
    \braket{\phi_{f,m}(R)|(R-R_0)|\phi_{i,n}(R)}
    \braket{\psi_f(r,R_0)|\frac{\partial H}{\partial R}|\psi_i(r,R_0)}
    \label{app:nonrad:eq:eph1}
\end{align}

Alternatively, the electron-phonon coupling can instead be expressed in terms of phonon modes $Q_k$,
\begin{eqnarray}
    V_{in,fm}
    &=&
    \sum_k \braket{\psi_f(r,R_0)|
    \frac{\partial H}{\partial Q_k}|\psi_i(r,R_0)}
    \braket{\phi_{f,m}(R)|(Q_k-Q_{k,0})|\phi_{i,n}(R)} \\
    &=& \sum_k C^k_{if}\braket{\phi_{f,m}(R)|
    \mathbf{Q}_k|\phi_{i,n}(R)} \label{app:nonrad:eq:eph2}
\end{eqnarray}
with
\begin{eqnarray}
    \mathbf{Q}_{i,k}=\frac{1}{\sqrt{M_k}}\sum_R M_R \mu_k(R) \mathbf{R}_i \nonumber \\
    \mathbf{Q}_{f,k}=\frac{1}{\sqrt{M_k}}\sum_R M_R \mu_k(R) \mathbf{R}_f
\end{eqnarray}
Here $\mathbf{R}_i = R_i - R_i(0)$ and  $\mathbf{R}_f = R_f - R_f(0)$ is the displacement of the atomic positions from equilibrium. $M_R$ is the mass of the atom located at position $R$, $M_k$ is the reduced mass in the $k$th phonon mode, and $\mu_k(R)$ is the phonon mode displacement vector at position $R$.

In Eq. (\ref{app:nonrad:eq:eph2}), we have defined the electron-electron coupling constants $C^k_{if}$.
\begin{equation}
    C^k_{if}
    = \braket{\psi_f(r,R_0)|
    \frac{\partial H}{\partial Q_k}|\psi_i(r,R_0)}
    =\sum_R \mu_k(R) \braket{\psi_f(r,R_0)|
    \frac{\partial H}{\partial R}|\psi_i(r,R_0)}
\end{equation}
Now for the full transition rate we have
\begin{eqnarray}
    \Gamma_{i\rightarrow f}&=& \frac{2\pi}{\hbar}\sum_{k_1,k_2} C^{k_1}_{if} C^{k_2}_{if}
    \bigg( \sum_{n,m} f(i,n) \braket{\phi_{i,n}(R)|
    \mathbf{Q}_{k_1}|\phi_{f,m}(R)} \nonumber \\
    && \hspace{6mm} \cdot
    \braket{\phi_{f,m}(R)|\mathbf{Q}_{k_2}|\phi_{i,n}(R)}
    \delta(\hbar\omega_{in}-\hbar\omega_{fm}-\Delta E_{if}) \bigg) \label{app:nonrad:eq:g3}
\end{eqnarray}
If we use the integral form of the dirac delta function $\delta(x)=\frac{1}{2\pi}\int_{-\infty}^{\infty}e^{ixt}dt$
then we can reduce the phonon-phonon coupling piece of Eq. \ref{app:nonrad:eq:g3} (see Appendix for derivation):
\begin{equation}
    \Gamma_{if}=\frac{2\pi}{\hbar} \sum_{k1,k2} C^{k_1}_{if}C^{k_2}_{if}\cdot A^{k1,k2}_{if} \label{app:nonrad:eq:g4}
\end{equation}
where,
\begin{align}
    A^{k1,k2}_{if}=\frac{1}{2\pi \mathcal{Z}}\int_{-\infty}^{\infty}\chi^{k_1,k_2}_{if}(t,T)e^{-it\Delta E_{if}/\hbar}\,dt \label{app:nonrad:eq:a}\\
    \chi^{k_1,k_2}_{if}(t,T) = \text{Tr}\left[ \mathbf{Q}_{k_1} e^{-it H_{f}/\hbar}
    \mathbf{Q}_{k_2}e^{-(\beta\hbar-it)H_{i}/\hbar} \right] \label{app:nonrad:eq:chi}
\end{align}


\section{Full-Phonon}
This section discusses details of computing Eq. \ref{app:nonrad:eq:a} \& \ref{app:nonrad:eq:chi} in practice,  following the implementation of [Shi 2015 PRB]. First of all we will assume that the phonon modes in states $i$ and $j$ are the same, so $k_1 = k_2 = k$. Next we introduce the following diagonal ($N_{vib} \times N_{vib}$) matrices:
\begin{align}
    a(\tau_\xi)_k = \frac{\omega_k}{\sinh{(i\hbar\omega_k\tau_\xi)}}\, , \quad
    c(\tau_\xi)_k = \omega_k\coth{(i\hbar\omega_k\tau_\xi/2)}\, , \quad \nonumber \\
    d(\tau_\xi)_k = \omega_k\tanh{(i\hbar\omega_k\tau_\xi/2)}\, .
\end{align}
Where, $\xi = (i,j)$, $\tau_i = -t-i\beta$, $\tau_j = t$, and $\omega_k$ is the frequency of the $k$th harmonic oscillator. We then define matrices:
\begin{align}
    C(\tau_i,\tau_j)_k = c(\tau_i)_k + c(\tau_j)_k\, , \quad
    D(\tau_i,\tau_j)_k = d(\tau_i)_k + d(\tau_j)_k\, .
\end{align}
And also
\begin{align}
    D_\text{HT} &= -D^{-1} d(\tau_j) \textbf{K} \, , \\
    A_\text{HT} &= \frac{1}{2} (D^{-1} - C^{-1}) + D_\text{HT}(D_\text{HT})^T\, ,
\end{align}
where
\begin{align}
    \textbf{K}_k = \Delta Q_{ij,k} = \frac{1}{\sqrt{M_k}} \sum_R M_R \mu_k(R) \Delta R_{ij}\, .
\end{align}
This gives the final form:
\begin{align}
    \chi_{ij}^k(t,T) &=
        \sqrt{
            \frac{
                \det{[a(\tau_j)]} \det{[a(\tau_i)]}
            }{
                (i\hbar)^{2N} \det{(C)} \det{(D)}
            }
        }\, \nonumber \\
        & \hspace{2cm} \times \exp \left[
            -\textbf{K}^T d(\tau_j)\textbf{K}
            + \textbf{K}^T d(\tau_j) D^{-1} d(\tau_i)\textbf{K}
        \right]
        (A_\text{HT}) \label{app:nonrad:eq:final}
\end{align}
One can then integrate Eq. \ref{app:nonrad:eq:a} to give the final phonon part.


\section{Linear Response Theory}

Here we consider the single particle Hamiltonian ($h$) to first order deviation in the one-dimensional effective coordinate ($Q$) as:

\begin{align}
    h=h_{a}+\frac{\partial h}{\partial Q} (Q-Q_{a})
\end{align}

We can consider the latter term as a perturbation on the system where only the term $\vhq$ acts on the electronic states ($Q$ acts on phonon states). Therefore, the first-order response of the electronic eigenstate ($\varphi_m$) is given by:

\begin{align}
    \ket{\Delta\varphi_m} = \sum_{n\neq m} \ket{\varphi_n} \frac{\bra{\varphi_n} \vhq \ket{\varphi_m}}{\varepsilon_m-\varepsilon_n} \label{app:nonrad:eq:p}
\end{align}

We now work to solve for $\bra{\varphi_n} \vhq \ket{\varphi_m}$, the term we want to replace in the current formalism. First consider a simple rewrite of Eq. \ref{app:nonrad:eq:p}.

\begin{align}
    \ket{ \Delta\varphi_m} =  \left(\sum_{n\neq m} \ket{\varphi_n}\bra{\varphi_n}\right)\frac{ \vhq \ket{\varphi_m}}{\varepsilon_m-\varepsilon_n}
\end{align}
Evoking the completeness relation gives
\begin{align}
    \ket{ \Delta\varphi_m} = \bigg(\mathbb{1}- \ket{\varphi_m}\bra{\varphi_m}\bigg)\frac{ \vhq \ket{\varphi_m}}{\varepsilon_m-\varepsilon_n}
\end{align}
Then taking the inner product with $\bra{\varphi_n}$ and implementing the orthogonality of these states $\braket{\varphi_n|\varphi_m}=\delta_{nm}$ (in this case $n$ and $m$ differ, so $\braket{\varphi_n|\varphi_m}=0$).
\begin{align}
    \braket{\varphi_n|\Delta\varphi_m} &= \bigg(\bra{\varphi_n}- \braket{\varphi_n|\varphi_m}\bra{\varphi_m}\bigg)\frac{ \vhq \ket{\varphi_m}}{\varepsilon_m-\varepsilon_n} \\
    &= \bigg(\bra{\varphi_n}\bigg)\frac{ \vhq \ket{\varphi_m}}{\varepsilon_m-\varepsilon_n} \\
    &= \frac{\bra{\varphi_n} \vhq \ket{\varphi_m}}{\varepsilon_m-\varepsilon_n}
\end{align}
This gives the final form we desired ($n=i$ initial state; $m=f$ final state)
\begin{align}
    \bra{\varphi_i} \vhq \ket{\varphi_f} = (\varepsilon_f-\varepsilon_i)\braket{\varphi_i|\frac{\partial \varphi_f}{\partial Q}}
\end{align}
Note that $\varphi_f$ is also considered to change first order in $Q$ and hence $\ket{\Delta\varphi_f}=\ket{\frac{\partial \varphi_f}{\partial Q}}$.


\section{Supplemental Derivations}

\subsection{S1}
Proof that
\begin{align}
    \braket{\psi_f(r,R_0)|H_0|\frac{\partial \psi_i}{\partial R}}
    + \braket{\frac{\partial \psi_f}{\partial R}|H_0 |\psi_i(r,R_0)}  =  0 .
\end{align}
Consider,
\begin{align}
    & \frac{\partial}{\partial R}\bigg(\braket{\psi_f(r,R_0)|H_0 |\psi_i(r,R_0)}\bigg) = \\
    & \hspace{2cm} \braket{\frac{\partial \psi_f(r,R_0)}{\partial R}|H_0 |\psi_i(r,R_0)}
    + \braket{\psi_f(r,R_0)|H_0 |\frac{\partial \psi_i(r,R_0)}{\partial R}} \nonumber \\
    & \hspace{2cm} + \braket{\psi_f(r,R_0)|\frac{\partial H_0}{\partial R} |\psi_i(r,R_0)} \nonumber \\
    & 0 = \braket{\frac{\partial \psi_f(r,R_0)}{\partial R}|H_0 |\psi_i(r,R_0)}
    + \braket{\psi_f(r,R_0)|H_0 |\frac{\partial \psi_i(r,R_0)}{\partial R}}
    + 0 \nonumber \\
    & \Rightarrow \braket{\frac{\partial \psi_f(r,R_0)}{\partial R}|H_0 |\psi_i(r,R_0)}
    + \braket{\psi_f(r,R_0)|H_0 |\frac{\partial \psi_i(r,R_0)}{\partial R}} = 0
\end{align}
Here the left hand side is zero because $\braket{\psi_{f}|\psi_{i}}=0$, while the final term on the right hand side is zero because $\partial H_0/\partial R = 0$.

\subsection{S2}
Below the mathematical steps which allow for the rewriting of Eq. (\ref{app:nonrad:eq:g3}) in terms of Eq. (\ref{app:nonrad:eq:g4}-\ref{app:nonrad:eq:chi}) are presented.
\begin{align}
    &\sum_{n,m} f(i,n) \braket{\phi_{i,n}(R)|
    \mathbf{Q}_{k_1}|\phi_{f,m}(R)} \cdot
    \braket{\phi_{f,m}(R)|\mathbf{Q}_{k_2}|\phi_{i,n}(R)}
    \delta(\hbar\omega_{fm}-\hbar\omega_{in}+\Delta E_{if}) \nonumber \\
    &= \frac{1}{2\pi \hbar\mathcal{Z}} \sum_{n,m} e^{-\beta \hbar\omega_{in}} \braket{\phi_{i,n}(R)|
    \mathbf{Q}_{k_1}|\phi_{f,m}(R)} \cdot
    \braket{\phi_{f,m}(R)|\mathbf{Q}_{k_2}|\phi_{i,n}(R)} \nonumber \\
    &\hspace{7cm}\cdot \int_{-\infty}^{\infty}
    e^{it(\omega_{in}-\omega_{fm}-\Delta E_{if}/\hbar)} \,dt\nonumber \\
    &= \frac{1}{2\pi \hbar\mathcal{Z}} \int_{-\infty}^{\infty} \bigg(\sum_{n,m} \braket{\phi_{i,n}(R)|
    \mathbf{Q}_{k_1}|\phi_{f,m}(R)} \cdot
    \braket{\phi_{f,m}(R)|\mathbf{Q}_{k_2}|\phi_{i,n}(R)} \nonumber \\
    &\hspace{7cm}\cdot
    e^{-(\beta\hbar-it)\omega_{in}-it\omega_{fm}-it\Delta E_{if}/\hbar)} \bigg) \,dt\nonumber \\
    &= \frac{1}{2\pi \hbar\mathcal{Z}} \int_{-\infty}^{\infty} \bigg(\sum_{n,m} \braket{\phi_{i,n}(R)|\mathbf{Q}_{k_1}e^{-it\omega_{fm}}|\phi_{f,m}(R)} \nonumber \\
    &\hspace{4.5cm} \cdot    \braket{\phi_{f,m}(R)|\mathbf{Q}_{k_2}e^{-(\beta\hbar-it)\omega_{in}}|\phi_{i,n}(R)}    e^{-it\Delta E_{if}/\hbar)} \bigg) \,dt\nonumber \\
    &= \frac{1}{2\pi \hbar\mathcal{Z}} \int_{-\infty}^{\infty} \bigg(\sum_{n} \bra{\phi_{i,n}(R)}\mathbf{Q}_{k_1} \nonumber \\
    &\hspace{3cm}\sum_m e^{-it \omega_{fm}}\ket{\phi_{f,m}(R)} \cdot    \bra{\phi_{f,m}(R)} \nonumber \\
    & \hspace{6.2cm} \mathbf{Q}_{k_2}e^{-(\beta\hbar-it)\omega_{in}}\ket{\phi_{i,n}(R)}    e^{-it\Delta E_{if}/\hbar} \bigg) \, dt \nonumber \\
    &= \frac{1}{2\pi \hbar\mathcal{Z}} \int_{-\infty}^{\infty} \bigg(\sum_{n} \bra{\phi_{i,n}(R)}\mathbf{Q}_{k_1} e^{-it H_{f}/\hbar} \mathbf{Q}_{k_2}e^{-(\beta\hbar-it)\omega_{in}}\ket{\phi_{i,n}(R)}    e^{-it\Delta E_{if}/\hbar} \bigg) \, dt \nonumber \\
    &= \frac{1}{2\pi \hbar\mathcal{Z}} \int_{-\infty}^{\infty} \text{Tr} \left[ \mathbf{Q}_{k_1} e^{-it H_{f}/\hbar}
    \mathbf{Q}_{k_2}e^{-(\beta\hbar-it)H_{i}/\hbar} \right]
    e^{-it\Delta E_{if}/\hbar} \, dt
\end{align}
Plugging this piece back into Eq. \ref{app:nonrad:eq:g3} gives the final condensed form of the transition rate.
\begin{equation}
    \Gamma_{if}=\frac{2\pi}{\hbar} \sum_{k1,k2} C^{k_1}_{if}C^{k_2}_{if}\cdot A^{k1,k2}_{if}
\end{equation}
Where we have shown that
\begin{align}
    A^{k1,k2}_{if}=\frac{1}{2\pi \mathcal{Z}}\int_{-\infty}^{\infty}\chi^{k_1,k_2}_{if}(t,T)e^{-it\Delta E_{if}/\hbar}\,dt \\
    \chi^{k_1,k_2}_{if}(t,T) = \text{Tr}\left[ \mathbf{Q}_{k_1} e^{-it H_{f}/\hbar}
    \mathbf{Q}_{k_2}e^{-(\beta\hbar-it)H_{i}/\hbar} \right]
\end{align}
