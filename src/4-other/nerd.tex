\section{Optimizing Workflow as a Computational Scientist}
% Non-scientific skills worth learning and sharing
% Why develop these non-scientific skills?
%     - improve productivity
%     - minimize work stress both mental and physical
%     - learning and taking advantage of these strategies makes work more fun!
%
% [1] VIM
%     - basic and advanced vim usage
%     - use hjkl
%     - use /, f, t
%     - v, shift-v, ctrl-v
%     - macros
%     - vim bindings command line and vimium
% [2] command line
%     - zsh - autocomplete and syntax highlighting
%     - fzf - better history search
%     - tmux - save and organize your ssh sessions
% [3] git
%     - for organizing projects (example hbn)
%     - backup!!
%     - same exact configuration on every machine
% [4] python
%     - pandas = easy way to organize and process data
%     - pw2py = atomgeo, qeinp, qeout, qesave examples...
%     - pymatgen
%     - how to use numpy (numpy dot product v. manual dot product v. c++ dot product)
% [5] other (useful commands)
%     - convert
%     - tree
%     - tail -f (perfect for watching output of a calculation)
%     - tee (another way of watching output of a calculation)
%     - watch (great for monitoring squeue)
%     - tldr (quick check how to use command, tar is a good example)
