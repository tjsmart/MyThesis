\begin{abstract}
    State-of-the-art first-principles calculations are implemented and utilized in order to optimize or give insight into material properties for various technologies and applications. These applications, and therefore the research, have two primary focuses. The first is energy conversion and storage using transition metal oxides and the second is quantum defects in two-dimensional materials as qubits and single photon emitters.

    Regarding the first, the limitation of transition metal oxides is in their poor carrier conduction due to the formation of small polarons. Here I have developed our understanding of how to compute small polaron properties such as optical absorption, polaron transport, and carrier concentrations, especially in the presence of dopants and defects. Polaron transport has been investigated under both macroscopic (dielectric continuum) and microscopic (explicit hopping) models within small polaron theory. Primary focus has been given to the \ce{Fe2O3} system, where I have extensively researched polaron formation in the presence of many intrinsic defects and external dopants. Furthermore, I have discovered a novel form of dopant clustering mediated by small polaron and dopant interactions which has been experimentally validated. Outside of \ce{Fe2O3}, I have demonstrated the origin of the optical gap in \ce{Co3O4} is due to small hole polaron formation, and in \ce{CuO} I have detailed the formation and transport of spin polarons. Lastly, in various other systems (\textit{e.g.} \ce{BiFeO3} and \ce{LaFeO3}) we have provided insight on how dopants impact polaron properties which are relevant to experimentally observed enhancements.

    For the second research focus, the design of single photon emitters and spin-based qubits in hexagonal boron nitride is presented. Here, both static and dynamic properties are computed for the first time for defects in two-dimensional materials. For static properties, we have developed methods to deal with more accurate electron correlation beyond standard density functional theory (e.g. GW approximation and hybrid functional) and charged defect interaction for systems with  highly anisotropic and weak dielectric screening. I have also implemented computing the zero-field splitting of $S \ge 1$ systems, an essential quantity in defect based-qubit systems like NV center in diamond. For dynamical properties, we included exciton-defect coupling for radiative lifetime from solving the Bethe-Salpeter equation and electron-phonon interactions for nonradiative lifetime. In addition, I have implemented computing intersystem crossing (necessary for spin initialization and readout) with spin-orbit coupling and electron-phonon interaction. With these computed static and dynamical properties,  we are able to predict spin qubits read-out efficiency and new quantum spin defect systems in hexagonal boron nitride which can be potential candidates for spin-based quantum technologies.
\end{abstract}
