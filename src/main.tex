\documentclass[12pt]{../include/ucthesis}
% This needs to be added before hyperref
\usepackage[papersize={8.5in,11in},inner=1.5in,outer=1.25in,top=1.25in,bottom=1.25in]{geometry}

\usepackage{fixltx2e}
\usepackage{url}
\usepackage{amsmath, amsthm, amssymb}
\usepackage{color}
\usepackage{xspace}
\usepackage{verbatim}
\usepackage{graphicx}
\usepackage{psfrag}
\usepackage{pbox}
\usepackage{epstopdf}
\usepackage{textcomp} % For copyright symbol
\DeclareGraphicsExtensions{.pdf,.eps,.png,.jpg}
\usepackage[font=singlespacing]{subcaption}
\captionsetup[table]{labelfont=bf,font=singlespacing}
\captionsetup[figure]{labelfont=bf,font=singlespacing}
\captionsetup[subfigure]{labelfont=bf,font=singlespacing}
\usepackage[titletoc,title]{appendix}
\usepackage{tabularx}
\usepackage{rotating}
\usepackage{hyperref} % Turns all internal references into links to pages within the document
\hypersetup{pdfborder={0 0 0}} % removes red box from links

\usepackage{lmodern} % Use the Latin Modern font
\usepackage[T1]{fontenc} % Use a modern font encoding

\makeatletter % Hack to get setspace working properly
\let\@currsize\normalsize
\makeatother
\usepackage{setspace}% http://ctan.org/pkg/setspace

% Set default spacing to double (yes, it should be 1.6 here...)
\linespread{1.6}

%%%%%%%%%%%%%%%%%%%%%%%%%%%%%%%%%%%%%%%%%%%%%%%%%%%%%%%%%%%%
% my added packages
\usepackage[version=4]{mhchem}
\usepackage{float}
\usepackage{placeins}
\usepackage{array}
\usepackage{threeparttable}
\usepackage[symbol]{footmisc}
\usepackage{physics}
\usepackage{makecell}
\usepackage{siunitx}
\usepackage{xr}
\usepackage{xstring}
\usepackage{multirow}
\usepackage{upgreek}
\usepackage{mathtools}
\usepackage{braket}
\usepackage{bbold}


\begin{document}

\title{First-Principles Computational Design of Materials: From energy conversion to quantum information}
\author{Tyler J. Smart}
\degreeyear{2021}
\degreemonth{June}
\degree{DOCTOR OF PHILOSOPHY}
\chair{Prof.\ Frank Bridges}
\committeememberone{Prof.\ Yuan Ping}
\committeemembertwo{Prof.\ Yat Li}
%\committeememberthree{Professor 4} % Uncomment if you have 4 committee members. Also change the next line to `\numberofmembers{4}`
\numberofmembers{3}
\deanlineone{Quentin Williams}
\deanlinetwo{Interim Vice Provost and Dean of Graduate Studies}
\deanlinethree{}
\field{Physics}
\campus{Santa Cruz}

\begin{frontmatter}
\maketitle

\copyrightpage

\tableofcontents

\listoffigures

\listoftables

\begin{abstract}
    State-of-the-art first-principles calculations are implemented and utilized in order to optimize or give insight into material properties for various techonologies and applications. These applications, and therefore the research, have two primary focuses. The first is energy conversion and storage using transition metal oxides and the second is quantum defects in two-dimensional materials. Regarding the first, the limitation of transition metal oxides is in their poor carrier conduction due to the formation of small polarons. Here, detail and insight are presented into the formation of small polarons, especially under atomic doping wherein doping can help to improve carrier conduction in these systems and offer better energy conversion. For the second application, an emphasis in the design of single photon emitters and spin-based qubits in hexagonal boron nitride is presented. In particular, defect ionization energies, optoelectronic properties, and nonradiative propreties, including intersystem crossing, are discussed. Finally, selected topics are presented which do not necessarily fall into these two categories but is otherwise interesting to discuss.
\end{abstract}

\begin{dedication}
\vspace*{\fill}
\begin{center}
    To my wife, \\
    Channey Lane Smart.
\end{center}
\vspace*{\fill}
\end{dedication}

\begin{acknowledgements}
    The research presented here is not the result of my effort alone. Here I make my best attempt to acknowledge those whose impact was pivotal in bringing this research to fruition.

    First, and foremost, I would like to thank my advisor Prof.\ Yuan Ping. Her passion, wisdom, and creativity is unparalleled, I am extremely grateful for all that she has taught me in these five years. I am thankful to Dr.\ Feng Wu, who was an amazing mentor to me, his patience and excitement in sharing his knowledge was very inspiring to me. I would also like to thank Mingpeng Chen, who not only provided me incredible insight into experiment but also was an amazing friend to share a meal with. I would like to thank Kejun Li, who I worked on various two-dimensional material projects and Valentin Urena Baltazar, who I worked with on various transition metal oxide projects. To the entire Ping Group I am thankful for the many conversations we had in which I learned so much.

    Next, I thank several collaborators outside of the Ping Group. Dr.\ Tuan Anh Pham, was an amazing mentor to me, going beyond just research advice and often providing me with invaluable life and career advice. I thank him and Dr.\ Tadashi Ogitsu both for providing me the opportunity to collaborate with LLNL and learn so much from them. I am thankful Dr.\ Marco Govoni at Argonne National Laboratory for facilitating the opportunity to be apart of the Quantum Information Science and Engineering Network. I am thankful of several experimentalist whose discussions and collaborations with I learned a lot from including, but not limited to, Prof.\ Kyoung-Shin Choi, Prof.\ Yat Li, Prof.\ Shaowei Chen, Prof.\ Frank `Bud' Bridges, and also their graduate students such as, Dr.\ Allison Cardiel, Dr.\ Tianyi Kou, Dr.\ Bingzhang Lu.

    I am thankful to my committee members, Prof.\ Frank `Bud' Bridges, Prof.\ Yat Li, and Prof.\ Yuan Ping, in particular Bud for agreeing to be the chair and my department advisor. There are also many members of the physics department whose friendship and insights were invaluable to my PhD experience. Specifically, I am thankful to Dr.\ Ben Miller, Dr.\ Katie Hellier, and Prof.\ Stephano Profumo who provided me with a much needed guidance during my PhD. I thank Carey Williams and all the students in my year, for the fun times and a terrific friendship.

    I am thankful to physics department faculty at Humboldt State University who taught me physics such as Prof.\ Monty Mola, Prof.\ Wes Blivens, and Prof.\ C.D.\ Hoyle. I still remember the day I decided I wanted to major in Physics from a short discussion with Monty.

    Lastly, I am thankful to my family. My mother and father, who taught me that trying your best is the best way to enjoy life. My wife, Channey for being my best friend, and the ultimate source of wisdom and support. And my son Alden, who made life enjoyable and fun, even as I finished out the PhD during the coronavirus pandemic.

    Oh, and thank you to the voice who told me that ``nobody enjoys their PhD'', proving you wrong lit a scorching fire inside of me and was my greatest success. :)
\end{acknowledgements}

\end{frontmatter}

% ------------------------------------------------------------
% intro
% //////////////////////////////////////////////////////////|
% ++++++++++++++++++++++++++++++++++++++++++++++++++++++++++|
% %%%%%%%%%%%%%%%%%%%%%%%%%%%%%%%%%%%%%%%%%%%%%%%%%%%%%%%%%%|
%    CHAPTER                                                |
% %%%%%%%%%%%%%%%%%%%%%%%%%%%%%%%%%%%%%%%%%%%%%%%%%%%%%%%%%%|
% ++++++++++++++++++++++++++++++++++++++++++++++++++++++++++|
% \\\\\\\\\\\\\\\\\\\\\\\\\\\\\\\\\\\\\\\\\\\\\\\\\\\\\\\\\\|
\chapter{Introduction}

% general "big picture" motivation, will introduce briefly both renewable energy and quantum information science motivations
Understanding and developing material properties is essential to solving some of societies greatest concerns. One such concern of particular interest is the desperate need for renewable energy sources.
Advancing novel technologies for energy harvesting, conversion, and storage is critical to ensure the economic viability of U.S. energy and chemical industries \cite{eia,renewable}. For many of these technologies, a detailed understanding of chemical processes at electrochemical interfaces is essential. For instance, optimizing water splitting reactions at the semiconductor-water interface in photoelectrochemical cells is key for improving the device efficiency and stability for generating hydrogen fuel from water and sunlight \cite{mccrory2015}. Alternatively, the causes of oxidation and corrosion at the interface can be illuminated via chemical degradation processes \cite{huang2019}. Last but not least, understanding the relationship between reactivity and electronic properties of liquid electrolytes at the interface with electrode materials is also one of the prerequisites for manipulating the electrochemical stability of electrode-electrolyte interfaces in ion batteries and supercapacitors \cite{wang2018}.

However, the development of renewable energy resources is only a single example of the type of problem in which material design is crucial. As another example, material science is unsurprisingly at the very heart of developing brand new technoligies, particularly those in the realm of quantum information science.
% IBM quantum computer – superconducting qubits


% copied bs.
Meanwhile, the development of innovative quantum technologies is immanent and will make broad impacts on our national technology sector \cite{quantum}. With the intention to expand these fields, the role of computational science has grown immensely alongside ever-growing supercomputing facilities. These facilities enable calculations of large-scale simulations which provide improved theoretical understanding of these aforementioned fields. In particular, first-principles simulations allow us to better understand the quantum mechanical nature of materials, which is an essential part of their application. These simulations have proven to be pivotal in the evolution of many fields all the way from renewable energy to quantum technology.

As such, my research in modeling materials from first-principles calculations is bifurcated into two branches of motivation: 1. renewable energy and energy conversion with transition metal oxides (TMOs) and 2. quantum information sciences in two-dimensional (2D) materials. In this proposal I will discuss my research within both of these fields by covering the motivation behind my research, discussing progress achieved thus far and finally how these efforts are naturally extending and culminating in my dissertation year.


%
The rational design of materials by accurate yet feasible computational approach can offer researchers the ability to discover and describe otherwise unattainable phenomena. Specifically in the era of supercomputers, large-scale calculations have reached the stage of being more obtainable than ever before, heightening the interest in the computational approach to science. Here, large-scale may refer to pushing the limits of computational resources in terms of processing power, computer memory, or simply time to complete the task. Specifically the past 10 years, the scale of these computational resources has grown exponentially. And perhaps to no surprise, so has humanities interest in utilizing these resources to achieve tasks once never believed possible.
% <May put here a graph of supercomputer resources etc.>


The task which is relevant to this dissertation is that of the multi-electron problem. In the interrelated field of physics, chemistry, and material science,  the multi-electron problem essentially refers to any problem beyond involving interactions between more than one electron, often in an external field.


% %%%%%%%%%%%%%%%%%%%%%%%%%%%%%%%%%%%%%%%%%%%%%%%%%%%%%%%%%%
%    section
% %%%%%%%%%%%%%%%%%%%%%%%%%%%%%%%%%%%%%%%%%%%%%%%%%%%%%%%%%%
\section{The Multi-Electron Problem}

\subsection{Hohenberg-Kohn-Sham Theory}

The many-electron problem refers to any



% %%%%%%%%%%%%%%%%%%%%%%%%%%%%%%%%%%%%%%%%%%%%%%%%%%%%%%%%%%
%    section
% %%%%%%%%%%%%%%%%%%%%%%%%%%%%%%%%%%%%%%%%%%%%%%%%%%%%%%%%%%
\section{Formalism of Charged Defect Formation}

Add intro here, may change dependening... Reword the below a bit.

Defect formation energies and charge transition levels are the fundamental properties of defects in solids, where the former represents how easily a defect can form at a specific chemical condition and the latter represents how easily the defect can be ionized or excited.

\subsection{Elemental Chemical Potentials}
% the discussion here is similar to https://pubs.rsc.org/en/content/articlepdf/2013/cp/c3cp53311e and https://pubs.acs.org/doi/pdf/10.1021/acs.jpcc.5b08081 but also https://pubs.acs.org/doi/pdf/10.1021/acs.chemmater.8b05362
The chemical potential of Fe and O was evaluated from first principles following the approach employed in Ref.~\cite{lee2013thermodynamics}.
Namely, for \ce{Fe2O3} in thermodynamic equilibrium growth conditions, the chemical potentials $\mu_{\rm Fe}$ and $\mu_{\rm O}$ must satisfy Eq.~\ref{eq:chem_fe2o3}$-$\ref{eq:chem_poor}:
\begin{align}
    2\Delta \mu_{\rm Fe} + 3\Delta \mu_{\rm O} = \Delta H_{\ce{Fe2O3}} \label{eq:chem_fe2o3} \\
    2\Delta \mu_{O} \leq  \Delta H_{\ce{O2}} \label{eq:chem_rich} \\
    3\Delta \mu_{\rm Fe} + 4\Delta \mu_{\rm O} \leq \Delta H_{\ce{Fe3O4}} \label{eq:chem_poor}
\end{align}
where $\Delta \mu_{\rm i}$ is the chemical potential of species $i$ referenced to its most stable phase.
In an O rich environment, \ce{Fe2O3} will be in equilibrium with \ce{O2} as in Eq.~\ref{eq:chem_rich}, whereas in an O poor environment (Fe rich), \ce{Fe2O3} will be in equilibrium with \ce{Fe3O4} as in Eq.~\ref{eq:chem_poor}.
In order to correct the well-known overbinding problem of O$_2$ by DFT, we use the experimental value of 5.23 eV for the binding energy of \ce{O2} in order to correct the total energy of \ce{O2} computed by DFT. \cite{wang2006oxidation}
Meanwhile, the chemical potential of each tetravalent dopant $X$ was computed as in Ref.~\cite{zhou2015understanding}, where after obtaining the chemical potential of O in the rich/poor limit, the chemical potential $\Delta \mu_{\rm X}$ can be computed from Eq.~\ref{eq:chem_x}.
\begin{align}
    \Delta \mu_X + 2\Delta \mu_{\rm O} \leq \Delta H_{X{\rm O}_2} \label{eq:chem_x}
\end{align}


\subsection{Defect Formation Energy and Ionization Energy}
The charge defect formation energy provides insight into the charge states of dopants providing some insight into the influence on carrier concentration.
We computed the charge defect formation energy ($E^f_q$) of each defect system according to:
\begin{align}
    E^f_q(X; \varepsilon_F) = E_q(X) - E_{prist} + \sum_i \mu_i \Delta N_i + q \varepsilon_F + \Delta_q,
    \label{eq:cfe}
\end{align}
where $E_q(X)$ is the total energy of the defect system ($X$) with charge $q$, $E_{prist}$ is the total energy of the pristine system, $\mu_i$ and $\Delta N_i$ are the chemical potential and change in the number of atomic species $i$, and $\varepsilon_F$ is the electron chemical potential. A charged defect correction $\Delta_q$ was computed for charged cell calculations with the JDFTx code \cite{JDFTx} by employing the techniques developed in Ref.~\cite{wu2017first,Ping2013}. Meanwhile, the chemical potentials were carefully evaluated against the stability of byproduct compounds as detailed above.
Finally, the corresponding charge transition levels of the defects can be obtained from the value of $\varepsilon_F$ where the stable charge state transitions from $q$ to $q'$.
\begin{align}
    \epsilon_{q|q'} = \frac{E^f_q - E^f_{q'}}{q' - q}
    \label{eq:ctl}
\end{align}
Typically, for a semiconductor or insulator the ionization energy of a p-type/n-type dopant is given by the value(s) of its charge transition level(s) referenced to the valence/conduction band edge of the host materials. However, in systems which form small polarons the ionization energy should be referenced to the free polaron state. \cite{seo2018role} Here the free electron small polaron level is defined as the $\epsilon_{0|-1}$ transition level in the pristine system. In this way the free polaron level was computed and all ionization energies of n-type defects/dopants in this paper are referenced to this level.



\subsection{Defect Concentration}
In order to relate the effects of doping directly to polaron concentrations, we can compute the defect concentrations self-consistently by evaluating charge neutrality based on the formation energy of each defect. Following the formalism presented in Ref.~\cite{freysoldt2014first}, the charged defect concentration ($c_q$) is computed as:
\begin{align}
    c_q(X; \varepsilon_F) = g \exp [- E^f_q(X; \varepsilon_F) / k_B T],
\end{align}
where $g$ is the degeneracy factor accounting for the internal degrees of freedom of the point defect, $k_B$ is the Boltzmann factor, and $T$ is temperature. In order to maintain neutrality, the introduction of defect $X$ with charge $q$ into the lattice must be compensated by defects of opposing charge or through the generation of free carriers. Specifically charge neutrality must be held:
\begin{align}
    \sum_{X,q} c_q(X; \varepsilon_F) + n_h - n_e = 0, \label{eq:neutrality}
\end{align}
where the concentration of free delocalized holes ($n_h$) and free delocalized electrons ($n_e$) can be evaluated via:
\begin{align}
    n_e - n_h = \int_{-\infty}^{\infty} dE \frac{D(E)}{1+\exp[(E-\varepsilon_F)/k_B T]}.
\end{align}
Here $D(E)$ is the electronic density of states of the pristine system. Eq.~\ref{eq:neutrality} can be evaluated by standard root-finding algorithms to obtain $\varepsilon_F$ where charge neutrality is held.
Note, in hematite free electrons will localize into small polarons, and so the formation of free electron small polarons is entered in a similar way to a defect, i.e.\ with a formation energy.
Finally, in order to relate to experimental measurements of \ce{Fe2O3} photoanodes, we computed the concentration first at a synthesis temperature of $T_S=1073$ K (800 $^\circ$C is a common synthesis temperature \cite{ling2011sn,tian2020electronic}), and then recomputed charge neutrality at normal operation (room) temperature $T_O=300$ K while maintaining defects total concentrations similar to methods employed by Ref.~\cite{lee2013thermodynamics}.



% ------------------------------------------------------------
% tmo
\chapter{Designing Efficient Transition Metal Oxides}

% %%%%%%%%%%%%%%%%%%%%%%%%%%%%%%%%%%%%%%%%%%%%%%%%%%%%%%%%%%
%    Section
% %%%%%%%%%%%%%%%%%%%%%%%%%%%%%%%%%%%%%%%%%%%%%%%%%%%%%%%%%%
\section{Overview}
Solar water splitting, the process of utilizing the Sun's energy by converting water to hydrogen fuel represents the most promising and sustainable means of harvesting renewable energies \cite{yao2018photoelectrocatalytic,tachibana2012artificial,walter2010solar}. In particular, many transition metal oxide (TMO) based photoelectrochemcial (PEC) devices (such as \ce{Fe2O3} and \ce{BiVO4}) are capable of achieving the highest solar-to-hydrogen efficiencies and offer the greatest avenue of developing this field, while simultaneously utilizing cheap and abundant sources \cite{kim2019elaborately,tamirat2016using,lee2019progress}.
Unfortunately, progress within this field has been stunted by the formation of small polarons (hereinafter referred to simply as polarons) which conduct via a thermally activated hopping mechanism and yield very poor conduction when compared to typical band conduction \cite{bosman1970small}. Hence, polarons are the chief bottleneck in the development of TMO-based PEC devices, as poor conduction leads to low efficiency.
% Small polarons represent carriers which induce large lattice polarization result in carrier wavefunctions which are spatially small (on the order of a unit cell).


Therefore, my research focuses on the development of our understanding of polarons in TMOs and the discovery of methods for facilitating polaron conduction in TMOs for improved solar-to-hydrogen efficiency. In particular, atomic doping in TMOs has been highly successful for improving the solar-to-hydrogen efficiency of TMOs. For example, in my first publication we studied how several defects in \ce{Fe2O3} influence carrier conduction \cite{smart2017effect}. This work formulated methods for studying polaron transport in doped systems and showed how Sn doping in \ce{Fe2O3} can improve carrier concentration. Following this, we developed mechanistic insights into how Li doping facilitates novel spin polaron conduction in CuO, which was validated by experimental measurement of enhanced photocathode performance by Li-doped CuO \cite{smart2018mechanistic}. Recently, our work demonstrating how the intrinsic band gap of \ce{Co3O4} had been misidentified due to the formation of optically active hole polarons caught a lot of attention \cite{smart2019optical}. In particular, this work received a Lawrence Livermore National Laboratory (LLNL) summer student poster symposium award, was highlighted in LLNL notable news, and was pivotal in my reception of the Graduate Student Scholar Program fellowship from LLNL. Finally, I have collaborated in many experimental works where our abilities to simulate polarons \cite{wheeler2019combined,radmilovic2020combined} and chemical reactions \cite{lu2017,kou2018,kou2019,lu2019} have proven vital in our understanding of how atomic doping enhances the performance of these materials.

\def\vo{\text{V}_\text{O}}

\section{Polaron Formation and Transport in \ce{Fe2O3}}

In our 2017 work published in Journal of Physics: Condensed Matter~\cite{smart2017effect}, we investigated polaron formation and transport in \ce{Fe2O3}, demonstrating the notation of a `free polaron' level and detailing polaron hopping in pristine and doped \ce{Fe2O3}.
Several transition metal oxides, such as TiO$_2$, BiVO$_4$, WO$_3$ and hematite \ce{Fe2O3}, have exemplified some of the desirable characteristics of an efficient photoanode for solar-driven photoelectrochemical (PEC) water splitting \cite{alexander2008metal,sivula2011solar}. In particular, hematite (\ce{Fe2O3}) is a low cost, earth abundant, n-type semiconductor with a relatively smaller band gap in the visible range ($\sim\,$2 eV) compared with other oxides; thus it has been established as a promising candidate for photoanodes. However, application of hematite as a photoanode has been hindered from low carrier denstiy, low carrier mobility and high electron-hole recombination rate. Collectively, this results in the solar to fuel efficiency being substantially lower than the theoretical value, i.e. hematite typically yields a solar to hydrogen efficiency of 1-2\% despite having a theoretical efficiency for water splitting of 13\% \cite{glasscock2007enhancement}.

Extensive research has been conducted on the introduction of certain defects or dopants into hematite which could enhance PEC performance. In particular it has been shown that the introduction of defects such as $\vo$, Sn and Ti can improve the photocurrents of \ce{Fe3O3}.~\cite{forster2015oxygen,ling2011sn,wang2011facile} Despite this, the underlying mechanism is not well understood. Specifically, whether the dopants improved the bulk properties or the interface charge separation, and how they affected light absorption, carrier concentration and carrier mobility of the bulk hematite, etc. Answering these questions is crucial for further optimization of chemical composition and morphology of hematite for higher PEC efficiency.
%%%%%%%%%%%%%%%%%%%%%%%%%%%%%%%%%%%%%%%%%%%%%%%%%%%%%%%%%%%%%%%%%%%%%%%%%%%%

\begin{figure}[t!]
\begin{center}
\includegraphics[keepaspectratio=true,scale=0.25]{2-tmo/figures-fe2o3/pristine}
\caption{The crystal structure of \ce{Fe2O3} with lattice coordinates a, b and c. The blue arrows are to indicate the anti-ferromagnetic ordering of the iron ab layers.} \label{fe2o3:fig:pristine}
\end{center}
\end{figure}

%%%%%%%%%%%%%%%%%%%%%%%%%%%%%%%%%%%%%%%%%%%%%%%%%%%%%%%%%%%%%%%%%%%%%%%%%%%%
Recently, the joint experimental and theory work by Ref.~\cite{kim2015simultaneous} provided a detailed explanation for how nitrogen doping and oxygen vacancies in bismuth vanadate have led to improved photo-to-current efficiency by a simultaneous improvement of carrier density, carrier mobility and visible light absorption. In Ref.~\cite{kim2015simultaneous}, it has been found that the enhanced mobility is tightly connected to the small polaron transport properties in BiVO$_4$, \textit{i.e.}\ the N doping lowered the small polaron hopping barrier and improved the hopping mobility. Similarly, small polaron transport is the main carrier conduction mechanism in hematite, which causes its carrier mobility and conductivity to be extremely low. In the past, certain defects such as oxygen vacancies in bulk hematite have been discussed by computing their defect formation energies \cite{lee2013thermodynamics,zhou2015understanding} but have not provided insights related to small polaron formation and conduction, which is the key to understand experimental observations (throughout this paper we will simply write polarons when referring to electron small polarons).

Therefore we plan to illuminate the mechanism responsible for the enhancement of efficiency in doped hematite and provide a quantitative depiction of the roles of these dopants in terms of the small polaron formation and transport mobility in bulk hematite. We will discuss the electronic structure, defect charge transition levels and ionization energies for these individual defects as well as defect complex. Although several studies have been carried out on the formation energies of doped hematite \cite{lee2013thermodynamics,zhou2015understanding}, their dependence on the choice of U parameters has not been examined in details. As we will show below, we have computed defect formation energies and ionization energies for different U parameters to investigate how the results vary as a function of U. Besides the defect ionization energies which are closely related to the carrier concentration, it is crucial to understand the effect of defects or dopants on the small polaron transport mobility in hematite which is another factor determining the carrier conductivity. To our best knowledge, discussion of small polaron transport in hematite from first principles is limited to pristine systems and there are few studies of small polaron transport in defective hematite \cite{liao2011electron}.In this work, we will examine how defects effect the small polaron hopping barriers for carrier conduction in doped hematite.
By this means we can answer important experimental questions related to the effects of defects on hematite and also provide extending insights for other doped transition metal oxides, supplying guidance for future experimental design of improved small polaron transport properties in metal oxides.


\subsection{Methods}
We obtained the electronic structure of pristine and doped hematite with Density Functional Theory (DFT) calculations including Hubbard U corrections in order to take into account of strong on-site d electron interactions. Total energy, geometry relaxation and electronic structure calculations were performed with open source, plane wave codes Quantum ESPRESSO \cite{QE1}. We used ultrasoft pseudopotentials \cite{gbrv} including Fe $3s$ and $3d$ semicore electrons, which in our case yielded identical results to norm-conserving alternatives \cite{ONCV1,ONCV2}. For the plane wave basis we individually used cutoff energies of 40 Ry for wave functions and 240 Ry for charge density. The Brillouin zone was sampled with a $4\times4\times2$ Monkhorst-Pack $k$-point mesh for geometry optimization of the 30 atom unit cell and the $k$-point mesh was doubled in density of states calculations. Methfessel-Paxton first-order spreading was used to expedite Brillouin zone integration with a smearing width of 0.001 Ry \cite{methfessel1989high}. Total energy was calculated self consistently until a convergence of $10^{-8}$ Ry was achieved and geometry was optimized until the net force per atom reached less than 10$^{-3}$ Ry/au. For the hopping activation barrier calculations, a linear extrapolation scheme of atomic structure between initial and final hopping centers was employed for pristine and defective systems.

\ce{Fe2O3} is arranged in the hexagonal corundum structure with a space group of R$\overline{3}$c \cite{pauling1925crystal} (Figure \ref{fe2o3:fig:pristine}). The unit cell consists of 30 atoms where iron, Fe$^{3+}$, is six coordinated and oxygen, O$^{2-}$, is four coordinated. The system is anti-ferromagnetic where the Fe$^{3+}$ ions have a high spin configuration of partially occupied $3d$ orbitals with aligned spin between ions within $ab$-planes and anti-aligned along the $c$ direction. We first optimized the initial geometry of pristine hematite with a variable cell relaxation which obtained cell parameters of $a=5.13\,${\AA} and $c=13.99\,${\AA} with bond lengths and angles of Fe-O$\ = 1.99,\, 2.14\,${\AA}, O-Fe-O$\ = 90.7,\,86.0,\,78.6\, ^\circ$, which agree well with other work in both experiment \cite{finger1980crystal} and theory \cite{adelstein2014density}. The pristine structure and fundamental values that we obtained are collected in Table \ref{fe2o3:table:pristine}.\\

\begin{table}[H]
    \footnotesize
    \centering
    \begin{tabular}{c c c}
         \hline
         \multicolumn{3}{c}{ Pristine Hematite Fundamental Values \vspace{0.5mm}} \\
         \hline
         Value & Present Work & Experiment~\cite{finger1980crystal} \\
         \hline
         $a$ ({\AA}) & 5.13  & 5.04 \\
         $c$ ({\AA}) & 13.99  & 13.75 \\
         Fe-O ({\AA}) & 1.99, 2.14  & 1.94, 2.11 \\
         O-Fe-O ($^{\circ}$) &  90.7, 86.0, 78.6 & 90.5, 85.0, 78.2 \\
         $\mu_{Fe}$ ($\mu_B$) & 4.0 & 4.6 \cite{coey1971study} \\
         $E_{gap}$ (eV) &  2.21  & 2.2  \cite{gilbert2009band}\\
         \hline
    \end{tabular}
    \caption{Computed and experimental lattice constants $a$ and $c$ ($b=a$), nearest neighbor bond lengths Fe-O, bond angles O-Fe-O and the band gap energy $E_{gap}$.}
    \label{fe2o3:table:pristine}
\end{table}

% Hematite is a charge transfer insulator \cite{drager1992high,catti1995theoretical} and in accordance with the strong correlation of electrons in $3d$ orbitals we apply the Hubbard correction \cite{dudarev1998electron} to the Fe $3d$ orbitals in the form of Eq.\ref{fe2o3:eq:U}. Here, $E_\text{DFT}$ is the energy obtained from standard Density Functional Theory methods which is corrected by the following term which includes the occupation matrix $\lambda_i^{I\sigma}$ ($I$ ranges over all Fe ions, $i$ ranges over $3d$ orbitals and $\sigma$ is for spin up or down).
%  \begin{align}
%     E_{\text{DFT}+\text{U}}=E_{\text{DFT}}+\frac{U_{\text{eff}}}{2}\sum_{I,\sigma}\sum_{i}\lambda_{i}^{I\sigma}(1-\lambda_{i}^{I\sigma}) \label{fe2o3:eq:U}
% \end{align}

Hematite is a charge transfer insulator \cite{drager1992high,catti1995theoretical} and in accordance with the strong correlation of electrons in $3d$ orbitals we apply the Hubbard correction \cite{dudarev1998electron} to the Fe $3d$ orbitals. The $U_{\text{eff}}$ parameter, which sets the strength of this correction, was determined empirically to be 4.3 eV, in order to best fit the band gap of hematite to be 2.2 eV as measured by soft x-ray spectroscopy \cite{gilbert2009band}. It is notable that Hubbard parameters used in previous calculations of this material vary, although most used a $U_{\text{eff}}$ between 4 eV to 4.3 eV.~\cite{lee2013thermodynamics,adelstein2014density,cava2013effects,zhou2015understanding,mosey2008rotationally,rollmann2004first,liao2011electron} In particular, one study also used a non-empirical method, on the basis of unrestricted Hartree Fock theory, to determine a converged value of $U_{\text{eff}} = 4.3$ eV for \ce{Fe2O3}, which they also showed to provide results in good agreement with experiment.~\cite{mosey2008rotationally}

%%%%%%%%%%%%%%%%%%%%%%%%%%%%%%%%%%%%%%%%%%%%%%%%%%%%%%%%%%%%%%

\begin{figure}
\begin{center}
\includegraphics[keepaspectratio=true,scale=1]{2-tmo/figures-fe2o3/vo_band_wlabel.pdf} % size 0.7,0.675 helvetica18
\vspace{2mm}
\includegraphics[keepaspectratio=true,scale=0.15]{2-tmo/figures-fe2o3/vo_defect.png}
\caption{(Left) The band structure of Vo:Fe$_2$O$_3$ with unperturbed band edges and the creation of three isolated defect states $\alpha$, $\beta$ and $\gamma$ (red curves: spin up; black curves: spin down; blue dashed lines mark VBM and CBM). Defect states are states which are away from the band edges and correspond to localized states as in this case. (Right) Isosurface plot of the two highest occupied orbitals present in Vo:Fe$_2$O$_3$ with an isosurface of $\sim\,$2\% the maximum value. These two degenerate small polarons form at nearest neighbor Fe of $\vo$ (grey box) and are referred to as $\beta$ polarons throughout this paper (green isosurface spin up and blue isosurface spin down).} \label{fe2o3:fig:vo}
\end{center}
\end{figure}

%%%%%%%%%%%%%%%%%%%%%%%%%%%%%%%%%%%%%%%%%%%%%%%%%%%%%%%%%%%%%%


% The $U_{\text{eff}}$ parameter, which sets the strength of this correction, was determined empirically to be 4.3 eV, in order to best fit the band gap of hematite to be 2.2 eV as measured by soft x-ray spectroscopy \cite{gilbert2009band}. It is notable that Hubbard parameters used in previous calculations of this material vary, although most used a $U_{\text{eff}}$ between 4 eV to 4.3 eV.~\cite{lee2013thermodynamics,adelstein2014density,cava2013effects,zhou2015understanding,mosey2008rotationally,rollmann2004first,liao2011electron} In particular, one study also used a non-empirical method, on the basis of unrestricted Hartree Fock theory, to determine a converged value of $U_{\text{eff}} = 4.3$ eV for \ce{Fe2O3}, which they also showed to provide results in good agreement with experiment.~\cite{mosey2008rotationally}


The introduction of any extra electrons into \ce{Fe2O3} will form strongly localized small polarons due to large electron phonon couplings (the electrons are self-trapped by local lattice distortions) and the small polaron radius is typically no farther than the next-nearest neighbors.~\cite{austin1969polarons} The formation of small polarons have been found in several metal oxides, such as Fe$_2$O$_3$, BiVO$_4$,~\cite{kim2015simultaneous} and ABO$_{3}$ perovskites \cite{eglitis2014ab}, where the small polarons must be thermally activated and then hop between metal sites in order to conduct currents. In this work we plan to study the relative stability and hopping barriers between different polaron configurations, where controlling the position of small polaron formation with/without defects is the key. We controlled the placement of the small polaron through local geometry adjustments that break the symmetry of the ground state system, such as moving the oxygen atoms surrounding an Fe ion away from it to ensure localization of polarons at that site and then relaxing the geometry to energy minimum. This bias initiates the localization of polarons which will expand the local metal-oxygen bonds slightly ($\sim$0.1 \AA).  The small polarons form both in pristine systems (without explicit defects) and defective systems. For the latter, the small polaron and defect center often interact and possibly form a complex as we will discuss later.

For calculations with defects and excess charge, a $2\times 2\times 1$ supercell with a $2\times2\times2$ $k$-point mesh was adopted and internal geometries were relaxed keeping cell parameters fixed to simulate an isolated defect within the periodic system.  We found the $2\times 2\times 1$ supercell is large enough to avoid periodic interactions between neutral defects. The charged defect formation energies have a slow convergence with supercell size (scale as 1/L) due to image charge interactions. We corrected the charge defect formation energies using the charge correction scheme similar to Ref.~\cite{freysoldt2014first} and our recent development in Ref.~\cite{PING2017JCP} where the proper treatment for electrostatic potential alignment at presence of strong geometry relaxation by defects is included, which has been a general issue for charged defect calculations \cite{kumagai2014electrostatics}.



\subsection{Electronic Structure of Defective Hematite}


% \subsubsection{Oxygen Vacancy}

Oxygen vacancy is an important intrinsic defect in \ce{Fe2O3}, whose concentration can be controlled through experimental conditions. Recently, it has been found that the introduction of oxygen vacancies in \ce{Fe2O3} can improve the overall photocurrents of hematite based photoanodes in solar water splitting cells.~\cite{forster2015oxygen} Yet, the role of oxygen vacancies is unclear, and two possible explanations are present: (1) oxygen vacancies improve the bulk carrier conductivity properties or (2) oxygen vacancies improve the electron hole separation at electrode/electrolyte interfaces. In this work we focus on (1), \textit{i.e.}\ the effect of oxygen vacancies on the bulk properties of \ce{Fe2O3}.


%%%%%%%%%%%%%%%%%%%%%%%%%%%%%%%%%%%%%%%%%%%%%%%%%%%%%%%%%%%%%%

\begin{figure}
\begin{center}
\includegraphics[keepaspectratio=true,scale=1]{2-tmo/figures-fe2o3/nvo_band_wlabel.pdf}
\includegraphics[keepaspectratio=true,scale=0.15]{2-tmo/figures-fe2o3/nvo_defect.png}
\caption{(Top) The band structure of N+Vo:Fe$_2$O$_3$ with a perturbed valence band edge (right below the lower dashed blue line), creating a reduced gap and an indirect to direct gap transition (red curves: spin up and black curves: spin down; blue dashed lines mark VBM and CBM). There are several defect states introduced in the gap (between two blue dashed lines), including occupied defect states which are primarily hybridized Fe $3d$ and N $2p$ states. (Bottom) Isosurface plot (blue) of the highest occupied orbital present in N+Vo:Fe$_2$O$_3$, a combination of a $\beta$ polaron from the presence of $\vo$ (grey box) and N $2p$ orbital (pink atom with arrow).} \label{fe2o3:fig:nvo}
\end{center}
\end{figure}

%%%%%%%%%%%%%%%%%%%%%%%%%%%%%%%%%%%%%%%%%%%%%%%%%%%%%%%%%%%%%%
The role of oxygen vacancies is simulated through the removal of one oxygen atom within our supercell (120 atoms) and allowing the internal geometry to relax. Since the oxygen within \ce{Fe2O3} has an oxidation state of O$^{2-}$, oxygen vacancy is an $n$-type dopant, which donates two electrons to the system. The band structure of hematite with one neutral oxygen vacancy (Figure \ref{fe2o3:fig:vo}) displays the creation of three defect states within the band gap $\alpha$, $\beta$, $\gamma$ in agreement with previous theoretical work \cite{lee2013thermodynamics}. The corresponding projected density of states is included in SI figure 2. Defect states are considered to be lozalized states (over a few nearest neighbor atoms -- see SI figure 7) and have no dispersion as seen by the band structure and density of states. We found the two electrons are spontaneously ionized from oxygen vacancy site and form localized small polarons at nearby Fe ion (no charge density appears at the center of the vacancy, unlike the case of SrTiO$_3$ \cite{janotti2014vacancies}). The $\beta$ state in the band structure corresponds to the two polaron states (one spin up and one spin down) in which the extra electrons from $\vo$ occupy (the wave functions are shown in Figure \ref{fe2o3:fig:vo}). They are $d_{x^2-y^2}$ type orbitals of the two nearest neighbors of the four-coordinated $\vo$. The $\alpha$ state is a perturbed valence state of O $2p$ orbitals surrounding the site at which the extra electrons go, while the $\gamma$ state is a perturbed conduction state which is unoccupied and corresponds to $e_g$ orbitals of the next nearest Fe neighbors of $\vo$. Overall, the oxygen vacancy did not modify the band edge positions and band dispersion of the pristine systems, except introducing isolated defect levels in the band gap. Therefore, we do not expect it will affect the main optical absorption spectra (as the defect states have low density of states) but it may affect the transport properties of the system, i.e. the carrier concentration or mobility as we will discuss later.


% \subsubsection{N-doped + Oxygen Vacancy}

Atomic nitrogen has been found to be a promising dopant to improve visible light absorption properties of metal oxides, such as in TiO$_2$ \cite{ansari2016nitrogen} and BiVO$_4$.~\cite{kim2015simultaneous} Recently, we found N doping in the presence of oxygen vacancies can also improve the carrier mobility of BiVO$_4$, where the small polaron hopping barrier was lowered. Yet, N doping in \ce{Fe2O3} with/without oxygen vacancies has not been studied both theoretically and experimentally to our best knowledge. We will firstly discuss the effects of N doping and then discuss the combination effect of N doping with oxygen vacancies.

For N-doped hematite we replaced one O in our system with N, a p-type dopant since N has one fewer valence electron than O. This substitution results in a deep defect state created corresponding to an unoccupied N $2p$ orbital and results in slightly perturbed valence band states (See SI Figure 3 for band structure and projected density of states.) The deep defect state introduced by N doping has minimal overlap with the O $2p$ states, which may introduce defect state involved transitions in the optical spectra but will not affect the main absorption edge. In the meanwhile, the VBM in N-doped hematite is raised by 0.1 eV due to the hybridization between N and O 2p states, and interestingly, the indirect band gap for the pristine \ce{Fe2O3} becomes a direct band gap after N doping. This could both shift the absorption edge to lower energy slightly and also enhance the absorption coefficient at the same energy range compared to the pristine case.

%%%%%%%%%%%%%%%%%%%%%%%%%%%%%%%%%%%%%%%%%%%%%%%%%%%%%%%%%%%%%%

\begin{figure}
\begin{center}
\includegraphics[keepaspectratio=true,scale=1]{2-tmo/figures-fe2o3/sn_band_wlabel.pdf} % size 0.725,0.7 helvetica18
\includegraphics[keepaspectratio=true,scale=0.1075]{2-tmo/figures-fe2o3/sn_defect.png}
\caption{(Top) The band structure of Sn:Fe$_2$O$_3$ with a perturbed conduction band edge of Fe $3d$ states creating a reduced gap of 2.0 eV (see SI figure 6 for details). There is a single isolated defect introduced in the gap which corresponds to a small polaron formed at Fe site near Sn. (Bottom) Isosurface plot (blue) of the highest occupied orbital present in Sn:Fe$_2$O$_3$, a small polaron formed directly below Sn (grey atom).} \label{fe2o3:fig:sn}
\end{center}
\end{figure}

%%%%%%%%%%%%%%%%%%%%%%%%%%%%%%%%%%%%%%%%%%%%%%%%%%%%%%%%%%%%%%

Next, we considered nitrogen doping coupled with oxygen vacancy in both a (1N:1$\vo$) ratio and a (2N:1$\vo$) charge-balanced ratio. In the (1:1) doping regime the overall effect is still n-type yet one of the polaron states is absorbed by the nitrogen into the previously unoccupied N $2p$ state (Figure \ref{fe2o3:fig:nvo}). We found that the defects prefer to be close-by, rather than far away, with $E_{near}-E_{far}\approx -0.67$ eV, with only small variations in the $E_{near}$ based on N and $\vo$ orientation and distance. Tested configurations had the distance between $\vo$ and N in the `near' case as about $\sim\,$2.5-3.0 \AA $\ $and $\sim\,$7-10 \AA $\ $in the `far' case. The resulting electronic structure of N+$\vo$:Fe$_2$O$_3$ is shown in Figure \ref{fe2o3:fig:nvo} and displays a slightly reduced band gap of 2.1 eV (see SI Figure 4 for details). Again, as in N-doped hematite, we see a shift at the VBM of N+$\vo$:Fe$_2$O$_3$, resulting in a direct band gap introduced from N-doping.

%%%%%%%%%%%%%%%%%%%%%%%%%%%%%%%%%%%%%%%%%%%%%%%%%%%%%%%%%%%%%%%%%%%%%%%%%%%%%%%%

\begin{table}[t!]
\begin{center}
\begin{tabular}{p{1.9cm}p{1.9cm}p{1.9cm}p{1.9cm}}
 \hline
 \multicolumn{4}{c}{ 2N:1$\vo$ Hematite configurations \vspace{0.5mm}} \\
 \hline
 $\beta$ polarons & N-N ({\AA}) & N-Fe-N ($^{\circ}$)& $\Delta E_{tot}$ (eV) \\
 \hline
 2  & 2.4025 & 74.578 & 0.44 \\
 1  & 2.7482 & 86.841 & 0.23 \\
 0  & 3.8606 & 151.269 & \hspace{1mm} 0 \\
 \hline
\end{tabular}
\end{center}
\caption{Summary of 2N:1$\vo$ configurations, displaying the relation between number of $\beta$ polarons from $\vo$ formed, distance between N (N-N ({\AA})), atomic angle (N-Fe-N ($^{\circ}$)) and the total energy difference ($\Delta E_{tot}$) with the most stable system (the configuration with 0 $\beta$ polarons formed).} \label{fe2o3:table:n2vo}
\end{table}

%%%%%%%%%%%%%%%%%%%%%%%%%%%%%%%%%%%%%%%%%%%%%%%%%%%%%%%%%%%%%%%%%%%%%%%%%%%%%%%%

We also tested nitrogen doping coupled with oxygen vacancy in a (2:1) ratio and we found three possible electronic configurations based on the orientation of the defects within the lattice (in each case they are attached to the same Fe ion). These configurations are nicely characterized by the number of $\beta$ type states ($\beta$ polarons) created by $\vo$. A summary of these configurations is collected in Table \ref{fe2o3:table:n2vo}. We found that if all three defects are as close as possible (with N-N distance 2.40\AA) then both small polarons are formed (the number of $\beta$ polarons is 2 as shown in Table \ref{fe2o3:table:n2vo}). If all three are still close yet the nitrogen atoms are on opposite sides of $\vo$ (with N-N distance 3.86\AA)then both small polarons are absorbed by the nitrogen (the number of $\beta$ polarons is 0 as shown in Table \ref{fe2o3:table:n2vo}; see detailed structures in SI figure 8). Lastly, it is possible to form a configuration somewhere in between these two states and have only one small polaron state formed (the number of $\beta$ polarons is 1 as shown in Table \ref{fe2o3:table:n2vo}). Ultimately, the most stable configuration for the (2:1) ratio is to have the nitrogen on opposite sides of $\vo$ so that they may absorb the two electrons of $\vo$ with minimal interaction between nitrogen sites. This configuration is minimal in energy because the absorption of these extra electrons allows the nitrogen to achieve filled valence bands while keeping their repulsive electrostatic interaction at a minimum. In any of these configurations, several defect states are still present within the band gap (see the corresponding band structure and projected density of states in SI figure 5).  Although both electrons donated from $\vo$ are absorbed by the nitrogen atoms in the most stable configuration of 2N:1$\vo$, these dopants still introduced extra defect states inside the gap. This is unlike the case of BiVO$_4$, where the charge balanced N doping with oxygen vacancy results in the shift of valance band edges but no isolated defect states formed in the band gap \cite{kim2015simultaneous}. Meanwhile, the valence band edge shifts up by 0.2 eV in this case resulting in a reduced band gap size of 2.0 eV, and is also accompanied by an indirect to direct
 gap transition.



% \subsubsection{Sn-doped}

Sn substitution has been shown experimentally to be a promising dopant for improving photocurrents in \ce{Fe2O3} photoanodes \cite{ling2011sn}. However, similar to the case of oxygen vacancies, it is unclear whether this is because the bulk carrier conductivity has been improved. Sn is a n-type dopant since the substitution Fe$^{3+}\rightarrow$ Sn$^{4+}$ donates an electron into the system. The atomic number of tin is nearly double that of iron (50:26) yet the atomic radius of Sn$^{4+}$ (83.0 pm) is only slightly larger than that of Fe$^{3+}$ (78.5 pm). Since the Fe ions in hematite have a high spin configuration of half-filled $3d$ orbitals, the main modification of the electronic structure upon the introduction of Sn substitution is that Sn has no unpaired spin which creates a hole in the magnetic ordering of the anti-ferromagnetic arrangement of \ce{Fe2O3}. Additionally, Sn valence orbitals are less localized than Fe $3d$ orbitals, which results in the excess electron easily moving away from Sn and localizing on a nearby Fe ion. Therefore, the excess electron contributed from Sn substitution forms a small polaron at the Fe site located either directly below or directly above Sn (depending on the location of Sn in the lattice, Figure \ref{fe2o3:fig:sn}). Alignment of Fe 3s semi-core states between the pristine and doped systems reveals the conduction band edge down-shift of unoccupied Fe $3d$ orbitals in Sn-doped hematite resulting in a slightly smaller band gap (2.0 eV) than pristine \ce{Fe2O3} (see Figure \ref{fe2o3:fig:sn} for the band structure and SI Figure 6 for the projected density of states). Overall, Sn doping causes one small polaron defect state in the gap and perturbs the states close to the band edges, unlike the case of oxygen vacancies which only results in several isolated defect states in the gap with no perturbation of band edges.


\subsection{Formation Energy of Charged Defects}

To understand whether the defects can be ionized easily at the room temperature (and therefore contribute to the carrier concentrations), we calculated thermodynamical charge transition levels $\epsilon_{0/+1}$ and $\epsilon_{+1/+2}$ of the doped systems considered.
% The charge transition levels are defined to be the Fermi level where the formation energies of two different stable charged states are equal. The formation energy is given by
% \begin{align}
% E_{for}(defect,q) = \Delta E_{tot} +\lambda\,\mu_x+q\,\varepsilon_{F}+\Delta_q \label{fe2o3:eq:ef}
% \end{align}
% where $\Delta E_{tot} = E_{tot}(defect,q)-E_{tot}(pristine)$ is the difference in total energy for the charged defect system to that of the pristine system, $\lambda\,\mu_x$ is the chemical potential of any removed or added species, $q\,\varepsilon_{F}$ is the electron reference with $\varepsilon_{F}$ measured from the valence band maximum (of the pristine system) and $\Delta_q$ is the charged cell correction to correct the spurious interactions between charges in periodic images. The charged cell correction is computed as discussed in the method section,~\cite{PING2017JCP} while
The chemical potentials of elements Fe, Sn, O and N were estimated as the total energy per atom of the natural occurrence of the element. In this paper we focus on the charge transition levels of the defects which will not depend on the chemical potential of elements (\textit{e.g.}\ oxygen poor or rich conditions).

%%%%%%%%%%%%%%%%%%%%%%%%%%%%%%%%%%%%%%%%%%%%%%%%%%%%%%%%%%%%%%


\begin{figure}[t!]
\begin{center}
\includegraphics[keepaspectratio=true,scale=0.72]{2-tmo/figures-fe2o3/ef_hematite.pdf} %Helevetica 24 and ps 1.75
\end{center}
\caption{Formation energy displaying the most stable charge states of the different doped systems considered with respect to the Fermi level. The zero Fermi level is the VBM of the pristine system.} \label{fe2o3:fig:ef}
\end{figure}

%%%%%%%%%%%%%%%%%%%%%%%%%%%%%%%%%%%%%%%%%%%%%%%%%%%%%%%%%%%%%

%%%%%%%%%%%%%%%%%%%%%%%%%%%%%%%%%%%%%%%%%%%%%%%%%%%%%%%%%%%%%

\begin{table}[t!]
\begin{center}
\begin{tabular}{p{2.5cm}p{2.5cm}p{2.5cm}}
 \hline
 \multicolumn{3}{c}{ Ionization Energies in Doped Hematite \vspace{0.5mm}} \\
 \hline
 Doping & (+1/+2) eV & (0/+1) eV\\
 \hline
 $\vo$ & 1.31  & 1.09 \\
 N+$\vo$& 1.82 & 1.14 \\
 Sn & -- & 0.74 \\
 \hline
\end{tabular}
\end{center}
\caption{Corresponding ionization energies obtained from the energy difference between the charge transition level to the conduction band for different doping cases.} \label{fe2o3:table:ef}
\end{table}


%%%%%%%%%%%%%%%%%%%%%%%%%%%%%%%%%%%%%%%%%%%%%%%%%%%%%%%%%%%%%


We obtained the charged defect formation energies and charge transition levels for the doped systems as displayed in Figure \ref{fe2o3:fig:ef}. Typically, one obtains the ionization energies of these states by the difference of the conduction band minimum 2.21 eV (of the pristine system) to the Fermi level at which the charge transitions occur, summarized in Table \ref{fe2o3:table:ef}. We see that $\vo$ introduces deep defects into the system with large ionization energies 1.31 eV (+1/+2) and 1.09 eV (0/+1), consistent with previous theoretical work which reported 1.08 eV for (0/+1) \cite{lee2013thermodynamics}. Meanwhile we see that N+$\vo$ deepens the (+1/+2) transition by 0.51 eV yet does not significantly change the (0/+1) ionization energy. Sn has the lowest ionization energy of 0.74 eV (0/+1), which is larger than previously reported values by 0.22 eV \cite{zhou2015understanding}; yet, they also reported a smaller band gap (2.15 eV) than ours (2.21 eV) due to a lower U value (4 eV). It is important to note that these ionization energies indirectly depend on the Hubbard U parameter. Specifically, a lower U parameter will lower the conduction band minimum position and hence result in lower defect ionization energies. We find that the charge transition levels are not expected to change much with respect to the valence band maximum, yet the band gap can have a range from 1.8 eV to 2.2 eV for a Hubbard U of 3.0 eV to 4.3 eV (see Figure \ref{fe2o3:fig:efu}). Nonetheless, the overall defect property does not change qualitatively, i.e. the oxygen vacancy has an ionization energy ranging from 1.09 eV (at U=4.3 eV) to 0.79 eV (at U=3.0 eV), which is still a deep impurity compared with the ionization energy of a small polaron in the pristine hematite (0.49 eV at U=4.3 eV and 0.09 eV at U=3.0 eV.)

%%%%%%%%%%%%%%%%%%%%%%%%%%%%%%%%%%%%%%%%%%%%%%%%%%%%%%%%%%%%%%


\begin{figure}[t!]
\begin{center}
\includegraphics[keepaspectratio=true,scale=0.72]{2-tmo/figures-fe2o3/ef_U.pdf} %Helevetica 24 and ps 1.75
\end{center}
\caption{Formation energy plot for pristine hematite and hematite with $\vo$ with different U parameters, 3.0 and 4.3 eV. The reference zero of the Fermi level is the VBM of the pristine system.
The charge transitions still occur at the similar Fermi level with respect to the VBM in the respective systems, however, a smaller U parameter results in a smaller band gap. In particular $E_{gap}=1.81$ eV in the $U=3.0$ eV system, 0.4 eV smaller than $E_{gap}=2.21$ eV in the $U=4.3$ eV system.
As a result, the defect ionization energy is lowered by 0.4 eV at U= 3.0 eV compared with U=4.3 eV.} \label{fe2o3:fig:efu}
\end{figure}

%%%%%%%%%%%%%%%%%%%%%%%%%%%%%%%%%%%%%%%%%%%%%%%%%%%%%%%%%%%%%

We note that although the absolute Sn-doping ionization energy is still large (0.74 eV), if we compare its ionization energy with the ionization energy of one small polaron in a pristine system (0.49 eV; black line in Figure \ref{fe2o3:fig:ef}), the difference is only 0.25 eV. This implies that it takes 0.25 eV to excite the small polaron from a defect-bound polaron to an unbound polaron (or `free' polaron), which can be considered as a shallow impurity. And because electrons naturally form small polarons even in the pristine \ce{Fe2O3} and it takes 0.49 eV to ionize the electrons from self-trapped to free electrons in the conduction bands, absolute `shallow' n-type defects (where the ionization energy relative to the CBM is at the order of $kT$) can hardly form in \ce{Fe2O3}. In particular,this is evident for the ionization energies of $\vo$ and N+$\vo$ which are still $\sim\,$0.6-0.7 eV away from the ionization energy of one small polaron in the pristine system. Therefore, even when considering the ionization energy of a small polaron in the pristine system, the polarons formed in these systems ($\vo$ and N+$\vo$) are still difficult to ionize and contribute to carrier concentrations at room temperature, unlike the case of Sn doping. \\

%%%%%%%%%%%%%%%%%%%%%%%%%%%%%%%%%%%%%%%%%%%%%%%%%%%%%%%%%%%%%

\subsection{Small Polaron Transport in Pristine and Defective Hematite}

From the ionization energy calculations above, we have found Sn is a promising dopant for carrier concentration improvement. As the carrier conductivity depends on both the carrier concentration and mobility, in this section we will discuss whether Sn doping improves the carrier mobility as well. Previous work \cite{rosso2003an} found that in pristine \ce{Fe2O3} the coupling of small polaron hopping sites (V$_{AB}$) is large relative to the reorganization energy $\Delta$G* (V$_{AB} > \Delta$G*/4), implying the small polaron hopping conduction of \ce{Fe2O3} is in the adiabatic regime. We then computed the small polaron hopping activation energy of pristine and Sn-doped hematite using a linear extrapolation technique, which includes the coupling V$_{AB}$ implicitly. We start from the initial configuration $q_{a}$ where the small polaron is located on a Fe ion and then using geometrical techniques described in the method section determine a configuration $q_{b}$ in which the small polaron is located on a nearest neighbor Fe ion in the same $ab$-plane. Then, in accordance with Eq. \ref{fe2o3:eq:ex} we linearly extrapolated from $q_{a}$ to $q_{b}$, allowing the electron density of the system to equilibrate at each step but keeping the geometry $q_{x}$ fixed.
\begin{align}
q_{x} = q_{a}(1-x)+x\,q_b \label{fe2o3:eq:ex}
\end{align}
The computed total energies (referenced to the starting configuration) as a function of reaction coordination are shown in Figure \ref{fe2o3:fig:transport}. The peak of the barrier is the saddle point between the two hopping sites and the true activation energy $E_{A}$ is obtained from relaxing the geometry from this point. In the pristine system the saddle sits at the $q_{0.5}$ reaction coordinate due to the symmetry of two hopping centers and in Sn:Fe$_2$O$_3$ it sits at the $q_{0.56}$ reaction coordinate. The latter relates to the different distances from the small polaron center to  the Sn defect center.\\

%%%%%%%%%%%%%%%%%%%%%%%%%%%%%%%%%%%%%%%%%%%%%%%%%%%%%%%%%%%%

\begin{figure}[t!]
\begin{center}
\includegraphics[keepaspectratio=true,scale=0.69]{2-tmo/figures-fe2o3/transport.pdf}

\vspace{2mm}

\includegraphics[keepaspectratio=true,scale=0.175]{2-tmo/figures-fe2o3/transport_sn}
\end{center}
\caption{(Top) Small polaron hopping barrier in the pristine and Sn-doped systems where the solid square and dot indicate the activation energy. (Bottom) Schematic of polaron hopping in the Sn-doped system (left is the initial $a$ configuration and $b$ is the final configuration).} \label{fe2o3:fig:transport}
\end{figure}

%%%%%%%%%%%%%%%%%%%%%%%%%%%%%%%%%%%%%%%%%%%%%%%%%%%%%%%%%%%%%


For the pristine system we obtain an activation energy of $E_{A,Fe} = 0.11$ eV, consistent with previously calculated results of 0.11 eV \cite{rosso2003an} and 0.13 eV \cite{adelstein2014density}. In the Sn-doped system we computed an activation energy of $E_{A,Sn}=0.16$ eV. This higher activation energy is quite significant when compared with $kT$ at room temperature ($k$ is the Boltzmann constant and $T$ is temperature). In particular the electron transfer rate is proportional to the exponential of the activation energy over $kT$ as shown in Eq.\ref{fe2o3:eq:mob}.
\begin{align}
\tau = A e^{-E_{A}/kT} \label{fe2o3:eq:mob}
\end{align}
where $\tau$ is the electron transfer rate and $A$ is the prefactor which depends on the number of nearest neighbors and the attempt frequency.
Considering that both systems have 3 nearest neighbors and Sn disrupts the geometry only slightly,  the prefactor $A$ of these two systems should be on the same order of magnitude. The hopping mobility is related to the electron transfer rates through the Einstein relation and one dimensional random walk i.e. $\mu=eR^2\tau/(2kT)$, where R is the electron transfer distance and $\tau$ is the electron transfer rate\cite{adelstein2014density,einstein}. Therefore we can determine the ratio of the mobility in these two systems to be solely dependent on the difference of their activation energies.

\begin{align}
\frac{\mu_{Sn}}{\mu_{Fe}} \simeq e^{-(E_{A,Sn}-E_{A,Fe})/kT} \label{fe2o3:eq:mobratio}
\end{align}
Eq.\ref{fe2o3:eq:mobratio} gives a ratio $\mu_{Sn}/\mu_{Fe} = 0.14$, which implies a decrease in carrier mobility in Sn:Fe$_2$O$_3$. We note that the effect of defects on the carrier mobility depends on the distance to the defect center, i.e. the hopping barriers between the pristine and Sn-doped systems will be similar when the hopping centers are far away from the defects (Sn). Therefore what we estimate here is the lower bound of carrier mobility at presence of Sn.\\

We also investigated the hopping barriers at presence of $\vo$ and N+$\vo$ by using similar procedures; yet we found the small polaron centers are unstable away from the defects (the electrons tend to localize closest to the defect centers, despite any applied local distortion at the positions far away from the defects.) This again indicates these defects in bulk \ce{Fe2O3} are deep and extra electrons from  $\vo$  and N+$\vo$  tend to form small polarons tightly bounded to the positive charged defect centers.

%%%%%%%%%%%%%%%%%%%%%%%%%%%%%%%%%%%%%%%%%%%%%%%%%%%%%%%%%%%%%

\subsection{Conclusions}

In summary, this work discussed the effects of defects on the small polaron formation, ionization and hopping transport properties in bulk \ce{Fe2O3} through first-principles calculations, where the choice of dopants is inspired by recent experimental and theoretical work\cite{forster2015oxygen,ling2011sn,wang2011facile,kim2015simultaneous}. Our calculations of the electronic structure of pristine and doped \ce{Fe2O3} show that the small polarons will naturally form at Fe ions if there are any excess electrons (unless there are dopants with stronger correlated electrons available than the $3d$ electrons of Fe). These small polarons can be tightly bounded to the defect centers (strongly preferred to locate at Fe closest to the defects) or possibly conduct through thermally activated small polaron hopping, depending on the corresponding ionization energies.

Electronic structure calculations show the defects we investigated here have different effects on the band structures: for $\vo$, only deep defect levels are introduced in the band gap and the bulk band structure did not change compared with the pristine hematite. For the cases of N,  (1N:1$\vo$) and (2N:1$\vo$) doping, we observe 1.\ defect levels appear in the band structure; 2.\ the valence band maximum shifts up by $0.1-0.2$ eV, accompanied by an indirect to direct gap transition, which could improve the visible light absorption at the same energy range as before, but also lower the absorption spectra edge than the pristine hematite. The enhancement of visible light absorption may be further confirmed by computing absorption spectra for these doped systems in the future work. For Sn doping, we found related defect states in the gap and also a band gap reduction by 0.2 eV by lowering of the conduction band position.

Formation energy as a function of Fermi level reveals most stable charge configurations of the several doped systems of hematite considered: $\vo$, N+$\vo$ and Sn. From our defect formation energy calculations, we see that $\vo$ and N+$\vo$ bind the small polarons strongly with relatively large ionization energies. In the Sn-doped system we see a much lower ionization energy (it takes 0.25 eV from a defect-bound polaron to a ``free'' polaron) compared to the pristine system. This shows Sn doping could contribute to the carrier concentrations at room temperature. We note that due to the spontaneous formation of small polarons even in pristine systems (with an ionization energy of 0.49 eV to become free electrons in CBM), an absolute shallow n-type impurity can hardly form in \ce{Fe2O3} since this would require an ionization energy relative to CBM more comparable to $kT$ at room temperature ($\sim\,$0.026 eV).

Calculations of the small polaron hopping activation energy were conducted in pristine and Sn-doped hematite and we found that the introduction of Sn may lower the hopping mobility due to a higher hopping activation barrier. However, the improved carrier concentration and possibly improved light absorption  by Sn doping can improve the overall photocurrents as observed experimentally. We note that although experimentally they also found $\vo$  improved the photocurrents of \ce{Fe2O3} \cite{forster2015oxygen,ling2011sn} (which is not supported by our theoretical results that $\vo$ is a deep impurity and does not modify the band structure of the pristine system except introducing defect bands), the measurements are based on an average of bulk and surface defects, at the presence of electrode/electrolyte interfaces. In order to further elucidate the roles of $\vo$, N+$\vo$ and Sn doping in \ce{Fe2O3} for solar water splitting applications, the ionization energies of charged defects at surfaces as well as the interaction between small polarons, defects and water at the \ce{Fe2O3}/water interface needs to be investigated in the future work.

% shortcut for vacancy
\def\vacancy{\text{V}}
% shortcut for general vacancy defect (V_X)
\newcommand{\vac}[1]{\vacancy_\text{#1}}
% shortcut for substitution (A_B) where A substitutes B
\newcommand{\sub}[2]{\text{#1}_\text{#2}}
% shortcut for interstitial (X_i)
\newcommand{\itl}[1]{\text{#1}_i}
\def\vo{\vac{O}}
\def\vfe{\vac{Fe}}
\def\oi{\itl{O}}
\def\fei{\itl{Fe}}

% shortcut for delta mu
\newcommand{\dmu}[1]{\Delta\mu_\text{#1}}
% shortcut for log O2
\def\po{p_{\rm O_2}}
\def\logp{\log (\po)}
% shortcut for EP (in case we want to change notation)
\def\ep{EP}


\section{Carrier Concentrations in \ce{Fe2O3}}

Outstanding questions remain in the pursuit of highly-efficient \ce{Fe2O3}-based devices with atomic doping. For example, the identity of intrinsic free carrier donor remains under debate.
Specifically, various experimental works have claimed that oxygen vacancies ($\vo$) are the source of extra carriers in $n$-type \ce{Fe2O3}, \cite{zhang2019interfacial} whereas theoretical works have shown that $\vo$ have an unreasonably large ionization energy,\cite{smart2017effect} which suggests that they cannot be the primary contributor of free carriers. Meanwhile, some theoretical works have supported that iron interstitials ($\fei$) are the major carrier donors due to a significantly smaller ionization energy than that of $\vo$.~\cite{lee2013thermodynamics} Another important question, which is more general to oxides than particularly \ce{Fe2O3}, is how to determine dopants which will yield the best performances of oxide based devices? Insights into the design of efficient oxide based devices by simple yet practical prediction of atomic doping are highly desired.~\cite{yang2017progress}

There have been some theoretical works on intrinsic defects and atomic doping in \ce{Fe2O3}, for example, defect formation energy and charge transition levels have been computed for \ce{Fe2O3}, which can help determine dopants with low ionization energy such as Sn, Ge, and Ti. \cite{smart2017effect,zhou2015understanding}
However, these works cannot yet address the above questions since directly obtaining carrier concentrations requires understanding the combined effects of dopant solubility and dopant ionization energy.
Furthermore, intrinsic defects should be considered simultaneously as they may compensate $n$-type dopants, and yet, the effects of intrinsic defects with or without external doping are not well-established in \ce{Fe2O3}, as mentioned above.

In this work, we provide answers to these questions by investigating carrier concentrations in \ce{Fe2O3} from first-principles. By careful evaluation of defect concentrations including the presence of small electron polarons we can reliably predict the concentrations of free carriers in \ce{Fe2O3}, in excellent agreement with experiments.
Further detailed computational analysis of dopant solubility, ionization energy, chemical condition, and synthesis temperature are provided in order to answer outstanding questions such as: What are the intrinsic carrier donors in pristine \ce{Fe2O3}? Which dopants are the best at raising carrier concentrations? What makes a dopant effective in raising carrier concentrations (\textit{e.g.}\ solubility or ionization energy)?
The work is organized as follows, first we briefly present our computational methodology. Second we discuss intrinsic defects in undoped \ce{Fe2O3}. Third we systematically study several tetravalent and pentavalent dopants idenitifying the best dopants for this system. Finally we analyze the importance of solubility against ionization energy in determining which dopants will be the best at enhancing carrier concentrations. The contribution of entropy to the formation energy is rigorously studied and the general trends of formation energy (ionization energy) within each group with respect to ionic radius is also presented.
This work provides detailed understanding of the interactions between intrinsic defects, extrinsic dopants, and small polarons in polaronic oxides.

Density functional theory calculations were performed in the open-source plane-wave code QuantumESPRESSO~\cite{QE1} using ultrasoft GBRV pseudopotentials~\cite{gbrv} and an effective Hubbard $U$~\cite{dudarev1998electron} value of 4.3 eV for Fe $3d$ orbitals \cite{smart2017effect,adelstein2014density}.
Plane-wave cutoff energies of 40 Ry and 240 Ry were used for wavefunctions and charge density, respectively. All calculations employed a $2\times 2\times 1$ supercell (120 atoms) of the hexagonal unit cell with a $2\times 2\times 2$ $k$-point mesh for integration over the Brillouin zone.
Charged defect formation energies and defect concentrations, including free carrier concentrations were evaluated from first-principles.


\subsection{Intrinsic Defect Contributions to Polaron Concentration}

\begin{figure}[H]
\centering
% \includegraphics[keepaspectratio=true,width=0.95\linewidth]{Figures/dc.png}
\includegraphics[keepaspectratio=true,width=1\linewidth]{2-tmo/figures-fe2o3/Figure1_v5.png}
\caption{Identification of the source of free carriers in undoped \ce{Fe2O3}.
(a) Local structure and electron wavefunctions (yellow cloud) of $\vo$ with two $EP$s.
(b) Local structure and electron wavefunctions of $\fei$ with three $EP$s (one of which forms at the interstitial site).
(c) Intrinsic defect concentrations as a function of \ce{O2} partial pressure in undoped hematite computed at room temperature with synthesis at $T_S = 873$, 1073, 1373 K.
(d) $\vo^+$, $\fei^+$ and $EP$ concentration at room temperature and $\po=1$ atm, as a function of synthesis temperature.
(e) Polaron concentration and (f) difference between $\fei^+$ and $\vo^+$ at room temperature as a function of synthesis temperature ($T_S$) and oxygen partial pressure ($\po$).
In the atomic plots, gold=Fe, red=O, and blue=$\fei$.
The yellow cloud is an isosurface of the polaron wavefunction with an isosurface level of 5\% its maximum.
}
\label{fig:intrinsic}
\end{figure}

In undoped hematite, intrinsic defects such as vacancies ($\vo$, $\vfe$), and interstitials ($\oi$, $\fei$) may form within the lattice along with the generation of carriers such as free small electron polarons ($EP$).
Since $\vo$ and $\fei$ are $n$-type defects, they introduce small electron polarons into the lattice as shown in Figure~\ref{fig:intrinsic}a-b.
In Figure~\ref{fig:intrinsic}c, intrinsic defect concentrations are provided at room temperature (300 K) as a function of oxygen partial pressure ($\po$) in pristine hematite for three synthesis temperatures ($T_S = 873$, 1073, and 1373 K).
First we note that the largest presence of any defect is $\vo$ (lowest formation energy in Table~\ref{table:intrinsic}), which grows monotonically with decreasing $p_{\rm O_2}$, and should be chiefly responsible for the non-stoichiometry observed in \ce{Fe2O3}. \cite{dieckmann1993point}
Second, we find that intrinsically excess electrons can form into free electron polarons (dashed red line labeled as $EP$) whereas the concentrations of free delocalized band electrons and holes are negligible (less than $10^8$ cm$^{-3}$), consistent with experimental measurements of photoexcited electrons in \ce{Fe2O3}.\cite{carneiro2017excitation} Also, the intrinsic $n$-type nature of \ce{Fe2O3} is consistent with the lower formation energy of $\vo$ and $\fei$ compared with $p$-type defects in Table~\ref{table:intrinsic}.

In terms of identifying the primary donor of these $EP$, the conclusions are dependent on the synthesis conditions and cannot be determined from formation energy or ionization energies alone. At $T_S=873$ K, it is the case that ionized oxygen vacancies ($\vo^+$, solid purple line) are the primary donor to free electron polaron concentrations (overlaps with the dashed red line labeled $EP$).
Interestingly, as the synthesis temperature is elevated, for example to $T_S=1373$ K, free electron polaron concentrations are not just more abundant, they are also originating from a different source, namely, Fe interstitials ($\fei^+$, solid light blue line) in Figure~\ref{fig:intrinsic}c.
The change in electron polaron concentration versus synthesis temperature is directly shown in Figure~\ref{fig:intrinsic}d at atmospheric pressure ($\po=1$ atm).
We find that for synthesis temperatures below a critical temperature of $\sim$1104 K, $\vo^+$ is the primary donor, whereas above this threshold $\fei^+$ will become the primary donor.
To simultaneously show the effect of oxygen partial pressure we plot a heat map of electron polaron concentration and the difference between $\fei^+$ and $\vo^+$ concentration in Figure~\ref{fig:intrinsic}e-f. The correspondence between the two figures reveals the importance of forming $\fei$ in achieving higher carrier concentrations in \ce{Fe2O3}.


\begin{table}[H]
\footnotesize
\centering
\begin{tabular}{c|cccc}
\hline \hline
Defect & $E^f$ (eV) &  $IE$ (eV) \\
\hline
$\vo$ &    2.06 &       0.70 \\
$\fei$ &    3.46 &      -0.01 \\
$\vfe$ &    4.14 &       --   \\
$\oi$ &    3.15 &       --   \\
\hline \hline
\end{tabular}
    \caption{The formation energy and first ionization energy of intrinsic defects at $\po=1$ atm in undoped \ce{Fe2O3}.}
\label{table:intrinsic}
\end{table}

The observation above is important in resolving the long-standing confusion about which defect acts as the major carrier donor in pristine hematite.
Our results here suggest that previous debate over the primary donor can be explained by the transition from $\vo^+$ to $\fei^+$ which has not been previously identified.
Furthermore, this transition highlights the varying importance of defect solubility versus ionization energy.
While the ionization energy of $\vo$ is 0.7 eV, it has the highest solubility amongst intrinsic defects, with a formation energy 1.4 eV lower than that of $\fei$.
At lower synthesis temperatures (\textit{e.g.}\ below $1100$ K), the formation of $\fei$ is sparse (less than 10$^{10}$ cm$^{-3}$ when $T_S=873$ K as shown in Figure~\ref{fig:intrinsic}c) and by consequence $\vo$ is the primary source of carriers.
In this situation the carrier concentrations are extremely low, $\sim$10$^{12}$ cm$^{-3}$ because the Fermi level is pinned at the first charge transition level of $\vo$ at $\sim$0.94 eV. This observation is in good agreement with recent measurements of undoped \ce{Fe2O3} which exhibit Fermi level positions between 0.8 and 1.2 eV.~\cite{lohaus2018limitation}
When the synthesis temperature is increased or the oxygen partial pressure is decreased, the formation of $\fei$ is more achievable and eventually it can act as the major donor in \ce{Fe2O3}.
In this situation, $\fei$ is always ionized $\fei^+$ due to a negative ionization energy, -0.01 eV, and so the carrier concentrations of hematite can be dramatically increased (red regions in Figure~\ref{fig:intrinsic}d-f where $\fei^+$ is the primary donor, $EP$ concentrations can reach $\sim$10$^{18}$ cm$^{-3}$). In this situation the Fermi level will approach the free polaron limit as experimentally observed.~\cite{lohaus2018limitation}
It has to be noted that there is some difference between our computed polaron concentration and expeimentally measured polaron concentration for pristine hematite, which could be related to two perspectives0.\cite{wang2011facile, ling2012influence} First, experimentally, hematite are often grown and measured on substrate such as fluorine doped hematite (FTO). The interface between hematite film and substrate could actually play an important role in shifting Fermi level (carrier concentration), i.e. low and high work function contact materials could raise and lower the Fermi level at the interface, respectively.\cite{lohaus2018limitation}
 Second, the carrier concentration (or Fermi level) of pristine system is very sensitive to temperature and pressure. The experimental vacuum synthesis condition is not easy to be correctly defined in simulation.

\subsection{Tetravalent and Pentavalent Dopant Raise Carrier Concentrations}

 \begin{figure}[H]
\centering
\includegraphics[keepaspectratio=true,width=1\linewidth]{2-tmo/figures-fe2o3/Figure2_v4.png}
\caption{
Atomic structures and defect concentrations of extrinsic dopants which are best at raising carrier concentrations in \ce{Fe2O3}.
(a) Atomic structures of \ce{Fe2O3} with neutral dopants Ti, Ge, Nb, and Sb. In the case of tetravalent and pentavalent the yellow cloud(s) represent the one or two nearby electron polaron(s), respectively.
(b) Room temperature defect, dopant, and carrier concentrations of Ti, Ge, Nb and Sb doped hematite at $p_{O_2}$ = 1 atm and as a function of synthesis temperature.
In the atomic plots, gold=Fe, red=O, and the remaining colored atom is the dopant as labeled within each figure.
The yellow cloud is an isosurface of the polaron wavefunction with an isosurface level of 5\% its maximum.
}
\label{fig:dopant}
 \end{figure}


 \begin{table}[h]
\footnotesize
\centering
\begin{tabular}{c|cccc}
\hline \hline
Dopant & $E^f$ (eV) &  $IE$ (eV) &  $\rho_{EP}$(${\rm cm}^{-3}$) &  $\rho_{EP}^{exp}$(${\rm cm}^{-3}$)  \\
\hline
Ti &        0.884 &           0.157 &        7.4$\cross 10^{19}$ &  1.4$\cross 10^{19}$-3.3$\cross 10^{20}$ \cite{wang2011facile,rioult2014single,malviya2017influence,hahn2010photoelectrochemical,zandi2013highly,glasscock2007enhancement}\\
Ge &        0.810 &           0.197 &        4.7$\cross 10^{19}$ &  3.0$\cross 10^{19}$ \cite{liu2014highly}\\
Sb &        0.546 &           0.247 &        4.1$\cross 10^{19}$ &  1.1$\cross 10^{20}$ \cite{annamalai2018influence}\\
Nb &        1.461 &           0.153 &        2.2$\cross 10^{19}$ &  5.0$\cross 10^{19}$ \cite{sanchez1988photoelectrochemistry} \\
Bi &        1.155 &           0.198 &        1.4$\cross 10^{19}$ &  N/A\\
As &        0.808 &           0.268 &        1.3$\cross 10^{19}$ &  N/A\\
Sn &        0.883 &           0.255 &        1.2$\cross 10^{19}$ &  6.5$\cross 10^{18}$-1.1$\cross 10^{20}$ \cite{li2017morphology,tian2020electronic,ling2011sn,yang2016acid}\\
Pb &        1.061 &           0.241 &        9.0$\cross 10^{18}$ &  N/A\\
Ta &        1.224 &           0.260 &        6.1$\cross 10^{18}$ &  N/A\\
Hf &        1.143 &           0.267 &        4.2$\cross 10^{18}$ &  N/A\\
Zr &        1.230 &           0.259 &        3.7$\cross 10^{18}$ &  N/A\\
V &        1.137 &           0.344 &        1.0$\cross 10^{18}$ &  N/A\\
P &        1.926 &           0.348 &        1.8$\cross 10^{17}$ &  N/A\\
Si &        1.949 &           0.165 &        1.4$\cross 10^{17}$ &  N/A\\
\hline \hline
\end{tabular}
\caption{Summary of representative dopants considered in this work with their formation energy ($E^f$), first ionization energy ($IE$), and induced polaron concentration ($\rho_{EP}$) computed at room temperature with synthesis at $T_S=1073$ K and $\po=1$ atm.
}
\label{table:summary}
\end{table}



 In order to achieve higher carrier concentrations and optimize the efficiency of \ce{Fe2O3}-based devices, extrinsic doping will be necessary.
 In order to broadly survey potential dopants and identify optimal doping strategies we studied all group IV, V, XIV, and XV elements. Intuitively substituting trivalent Fe ions by tetravalent or pentavalent ions will donate electrons due to the increased valence count.
 In Figure~\ref{fig:dopant} we show the results of four dopants (Ti, Ge, Nb, Sb) that we found enhance carrier concentration of \ce{Fe2O3} the most, at room temperature and under typical synthesis conditions, $\po=1$ atm and $T_S=1073$ K.~\cite{tian2020electronic}
 We find dramatic enhancement in electron polaron concentration of \ce{Fe2O3} after atomic doping (dashed red line in Figure~\ref{fig:dopant}b).
 Specifically, whereas in undoped \ce{Fe2O3} free electron polaron concentrations are maximally $\sim$10$^{18}$ cm$^{-3}$, many dopants can raise electron polaron concentrations higher.
 When comparing all of the studied dopants together (see Table~\ref{table:summar}) we find that Ti-doped \ce{Fe2O3} has the highest yield for enhancing polaron concentration which exhibits concentrations of $\rm 10^{20}\ cm^{-3}$, in great agreement with experimental measurement. \cite{hahn2010photoelectrochemical,glasscock2007enhancement,rioult2014single,fu2014highly,malviya2017influence}
 Meanwhile our predictions of Ge and Sb as effective dopants in raising carrier concentrations to $\rm 10^{19}-10^{20}\ cm^{-3}$ are consistent with experimental measurements as well.\cite{annamalai2018influence,liu2014highly}
 The excellent agreement with experiments on various dopants highlight the robustness of our first-principles prediction of defect properties and the depiction of carriers as small electron polarons.
 Based on our carrier concentration calculation results, we predict that Ti, Ge, Nb and Sb are excellent candidates for further investigation as potentially promising dopants for enhancing PEC performance in hematite.


 In Table~\ref{table:summary}, some correlation between the dopant formation energy and the carrier concentration ($\rho_{EP}$) is apparent, while the ionization energy seems less important.
 % however, it is worth noting that this trend actually breaks for certain cases where the formation energy  ($E^f$) should be considered.
 The best example of this is the case of P and Si which have nearly identical formation energies (1.926 and 1.949 eV), and despite a remarkable difference in their ionization energy (0.35 and 0.17 eV), P doping is predicted to have only slightly larger electron polaron concentrations than Si (1.8$\times\ 10^{17}$ and 1.4$\times\ 10^{17}$ cm$^{-3}$).
 Meanwhile, in Figure~\ref{fig:trends}a we show there is a dramatic change in carrier concentrations for Si doping as a function of synthesis temperature, which underperforms P doping at low temperatures but then outperforms at higher temperatures.
 This result is reminiscent of Figure~\ref{fig:intrinsic}d, and touches on the outstanding question of what is most important in determining the ability for a dopant or defect to raise carrier concentrations, \textit{e.g.}\ low formation energy or low ionization energy?

\subsection{Critical Role of Synthesis Temperature}

 In order to directly address this question, we performed linear regression on the larger data sets obtained from dopant calculations to analyze the importance of dopant formation energy or solubility against that of ionization energy. % Furthermore we computed our data sets for various synthesis temperature to highlight the importance of experimental conditions.
 Figure~\ref{fig:trends}b shows the predictive score (\textit{e.g.}\ the coefficient of determination $R^2$) of modeling the data sets such as those shown in Figure~\ref{fig:trends}c-e by three cases. First (red line in Figure~\ref{fig:trends}b) the polaron concentration is predicted by a single exponential term from the dopants ionization energy ($e^{IE}$). Second (in blue) a single exponential from the dopants formation energy ($e^{FE}$) was used to predict the polaron concentration. Finally (in black) the polaron concentration is predicted by both terms ($e^{IE}+e^{FE}$).
 When using both terms (black), the predictive score is typically exceeding 0.85 which signifies that these two dopant properties uniquely determine the induced electron polaron concentrations. Formation energy determines how much dopants can be incorporated into the crystal lattice of hematite while ionization energy determins how much dopants can be ionized.

 \begin{figure}[H]
\centering
\includegraphics[keepaspectratio=true,width=1\linewidth]{2-tmo/figures-fe2o3/Figure3_v8.png}
\caption{
Resolving the importance of dopant formation energy and ionization energy in determining carrier concentrations of \ce{Fe2O3}.
(a) Electron polaron concentration change of Si (red) and P (blue) doped hematite at room temperature and $\po=1$ atm as a function of synthesis temperature ($T_S$).
(b) Predictive score of linear regression models on the induced carrier concentrations using dopants formation energy (blue), dopants ionization energy (red), or both (black).
Electron polaron concentrations at room temperature and $\po=1$ atm for various synthesis temperatures: (c) 873, (d) 1073, and (e) 1373 K, plotted against dopant formation energies and with dopant ionization energies distinguished in colors. (f) The correlation between formation energy and ionic radius of different dopants. (g) The correlation between ionization energy and ionic radius of different dopants.
}
\label{fig:trends}
 \end{figure}


More importantly, the decomposition into dopant formation energy (blue) or dopant ionization energy (red) reveals the relative importance of these two factors in determining the induced carrier concentrations.
 For lower temperatures the solubility of the dopant (formation energy, $R^2 \sim 0.8$) almost uniquely determines how well the dopant is able to raise carrier concentrations, while ionization energy is significantly less important ($R^2 \sim 0.1$).
 This explains the above observation that P and Si, which have similar formation energy (1.926 and 1.949 eV), have nearly identical polaron concentrations ($\sim$10$^{17}$ cm$^{-3}$) at low synthesis temperature, despite significant difference in their ionization energy (0.348 and 0.165 eV).
 It can also explain how $\vo$, despite a significantly larger ionization energy (0.7 eV) than $\fei$ (-0.01 eV), is still the major donor in \ce{Fe2O3} at lower synthesis temperatures due to its lower formation energy (2.06 vs. 3.46 eV).
 Meanwhile, as the synthesis temperature is elevated poor solubility can be overcome and dopants ability to ionize is weighted equally with its formation energy (blue and red lines approach $R^2 \sim 0.5$ as synthesis temperature is increased in Figure~\ref{fig:trends}b).
 This helps to explain the dramatic increase in polaron concentration under Si doping shown in Figure~\ref{fig:trends}a, as well as the transition from $\vo$ to $\fei$ as the source of carriers in undoped \ce{Fe2O3} shown in Figure~\ref{fig:intrinsic}c-f.
 Beyond this we conclude that less soluble dopants such as Si, require higher synthesis temperatures to reach carrier concentrations seen for more soluble dopants such as Ti, Ge, and Sb (Figure~\ref{fig:dopant}b).

 We also explicitly consider the effect of entropy to the total formation energy. Entropy contributes formation energy from two aspects: configurational entropy and vibrational entropy. Configurational entropy is depending on the number of different possible configurations for the defect to be placed in hematite lattice, which always stabilizes the dopant formation. It can be separated into two parts, entropy from ideal solution and excess entropy of mixing. Since the second term is small compared to the first term,\cite{ye2015generalized, ye2016high} in this work, the configurational entropy of ideal solution is used to approximate the total configurational entropy. On the other hand, the vibrational entropy is computed by taking the difference of vibrational entropy between doped species and undoped species. We chose Sn and Nb as two representatives for group IV and V elements and computed the entropy correction of them to the formation energy and found that their contribution to the total formation energy is 0.1-0.2 eV, which does not affect the defect concentration significantly (Figure ~\ref{fig:Snentropy} and ~\ref{fig:Nbentropy}). Based on that, formation energy without entropy correction is used across the whole paper.

 Some general trends between formation energy/ionization energy and ionic radius of dopants within each group are also observed. The formation energies of dopants in each group have a parabolic shape with respect to their ionic radius (Figure~\ref{fig:trends}f). Some dopants such as Ti and Sb have small formation energies, while some others such as P and Pb have much lower solubility. A radius around 60 pm seems to be a sweet spot, which is slightly smaller than that of the ionic radius of $Fe^{3+}$ (64.5 pm) due to the repulsion caused by polaron formation. A dashed line at 60 pm is drawn to guide the eyes. On the other hand, in Figure~\ref{fig:trends}g, it can be noted that the trends for group IV and group V elements are different. For group V elements, ionization energy is generally getting smaller with ionic radius. However, the trend is opposite for group IV elements, ionization energy is generally increasing with ionic radius, which could be related to the electronic configurations of the dopants after ionization. For instance, group IV dopants have inert gas electronic configurations after ionization energy, which are very stable and lower the ionization energy compared to group V dopants overall. In addition, with the increase of ionic radius, the electronic configurations of ionized group IV dopants are getting less stable, which explains the upside trend of them.
 These trends could help experimentalists make reasonable choices about what dopants to choose when there is only very limited information such as ionic radius.

\subsection{Conclusions}

 In summary, this work demonstrates the role of defects, dopants, and small polarons in determining carrier concentrations in a prototypical oxide, \ce{Fe2O3}. This work identifies from first-principles calculations that the critical role of synthesis temperature on small polaron carrier concentration in hematite, both pristine and doped. For pristine hematite, the major electron polaron donor switches from $V_O$ to $Fe_i$ with the ramp up of synthesis temperature because the high formation energy of $Fe_i$ can be overcome at higher temperature and its low ionization energy makes it easier to be ionized compared to $V_O$. From our survey of all group IV and V dopants, we find that Ti, Ge, Sb, and Nb, are able to achieve the highest free carrier concentration in \ce{Fe2O3}.
 The linear regression on our data set of these dopants under different chemical conditions and synthesis temperatures reveals that dopant solubility is more important in determining the improvement on carrier concentration in \ce{Fe2O3}.
 Our study suggests that lower solubility dopants such as Si will require elevated synthesis temperatures for them to be incorporated into the lattice.
 This work addresses several outstanding questions for hematite but will also be applicable to other polaronic oxides, therefore, our study has broader interests to fundamentally designing more efficient oxide-based energy conversion and storage devices.

\def\be{\Delta_{quad}}

\section{Dopant Clustering in \ce{Fe2O3}}

In our 2021 work published in Chemistry of Materials~\cite{smart2021doping}, we investigated the doping bottleneck in \ce{Fe2O3} and demonstrated it is caused by a novel form of dopant clustering which is mediated by dopant and small polaron interactions which give rise to dopant pairs which resemble electric quadrupoles and can form at very low concentrations.
As we have discussed above, applications of hematite and other transition metal oxides are limited by small polaron formation.
To overcome these limitations, several efforts have been made to dope hematite by tetravalent ions that yield improved photoelectrochemical performance of hematite photoelectrodes
\cite{lohaus2018limitation,biswas2020tuning,ling2011sn,li2017morphology,kumar2011electrodeposited,fu2014highly,malviya2017influence,yang2013new,tian2020electronic,liu2013ge}.
While the performance can be moderately improved via group IV and XIV dopants, the optimal doping concentration strongly varies with each individual dopant,\cite{kumar2011electrodeposited,fu2014highly,malviya2017influence,yang2013new,tian2020electronic,liu2013ge} thereby requiring extensive experimental testing each time \cite{walsh2017instilling}.
%exact optimization of doping remains obscure due to physical limitations of doping \cite{walsh2017instilling}.
%Specifically, the doping of tetravalent ions in hematite shows consistently non-monotonic behavior \cite{kumar2011electrodeposited,fu2014highly,malviya2017influence,yang2013new,tian2020electronic,liu2013ge}, although the exact doping concentration  optimal for improving performance, is seemingly random amongst dopants.

For example, researchers\cite{fu2014highly,malviya2017influence} have found that Ti-doped hematite photoanodes had the highest carrier density and photocurrent at a doping concentration of around 0.1\%. Meanwhile, several works\cite{yang2013new,tian2020electronic,bindu2018electrical} have found that optimal PEC performance with Sn-doped hematite photoanodes was achieved at 3\% Sn doping concentration. In all of these cases there is a direct correlation between optimizing carrier density and PEC performance; however, the mystery of extremely low optimal doping concentration for certain dopants remains elusive.
% There are two logical mechanisms for the stifling of the dopant effects: compensation by opposite-charged impurities or the clustering of dopants.
Two possible mechanisms could be responsible for the doping bottleneck: compensation by oppositely-charged defects or the clustering of dopants.
However, the concentration of intrinsic p-type defects is expected to be negligible in \ce{Fe2O3}\cite{lee2013thermodynamics}, which leaves a strong rationale for the clustering of dopants being the cause of low optimal doping concentration.
 % TS: I shortened this part and combined with the previous paragraph
On the theoretical side, group IV and XIV dopants in \ce{Fe2O3} have been previously investigated from first-principles, focusing on their
electronic structure, formation energy, and polaron hopping barrier\cite{smart2017effect,zhou2019silicon,zhou2015understanding,liao2011electron}.
% Although certain dopants were suggested to improve carrier concentration based on their reasonably low ionization energies\cite{smart2017effect,zhou2015understanding}, their equilibrium carrier concentrations were not computed and the interplay between dopants and intrinsic defects were not taken into account.
However, these studies cannot explain the low optimal doping concentration observed experimentally.

In this work, we will reveal the origin of the extremely low optimal doping concentration in \ce{Fe2O3} through a joint theoretical and experimental study. We suggest a novel form of dopant clustering in polaronic oxides and conclude its critical role on determining carrier concentration.
We begin by detailing our computational methodology, including our proposed model for disentangling the effects of dopant clustering.
Next, the electronic structure of isolated and clustered Sn dopant formation is provided, which resembles an electric dipole and quadrupole, respectively. The binding energy of the clustered dopants as quadrupoles is computed to validate their thermodynamic stability. Then, the formation of the theoretically predicted Sn-Sn pairs are confirmed by experimental EXAFS, and
their mechanistic origin is unraveled theoretically in terms of electrostatic, magnetostatic and strain effects.
Finally, carrier concentrations of \ce{Fe2O3} with and without dopant clustering are computed to elucidate the underlying mechanism of the doping bottleneck. At the end, essential design principles are provided to yield higher conductivity in polaronic oxides for the advancement of energy conversion applications.

\subsection{First-Principles Calculations}
All Density Functional Theory (DFT) calculations were carried out using the open source plane-wave code QuantumESPRESSO \cite{QE1} with ultrasoft pseudopotentials \cite{gbrv} and an effective Hubbard $U$ \cite{dudarev1998electron} value of 4.3 eV for Fe $3d$ orbitals \cite{smart2017effect,adelstein2014density}.
This $U$ value is chosen for its ability to reproduce the bandgap of hematite ($\sim2.21$ eV) but also has shown to capture physics of small polarons such as the polaron hopping barrier~\cite{adelstein2014density,smart2017effect}.
Plane-wave cutoff energies of 40 Ry and 240 Ry were used for wavefunctions and charge density, respectively. All calculations were performed with a $2\times 2\times 1$ supercell (120 atoms) of the hexagonal unit cell with a $2\times 2\times 2$ $k$-point mesh for integration over the Brillouin zone. A $3\times 3\times 1$ supercell was also tested to ensure convergence with supercell sizes (see supercell convergence in SI Table S1).
The consistency between these supercell sizes also validates that the present calculations are in the dilute limit and while the Sn at Fe concentrations of the aforementioned supercells are 2.08\% and 0.93\%, respectively, these concentrations do not yield interactions between dopant periodic images (even for systems with two dopants, see SI Figure S1). The actual concentrations of dopants are determined by evaluating charge neutrality directly from dopant formation energies at a synthesis condition as discussed later in this section.
Finally, we note that we use the same $U$ value for systems with dopants. While it is true that changing $U$ will vary the bandgap (in this case the conduction band shifts due to $U$ correction on Fe $3d$), it has been shown that the formation energies computed with different $U$ values were very similar and the ionization energies changed little when referenced to the free polaron level instead of the CBM for \ce{Fe2O3}.~\cite{smart2017effect}
%Any non-neutral calculations (such as charged formation energy calculations) included a charged defect correction was computed using the JDFTx code \cite{JDFTx} following the scheme developed in Ref.~\cite{ping2013}, which is necessary in order to remove the spurious interactions of the polarons with their periodic images and with the uniform compensating background charge \cite{kokott2018first}.


\subsection{Electrostatic, Magnetostatic, and Strain Model of Binding Energy}
We will later demonstrate that single dopants resemble dipoles, while dopant-pairs resemble quadrupoles. Here, we demonstrate our analysis on the physical contributions to the quadrupole binding energies computed from first-principles as we will discuss later, by using an electrostatic, magnetostatic, and strain (EMS) model.
In this model, the binding energy is obtained by separately computing electrostatic ($\be^{elec}$), magnetostatic ($\be^{mag}$), and strain ($\be^{strain}$) contributions.
For the electrostatic effect, we compute the electrostatic potential contribution to binding energy ($\be^{elec}$) by taking the difference between the quadrupole
(quad) and twice the dipole configuration (dipole):

\begin{align}
    \be^{elec} = \frac{k\alpha}{2 \epsilon_r} \left( \sum_{ij}^{quad} \frac{q_i q_j}{r_{ij}} - 2\sum_{ij}^{dipole} \frac{q_i q_j}{r_{ij}} \right).
    \label{eq:be_elec}
\end{align}
% k = 14.4   (eV*Å/e)
% eps_r = 22.9 for hematite
% alpha = 4.1719 for corundum
Here $k$ is the Coulomb constant, $\alpha$ is the Madelung constant, and $\epsilon_r$ is the relative permeability (22.9 for hematite). \cite{onari1977infrared} The summation goes over all polarons and dopants $i$ and $j$, with relative charges $q_{i}$ and $q_{j}$, and physical separation $r_{ij}$.
%\textcolor{red}{not sure this is necessary:}For example, in the dipole system there will be one term for the electrostatic interaction between the single Sn ($q=+1$) and the single $EP$ ($q=-1$). Meanwhile in the quadrupole system there will be six terms corresponding to four Sn-$EP$ interactions, one $EP$-$EP$ interaction, and one Sn-Sn interaction.

Magnetic effects were computed using the Heisenberg Hamiltonian $H_{spin}=-\frac{1}{2}\sum_{ij}J_{ij}\hat{S}_i\cdot\hat{S}_j$, where $J_{ij}$ is the magnetic coupling between the spins of the $i^{\rm th}$ and $j^{\rm th}$ ion, and $\hat{S}_i$ is the spin of the $i^{\rm th}$ ion. Here we use the magnetic exchange coupling constants computed from Ref.~\cite{nabi2010magnetic}, which provided magnetic couplings for both the superexchange between two $\rm Fe (3+)$ or between two $\rm Fe (2+)$, as well as the double-exchange between $\rm Fe (3+)$ and $\rm Fe (2+)$. For high-spin $\rm Fe (3+)$ and $\rm Fe (2+)$, the value of $\hat{S}_i$ is $5/2$ and $2$, respectively, while the spin of the tetravalent dopants is zero (hence the magnetic interaction with these dopants is always zero). In this way, we can compute the magnetic contribution to the binding energy from the magnetic energy of the quadrupole system subtracted by two times of the dipole system:
\begin{align}
    \be^{mag} = -\frac{1}{2}\left( \sum_{ij}^{quad} J_{ij} \textbf{S}_i \cdot \textbf{S}_j - 2\sum_{ij}^{dipole} J_{ij} \textbf{S}_i \cdot \textbf{S}_j\right).
    \label{eq:be_mag}
\end{align}
In this work, we assume $J_{ij}$ between two Fe ions before and after doping are the same, as lattice distortions are generally small compared to Fe distances.
%and make the approximation that changes in $J_{ij}$ due to lattice distortions are likely negligible.
With the above consideration and the non-magnetic nature of dopants, our computed magnetic energy is identical for all the dopants.
% Note the magnetic energy does not depend on the dopant when the spin of the dopant is zero. Also, we make the approximation that changes in the coupling parameters $J$ between Fe ions due to lattice distortion is negligible., our computed magnetic energy is identical for all non-magnetic dopants.

Finally, in order to compute the strain contribution to the binding energy, we evaluated the change in energy induced by lattice distortions before and after doping. As usual, taking this energy for the quadrupole system and subtracting twice the dipole system:
\begin{align}
    % \be^{strain} = \left[E_{prist} - E_{prist}(R_{quad})\right] - 2\left[E_{prist} - E_{prist}(R_{dipole})\right].
    \be^{strain} = E_{quad}^{strain} - 2 E_{dipole}^{strain}.
    \label{eq:be_strain}
\end{align}
Here $E_{X}^{strain}$ is the strain energy of system $X$, computed as a difference of total energy of the pristine system with its equilibrium geometry and with relaxed geometry from the doped system (first relax with dopants then substitute back Fe atoms to keep the same composition as pristine \ce{Fe2O3}). We note a similar approach was used in Ref.~\cite{bao2017first} to evaluate strain energies.
% Here $E_{prist}$ is the total energy of the pristine system with a relaxed structure and $E_{prist}(R_{X})$ is the total energy of the pristine system with a fixed structure corresponding to the atomic positions in the doped system. We note a similar approach was used in Ref~\cite{bao2017first}.

% \subsection{Charged Defect Formation Energy and Concentration}
% We computed the formation energy ($E^f_q$) of each defect at a charge state $q$ according to:
% \begin{align}
%     E^f_q(X; \varepsilon_F) = E_q(X) - E_{prist} + \sum_i \mu_i \Delta N_i + q \varepsilon_F + \Delta_q,
%     \label{eq:cfe}
% \end{align}
% where $E_q(X)$ is the total energy of the defect system ($X$) with charge $q$, $E_{prist}$ is the total energy of the pristine system, $\mu_i$ and $\Delta N_i$ are the chemical potential and change in the number of atomic species $i$, and $\varepsilon_F$ is the electron chemical potential. A charged defect correction $\Delta_q$ was computed with techniques developed in Refs. \cite{PING2017JCP, wu2017first} and implemented in the JDFTx code \cite{JDFTx}. The chemical potentials were carefully evaluated against the stability of byproduct compounds as detailed in the elemental chemical potential section of the SI.
% The corresponding charge transition levels of the defects were obtained from the value of $\varepsilon_F$ where the stable charge state transitions from $q$ to $q'$.
% \begin{align}
%     \epsilon_{q|q'} = \frac{E^f_q - E^f_{q'}}{q' - q}
%     \label{eq:ctl}
% \end{align}
% The ionization energies are computed by referencing the CTLs to the free polaron state.
% \cite{seo2018role,radmilovic2020combined,zhou2020interstitial,lee2020electrochemical}
% Namely in \ce{Fe2O3}, it has been experimentally observed that photoexcited carriers relax on picosecond timescale to form small polarons~\cite{carneiro2017excitation}, which have been measured to form at energies $\sim$0.5 eV below the conduction band minimum (CBM)~\cite{lohaus2018limitation,pastor2019situ}. Theoretically this free polaron level is computed as the charge transition level from $q=0$ to $q=-1$ in the pristine system, $\varepsilon_{FP} = \varepsilon_{-1|0}^{prist} = E^f_{-1}(prist) - E^f_0(prist)$. By this method we obtain that the free polaron level is positioned at 0.497 eV below the CBM in excellent agreement with experimental observation.~\cite{lohaus2018limitation,pastor2019situ}
%
% From charge defect formation energies, charged defect concentration ($c_q$) can be computed as:
% \begin{align}
%     c_q(X; \varepsilon_F) = g \exp [- E^f_q(X; \varepsilon_F) / k_B T],
% \end{align}
% where $g$ is the degeneracy factor accounting for the internal degrees of freedom of the point defect, $k_B$ is the Boltzmann factor, and $T$ is temperature.
% Concentrations including intrinsic defects, extrinsic dopants and free electron polarons were computed by determine their charge neutrality condition~\cite{lee2013thermodynamics,freysoldt2014first}
% (additional details are provided in the defect concentration section of the SI).
% To best relate to experimental measurements of \ce{Fe2O3} photoanodes, we computed the concentration first at a synthesis temperature of $T_S=1073$ K (800 $^\circ$C is a common synthesis temperature \cite{ling2011sn,tian2020electronic}), and then recomputed charge neutrality at normal operation (room) temperature $T_O=300$ K.
% The partial pressure of oxygen gas ($p_{\rm O_2}$) corresponding to atmospheric condition (1 atm) is used.

% In order to maintain charge neutrality, the introduction of defect $X$ with charge $q$ into the lattice must be compensated by defects of opposite charge or through the generation of free carriers. Specifically charge neutrality must be held:
% \begin{align}
%     \sum_{X,q} c_q(X; \varepsilon_F) + n_h - n_e = 0, \label{eq:neutrality}
% \end{align}
% where $n_h$ denotes the concentration of free delocalized holes, and $n_e$ denotes free delocalized electrons. However, in hematite free electrons will nearly entirely localize into small polarons, so the concentration of free small electron polarons is entered in Eq.~\ref{eq:neutrality} in a similar way to that of a defect, i.e.\ with a specific formation energy.\cite{lee2013thermodynamics}
% Furthermore, in order to relate to experimental measurements of \ce{Fe2O3} photoanodes, we computed the concentration first at a synthesis temperature of $T_S=1073$ K (800 $^\circ$C is a common synthesis temperature \cite{ling2011sn,tian2020electronic}), and then recomputed charge neutrality at normal operation (room) temperature $T_O=300$ K, considering thermodynamic equilibrium at each temperature (additional details can be found in the sections of defect formation energy and concentration in SI). In the end, the only variable of this approach is the partial pressure of oxygen gas ($p_\ce{O2}$).



% \section{Results and Discussion}

% \subsection{Formation of dopant-polaron quadrupoles}
% Electronic and local structure
% Binding energy (scatter plot)
% EMS model

% prelude to the section

\begin{figure}
    \centering
    \includegraphics[keepaspectratio=true,width=0.8\linewidth]{2-tmo/figures-fe2o3/electronic_500.png} % high resolution
    \caption{Electronic structure of Sn-doped \ce{Fe2O3}. (\textbf{a}) Wavefunction of the small electron polaron ($EP$) in the single Sn-dopant system where the $EP$ and Sn form a dipole. (\textbf{b}) Band structure and projected density of states (PDOS) of the dipole Sn system. (\textbf{c}) Wavefunctions of the two $EP$ in the two Sn-dopant system where the two $EP$ and two Sn form a quadrupole.
    The Sn-Sn separation is 3.784 {\AA}.
    (\textbf{d}) Band structure and PDOS of the quadrupole Sn system. For the atomistic plots, gold=Fe, red=O, grey=Sn, and the yellow/blue ($+$/$-$) cloud is the isosurface of the polaron wavefunction (the isosurface level is 1\% of the maximum). In the band structures, dark/light blue is spin up/down and $\varepsilon_F$ is the Fermi energy.
    }
    \label{fig:electronic}
\end{figure}

% \subsection{Stable Multipole Formation by Dopant Clustering}
\subsection{Dopant Clustering by Multipole Formation}
% . experimentally there is strong indication of clustering from XAS; showing certain high Sn-Sn distance at very low concentration.
% . We found quadrupole can be formed with stable binding energy and agree with experimental XAS.
%
% -> Propose this interpretation of clustering
% -> Figure 1 goes here ;  describe the results of one dopant and two dopants; and binding energy (mention negative values)


Substitutional doping by group IV and XIV elements was investigated theoretically by replacing a single Fe site by the dopant ($X$). Consistent with previous studies \cite{smart2017effect,zhou2015understanding} and experimental observation,\cite{ling2011sn,li2017morphology} we found this process yields the formation of small electron polarons corresponding to the identification of $\rm Fe (2+)$ after replacing $\rm Fe (3+)$ by the tetravalent dopant $X (4+)$. (Note in this paper we use the notation $X (i)$ to denote an ion $X$ with valency $i$). The electronic structure of the single Sn-doped system is shown in Figure~\ref{fig:electronic}b right panel, wherein the band structure exhibits a flat isolated occupied state in the gap corresponding to the small electron polaron ($EP$) with tight spatial localization similar in size to the Fe$-$O bond lengths. Likewise, the projected density of states (PDOS) in Figure~\ref{fig:electronic}b shows a sharp isolated peak composed mostly by Fe $3d$. The wavefunction of the $EP$ is shown in Figure~\ref{fig:electronic}a with a clear $d_{z^2}$ character. The $EP$ forms at the Fe site nearest to the Sn dopant with a Sn$-EP$ distance of $2.981$ {\AA} ($d_0$ in Figure~\ref{fig:electronic}a).


To investigate dopant-dopant interactions, we placed a second Sn dopant in the lattice. All possible Sn-Sn pair configurations were tested, and the lowest energy configuration was clearly identified (see SI Figure S2, S3 and Table S2). The electronic structure of this corresponding configuration is shown in Figure~\ref{fig:electronic}c-d, which exhibit the formation of two $EP$ states.
% with a noticeable energetic difference of 0.12 eV.
We find the asymmetry of the local structure, which %differing distances of the polaron to the two Sn atoms
is a natural consequence of the corundum crystalline form, causes a noticeable energetic difference of 0.12 eV between $EP_1$ and $EP_2$.
Specifically, in Figure~\ref{fig:electronic}c, $EP_1$ has distances to the two adjacent Sn of $d_1=3.011$ and $d_2=3.129$ {\AA}, whereas $EP_2$ has distances of $d_3=3.510$ and $d_4=4.112$ {\AA}. The proximity of $EP_1$ to the Sn yields a lower energy state relative to $EP_2$.
Lastly, the theoretically predicted Sn-Sn distance of $3.784$ {\AA} closely matches experimentally observed Sn-Sn peak in EXAFS data of Sn-doped hematite samples (as discussed in next section).
% closely matches experimentally observed Sn-Sn peak in EXAFS data of Sn-doped hematite samples when the dopant concentration exceeds 3\%, which might imply a similar dopant clustering. \cite{annamalai2015activation}
% TS: I modified the above sentence since we have our own EXAFS data in the next section. That section also refers back to this section

The remaining group IV (Ti, Zr, Hf) and XIV (Si, Ge, Sn, Pb) dopants were also simulated in both single and pair dopant configurations with negligible differences in their electronic structure
and polaron configurations from Sn (all electronic structures are presented in SI Figure S4-S9).
Note for the present study, the configuration of two dopants is chosen to be the same for all dopants for the purpose of discussing chemical trends, as predicted by the case of Sn. It is possible that dopants may vary in their exact pair dopant configuration, for example see SI Table S3. This variation does not affect the main implications on carrier concentration we conclude later.
Most importantly, the stable configuration of a single tetravalent dopant (such as Sn) resembles an electric dipole where the Sn and $EP$ represent positive and negative charge centers, respectively. In this way the system with two dopants resembles an electric quadrupole (two positive Sn centers and two negative $EP$ centers).
Therefore, we will denote the single doped system as a first-order multipole (dipole) system and the pair doped systems as a second-order multipole (quadrupole) system.
%Expanding on this idea, one may consider that larger order multipoles in \ce{Fe2O3} may form, wherein the negative charge of the polarons and the positive charge of the dopants aggregate together analogous to an electric multipole expansion.
To examine the thermodynamic stability of dopant-polaron quadrupole, we studied their binding energy ($\be$) from two separate dipoles:
%by the difference in the formation energy of the quadrupole and that of two separate dipoles:
\begin{align}
    \be = E^f(quad) - 2 E^f(dipole).
    \label{eq:be1}
\end{align}
Here, $E^f(X)$ is the formation energy of the system with neutral dopants in a configuration $X$ (e.g.\ quadrupole or dipole dopant system).
%and is defined as $E^f(X) = E(X) - E_{prist} + \sum_i \mu_i \Delta N_i$. Here $E(X)$ and $E_{prist}$ is the DFT total energy of the doped and pristine systems, respectively, and $\mu_i$ is the chemical potential of atomic species $i$ with change $\Delta N_i$.
%Substituting this definition into Eq.~\ref{eq:be1} simplifies the binding energy into a simple expression of DFT total energy,
%\begin{align}
%    \be = E(quad) - 2 E(dipole) + E_{prist}.
%    \label{eq:be2}
%\end{align}
The quadrupole binding energy was evaluated for all group IV and XIV dopants considered in this study. We observed that the binding energy for all dopants is negative ($\sim-$0.1 to $-$0.2 eV), as shown in Table~\ref{table:ems} ($\be^{DFT}$), indicative of a strong tendency for dopants and polarons to aggregate.
We note that we expect dopant clustering occurs during the cooling process from synthesis temperature (over one thousand K here) down to room temperature. At a synthesis condition, dopants will be all ionized and the binding of dopants into quadrupoles will not occur.
Since all the binding energies are lower than $kT$ at room temperature, it is expected that the quadrupoles are stable at room temperature.
% All in all, the negative binding energy is indicative of a strong tendency for dopants and polarons to aggregate.


% \subsection {Experimental evidence for Sn pair formation: EXAFS results}
\subsection{Experimental Evidence for Dopant-Pair Formations}

Extended x-ray absorption fine structure (EXAFS) data at the Sn edge were collected at SSRL for two Sn doped \ce{Fe2O3} samples, with Sn nominal concentrations of 0.1\% and 1.0\%. (Sn concentration of 1\% corresponds to replacing 1 out of 100 Fe with Sn; measured concentrations are 10-20\% lower, see Table S4 in SI for details). Synthesis methods are detailed in the SI. A standard fluorescence set-up (32 element Ge fluorescent detector) was used with the sample set at 45$^{\circ}$ to the beam, and an Oxford helium cryostat maintained the temperature at 10 K. Details about the data collection and reduction are in the SI section on EXAFS characterization. The $r$-space data are plotted in Figure~\ref{exafs} for the 0.1\% Sn and 1.0\% Sn samples. For the 0.1\% Sn sample (Figure~\ref{exafs}a), the amplitudes of the further neighbor peaks are quite large and the data can be well fit (solid orange line) to the hematite structure, with a small expansion for the Sn-Fe pairs compared to hematite;  roughly 0.1 {\AA} for closer pairs but only 0.02 {\AA} for Fe neighbors near 3.7 {\AA}. This is the expected behavior around a substitutional dopant site when the dopant valence $\rm Sn(4+)$ is higher that the host valence $\rm Fe(3+)$, and this behavior has been observed in other similar situations.\cite{mackeen2018substitution}. The further neighbor Sn-O peaks are expected to contract very slightly, but because these small peaks overlap the larger Sn-Fe peaks, the pair-distances fluctuate too much. The first O shell, although split in hematite, collapses to a single peak with an average Sn-O distance of 2.05 {\AA}, very close to the averaged first neighbor distance in hematite, 2.03 {\AA}; this is a competition between a larger ionic radius for $\rm Sn(4+)$, and larger electrostatic force between $\rm Sn(4+)$ and $\rm O(2-)$.

On the other hand, the EXAFS $r$-space plot for 1\% Sn sample is quite different (Figure~\ref{exafs}b). The data up to 3 {\AA} are very similar to that for 0.1\% Sn - i.e.\ the phase of the real part of the Fourier transform, R($r$), is the same. However in the range 3-3.8 {\AA}, the phase changes dramatically and a dip develops in the amplitude near 3.3 {\AA} which has the  shape of an interference dip. It occurs close to the expected position for the Sn-Fe peak in an EXAFS plot (actual distance $\sim$3.7 {\AA}: note that there is a calculable phase shift of peaks in $r$-space plots to lower $r$). These data can't be fit to a simple distorted hematite model and the shape of R($r$) suggests that another peak is present (see SI Figure S11). Consequently, an additional peak, corresponding to one Sn-Sn pair with a distance close to 3.7 {\AA}, was included in the fit. The number of Sn-Fe pairs was correspondingly reduced from 6 to 5. This yielded the good fit (details in SI) shown in Figure~\ref{exafs}b, and is a clear evidence that Sn-Sn pairs have formed.
Remarkably, this Sn-Sn pair distance ($\sim$3.7 {\AA}) matches the theoretically predicted distance of Sn-Sn pair (3.784 {\AA}) shown above.
%which forms due to quadrupole formation.
% Again the Sn-O peaks are slightly contracted while the Sn-Fe pairs are slightly expanded; the long Sn-Sn peaks is at 3.67 {\AA}.

%***************************** Fig. X ******************************
\begin{figure}[H]
    \centering
    \includegraphics[width=0.47\linewidth]{2-tmo/figures-fe2o3/exafs_500.png}
    \caption{EXAFS $r$-space data at the Sn K edge, for (\textbf{a}) 0.1\% and (\textbf{b}) 1\% Sn in Fe$_2$O$_3$. The plot for 0.1\% Sn also shows a fit to the hematite structure; good agreement is obtained with a slight contraction of the Sn-O pairs and a slight expansion of the Sn-Fe pairs. At 1\% Sn, the EXAFS changes significantly. Although the first two peaks are very similar, the region from 3-3.8 \AA{} is quite different, particularly the shape of the phase (fast oscillating function), and a dip develops near 3.3 \AA{}. These data cannot be fit to the hematite structure. The data suggest that there is another peak present; in the fit shown in part (\textbf{b}), one of the Fe neighbors at $\sim$3.7 \AA{} is replaced with a Sn atom, forming a Sn-Sn pair. This leads to the excellent fit shown in (\textbf{b}).
    Fourier transform range, 3.5-13 \AA$^{-1}$; fit range in $r$-space, 1.1-4.2 \AA{} for both plots.
    In both figures, the blue and gold bars at the bottom indicate the position of Sn-O and Sn-Fe peaks, respectively, in undistorted hematite. The bar positions include the known shifts in $r$.
    }
    \label{exafs}
\end{figure}
%***************************** Fig. X ******************************


\subsection{Mechanisms of Dopant-Polaron Binding into Quadrupoles}
% . There are three parts:  electrostatic, strain,  and  magnetic. You need to explain briefly why physically important and how you compute them.
% . Then show results agree with DFT.
%
% One figure a,b,c
% -> first dipole can be explained by electrostatic
% -> however quadrupole we notice it has a correlation with strain (ionic radius)
% -> so we consider a model with three parts (ems)

% Considering the electron polaron carry a net negative charge, while the tetravalent dopants carry a net positive charge (with respect to $\rm Fe (3+)$), this system closely resembles the formation of an electronic dipole where the charge of the electron polaron and dopant are attracted electrostatically. % This was already said above
After confirming the existence of Sn-Sn clustering both theoretically and experimentally, we turn to investigate the mechanisms of their formation.
Above the analogy was made between the single Sn-doped system and electric dipoles,
so in order to probe this electrostatic interaction, we plotted the total energy of the single Sn-doped system as a function of Sn-polaron distance, as shown in Figure~\ref{fig:mechanisms}a.
The computed total energies were fitted to a Coulomb potential ($-a/r+b$) with the fitted values of $a=0.658\ {\rm eV\cdot\text{\AA}}$ and $b=0.230$ eV, and a coefficient of determination ($R^2$) of 0.85.
This validates a clear electrostatic attraction between the two bodies with opposite charges like a dipole.
%where the interaction is analogous to two point charges of opposite charge and the denotation of this system as a first-order multipole (dipole).
Furthermore, we find that $b$ is close to the value of ionization energy of Sn (0.25 eV), as expected.

%We note this Coulombic behavior is not universal, as non-Coulombic behavior was observed in,
However, this Coulombic interaction competes with other factors. For example
in Mo doped \ce{BiVO4}\cite{wu2018combining}, strain causes short-range repulsion between the dopant and polaron, and dominates over the Coulombic attraction.\cite{wu2018combining,zhang2018unconventional}
We find this was also reflected in the computed quadrupole binding energies ($\be$), as shown in Figure~\ref{fig:mechanisms}b, where we plotted them versus the ionic radius ($R_I$) of each dopant.
Specifically, there is a roughly positive correlation between the ionic radius and the quadrupole binding energy. We attribute this to the compensatory size effects of the dopant and the polarons. Namely, the replacement of Fe(3+) with ionic radius of 64.5 pm by $\rm Fe (2+)$ with ionic radius of 78.0 pm, yields an expansion strain at the lattice site. This strain can be reduced by smaller radii dopants (e.g.\ Ti, Ge, Si) that will increase the magnitude of $\be$ as shown in Figure~\ref{fig:mechanisms}b, or enhanced by larger radii dopants (e.g.\ Sn, Hf, Zr, Pb) that will decrease the magnitude of $\be$ towards zero.
We note that the case of Ti, an outlier in Figure~\ref{fig:mechanisms}b, possess stronger correlated $3d$ orbitals, which in turn exhibit stronger electron localization, may compensate local expansion from small polarons  and lower its binding energy (similar to Si which intrinsically is smaller than the rest).

\begin{figure}[H]
    \centering
    \includegraphics[keepaspectratio=true,width=1.0\linewidth]{2-tmo/figures-fe2o3/mechanisms_500.png} % high quality
    \caption{
    Mechanisms of dopant-polaron binding in \ce{Fe2O3}.
    (\textbf{a}) Total energy of Sn-doped hematite system as a function of the Sn$-EP$ distance.
    % The black curve represents a simple Coulomb potential fit with fit parameters $a=0.658\ {\rm eV/\AA}$ and $b=0.230$ eV and an $R^2$ value of 0.85. The accurate fit of the Coulomb potential strongly reflects the electrostatic attraction between the Sn (positive charge) and $EP$ (negative charge).
    The black curve represents a simple Coulomb potential fit which has an $R^2$ value 0.85, supporting the intuition of an electrostatic interaction between the Sn (positive charge) and $EP$ (negative charge) centers.
    (\textbf{b}) Quadrupole binding energy ($\be$) of group IV and XIV dopants in hematite computed by Eq.~\ref{eq:be1}, plotted against the ionic radius of the dopant~\cite{shannon1976revised} ($R_I$; valency 4+, coordination VI).
    % Additionally, the ionization energy ($IE$) is shown by the color of each point.
    % TS: IE is discussed later so I decided to remove it here
    (\textbf{c}) Computed $\be$ by Eq.~\ref{eq:be1} plotted against those computed with the EMS model in Eq.~\ref{eq:ems}. The linear fit relation ($f(x)$) between these models is shown in the inset box.
    }
    \label{fig:mechanisms}
\end{figure}


Therefore, it is necessary to consider electrostatic and strain effects simultaneously, and also include magnetic effects (binding energies may also be modified by the antiferromagnetism of \ce{Fe2O3}).
%In order to put both electrostatic and strain effects together and additionally consider magnetic effects which can be considerable for \ce{Fe2O3},
Hence, we propose a model of the quadrupole binding energy based on electrostatic, magnetostatic, and strain effects (abbreviated to EMS), in order to analyze the importance of each contribution:
% In order to disentangle the various competing effects which dictate the formation of dopant-polaron quadrupoles, we propose a model of the binding energy based on electrostatic, magnetostatics, and strain (abbreviated to EMS):
\begin{align}
    \be^{EMS} = \be^{elec} + \be^{mag} + \be^{strain}.
    \label{eq:ems}
\end{align}
Here $\be^{elec}$, $\be^{mag}$, $\be^{strain}$, correspond to electrostatic, magnetostatic, and strain contributions to the quadrupole binding energy, respectively.
The exact formulation for each component of the EMS model is detailed in the methods section (Eq.~\ref{eq:be_elec}$-$\ref{eq:be_strain}) and the results are summarized in Table~\ref{table:ems}.
We evaluate how well this model reproduces DFT calculations by plotting them against each other in Figure~\ref{fig:mechanisms}c.
%In order to gauge the EMS model, we first consider its ability to reproduce the results from our DFT calculations which are plotted against each other in Figure~\ref{fig:mechanisms}c.
The linear fitting shows adequate agreement between the simple EMS model and our exact DFT calculations (with a slope near unity and an $R^2$ value of 0.864), which justifies its use for interpreting the DFT binding energies.
%Thus, the EMS model is able to reproduce the DFT results rather well and allows us to disentangle them into the three different components and interpret them separately.

As shown in Table~\ref{table:ems}, each of the three components contributes significantly to the overall quadrupole binding energy. The electrostatic interaction ($\be^{elec}$) is typically the most dominant factor, and intuitively it is chiefly responsible for the attraction of dopants into the quadrupole configuration.
Interestingly, we also found a non-negligible contribution from magnetostatic interactions ($\be^{mag}$, 42 meV) which further participates in the binding of quadrupoles.
This effect is non-trivial but is an indirect consequence of placing non-magnetic dopants next to each other, which in-turn breaks fewer antiferromagnetic interactions and yields a lower energy configuration when dopant-pairs form. % \textcolor{red}{dopant clustering} occurs.
In contrast, the effect of strain ($\be^{strain}$) typically mitigates the formation of dopant-pairs % \textcolor{red}{dopant clustering}
(increases the system's energy with clustering) due to the accumulation of lattice distortion. However, as aforementioned, this effect can be compensatory in the dopant cases with smaller ionic radii than Fe(3+), which can pack more efficiently next to the polarons with larger ionic radius as specified in Figure~\ref{fig:mechanisms}b.
% TS: I suppose the best solution is to just remove this sentence: \textcolor{red}{This is particularly evident in the case of Si which was found to be an outlier with a large negative strain component due to its significantly smaller ionic radius. -

\begin{table}[H]
    \footnotesize
    \centering
    \begin{tabular}{c|ccc|c|c}
    \hline \hline
    Dopant & $\be^{elec}$ & $\be^{mag}$ & $\be^{strain}$ & $\be^{EMS}$ & $\be^{DFT}$ \\
    \hline
    Si & -0.184 & -0.042 & -0.33 & -0.553 & -0.227 \\
    Ge & -0.188 & -0.042 &  0.07 & -0.160 & -0.117 \\
    Ti & -0.175 & -0.042 & -0.04 & -0.256 & -0.233 \\
    Sn & -0.188 & -0.042 &  0.18 & -0.054 & -0.029 \\
    Hf & -0.188 & -0.042 &  0.13 & -0.098 & -0.055 \\
    Zr & -0.192 & -0.042 &  0.13 & -0.103 & -0.080 \\
    Pb & -0.204 & -0.042 &  0.11 & -0.135 & -0.039 \\
    \hline \hline
    \end{tabular}
    \caption{Collected values of the binding energy for group IV and XIV dopants in \ce{Fe2O3} computed by DFT ($\be^{DFT}$) or with the EMS model ($\be^{EMS}$) as in Eq.~\ref{eq:ems}. The various components of the EMS model are tabulated as well including the electronic ($\be^{elec}$), magnetic ($\be^{mag}$), and strain ($\be^{strain}$). All values are given in eV.}
    \label{table:ems}
\end{table}


\subsection{Effects of Dopant Clustering on Polaron Concentrations}
% \subsection{Dopant Clustering Negatively Impacts Free Polaron Concentration}

\begin{figure}
    \centering
    \includegraphics[keepaspectratio=true,width=0.9\linewidth]{2-tmo/figures-fe2o3/quad_500.png}
    \caption{
    Effects of quadrupole binding on carrier concentration in \ce{Fe2O3}.
    (\textbf{a}) Band diagram of various doped systems either single-doped (dipole) or with two dopants (quadrupole), the solid horizontal lines represent the thermodynamic charge transition levels (CTLs), and the ionization energy corresponds to the separation of the CTLs to the free polaron line (dashed grey line).
    (\textbf{b}) Change in the first ionization energy when quadrupoles are formed vs.\ ionic radius ($R_I$). Notably, when the ionic radius of dopants is below that of $\rm Fe (3+)$, the ionization energy is increased (as shown in blue bars), whereas it is decreased when the dopant radius is larger (as shown in orange bars).
    (\textbf{c}) Computed free electron polaron concentration as a function of dopant concentration for Sn, Ge, and Ti, with and without the effect of clustering (i.e.\ quadrupole formation).
    }
    \label{fig:quad}
\end{figure}

Lastly, we discuss the effects of dopant clustering, which we will show to be responsible for the low optimal doping concentrations of \ce{Fe2O3} observed experimentally.
The computed thermodynamic charge transition levels (CTLs) are displayed in Figure~\ref{fig:quad}a for both the dipole (single dopant) and quadrupole (pair dopants) systems. In quadrupole systems, there are two charge transition levels (gold and orange lines) which correspond to the ionization energies of two electron polarons (as shown in Figure~\ref{fig:electronic}c).
Because quadrupoles yield consistently lower second CTLs, they possess very high second ionization energy ranging from 0.34 eV to 0.48 eV (the orange lines in Figure~\ref{fig:quad}a, nearly doubled the ionization energies of corresponding single-doped systems denoted by the blue lines). Therefore, the ionization of both electrons after quadrupole formation is nearly impossible. On the other hand, the first ionization energies of quadrupoles compared to the ones of dipoles shift in a manner correlated with the ionic radius as shown in Figure~\ref{fig:quad}b.
In particular, dopants with a radius larger than the host site ($\rm Fe (3+)$) have a lower first ionization after quadrupole formation (orange bars in Figure~\ref{fig:quad}b) and vice versa.


To show how dopant clustering affects carrier concentration quantitatively, we compute the polaron concentrations with and without the effect of dopant clustering for three representative cases: Sn, Ge, and Ti in Figure~\ref{fig:quad}c.
Equilibrium polaron concentrations are computed following the defect formation energy and charge neutrality approach as detailed in the method section.\cite{lee2013thermodynamics,freysoldt2014first}.
We then introduce clustering in the theoretical synthesis process, by considering the probability at which two dopants form into a quadrupole following a thermal Boltzmann distribution (details in the SI section on probability of quadrupole formation).
As dopant-dopant pairs form during cooling process from synthesis temperature to room temperature, an intermediate temperature of $T=500$ K is chosen as an approximate temperature where pairs start to form. We show the trends among different dopants remain the same for different temperatures (see SI Figure S13).
%% Defect and polaron concentrations of Sn-doped, Ge-doped, and Ti-doped \ce{Fe2O3} are shown in \textcolor{red}{Figure~\ref{fig:dc}b-d} as representative examples (other simulated doped systems in group IV and XIV are shown in SI Figure S11). Although in undoped \ce{Fe2O3} free electron polarons are mostly generated by $\rm Fe_{i}^{+}$, in doped \ce{Fe2O3} they are mostly generated  from tetravalent group IV and XIV dopants (more than six orders of magnitude higher than the ones generated by $\rm Fe_{i}^{+}$).%%
In Figure~\ref{fig:quad}c, we show that
polaron concentration is reduced due to clustering (solid lines) compared to the case without clustering(dashed lines), and the magnitude of this reduction is closely related to their binding energy ($\be$) in Table \ref{table:ems}. Specifically, dopants such as Ge and Ti with large binding energies show significant trapping of polarons due to clustering in the second and third panels of Figure~\ref{fig:quad}c. %(the large difference between solid and dashed lines in the second and third panels of Figure~\ref{fig:quad}c).
% TS -- I added a paragraph break here

In order to avoid the adverse effects of clustering, Ti will be better suited to be doped at very low concentrations into hematite.
% On the other hand, Ti will be better suited to be doped at very low concentrations into hematite in order to avoid the adverse effects of clustering.
This explains the experimentally observed very low optimal doping concentration of Ti in hematite (about 0.1\%).\cite{fu2014highly,malviya2017influence} Furthermore, co-doping Ti with another dopant less prone to clustering may offer higher performance in hematite, which explains the success of recent co-doping strategies.\cite{zhang2010improved,mirbagheri2014visible,pan2015ti}
On the other hand, the small quadrupole binding energy of Sn makes it easier to dope Sn into hematite to higher concentrations without experiencing an immediate bottleneck.
This explains why the optimal doping of Sn into hematite (3\%) is an order of magnitude larger than Ti.\cite{yang2013new,tian2020electronic,bindu2018electrical}
If neglecting the effect of clustering, Ti would be a better dopant than both Sn and Ge (see dashed lines in Figure 4c, which shows that Ti has predicted to contribute the highest carrier concentration without clustering). Therefore, to unlock the potential of Ti, it is necessary to mitigate its strong tendency of clustering.
A strategy of co-doping Sn and Ti in hematite may offer the most effective strategy for maximizing performance of hematite photoanodes.
Alternatively, co-doping of divalent dopants, such as Mg, has been shown to relieve lattice distortion and can also offer further improvement to PEC performance~\cite{cai2020engineered}.
Ultimately, Ge is seen as the best dopant in group IV and XIV with the highest polaron concentration, which outperforms Sn and Ti in Figure~\ref{fig:quad}c.
%due to its ideal solubility and ionization energy, along with a lower tendency for clustering (compared to Ti).
%not the best for any of the property here

Finally, we remark that here we have focused on formations of dopant clustering at an early stage, which dominate at relatively low concentrations of doping in hematite. We also tested higher-order multipole clustering such as `hexapole' formation in hematite (e.g.\ three Sn dopants with the three introduced $EP$, details in SI section on higher-order multipoles and SI Figure S14). We found that hexapoles also have a negative binding energy, and thus
it is entirely possible that dopant clustering may grow even larger than the second-order multipoles we considered here.
However, there will also be larger and more complicated strain effects and configurational entropy which can compensate binding energies of larger sized aggregations.
Eventually, at even larger doping concentrations, the precipitation of different phases may occur, for example at 6\% Sn-doping in hematite XRD shows \ce{SnO2} secondary phases form \cite{tian2020electronic}.
% In addition,  in experiment, carrier conductivity is often measured instead of carrier concentration alone, where the carrier mobility (specifically polaron mobility here) will also play a significant role. We have studied doping effects on polaron mobility in transition metal oxides in our past work\cite{wu2018combining,zhang2018unconventional,smart2017effect,smart2018mechanistic}, and we will investigate the effects of dopant clustering on polaron mobility in our future study.
Future theoretical work should provide further insights to PEC experiments by investigating the role of dopants and dopant clustering on polaron mobility via small polaron hopping~\cite{wu2018combining,zhang2018unconventional,smart2017effect,smart2018mechanistic}
and optical absorption\cite{smart2019optical,wheeler2019combined,zhou2020interstitial}, which together, along with carrier concentrations studied here, directly impact the photoconductivity of transition metal oxide based photoelectrodes.
% Our future theoretical work will provide further insights to PEC experiments by investigating the role of dopants and dopant clustering on polaron mobility\cite{wu2018combining,zhang2018unconventional,smart2017effect,smart2018mechanistic} and optical absorption\cite{smart2019optical,wheeler2019combined,zhou2020interstitial}, which together, along with carrier concentrations studied here, directly impact the photoconductivity of transition metal oxide based photoelectrodes.


\subsection{Conclusions}
This work discovers a new mechanism of dopant clustering via the aggregation of n-type dopants and electron polarons into dopant-pairs % \textcolor{red}{bound dopant-polaron pairs}
which resemble electric multipoles.
% We find doping in \ce{Fe2O3} is limited by the formation of bound dopant-polaron pairs which resemble electric multipoles and trap electron polarons.
These pairs are thermodynamically stable due to several contributions % and they severely slow down the carrier concentration increase with doping concentration.
which we disentangle by a simple model involving three components: electronic, magnetic, and strain. Our model illuminates that binding occurs predominantly through electrostatic interactions but surprisingly is also mediated by magnetic interactions which together overcome strain to yield the consistently negative binding energies of tetravalent dopants in \ce{Fe2O3}. EXAFS experiments confirm the existence of these Sn-Sn pairs which formed at 1\% doping and have an identical interatomic distance compared to those predicted theoretically  ($\sim3.7$ {\AA}).

The effect of doping with and without clustering on carrier (small electron polaron) concentration is carefully examined.
We find doping in \ce{Fe2O3} is limited by dopant clustering which traps electron polarons and severely lowers the carrier concentration with respect to doping concentration. This clustering is shown to be responsible for the doping bottleneck in hematite, where dopants such as Ti exhibit extremely low optimal doping concentration (i.e.\ 0.1\%) for PEC application.
% We exclude the possibility of intrinsic compensating defects such as iron vacancy as the cause of doping clustering through our first-principles prediction of intrinsic and extrinsic dopant concentrations.
%and extrinsic dopants based on charge neutrality condition at thermodynamic equilibrium.
% Meanwhile, the binding energy of dopants into quadrupoles was shown to be responsible for the low optimal doping in hematite, where dopants such as Ti with a large binding energy exhibit extremely low optimal doping for PEC application.
Strategies to overcome this doping bottleneck are proposed; specifically codoping with dopants that exhibit low binding energies for clustering (e.g.\ Sn-Ti codoping) is seen as an ultimate solution.
Lastly, for single-type doping in group IV and XIV, we found Ge is the best dopant which can contribute the highest polaron concentrations even at presence of dopant clustering.
%Lastly, we demonstrated that dopants of slightly smaller ionic radii than the host site possess high solubility as well as low ionization energy making them better choices than other single dopants for optimal carrier concentrations.}
These findings provide a cohesive picture of the doping bottleneck in hematite and help to establish an improved rationale for further development of hematite photoanodes usage in renewable energy applications.


\section{Spin Polaron Conduction in CuO}

In our 2018 work published in npj Computational Materials~\cite{smart2018mechanistic}, we investigated hole transport in cupric oxide (CuO), a p-type semiconductor. Due to its relatively small bandgap (1.2-1.8 eV) and a conduction band minimum located at a more negative potential than that of water reduction,
\cite{jang2015tree,koffyberg1982photoelectrochemical,chiang2012copper,izaki2011electrodeposition,sagu2014rapid,masudy2016nanocrystal,lee2016scalable,guo2014cuo,emin2014novel,septina2017stabilized}
it has the potential to serve as an inexpensive and environmentally benign photocathode for a water splitting PEC.
However, like other TMO's, CuO suffers from poor carrier conductivity, which limits the effectiveness of CuO-based devices.\cite{guo2014cuo,emin2014novel,wong2016current,masudy2016nanocrystal}
Additionally, cathodic photocorrosion of CuO can also limit the use of CuO for photoelectrochemical applications. Fortunately, a recent study demonstrated that the photocorrosion of CuO can be effectively suppressed by depositing a thin protection layer that prevents direct contact of CuO and the electrolyte,\cite{septina2017stabilized} which encourages studies on further improving charge transport and photoelectrochemical properties of CuO. Facilitated charge transport in CuO can also be advantageous for the use of CuO in other electrochemical devices such as gas sensors.\cite{poloju2018improved,rydosz2013nano}

The development of charge transport in CuO depends on the understanding and optimization of the small polaron hopping process. Strong electron-phonon coupling in many transition metal oxides (Fe2O3, BiVO4, TiO2) leads to the localization of carriers into polarons, a quasi-particle representing the carrier and local lattice distortion.\cite{rettie2016unravelling,rettie2013combined,carneiro2017excitation}
Due to this localization, carriers are no longer transported through the system via typical band mechanisms. Rather, carriers must be thermally activated in order to “hop” between sites, a process known as polaron hopping conduction.\cite{devreese1996polarons,mott1968conduction}
This type of conduction leads to an extremely low carrier mobility (e.g. 0.1 $\rm cm^2/V/s$ for CuO)~\cite{rettie2016unravelling} several orders of magnitude lower than band-like semiconductors such as Si (~1000 $\rm cm^2/V/s$). Previous experimental studies have indicated that polaron formation also occurs in CuO.~\cite{samokhvalov1993low,wu2014charge,zheng2001evidence,jeong1996nonstoichiometry,zheng2004fast}
Interestingly, more exotic properties such as ``one-dimensional charge stripes'' and ``spin polarons'' have been found in CuO due to strong spin-charge-lattice interactions,\cite{zheng2001evidence} which distinguishes its conduction mechanism from the common electron polaron hopping conduction in non-magnetic oxides such as BiVO4.  However, there has yet to be a theoretical investigation on the existence and transport of polarons in CuO, which would provide deeper understanding of carrier transport and therefore offer effective doping strategies to improve the carrier transport properties in CuO and other magnetic oxides in general.
Finally, although there have been a few experimental doping studies of CuO to date,\cite{masudy2015titanium,zheng2004fast,choi2017p,gao2007ferromagnetic,masudy2015optical,zheng2003effect,chiang2016dopant,chiang2011li,chand2014structural} the role of dopants in improving hole conduction in CuO has not been clearly understood.

In this study, we address these fundamental questions by comparatively investigating hole conduction in pristine and Li-doped CuO. Our focus is on the elucidation of the mechanisms by which Li doping improves hole concentration and mobility through a combined theoretical and experimental effort. Our work is organized as follows, first we provide theoretical background on CuO and discuss the mechanism of hole conduction which involves a unique spin-flip hopping process of spin polarons. Second, we show how Li doping enhances hole concentrations and hole mobility in CuO. Finally, we confirm our theoretical results by preparing CuO and Li-doped CuO electrodes and experimentally compare their photoelectrochemical properties.

% Polaron formation & hole conduction in CuO from first-principles
\subsection{Polaron formation and hole conduction in CuO from first-principles }
Several experimental studies have shown that CuO has an Arrhenius dependence of conductivity to temperature.~\cite{samokhvalov1993low,wu2014charge,zheng2001evidence,jeong1996nonstoichiometry,zheng2004fast}
This dependence is expected for materials which form small polarons (a trapped electron or hole due to local lattice distortion) which must be thermally activated in order to hop between lattice sites in the material, a process known as polaron hopping.~\cite{mott1968conduction}
The mobility of small polaron hopping follows the relationship shown in Eq.~\ref{cuo:eq:mobility},
\begin{align}
    \mu \propto e^{-E_a / kT}
    \label{cuo:eq:mobility}
\end{align}
where $E_a$ is the activation energy, $k$ is the Boltzmann constant, $T$ is the temperature, and $\mu$ is the carrier mobility which is related to the conductivity $\sigma$, by $\sigma = en\mu$ ($n$ is majority carrier concentration, which are holes in this case). To confirm the presence of polarons in CuO, we computed the electronic structure of pristine CuO with a single electron removed from a 96 atom system ($2\times 3\times 2$ supercell), corresponding to a hole concentration of 2\% (at.\ \% of hole = 100 $\times$ [mol of hole] / [mol of O]). Detailed descriptions of our electronic structure calculations can be found below under the Computational Methods section and SI. From the density of states and wave function of the hole state, we determine that holes form localized polaron states. These polaron states are predominantly O $2p$ mixed with Cu $3d$ as seen from the partial density of states (Figure S2) in agreement with previous studies on the electronic structure of CuO.~\cite{ghijsen1988electronic}

\begin{figure}
    \centering
    \includegraphics[keepaspectratio=true,width=0.9\linewidth]{2-tmo/figures-cuo/figure1_spin_polaron_formation_in_cuo}
    \caption{
        \textbf{Spin Polaron Formation in CuO.}
        \textbf{a}. Pristine Cu and O chain with antiferromagnetic (AFM) ordering.
        \textbf{b}. If a hole forms (without a spin flip) it will be highly localized as it can only distribute on one Cu atom (other neighboring Cu atoms do not have an available state of the appropriate spin as explained in the main text).
        \textbf{c}. After a Cu’s moment flips, the hole can redistribute over several Cu atoms lowering the kinetic energy.
        \textbf{d}. The wavefunction of a hole in CuO which has formed a SP with a flipped Cu spin so that it may redistribute over several atoms, in accordance with panel C. An isosurface of 10\% of the maximum is used. (Blue ball = Cu with up spin, Grey ball = Cu with down spin, and Red ball = O). (In the panels a-c large arrows denote the unpaired spin of Cu, small arrows denote two spin states of O which are often paired, dashed arrows denote states with a shared hole, blue/grey arrows= up/down, and green arrow= flipped Cu spin.)
    }
    \label{cuo:fig:formation}
\end{figure}

An intriguing consequence of hole localization around a single Cu $3d^9$ ion is that the Cu magnetic moment will flip, forming a ``spin polaron'' (SP).\cite{de1962interactions,vigren1973mobility,mott1993polaron,chernyshev2003models,lu1990spin,yonemitsu1992sensitivity}
Such states are common in copper oxides as the combination of two SPs will create a spinless Cooper pair state which obeys Bose-Einstein statistics and is the basis of superconducting.\cite{mott1993polaron,alexandrov1994bipolarons} In general, a SP forms in polaronic materials where the kinetic energy of the state can be lowered substantially from the increased delocalization of the electron or hole wavefunction after the spin-flip occurs.~\cite{wood1992d} For example, after an electron at a spin-up state is removed, a hole is created at the same spin state. As a fermion, the hole obeys the Pauli exclusion principle like electrons and can only be added to a state which is already occupied (i.e. a state must be occupied by an electron of the same spin for a hole to form). Therefore, in an anti-ferromagnetic system, the delocalization of a spin-up hole is limited by the availability of neighboring atoms’ spin-up occupied states. As shown in Figure~\ref{cuo:fig:formation}B, with anti-ferromagnetic ordering, the hole polaron may form into a highly localized state (limited to forming over a single Cu atom and its bonding O atoms that have up spins, as neighboring Cu atoms do not have an available spin-up state at which the hole can form). But after a neighboring Cu ion's moment flips (Figure~\ref{cuo:fig:formation}C) an extra channel is created, and the spin-up hole can redistribute over several sites that all have an up spin, lowering the kinetic energy of the hole polaron. The resulting flipped Cu ion with a distributed polaron state over several Cu and O atoms is shown in Figure~\ref{cuo:fig:formation}D. As discussed in Ref.~\cite{wood1992d}, this lowering of kinetic energy through wavefunction delocalization dominates over the energy cost of the spin flip and facilitates the formation of  spin polarons in CuO. Additional explanations can be found in the SI (Figures S3-S4).

Considering that the magnetic couplings between Cu ions in CuO are significantly large ($J\sim 100$ meV), spin-spin interactions will have an important effect on the conduction of holes in CuO. To address this point, we consider the total kinetic rate $\kappa$ of the hopping process in Eq.~\ref{cuo:eq:kinetic},
\begin{align}
    \kappa = \left( \sum_i e^{-E_i/kT} \kappa_i \right) / \left( \sum_i e^{-E_i/kT} \right)
    \label{cuo:eq:kinetic}
\end{align}
where $E_i$ is the energy of the $i^{\rm th}$ configuration and $\kappa_i$ is the hopping rate between configurations (for full details see SI). For the case of CuO, we found that the formation of a polaron at a site without a flipped spin was not possible, and, presumably, the total energy of such a state is very high, which reduces the possible configurations entering Eq.~\ref{cuo:eq:kinetic}. Then we define hopping that does not involve a spin-flip process to have a rate given by $\kappa_i=e^{-E_a^{e-ph}/kT}$, with $E_a^{e-ph}$ being the usual hopping activation energy barrier due to electron-phonon interactions. We have shown that the Boltzmann factors that are related to the energies of different spin configurations will be dominated by the most probable hopping path (Figure S9). As a result, the full hopping rate $\kappa$ then reduces to Eq.~\ref{cuo:eq:eph_spin},
\begin{align}
    \kappa \sim e^{-(E_a^{e-ph}+E_a^{spin})/kT}
    \label{cuo:eq:eph_spin}
\end{align}
Namely as holes are conducted through the system they will invoke a spin-flip process which will cost energy equal to the cost of flipping a spin of a Cu ion (illustrated in Figure~\ref{cuo:fig:hopprist}). In final, we see that the energy of this spin-flip process $E_a^{spin}$ can be simply added to the electron-phonon process $E_a^{e-ph}$ to give the full activation energy $E_a$, given in Eq.~\ref{cuo:eq:eph_spin2}.
\begin{align}
    E_a = E_a^{e-ph}+E_a^{spin}
    \label{cuo:eq:eph_spin2}
\end{align}
Intuitively, a spin-flip hopping process will not have a well-defined transition state; if there was a transition state, it would be a spin delocalized state which is not favored to form in a polaronic oxide. To confirm this point, we employed the newly-developed constrained density functional theory (CDFT) technique for solids in which an external potential is added to the Kohn-Sham potentials, and its strength is varied self-consistently in order to localize a desired number of charges on a specific site.~\cite{goldey2017charge,kaduk2012constrained} This allows for a direct calculation of the electronic coupling constant between initial and final states $|H_{ab}|$ in CuO, which we obtained to be 1.01 meV (the numerical accuracy is 0.01 meV). This is two orders of magnitude smaller than the computed activation energy (shown later), implying that transport in CuO is indeed non-adiabatic, which cannot be described by a semi-classical transition state theory.

\begin{figure}
    \centering
    \includegraphics[keepaspectratio=true,width=0.9\linewidth]{2-tmo/figures-cuo/figure2_spin_polaron_hopping_in_pristine_cuo}
    \caption{\textbf{Spin Polaron Hopping in Pristine CuO.} Diagram describing the interplay of spin and polaron hopping in CuO. As opposed to Figure~\ref{cuo:fig:formation}, only Cu spins are shown here for simplicity.
    \textbf{a}. Initially the spin polaron has formed at the initial site (IS), while the moment of Cu ions at the final site (FS) are aligned anti-ferromagnetically (AFM) ($-J⁄4$).
    \textbf{b}. After the polaron hops to the final site (FS) the center Cu moment is flipped, costing energy according to the strength of $J$. (Blue = Cu with up spin, Grey = Cu with down spin, Green = Cu with flipped spin, Dashed Light Blue box = polaron state).
    }
    \label{cuo:fig:hopprist}
\end{figure}

The energy of this spin-flip ($E_a^{spin}$) can be obtained directly using the Heisenberg Hamiltonian $H_{spin}=-\sum_{i<j}J_{ij}\hat{S}_i\dot \hat{S}_j$, where $J_{ij}$ is the magnetic coupling between the spin of the $i^{\rm th}$ and $j^{\rm th}$ Cu ion and $\hat{S}_i$ is the spin of the $i^{\rm th}$ Cu ion (taken to be $1/2$ as Cu is in a $3d^9$ configuration with one unpaired electron). The use of this model is well-established in accurate modeling of the magnetic couplings of CuO,~\cite{rocquefelte2010short,rocquefelte2012theoretical,rocquefelte2013room} and our calculations show that fitting the total energy of different magnetic configurations of CuO with this Hamiltonian yields an R-squared of 0.999 (Figure S7). Following a previous work,~\cite{rocquefelte2010short} we considered five magnetic couplings in CuO: $J_z$, $J_x$, $J_a$, $J_b$, $J_2$. Of these five couplings, the coupling $J_z$ is dominant over the rest with a value of $-111$ meV and is responsible for the long-range antiferromagnetic transition of CuO at 230 K. Note that in CuO the magnetic correlation length remains large at temperatures above the transition temperature $T_N$, so magnetic coupling is still relevant to our discussion of hole conduction in CuO at room temperature.~\cite{zheng2001evidence,yang1989magnetic}
Second is the super-superexchange $J_2$ which is $-39$ meV yet is still three times smaller than $J_z$. The remaining values are $-17.4$ meV for $J_a$, $+3.0$ meV for $J_x$, and $+2.6$ meV for $J_b$. From this we can directly compute $E_a^{spin}$ according to Eq.~\ref{cuo:eq:easpin}.
\begin{align}
    E_a^{spin} = -\sum_{i<j}J_{ij} \Delta \left(\hat{S}_i\dot \hat{S}_j\right)
    \label{cuo:eq:easpin}
\end{align}
In CuO, $-\sum_{i<j}J_{ij} \Delta \left(\hat{S}_i\dot \hat{S}_j\right) = -(J_z+J_x+J_a+2J_b+J_2)$, which gives $E_a^{spin}$ to be 160 meV, a similar magnitude to $E_a^{e-ph}$ (99 meV) as shown in Table~\ref{cuo:table:eph}. This result validates that $E_a^{spin}$ contributes significantly to the overall activation energy. Therefore, this result suggests that dopants that can reduce the magnetic coupling contribution $E_a^{spin}$ as well as the electron-phonon contribution $E_a^{e-ph}$ to the activation energy can more effectively improve hole mobility in CuO.
Note that here we consider hopping along the ferromagnetic (FM) [101] direction (see Figure~\ref{cuo:fig:hopprist}) due to shorter Cu-Cu distances and superior orbital overlap between initial and final states. Meanwhile, we find that hopping along the anti-ferromagnetic (AFM) [10$\bar{1}$] direction is energetically unlikely to occur (Figure S10-S12).


\begin{table}[H]
    \footnotesize
    \centering
    \begin{tabular}{ccccc}
    \hline \hline
        Li (\%) & $\varepsilon_\infty$ & $\varepsilon_0$ & $\varepsilon_p$ & $E_a^{e-ph}$ (meV) \\
    \hline
    0         & 6.4  & 11.0  & 15.5  & 99 \\
    6.25      & 7.9  & 13.0  & 16.7  & 92 \\
    12.5      & 8.4  & 16.3  & 17.5  & 88 \\
    \hline \hline
    \end{tabular}
    \caption{\textbf{Electron-Phonon Activation Energy.} Effect of Li doping on the electron-phonon activation energy from Eq.~\ref{cuo:eq:eaeph}.}
    \label{cuo:table:eph}
\end{table}

% Spin polaron conduction in Li-doped CuO from first-principles
\subsection{Spin polaron conduction in Li-doped CuO from first-principles}
Our experimental work (discussed later) shows Li-doped CuO electrodes have significantly increased photocurrent and show a positive shift of onset potential, while also retaining a similar crystallinity and photon absorption to the pristine CuO electrodes. Thus, it is anticipated that Li doping in CuO improves electron-hole separation and/or carrier conduction (concentration and/or mobility). To confirm this postulation, we applied our theoretical techniques discussed above to clarify how Li doping improves hole conduction in CuO.

The enhancement of carrier concentration after Li doping can be confirmed by the low hole ionization energy in Li-doped CuO, which is comparable to $kT$. Specifically, the ionization energy for a $p$-type dopant is defined by the difference between its charge transition level (CTL) and the valence band maximum. The CTL ($\varepsilon_{q|q'}$) is defined as the value of electron chemical potential at which the stable charge state of the defect changes from $q$ to $q'$, given by Eq.~\ref{cuo:eq:ctl}.
\begin{align}
    \varepsilon_{q|q'} = \left( E_q^f - E_{q'}^f \right) / (q' - q)
    \label{cuo:eq:ctl}
\end{align}
And the formation energy $E_q^f$ of the charge state $q$ is defined by Eq.~\ref{cuo:eq:cfe},
\begin{align}
    E_q^f [\varepsilon_F ]=E_q-E_{prist}+\sum_i \mu_i \Delta N_i +q\varepsilon_F
    \label{cuo:eq:cfe}
\end{align}
where $E_q$ is the total energy of the system with the charged defect, $E_prist$ is the total energy of the pristine system, and the third term on the right side accounts for the change in number of atoms of each species $i$ between these two configurations ($\Delta N_i$), with $\mu_i$ being the atomic chemical potential of that element in its stable form. From Eq.~\ref{cuo:eq:ctl}-\ref{cuo:eq:cfe} we computed the hole ionization energy of Li-doped CuO to be 55 meV (corresponding to the $-1|0$ transition level). Since this energy is small and comparable to $kT$ at room temperature, it indicates that Li introduces shallow hole states which can be ionized at room temperature to increase the hole concentration. This indicates a shift of the Fermi level towards the valence band maximum as has been experimentally shown with a positive shift of the onset potential by $\sim 210$ mV. The introduction of shallow states from Li doping is also in agreement with previous theoretical and experimental works.\cite{zheng2004fast,choi2017p}

To investigate the effects of Li doping on the transport of holes in CuO (\textit{i.e.}\ the effect of Li on hole hopping mobility) we first focused on the electron-phonon contribution to the activation energy, $E_a^{e-ph}$. For this part, we computed the electron-phonon activation energy of pristine and Li-doped CuO via Eq.~\ref{cuo:eq:eaeph}. This method represents an averaged doping effect in a continuum polarization medium and avoids the sampling of all possible doping configurations and hopping paths.~\cite{austin1969polarons}
\begin{align}
    E_q^{e-ph} = \frac{e^2}{4\varepsilon_p} \left( 1/r_p - 1/R \right)
    \label{cuo:eq:eaeph}
\end{align}
Here $r_p$ is the polaron radius which is approximated as $r_p=1/2 (\pi/6)^{1⁄3} V^{1⁄3}$, $R$ is the average hopping distance, and $1/\varepsilon_p =1/varepsilon_\infty -1/\varepsilon_0$ where $\varepsilon_\infty$ is the high frequency dielectric constant and $\varepsilon_0$ is the static dielectric constant. The results of this calculation (Table~\ref{cuo:table:eph}) show that Li doping increased the high frequency dielectric constant ($\varepsilon_\infty$) due to increased carrier concentrations after Li doping. Although the static dielectric constant ($\varepsilon_0$) is also increased due to weaker Li-O bonds ($\varepsilon_0$ is inversely proportional to the bonding energy squared)\cite{gonze1997dynamical}, an increased $\varepsilon_\infty$ dominated and resulted in an overall lower barrier ($E_a^{e-ph}$). For example, the barrier decreases by 11 meV after 12.5\% Li doping, which corresponds to 1.5 times improvement on hopping mobility based on the $\mu \propto e^{-E_a/kT}$ relation between $E_a$ and mobility $\mu$. Therefore, Li doping assists the electron-phonon kinetics of carriers in CuO.

\begin{figure}
    \centering
    \includegraphics[keepaspectratio=true,width=0.9\linewidth]{2-tmo/figures-cuo/figure3_spin_polaron_hopping_in_li-doped_cuo}
    \caption{
    \textbf{Spin Polaron Hopping in Li-Doped CuO.}
    Diagram describing the interplay of spin and polaron hopping in CuO after Li doping (Orange = Li). As the spin polaron hops through the lattice, its interaction with Li will result in a lower magnetic barrier $E_a^{spin}$ due to broken magnetic couplings between Cu ions and non-magnetic Li ions ($E_{spin}$=0).
    }
    \label{cuo:fig:hopli}
\end{figure}

To consider the effect of Li on the magnetic contribution to the activation energy $E_a^{spin}$, we first recalculated the magnetic couplings in CuO after a significant amount of Li doping (12.5\%) using the same methods as before (Figure S8, Table S3-S4). We find that the predominate magnetic coupling $J_z$ is nearly the same after Li doping, although overall Li suppresses the anti-ferromagnetism of CuO due to the spinless character of Li, which has also been seen experimentally.~\cite{zheng2004fast} The resulting energy of the spin-flip process in Li-doped CuO from Eq.~\ref{cuo:eq:easpin} (assuming that there are no Li near the polarons) would be 137 meV, which is smaller than 160 meV in pristine CuO (mentioned above). We note that the largest benefit of Li doping is seen when we consider the interaction of neighboring SPs and Li. An analogue of the spin-flip hopping process after Li doping in CuO is shown in Figure~\ref{cuo:fig:hopli}. Since Li is non-magnetic, it does not interact with a SP when it passes by, and a single Li site can reduce the local hopping barrier of the SP by up to 55 meV which corresponds to approximately 9 times improvement of hopping mobility based on the $\mu \propto e^{-E_a/kT}$ relation (the case of Li breaking $J_z$ coupling). This larger effect of Li doping on the activation energy describes how Li doping significantly enhances the hole mobility in CuO, in agreement with previous experimental measurements of the activation energy of Li-doped CuO.\cite{zheng2004fast,gao2007ferromagnetic,zheng2003effect,chiang2016dopant} For example in Ref.~\cite{zheng2004fast}, a monotonic decrease of activation energy has been observed as a function of Li doping concentration (up to 16\%), accompanied by strong suppression of the anti-ferromagnetism of CuO. The activation energy decreases from 0.23 eV for pristine CuO to 0.035 eV for $\rm Cu_{0.92}Li_{0.08}O$, which leads to a three order of magnitude decrease of resistivity in experiments.~\cite{zheng2004fast} Since what we have discussed is relevant for isolated Li doping (Li-Li interaction is neglected in our magnetic interaction models), even a small amount of Li doping will have a significant impact on the conduction of holes in CuO and can dramatically increase the photocurrent density of CuO as we have seen in our experimental investigation.~\cite{chiang2016dopant}


% Experimental comparison of photoelectrochemical properties of CuO and Li-doped CuO electrodes
\subsection{Experimental comparison of CuO and Li-doped CuO electrodes}
Since the direct measurement of charge transport properties of our high surface area, nanofibrous polycrystalline electrodes was not possible, the effect of Li doping on charge transport properties was evaluated by comparing photocurrent generation of CuO and Li-doped CuO electrodes.  Since these electrodes have the same absorbance (Figure S15B), the number of electron-hole pairs generated in these electrodes under illumination must be identical.  Then, if an interfacial charge transfer reaction that can quickly consume almost all the surface reaching electrons is chosen for photocurrent measurement, any change in photocurrent generation caused by Li doping must be due to a change in the number of surface-reaching electrons caused by a change in the charge transport properties, which affects electron-hole separation.

\begin{figure}
    \centering
    \includegraphics[keepaspectratio=true,width=0.4\linewidth]{2-tmo/figures-cuo/figure5_j-v_plots_of_cuo_and_li-doped_cuo}
    \caption{
    \textbf{J-V Plots of CuO and Li-Doped CuO.}
    J-V plots (scan rate = 10 mV/s) of CuO (black) and Li-doped CuO (red) electrodes for O2 reduction in 0.1 M KOH (pH 13) solution with O2 purging under AM1.5G,100 ${\rm mW/cm^2}$ illumination. The inset shows the enlarged current in the potential region near the photocurrent onset potentials indicated by arrows.
    }
    \label{cuo:fig:exp}
\end{figure}

For this purpose, comparing photocurrent for water reduction may not be proper because the surface of CuO is not catalytic for water reduction, and a significant portion of the surface reaching electrons can be lost to surface recombination, making it difficult to accurately evaluate the change in the number of surface-reaching electrons. In this study, we used oxygen reduction as the photoelectrochemical reduction reaction that occurs on the CuO surface as the kinetics of this reaction is typically much faster than water reduction on oxide-based photocathodes.~\cite{cardiel2017electrochemical,kang2016photoelectrochemical,read2012electrochemical,wheeler2017photoelectrochemical} The J-V plots of CuO and Li-doped CuO for oxygen reduction obtained in a 0.1 M KOH (pH 13) solution purged with \ce{O2} under standard illumination conditions (AM1.5G, 100 mW/cm2) are shown in Figure~\ref{cuo:fig:exp}.

The pristine CuO electrode already shows efficient photocurrent generation for ce{O2} reduction as its bandgap allows for the utilization of a great portion of the visible solar spectrum, and its nanostructure reduces bulk electron-hole recombination.  For example, it achieved a photocurrent density of $\sim 1.2\ {\rm mA/cm^2}$ at a potential as positive as 0.8 V vs. RHE, and it increased up to $\sim 4.0\ {\rm mA/cm^2}$ when the potential was swept to 0.6 V. (The dark current initiating around 0.65 V vs. RHE is due to electrochemical reduction of ce{O2} which was subtracted from the photocurrent to determine overall photocurrent.) The photocurrent observed for \ce{O2} reduction can be considered the upper limit of photocurrent that can be observed for water reduction when an efficient hydrogen evolution catalyst is placed on the CuO surface to improve the water reduction kinetics.

The Li-doped CuO electrode significantly enhanced photocurrent generation. For example, the Li-doped CuO electrode achieved a photocurrent density of $\sim 2.0\ {\rm mA/cm^2}$ at a potential as positive as 0.8 V vs. RHE, and it increased up to $\sim 5.6\ {\rm mA/cm^2}$ when the potential was swept to 0.6 V. In addition to the evident increase in magnitude of photocurrent density, Li-doped CuO electrodes demonstrated a considerable shift in photocurrent onset to the positive direction by $\sim 210$ mV. The photocurrent onset potential for a reaction that has high interfacial charge transfer kinetics, such as \ce{O2} reduction on an oxide photocathode, can be considered the flatband potential. This is because for such reactions the loss of the surface-reaching minority carriers to surface recombination is negligible.  In this case, it can be assumed that photocurrent disappears when the applied potential is the same as the flatband potential, where electron-hole separation is no longer possible.  The fact that the J-V plots of CuO and Li-doped CuO electrodes measured with chopped illumination do not show any transient photocurrent even when the applied potential is near the photocurrent onset potential is a good indication that recombination on the CuO surface during \ce{O2} reduction is negligible. This confirms that the photocurrent onset potentials of these electrodes can be regarded as their flatband potentials. Since the flatband potential is the same as the Fermi level after accounting for the Helmholtz layer potential drop at the semiconductor/electrolyte interface, and the Helmholtz layer potential drop should not be altered by 0.1 at. \% Li doping, the shift of the onset potential of Li-doped CuO directly indicates that Li doping shifted the Fermi level of CuO to the positive direction, closer to the valance band maximum.~\cite{nozik1978photoelectrochemistry}

These experimentally obtained results agree well with computational results that Li doping generates shallow acceptors that effectively increase the hole concentration. The increase in hole concentration, which improves the hole conductivity, can reduce electron-hole recombination in the bulk or in the space charge region, increasing the number of minority carriers reaching the surface to perform oxygen reduction. Also, the increase in hole density that changed the Fermi level was confirmed by the shift of the flat band potential to the positive direction. Finally, according to our computational results, a simultaneous decrease in $E_a$ by Li doping also contributed to photocurrent enhancement by improving the hole mobility of CuO.

While the impact of Li doping on the activation energy $E_a$ and the impact on carrier density cannot be easily separated in our photocurrent measurements, the impact of Li doping on the activation energy has been discussed explicitly in dark resistivity measurements of Li doped CuO between a few K to 300 K.~\cite{zheng2004fast,gao2007ferromagnetic} An order of magnitude decrease of the hopping activation energy with 16 at.\ \% Li doping clearly confirmed the combined effect of $E_a^{spin}$ and $E_a^{e-ph}$ being lowered by Li doping, as the effect of $E_a^{e-ph}$ alone cannot explain the observed order of magnitude decrease in the hopping activation energy based on our calculations.~\cite{zheng2004fast} This study clearly demonstrated that the effect of Li-doping on $E_a^{spin}$ is still considerable at 300 K (because of the short-range magnetic couplings remaining above N\'eel temperature)\cite{zheng2001evidence,yang1989magnetic} and that Li-doping can play a critical role in improving the mobility of CuO at room temperature, which is relevant for its PEC applications.

\subsection{Conclusions}
In conclusion, we have studied in-depth hole conduction in pristine and Li-doped CuO by first-principles calculations accompanied by the PEC performance of experimentally prepared CuO and Li-doped CuO electrodes. In pristine CuO, we have verified the existence of spin polarons (SP), which occur via the flip of a single Cu ion’s spin so that the polaron may redistribute over several atoms, and this delocalization effect lowers the energy of the polaron state. We then showed how transport of SPs in CuO will involve a spin-flip hopping process and developed a theoretical framework of computing the activation energy which involves both electron-phonon and magnetic coupling contributions. Next, we displayed how Li doping in CuO generates shallow states above the valence band which pushes the Fermi level closer to the valence band maximum and improves hole concentrations in CuO. Then, we showed how Li doping improves hole hopping mobility in CuO by lowering the electron-phonon coupling contribution to the activation energy due to higher electronic screening. More importantly, we demonstrated that Li doping lowers the magnetic coupling contribution to the activation energy due to the destruction of magnetic interactions through the replacement of Cu ions with non-magnetic Li ions, culminating in a significantly lowered hopping barrier and increased hole mobility in Li-doped CuO. Finally, we prepared CuO and Li-doped CuO electrodes and compared their photoelectrochemical properties for \ce{O2} reduction, where the changes in photocurrent and the onset of photocurrent can be directly related to changes in charge transport properties and the Fermi level, respectively.  The experimental results show that Li doping enhances charge transport properties and shifted the Fermi level toward the valence band maximum while not affecting photon absorption, which agrees well with computational results. This work provides important insights on the mechanisms of the formation and transport of SPs and their effect on the charge transport properties of CuO and Li-doped CuO.  Similar to Li doping, doping with other non-magnetic shallow acceptors may also simultaneously improve carrier concentration and hopping mobility of magnetic oxides. In this case, shallow dopants can be ionized easily to increase carrier concentrations and increase dielectric screening, which weakens the charge-lattice interactions. Most importantly, non-magnetic dopants can break the magnetic couplings and lower the hopping barrier for SPs significantly, which is critical for improvement of hopping mobility.  These insights offer effective strategies for the improvement of hopping conduction in magnetic oxides through atomic doping, which provides important guidance for materials design.

% COMPUTATIONAL METHODS
\subsection{Computational Methods}
Cupric oxide (CuO) assembles in a monoclinic structure with C2/c symmetry and a geometric unit cell consisting of only 8 atoms. To consider the correct magnetic interactions prevalent in CuO, a $\sqrt{2}\times 1\times \sqrt{2}$ unit cell containing 16 atoms needs to be implemented (Figure S1).\cite{rocquefelte2012theoretical,yang1989magnetic,forsyth1988magnetism} Meanwhile, a $2\times 3\times 2$ supercell of 96 atoms was used for calculations considering polaron formation and doping. A final supercell of $2\sqrt{2}\times 3\times 2\sqrt{2}$ with 192 atoms was used to confirm spin polaron formation size and charged cell correction.

It is well known that both local and semi-local exchange and correlation functionals in DFT cannot accurately describe the electron correlation in magnetic insulators, which results in a qualitatively incorrect electronic structure. To account for this issue, we applied the Hubbard $U$ correction\cite{dudarev1998electron} with $U$ = 7.5 eV, a well-established model for this material.~\cite{peng2012density,peng2014ab,heinemann2013band,debbichi2012vibrational,wu2006lsda} All calculations were carried out in the open-source plane wave code Quantum ESPRESSO64 with ultrasoft LDA pseudopotentials,\cite{gbrv} unless otherwise noted. The choice of LSDA+U instead of GGA+U was made because GGA+U was unable to give the correct monoclinic structure of CuO, while LSDA+U yielded a geometry of CuO within 5\% of experimental values. Nonetheless, LSDA+U and GGA+U provide similar electronic structures (with the experimental geometry) and overall gave results in agreement with experimental expectations.  Our calculations yielded that Cu$^{2+}$ ions have a magnetic moment of 0.57 $\mu_B$ with a magnetic ordering according to Figure S1. Notably, the O atoms in this system share a non-negligible magnetic moment of 0.14 $\mu_B$ (in agreement with previous experiments\cite{forsyth1988magnetism} and theory\cite{peng2012density}). We were also able to replicate the computed magnetic couplings at higher levels of theory with the LSDA+U method (see Table S1-S2).

\def\co{Co$_\text{3}$O$_\text{4}$}
\def\vo{V$_\text{O}$}
\def\vco{V$_\text{Co(O)}$}
\def\vct{V$_\text{Co(T)}$}

\section{Small Polarons Inducing Optical Transitinos in \ce{Co3O4}}

In our 2019 work published in Physical Review Materials as a Rapid Communicaion~\cite{smart2019optical}, we demonstrated the origin and misconceptions behind optical transitions in \ce{Co3O4} due to the presence of small hole polarons which have active optical transitions below the energy of the true band-band transition.
Polarons, conduction electrons or holes with self-induced lattice polarization, are known to exist in most transition metal oxides (TMO) and deeply affect their optical and carrier transport properties \cite{reticcioli2019small}. In these materials, much of the interest has been related to the role of small polarons (SPs) that form when the induced lattice polarization is localized in a volume on the order of the unit cell. In particular, for many important TMOs, including \ce{Fe2O3}~\cite{sivula2011solar,ling2011sn,smart2017effect}, \ce{NiO}~\cite{gong2014nanoscale,hu2014efficient}, \ce{Co3O4}~\cite{wang2018phosphorus,aijaz2016}, \ce{MnO}~\cite{jin2015partially}, \ce{BiVO4}~\cite{wu2018combining,zhang2018unconventional,seo2018role,kim2015simultaneous}, \ce{CuO}~\cite{cardiel2017electrochemical,smart2018mechanistic}, it has been found that the formation of SPs is responsible for the low carrier mobility and conductivity, which hinders their practical application as electrochemical catalysts and photoelectrochemical (PEC) electrodes \cite{lee2019progress,tachibana2012artificial,roger2017earth,yan2016review, liao2013new}. It is also well-established that the transport of SPs in these TMOs can be characterized through the thermally activated hopping conduction mechanism and a logarithmic temperature dependence of the materials carrier mobility \cite{mott1968conduction}.

Unlike the distinct signature of SPs on the carrier conduction discussed above, the effect of polarons on electronic structure and optical transitions in TMOs is rather complex and difficult to elucidate. For example, in several TMOs, such as WO$_3$~\cite{gerosa2018role,ping2013}, TiO$_2$~\cite{TiO22018,Moser2013,Kang2010} and SrTiO$_3$~\cite{verdi2017origin,Wang2016}, the presence of large polarons that are delocalized over several unit cells may lead to a strong band gap renormalization through electron-phonon coupling. By contrast, the formation of SPs may introduce isolated gap states away from the band edges due to their spatially localized nature, which could be easily misinterpreted as band edges that define the fundamental band gap. A prime example is \ce{Fe2O3}, where recent time-resolved spectroscopy experiments have shown that its mid-gap states are indeed associated with optically active polarons, which in turn lead to transition energies that are significantly lower than the fundamental band gap \cite{carneiro2017excitation,biswas2018ultrafast}. This conclusion is also consistent with Lohaus \textit{et al.} \cite{lohaus2018limitation}, where the authors showed that while the band gap of \ce{Fe2O3} is 2.2~eV, an effective gap of 1.75~eV is observed due to the formation of small polarons. Nonetheless, distinguishing mid-gap states due to SP formation from other sources such as defect-bound states \cite{huda2010electronic,neufeld2015platinum,sanson2015polaronic} and surface states \cite{yatom2015toward} is still not well understood in literature. Despite extensive experiments on these TMOs, theoretical studies of SP effects on electronic structure and optical properties of TMOs are limited. More importantly, challenges remain in first-principles methods that accurately describe polarons and electronic structure of TMOs.

\begin{figure}[t]
\begin{center}
\includegraphics[keepaspectratio=true,width=6cm]{2-tmo/figures-co3o4/diag_poly_label.png}
\caption{Normal spinel atomic structure of \co. Octahedral Co are shown in green, tetrahedral Co are shown in blue/light blue (distinguishing spin polarization direction), and O are shown in red.}  \label{co3o4:fig:struct}
\end{center}
\end{figure}

In this paper, we discuss the role of SPs (hereinafter referred to as ``polarons'' for simplicity) in tricobalt tetraoxide (\co), an anti-ferromagnetic oxide with a normal spinel structure. Despite that \co\ has been extensively investigated for a wide range of technologies \cite{wu2010,li2005,ma2014,wang2018,hamdani2010,aijaz2016,meher2011,xia2011}, a fundamental understanding of the optical properties of this material remains largely lacking, and conflicting results have been reported, e.g., for the band gap of bulk \co. For instance, a value of 1.5-1.7~eV has been commonly reported for the optical gap of \co~\cite{lohaus2016,jiang2014,waegele2014,shinde2006,belova1983,lima2014,cheng1998}. On the other hand, several experimental studies conclude that, despite a transition being observed around 1.5-1.7~eV, the true band gap of \co\ is significantly smaller, yielding a value of around 0.7-0.9~eV~\cite{qiao2013,singh2014,sousa2019,martens1985}. This conclusion, however, is not supported by time-resolved optical spectroscopy measurements, which suggest the state at $\sim$0.8~eV above the valence band maximum is a localized polaron state \cite{waegele2014,jiang2014,wheeler2012}. Along this direction, other experiments have indicated that the intrinsic carriers in \co\ are hole polarons that are characterized by a nearest-neighbor hopping conduction mechanism; and such a signature implies that hole polarons may affect the optical properties of \co\ in a similar way as in \ce{Fe2O3} \cite{ngamou2010,tronel2006,shinde2006,lohaus2016,sahoo2013,sparks2018,cheng1998,koumoto1981}. Collectively, the existing results indicate that much is left to be understood regarding the nature of the optical transitions near the band edge of \co, and how it is related to polaron formation.

The aim of this work is to resolve the conflicting results in the literature on \co, and provide a coherent description of its electronic structure, carrier conduction, and optical properties through first-principles calculations. In particular, we discuss the level of theory needed for a proper description of the electronic structure of the material. In addition, we elucidate the role of polaron formation on the electronic band gap and optical spectra of $p$-doped \co, and we discuss how uniaxial strain can be used to distinguish polaron related transitions in the optical spectra. This work provides a straightforward method for considering SP effects in the optical absorption, alongside unambiguous SP peak assignment in agreement with several previous experimental studies. This work will help to distinguish SP formation from other optical effects often considered instead (e.g. exciton formation, thermal broadening of optical spectra and electron-phonon renormalization of band edges) for TMOs in general. Our study presents a roadmap for first-principles calculations in the investigation of SP effects on optical absorption.

\subsection{Computational Methods}

\begin{figure}[t]
\begin{center}
\includegraphics[keepaspectratio=true,width=9cm]{2-tmo/figures-co3o4/koop.png}
\caption{\textbf{a.} Generalized Koopmans' condition for electron polaron (EP) and hole polaron (HP) in \co. The exact exchange $\alpha$ for the PBE0($\alpha$) method is varied until the condition $\text{HOMO}_q=\text{LUMO}_{q+1}$ (at fixed geometry where polaron has formed) is met. In both cases, we find that at an exact exchange of 0.12 Koopmans' condition is satisfied. The corresponding pristine gap is computed to be 1.70 eV. \textbf{b.} Localized and \textbf{c.} delocalized hole wavefunction, subject to the value of the exact exchange. Isosurface plots use a cutoff of 10\% the maximum.}
\label{co3o4:fig:IPEA}
\end{center}
\end{figure}

We begin by discussing our computational strategy for addressing the electronic properties of \co{} (spinel structure shown in Figure \ref{co3o4:fig:struct}). It is well known that density functional theory (DFT), with conventional local or semi-local exchange-correlation functionals, is not sufficient to provide a proper description of polarons in transition metal oxides and often severely underestimates the band gap of these materials \cite{dudarev1998electron}. In order to mitigate this issue, several approaches have been proposed, including DFT+$U$ with an orbital specific Hubbard $U$ correction, and hybrid functional that includes a fraction of Hartree-Fock exchange ($\alpha$), hereinafter denoted as PBE0($\alpha$). However, these calculations are known to highly depend on the choice of the Hartree-Fock exchange or Hubbard $U$ correction. For instance, a wide range between 0.8 and 2.0~eV has been reported in the literature for the band gap of \co, depending on the level of theory employed \cite{singh2014}. In this context, it is also necessary to emphasize that, despite significant development has been made, establishing a first-principles approach that allows for an accurate prediction of the electronic properties of TMOs remains a significant challenge \cite{kent2018toward}.

Here, we implement both hybrid functional and DFT+$U$ calculations to provide an unbiased description of the electronic properties of \co. Notably, in variation with previous hybrid functional calculations, we invoked the generalized Koopmans' condition to determine the value of $\alpha$ from first-principles. We stress that this strategy has been shown to successfully predict the band gap of materials with band gaps up to 14 eV and has been particularly successful for polaronic systems \cite{miceli2018nonempirical,smart2018fundamental,liu2018electron,lany2011predicting}. Specifically, the generalized Koopmans' condition enforces the condition $\text{IP}_q=\text{EA}_{q+1}$ at a fixed geometry for an isolated state in the materials, where $\text{IP}_q$ is the ionization potential of an occupied state at the charge state $q$, whereas $\text{EA}_{q+1}$ is the electron affinity of the same state when it is unoccupied at the charge state $q+1$.  Here, we determine the value of $\alpha$ by enforcing the condition $\text{IP}_q=\text{EA}_{q+1}$ for both the hole and electron polaron (see Figure~S1 for more details of the electron polaron), and we obtained a value of 0.12 for the Hartree-Fock exchange $\alpha$ in both cases, as shown in Figure \ref{co3o4:fig:IPEA}. We note that hole polarons do not form for $\alpha$ below 10\%, as illustrated in Figure \ref{co3o4:fig:IPEA}b-c). The value of $\alpha$ determined in this manner is an intrinsic property of the bulk system, and the choice of defect used to enforce the Koopmans’ condition is proper as long as it has minimum hybridization with the bulk Bloch states \cite{bischoff2019adjustable,miceli2018nonempirical,smart2018fundamental}.

We then determined $U$ parameters based on the hybrid functional results, and the stable formation of electron and hole polarons (see Table~S1 for more details). In particular, we find that the use of $U$ values of $U_\text{Co(O)}=4$ eV and $U_\text{Co(T)}=3$ eV provides consistent results with the Koopmans' compliant hybrid functional (band gap agrees within 0.1~eV), as well as a proper description of the electronic properties of the system, as discussed later in this communication. We also note that hole polarons do not form for $U$ values below 2.5 eV (see Figure~S2).

\begin{figure}[t]
\begin{center}
\includegraphics[keepaspectratio=true,width=8cm]{2-tmo/figures-co3o4/crystal.png}
\caption{\textbf{a.} Projected density of states (PDOS) of Co(O) on $t_{2g}$ and $e_g$ orbitals. Schematic representation of the electronic configuration of \textbf{b.} octahedral Co$^{3+}$ and \textbf{c.} tetrahedral Co$^{2+}$ in \co\ due to a crystal field splitting ($\Delta_{oct}$ and $\Delta_{tet}$, respectively). \textbf{d.} PDOS of Co(T) on $t_2$ and $e$ orbitals. All the PDOS was computed with DFT+$U$.}
\label{co3o4:fig:split}
\end{center}
\end{figure}

All the calculations were then carried out using the plane-wave code Quantum ESPRESSO \cite{QE1} with norm-conserving pseudopotentials \cite{ONCV1}. A plane  wave cutoff of 100~Ry was used in all PBE+$U$ calculations, while a reduced cutoff of 50~Ry was implemented for the more demanding hybrid functional calculations (geometry and electronic structure are converged at 50 Ry). The calculations were generally performed within the 56 atom cubic cell with a $2\times 2\times 2$ k-point mesh for integration over the Brillouin zone. In addition, a $\sqrt{2}\times\sqrt{2}\times 2$ supercell with 224 atoms was utilized for comparison, particularly to understand the finite-size effects on polaron formation. Nevertheless, we find that the wavefunction character, energy level splitting, and band structure look largely unchanged between the 56 atom cubic cell and the 224 atom supercell (see Figure~S3 for further details). In all calculations with electron or hole polarons, the charged cell correction scheme as developed in Ref.~\cite{PING2017JCP} was employed, which is necessary in order to remove the spurious interactions of the polarons with their periodic images and with the uniform compensating background charge \cite{kokott2018first}. For the rest of the manuscript, unless otherwise noted, the results presented here were obtained at the DFT+$U$ level of theory that is computationally less demanding compared to hybrid functional calculations.

%\section{Results and Discussion}

\subsection{Electronic Structure of Pristine \co{}}

To set a baseline for the discussion of polaron effects in \co, we briefly summarize the electronic structure of the pristine system. As shown in Figure \ref{co3o4:fig:struct}, \co\ assembles in a normal spinel structure, where two thirds of Co occupy octahedral sites (denoted by Co(O)) and the remaining third occupy tetrahedral sites (denoted by Co(T)). We find that Co(O) exhibits a low-spin $3d^6$ orbital configuration with the $O_h$ symmetry, leading to filled $t_{2g}$ and empty $e_g$ bands, as shown in the calculated projected density of states (PDOS) and band diagram presented in Figure~\ref{co3o4:fig:split} a-b. On the other hand, Co(T) with the $T_d$ symmetry forms a high-spin $3d^7$ configuration with a half-filled $e_g$ band, yielding an overall magnetic moment of $\sim$3.2 $\mu_B$ as already reported in experiments \cite{roth1964}. These Co(T) sites experience an anti-ferromagnetic interaction mediated through a super-exchange of mutually bonded oxygen; in addition, the presence of a large Hund's spin exchange results in a splitting of the $t_{2}$ band. This is shown in Figure \ref{co3o4:fig:split} c-d, where we find that the $t_{2}$ minority spin states are formed at a higher energy level, whereas all majority spin states occur at lower and similar energies \cite{lima2014,chen2011} (also see Figure~S4 for further details). Overall, the behavior of the spin states and band splitting presented here are consistent with results reported in existing theoretical studies \cite{chen2011,wu2016}.

\subsection{Formation of Hole Polarons with Mid-Gap States}

Next, we discuss the nature of polaron formation in \co. Our calculations show that, among the Co(O) and Co(T) sites where hole polarons can form, the total energy of a hole polaron forming at Co(O) is lower than that of Co(T) by at least 70~meV. A more stable formation of the polaron on Co(O) is also reflected in the calculated density of states of \co\, where a larger contribution of Co(O) $d$ states is found at the valence band edge compared to the Co(T) $d$ states (see Figure~S4). Finally, our conclusion is consistent with the experimental study reported by Ngamou \textit{et al.} \cite{ngamou2010}, where the authors show that the hopping of polarons takes place in the octahedral sites and are responsible for driving the electrical transport in the oxide. Collectively, these results indicate that the computational approach employed here provides a proper description of polaron formation in \co.

\begin{figure*}%[t]
\begin{center}
\includegraphics[keepaspectratio=true,width=0.98\linewidth]{2-tmo/figures-co3o4/hole_diag_224.png}
\caption{\textbf{a.} Pristine band structure of \co\ with a 224 atom supercell (primitive cell band structure shown in SI Figure~S5). \textbf{b.} Hole polarons create a low-spin (LS) $d^5$ configuration at Co(O) along with a Jahn-Teller (JT) distortion which results in a $D_{4h}$ configuration and the creation of several mid-gap states. \textbf{c-e.} Wavefunction isosurface plots (yellow cloud) of the three polaron induced states under hole formation of $a_1(\uparrow)(d_{x^2-y^2})$, $b_1(\uparrow)(d_{z^2})$, and $b_2(\downarrow)(d_{xy})$ character, respectively. Isosurface plots use a cutoff value of 10\% the maximum. \textbf{f.} Band structure of \co\ with a hole polaron which shows several induced gap states (blue = spin up, black = spin down).}  \label{co3o4:fig:hole}
\end{center}
\end{figure*}

Beyond the findings on the thermodynamical stability of hole polaron formation in \co, our calculations show that the hole polaron at Co(O) leads to several mid-gap states. As shown in Figure \ref{co3o4:fig:hole}, we find that upon the hole polaron formation, Jahn-Teller (JT) distortion occurs at the Co(O) site due to an uneven occupation of the $t_{2g}$ band, and splits the degeneracy of the $O_h$ states. In addition, the uneven occupation of up/down states splits the spin degeneracy due to the on-site Coulomb repulsion of the $d$ orbitals. Specifically, we find that the majority spin states (e.g. $b_2(\uparrow), a_1(\uparrow), b_1(\uparrow)$) are located at a lower energy, whereas the minority spin states (e.g. $b_2(\downarrow), a_1(\downarrow), b_1(\downarrow)$) are pushed higher in the energy. Such splitting and ordering is consistent with the previous time-resolved spectroscopy measurements of Co(O) \cite{wu2016,belova1983}.
In addition, these mid-gap states are consistent with the experiment reported in Ref.~\cite{jiang2014}, where it was found that mid-gap excitations are associated with $a_1(\uparrow)$ and $b_1(\uparrow)$ states of Co(O). As a result, our analyses point to significant effects of hole polarons on the electronic structure of \co, most notably in the formation of mid-gap states in a similar way as found in \ce{Fe2O3}.

\subsection{Small Polaron Induced Optical Transitions}

We now turn to a more quantitative discussion of polaron effects on the electronic structure of \co. In particular, we obtain a band gap of 1.6~eV and 1.7~eV with the current choice of $U$ and $\alpha$ (respectively), which is in excellent agreement with the value of 1.5-1.7~eV reported in Refs.~\cite{lohaus2016,jiang2014,waegele2014,shinde2006,belova1983,lima2014,cheng1998}. However, this is significantly larger than the result of 0.7-0.9~eV claimed by other experiments. \cite{qiao2013,singh2014,sousa2019,martens1985}
We note that exciton binding energies are usually less than 150 meV in many TMOs \cite{ping20132electronic,le2014,ping2013}, and we show later that the calculated optical spectra, by including excitonic effects, cannot explain the low energy transition at 0.7-0.9~eV.
Interestingly, as shown in Figure~\ref{co3o4:fig:hole}b, at the current level of theory, we find that the mid-gap states are located at 0.8~eV away from the valence band maximum, and are associated with the hole polaron formation. This observation suggests that a scenario similar to the one observed for \ce{Fe2O3} may also occur in \co, i.e., the true gap of \co\ is $\sim$1.6~eV, and the mid-gap states are responsible for the transitions found at $\sim$0.8 eV in the experimental optical absorption spectra \cite{qiao2013,singh2014,sousa2019,martens1985}.
In order to verify our hypothesis and to further elucidate the role of polaron formation, we calculated the optical absorption spectra of \co\ with and without a hole polaron, and we compared the results with available experimental data. Absorption spectrum with a hole polaron was computed with a 56-atom supercell \cite{concentrationNote}, which better represents experimental hole concentrations of $p$-type \co\ \cite{waegele2014,lohaus2016} due to abundant cation vacancies \cite{tronel2006,godillot2013}.

Therefore, we computed the imaginary part of the dielectric function in the random phase approximation (RPA) with local field effects (as shown in Figure 5) and solving the Bethe-Salpeter equation that includes excitonic effects (as shown in SI Figures S8 and S9), as implemented in the YAMBO-code \cite{YAMBO}, using the single particle eigenvalues and wavefunctions derived from DFT+$U$. We note that this choice of starting point considers the balance between accuracy and computational cost, similar to this work in Ref.~\cite{Claudia2012}. In addition, for a direct comparison with experimental measurements, we calculated the absorption coefficient from the dielectric function \cite{jackson1999}.
%
\begin{align}
    A(\omega)=\frac{\omega}{c}\frac{\epsilon_2(\omega)}{\sqrt{\frac{\epsilon_1(\omega)+\sqrt{\epsilon_1(\omega)^2+\epsilon_2(\omega)^2}}{2}}}
\end{align}
%
\begin{figure}[t]
\begin{center}
\includegraphics[keepaspectratio=true,scale=1]{2-tmo/figures-co3o4/abs.png}
\caption{Optical absorption of \co\ in the pristine system (black) and the $p$-doped system (blue). Notably, only $p$-type doping i.e. the formation of holes, will cause mid-gap transitions below 1.6 eV, in agreement with experimental optical spectrum of \co\ shown in green \cite{qiao2013}. The inset image displays the mid-gap transitions which are labeled according to the states formed from hole polaron formation as in Figure \ref{co3o4:fig:hole}.
(Theoretical spectrum is an average of spectra with light polarized in the [100], [010], and [001] directions.)}  \label{co3o4:fig:abs}
\end{center}
\end{figure}

The calculated optical absorption spectra of \co\ are shown in Figure~\ref{co3o4:fig:abs}, together with the experimental spectrum. We find that the introduction of hole polarons leads to the formation of several lower energy optical transition peaks between 0.6 and 1.2 eV in \co. More importantly, we find that, in sharp contrast to the result obtained for the pristine system, the spectrum computed for \co\ with a hole polaron is in very good agreement with experimental data, where three lower lying transitions were also found between 0.7 and 1.1~eV \cite{qiao2013,lohaus2016,jiang2014,waegele2014}. Our analysis indicates that these transitions can be associated with those occur between the $p$-$d$ hybridized dispersive valence states and the localized $d$ states formed at the hole polaron site, for which the wavefunctions are illustrated in Figure \ref{co3o4:fig:hole} c-e. Collectively, these results clearly support the interpretation that the true optical gap of \co\ is $\sim$1.6 eV and that the optical transitions observed at $\sim$0.8 eV are due to hole polaron formation at Co(O) sites.
We note that electron polarons do not lead to the formation of low lying transitions (see Figure~S6), indicative of the $p$-doped nature of the experimental system.

In order to rule out the possibility that the low energy transitions $\sim$0.8 eV are caused by large excitonic effects~\cite{Laskowski2009,Bruneval2006,Wiktor2018,Rodl2012}, we also computed absorption spectra of pristine \co\ including excitonic effects by solving the Bethe-Salpeter Equation, as shown in SI Figure~S7, S8. More detailed discussions can be found in SI. Overall, no extra peaks in the BSE spectra show up at the energy range below 1 eV for pristine \co\ (no hole polaron included), which confirms the excitonic effects do not explain the low energy transitions in the absence of SPs.

In addition, we note that we neglect electron-phonon coupling and thermal expansion effects on the band edge positions and absorption spectra at finite temperature, as discussed in Refs.~\cite{monserrat2018phonon,bravic2019finite}. For example, the absorption edge may be subject to a red-shift, in addition to an overall broadening of the spectra. This offers additional directions for future works. In any case, these effects will not lead to an additional peak well separated from the main absorption in a direct band-gap semiconductor like \co; therefore, our conclusion on the small polaron contribution to the low energy optical transitions still holds.

\subsection{Detecting Hole Polaron Transitions via Strain}

Finally, we propose an experimentally viable method for distinguishing optical transitions involving localized polaron states from traditional band-band bulk state transitions. For \co, the JT distortion upon the introduction of the hole polaron extends the Co-O bonds along the $C_4$ axis, and this distortion may occur along any of the bond axis that aligns with the [100], [010], [001] directions of the cubic unit cell. Such a three-fold degeneracy can be broken if uniaxial strain is applied to the system along one of the crystal lattice directions, which in turn may affect the optical absorption spectrum. In this regard, monitoring the change in the optical spectrum of \co\ in the presence of an uniaxial strain could potentially provide signatures of hole polarons associated with a specific JT distortion.

For demonstration, we considered a 1\% tensile strain applied along the [100] direction. We find that the three-fold degeneracy in the polaron states is broken upon the introduction of the strain; in particular, polaron formation with the JT elongation along the [100] direction is  lowered  in  the  energy  by  5  meV  compared  to  those  associated  with  the  [010]  or  [001] directions.   Here,  to  investigate  the  collective  and  individual  effects  of  these  polarons  on the absorption spectrum, we computed a thermally averaged ensemble spectrum by using a Boltzmann probability distribution of the optical absorption obtained for each case:
\begin{align}
    P_i = \frac{e^{E_i/kT}}{\sum_i e^{E_i/kT}} \quad , \quad A(\omega)=\sum_{i}P_i\,A_i(\omega)\label{co3o4:eq:bolt},
\end{align}
where $E_{i}$ and $A_{i}$ are the energy and absorption spectrum of the system containing a polaron in the state $i$ ($i$ denotes different JT elongation direction), respectively.

\begin{figure}%[t]
\begin{center}
\includegraphics[keepaspectratio=true,width=7cm]{2-tmo/figures-co3o4/strain.png}
\caption{Optical absorption of $p$-doped \co\ under 1\% uniaxial tensile strain along the [100] direction. Temperature dependence determines the probability for which direction the Jahn-Teller elongation will occur (as the degeneracy is removed under strain) and results in a red-shift of optical peaks related to the hole polaron.}  \label{co3o4:fig:strain}
\end{center}
\end{figure}

The calculated optical absorption spectrum presented in Figure \ref{co3o4:fig:strain} clearly shows a red-shift in the first peak that is associated with polarons. In particular, we find that at lower temperatures where $kT$ is on the order of 5 meV or less, the resulting optical spectra follow that of the lowest energy polaron associated with JT elongation along the [100] direction. At higher temperatures, the clear red-shift remains, although a high temperature of 300~K is sufficient to quench the 5~meV energy difference between polaron states. In contrast to the transition associated with hole polarons, we find that bulk band-band transitions at higher energy (above 1.5 eV) remain mostly unchanged upon uniaxial strain (see Figure~S9). Accordingly, this allows one to clearly distinguish the local polaron state involved in optical transitions, whose JT distortion renders them quite sensitive to strain, from that of the band-band bulk state transitions which are insensitive to strain.

\subsection{Conclusions}

To summarize, we present a detailed investigation of the electronic structure and polaronic induced optical transitions in \co\ based on first-principles calculations. We resolved several contradicting findings in the literature related to the character of the charge carrier and band gap of the material. In particular, we show that the optical gap of pristine \co\ is 1.6 eV, whereas the lower lying transition around $\sim$0.8~eV is associated with the hole polaron, which was misinterpreted as the band edge of the material. We also demonstrated the important effects of uniaxial strain on the optical spectra of \co, which in turn can be used to reveal the localized character of polaron-induced electronic states.

Our study also suggests a strategy for establishing a potential first-principles approach that can simultaneously achieve an accurate description of polaron states, electronic band structure and optical properties in polaronic magnetic oxides. Specifically, the generalized Koopmans' condition can be utilized to derive the fraction of exact exchange from first-principles, which in turn can be used in hybrid functional for investigating the electronic structure of the oxide. These hybrid functionals can also be used for benchmarking DFT-$U$ calculations, which offer a much lower computational cost, or to provide inputs for higher level electronic structure methods, such as many-body perturbation theory within the \textit{GW} approximation.


\section{Combining Theory and Experiment}

\subsection{\ce{BiFeO3}}

Bismuth iron oxide, \ce{BiFeO3}, is a semiconductor with a rhombohedrally distorted perovskite structure that yields a large ferroelectric effect.~\cite{choi2009switchable} For this reason, it has been investigated as one of the most promising candidates for ferroelectric diode devices and ferroelectric photovoltaics.~\cite{choi2009switchable,yi2011mechanism,you2018enhancing,spanier2016power} Recently, \ce{BiFeO3} was also reported as a photoelectrode in a solar water-splitting cell.~\cite{shen2017dual,yilmaz2016perovskite,moniz2015visible,song2018domain,liu2016enhanced} In these reports, \ce{BiFeO3} was demonstrated to have a relatively narrow bandgap of $sim$2.2 eV and conduction band minimum (CBM) and valence band maximum (VBM) positions that straddle the water reduction and oxidation potentials,~\cite{shen2017dual,yilmaz2016perovskite} all of which are very attractive features for a photoelectrode in a water-splitting photoelectrochemical cell (PEC). Considering that \ce{Fe2O3}, an extensively studied photoanode with a similar bandgap, has a CBM that is $sim$200 mV more positive than the water reduction potential,~\cite{sivula2016semiconducting} the shifts in the CBM and VBM to the negative direction constitute an important advantage of \ce{BiFeO3} over \ce{Fe2O3}.~\cite{lee2019progress}

To date, both $n$-type and $p$-type \ce{BiFeO3} photoelectrodes have been reported,~\cite{shen2017dual,yilmaz2016perovskite,moniz2015visible,song2018domain,liu2016enhanced} meaning that \ce{BiFeO3} can serve as a photoanode or a photocathode, respectively. In these studies, the doping type varied without the introduction of external dopants, suggesting that the defects responsible for $n$-type and $p$-type \ce{BiFeO3} can both readily form.~\cite{lee2019progress} The defects that cause $n$-type behavior include oxygen vacancies,~\cite{paudel2012intrinsic} and the defects that cause $p$-type behavior include Bi vacancies.~\cite{rojac2017domain}

Despite the many interesting and advantageous features shown for \ce{BiFeO3} as a photoelectrode, there have been very few systematic investigations on the photoelectrochemical properties of \ce{BiFeO3}. Because of its popularity as a ferroelectric material, most studies on \ce{BiFeO3} photoelectrodes have focused on how the application of an external electric field on \ce{BiFeO3} affects its photocurrent generation or on the conversion between $n$-type and $p$-type photocurrent.~\cite{song2018domain,liu2016enhanced,cao2014switchable,huang2016tunable} From a careful analysis of these papers, it appears that the \ce{BiFeO3} electrodes used in these studies were very lightly doped.~\cite{lee2019progress} For the purpose of accurately evaluating the potential of \ce{BiFeO3} as a photoanode or a photocathode, optimally doped $n$-type and $p$-type \ce{BiFeO3} electrodes need to be prepared and examined individually.

Considering that charge transport in many oxide-based photoanodes involves small polaron hopping,~\cite{wheeler2019combined,smart2019optical,smart2018mechanistic,smart2017effect,seo2018role,kim2015simultaneous} understanding the formation and transport of small polarons in \ce{BiFeO3} is also critical. Unfortunately, while numerous theoretical studies on \ce{BiFeO3} have been published to date, they have focused on its bulk polarization,~\cite{kan2011chemical,neaton2005first} photovoltaic effects,~\cite{young2012first,liu2013development} and multiferroic effects.~\cite{wang2012atomistic,chang2016prediction} Small electron polaron formation and its effects on dopant ionization energies and concentration of free carriers in \ce{BiFeO3} have not yet been investigated theoretically.

In this work, we conducted combined experimental and theoretical studies on $n$-type \ce{BiFeO3} photoanodes. For the experimental investigation, we prepared highly uniform \ce{BiFeO3} photoanodes by electrodeposition and examined their photoelectrochemical properties and stability for use in a PEC. We then intentionally introduced oxygen vacancies into the pristine \ce{BiFeO3} lattice and examined the effect on carrier concentration and photocurrent generation. The experimental results were compared with those of a computational study, which examined the formation of small polarons in \ce{BiFeO3} and the effect of oxygen vacancies on small polaron formation and free carrier generation in \ce{BiFeO3} for the first time. The new experimental and computational results discussed in this study will significantly increase our fundamental understanding of \ce{BiFeO3} for use as a photoanode material.

\begin{figure}
    \centering
    \includegraphics[keepaspectratio=true,width=0.9\linewidth]{2-tmo/figures-bifeo3/exp}
    \caption{
    (\textbf{a}) J-V plots and (\textbf{b}) J-t plots at 0.8 V vs. RHE for pristine \ce{BiFeO3} (red) and \ce{N2}-treated \ce{BiFeO3} (blue) for sulfite oxidation. All measurements were obtained in pH 9.2 borate buffer containing 0.7 M sulfite under 1 sun illumination (100 mW/cm$^2$, AM 1.5 G).
    }
    \label{bifeo3:fig:exp}
\end{figure}

Density functional theory calculations were performed using the Quantum ESPRESSO package~\cite{QE1} with PBE+U exchange correlation functional, ultrasoft pseudopotentials,~\cite{gbrv} and Hubbard $U$ parameters of 2 eV on O $2p$ and 3 eV on Fe $3d$. All calculations were done using the hexagonal \ce{BiFeO3} cell (space group: R3c), which we expanded to a $2\times 2\times 1$ supercell to avoid spurious interactions during defect calculations. A $2\times 2\times 2$ $k$-point grid was used to calculate the charge density, and a $4\times 4\times 4$ $k$-point grid was used for density of states. We applied a newly developed charge correction scheme~\cite{PING2017JCP} to calculations containing excess charge as implemented in JDFTx.~\cite{JDFTx} For the calculations used to investigate the effect of oxygen vacancies, a single oxygen atom was removed from a 120-atom supercell. Because 72 oxygen atoms are present in this supercell, this is equivalent to removing 1.39 atomic \% oxygen (1.39 oxygen atoms out of 100 oxygen atoms), leading to the empirical formula of BiFeO$_{2.96}$.

Theoretical optical absorption spectra of \ce{BiFeO3} with and without $\rm V_O$ were obtained by computing the imaginary part of the dielectric function in the random phase approximation with local field effects as implemented in the YAMBO code.~\cite{YAMBO} The input of this calculation came directly from our single particle eigenvalues and wavefunctions from DFT+$U$ computed in Quantum ESPRESSO. The absorption spectrum ($\alpha$) is related to the real and imaginary parts of dielectric function ($\epsilon_1$ and $\epsilon_2$, respectively) as shown in the equation below.~\cite{jackson1999}
\begin{align}
    \alpha(\omega) = \frac{\omega}{c} \frac{\epsilon_2(\omega)}{\sqrt{\frac{\epsilon_1(\omega) +\sqrt{\epsilon_1(\omega)^2+\epsilon_2(\omega)^2}}{2}}}
    \label{bifeo3:eq:alpha}
\end{align}

To gain additional insight into the photoelectrochemical properties of \ce{BiFeO3}, we conducted density functional theory (DFT) calculations to investigate the small polaron formation and its effect on defect ionization energy and free carrier concentration in \ce{BiFeO3}. First-principles calculations were carried out on \ce{BiFeO3} using the DFT+$U$ method (see Computational Methods for more information). The computed bandgap of \ce{BiFeO3} was 2.2 eV, which is in great agreement with the experimentally measured value of the bandgap.

Due to strong electron-phonon interactions, carriers in many transition metal oxides are trapped by their self-induced lattice distortions, forming small polarons.~\cite{geneste2019polarons} Small polarons conduct through the system via a thermally-activated hopping mechanism unlike carriers in covalent semiconductors, which conduct through conventional band mechanisms.~\cite{lu2010room,gheorghiu2013preparation} Therefore, understanding and facilitating small polaron hopping are critical for the development of oxide-based photoelectrodes.~\cite{wheeler2019combined,smart2019optical,smart2018mechanistic,seo2018role,wu2018combining,zhang2018unconventional}

% Small Polaron Formation.

\begin{figure}
    \centering
    \includegraphics[keepaspectratio=true,width=0.9\linewidth]{2-tmo/figures-bifeo3/prist}
    \caption{
        (\textbf{a}) Norm-squared wavefunction of the electron polaron (yellow cloud) shown as an isosurface in the \ce{BiFeO3} lattice (purple = Bi, gold = Fe, red = O). Isosurface value is 1\% of the maximum amplitude of the wavefunction; (\textbf{b}) projected density of states (PDOS) for \ce{BiFeO3} before (left) and after (right) a single electron-polaron is introduced in a 120-atom supercell.
    }
    \label{bifeo3:fig:prist}
\end{figure}

The formation of an electron polaron in pristine \ce{BiFeO3} was simulated by adding one extra electron into the pristine \ce{BiFeO3} system and allowing the system to relax. We observed that the extra electron spontaneously localizes on a single Fe site, forming a small electron polaron as shown in Figure~\ref{bifeo3:fig:prist}a. This creates a deep, localized state that lies 1 eV below the CBM of the pristine \ce{BiFeO3} as shown in Figure~\ref{bifeo3:fig:prist}b. The formation of similar localized electron polaron states has been observed in other Fe$^{3+}$-based oxides such as \ce{Fe2O3}.~\cite{smart2017effect}

The Fe$^{3+}$ ions in \ce{BiFeO3} have $O_h$ crystal field splitting with $3d^5$ high-spin electron configuration.~\cite{baettig2005first} Thus, the extra electron occupies an Fe $t_{2g}$ state, and the small polaron state has mainly $t_{2g}$ character. Furthermore, the presence of a small polaron on the Fe ion, which lowers its valency from +3 to +2, perturbs valence states and creates additional localized states above the VBM of the pristine \ce{BiFeO3} (Figure~\ref{bifeo3:fig:prist}b). These localized states have character of $e_g$ orbitals of Fe$^{2+}$ and $2p$ orbitals of oxygen. (The corresponding wavefunctions are shown in Figure S6.)


% Electronic structure of $\rm V_O$ in \ce{BiFeO3}.
Previous theoretical studies on defect formation in \ce{BiFeO3} reported that the oxygen vacancy is a very deep donor with an ionization energy greater than 1 eV,~\cite{paudel2012intrinsic,kay2006new,zhang2010density,shimada2016multiferroic} meaning that the oxygen vacancy cannot contribute to the generation of $n$-type carriers at room temperature. This disagrees with our experimental observation that the \ce{N2}-treated \ce{BiFeO3} that contains more oxygen vacancies has a higher carrier density and generates significantly more photocurrent at room temperature. We note that previous theoretical studies did not consider the formation of small polarons in \ce{BiFeO3} and their effects on defect ionization energies and carrier concentration. Employing recently developed methods,~\cite{smart2017effect,seo2018role} we revisited the formation of oxygen vacancies in \ce{BiFeO3} to investigate their ionization energies with respect to the free polaron level in pristine \ce{BiFeO3}.

When an oxygen vacancy ($\rm V_O$) is formed in the lattice of \ce{BiFeO3}, it introduces two electrons that spontaneously generate two small electron polaron states. These two states correspond to the two peaks shown $sim$0.8 eV below the CBM in the PDOS of \ce{BiFeO3} in Figure~\ref{bifeo3:fig:vo}a. As in the case of introducing a free electron-polaron, these polaron states have mainly $t_{2g}$ character of Fe$^{2+}$. Due to attractive electrostatic interactions between the electron polarons and the $\rm V_O$ site, the most thermodynamically stable configuration is the one with the two electron-polarons located at the Fe sites nearest to the $\rm V_O$, as shown in Figure~\ref{bifeo3:fig:vo}b. The difference in energy of the $\rm V_O$ polarons seen in the PDOS is a result of their differing distances from the $\rm V_O$ (Figure S7). The introduction of $\rm V_O$ and the resulting small electron polarons also generate perturbed valence states above the VBM. These perturbed states are mainly composed of the $e_g$ orbitals (\textit{i.e.}\ $d_{x^2-y^2}$) of Fe$^{2+}$ and $2p$ orbitals of oxygen (Figure S8). When absorption spectra of \ce{BiFeO3} with and without $\rm V_O$ were simulated and compared (Figure S9), we found that the presence of the perturbed states above the VBM did not affect the absorption of \ce{BiFeO3}. This agrees with our experimental results.


\begin{figure}
    \centering
    \includegraphics[keepaspectratio=true,width=0.9\linewidth]{2-tmo/figures-bifeo3/vo}
    \caption{
    (\textbf{a}) Projected density of states (PDOS) for \ce{BiFeO3} with a single oxygen vacancy ($\rm V_O$) introduced into a 120-atom supercell; (\textbf{b}) Norm-squared wavefunction of the two electron-polarons (yellow regions surrounding Fe). Isosurface value is 1\% of the maximum amplitude of the wavefunction. The $\rm V_O$ is indicated by a single empty red circle between the two electron-polarons; (\textbf{c}) Charge formation energy (FE) diagram of $\rm V_O$ in \ce{BiFeO3}.
    }
    \label{bifeo3:fig:vo}
\end{figure}



% $\rm V_O$ as a shallow donor in \ce{BiFeO3}.
In order to consider the effects of oxygen vacancies on the carrier concentration in \ce{BiFeO3}, it is necessary to compute the formation energy of the defect in each of its charge states q.
\begin{align}
    E^f_q = [\varepsilon_F] = E_q - E_pst + \sum_i \mu_i \Delta N_i + q\varepsilon_F + \Delta_q
\end{align}
Here, $E^f_q$ is the formation energy of a defect with charge $q$, $E_q$ is the total energy of the defect system with charge $q$, $E_{pst}$ is the total energy of the pristine system, $\Delta N_i$ is the change in the number of atomic species $i$ with chemical potential $\mu_i$, $\varepsilon_F$ is the Fermi energy, and $\Delta_q$ is the defect charge correction consistent with recent developments~\cite{PING2017JCP,JDFTx} that serves to remove spurious interactions of the charged defect with its periodic images and with the uniform compensating background charge. The value of the Fermi level ($\varepsilon_F$) in which the system undergoes a transition of charge state $q$ to $q'$ defines the charge transition level $\varepsilon^{q|q'}$.
\begin{align}
    \varepsilon^{q|q'} = \frac{E^f_q-E^f_{q'}}{q'-q}
\end{align}
Typically, the charge transition level of an electron donor from one charge state to a more positive charge state referenced to the CBM defines the ionization energy of the defect. However, in polaronic oxides, the feasibility of polaron hopping is determined not by the ionization energy of the defect with respect to the CBM, but by the ionization energy of the defect with respect to a free polaron state where the polaron is not bound to a defect.~\cite{smart2017effect,seo2018role} Therefore, the true ionization energy of small polarons is equal to the energy difference between the charge transition levels of the defects (solid red dots in Figure~\ref{bifeo3:fig:vo}c) and the free polaron level (grey dashed line in Figure~\ref{bifeo3:fig:vo}c). The free polaron energy level can be obtained from the formation energy of the pristine system with ($q=-1$) and without ($q=0$) an extra electron. The Fermi level corresponding to the $\varepsilon^{0|-1}$ transition in the pristine system defines the free polaron level.~\cite{smart2017effect,seo2018role} The energy difference between the free polaron level and the CBM is the polaron binding energy.

Under this model, we computed the charge formation energy diagram of the $\rm V_O$ and its corresponding ionization energies. In an oxygen-rich environment, the formation energy of a neutral $\rm V_O$ is 3.4 eV, which agrees with previous calculations.~\cite{paudel2012intrinsic,kay2006new,zhang2010density,shimada2016multiferroic} We find that the $\rm V_O$ has two distinct charge transition levels corresponding to the (0/+1) and (+1/+2) transitions. The energies of these charge transition levels relative to the VBM (1.57 eV and 1.22 eV, respectively) are in excellent agreement with recent DFT+$U$ calculations.~\cite{geneste2019polarons} On the other hand, the positions of charge transition levels relative to the CBM of \ce{BiFeO3} vary drastically depending on the choice of $U$,~\cite{paudel2012intrinsic,geneste2019polarons,zhang2010density} which is a known effect. However, their positions with respect to the free polaron level are relatively insensitive to the choice of $U$, which is similar to what was reported for the case of Sn-doped \ce{Fe2O3}.~\cite{smart2017effect} Comparing the charge transition levels of $\rm V_O$ to the free polaron level, we found energy differences of 99 and 438 meV for the first and second charge transition levels, respectively. The energy of the first charge transition level (0/+1) relative to the free polaron level is comparable to $kT$ at room temperature (26 meV) and indicates that a fraction of the oxygen vacancies in \ce{BiFeO3} can ionize at room temperature and contribute to an increased carrier concentration. A simple thermodynamic calculation (assuming a Boltzmann-like distribution) suggests that $sim$2.05\% of oxygen vacancies will be ionized to their +1 state at room temperature at thermal equilibrium. This result differs from previous reports that $\rm V_O$ in \ce{BiFeO3} is a deep donor and cannot increase the carrier concentration.~\cite{paudel2012intrinsic,kay2006new,zhang2010density,shimada2016multiferroic}

Our theoretical result has clarified the role of oxygen vacancies in enhancing carrier concentrations in \ce{BiFeO3}, and it is consistent with our experimental findings. This study emphasizes that in polaronic oxides, the defect ionization energies need to be considered with respect to the free polaron level and not to the CBM to more accurately understand the role of defects in the charge transport properties.


% Conclusions

To summarize, we performed combined experimental and theoretical investigations on $n$-type \ce{BiFeO3} to evaluate its properties relevant to its use as a photoanode in a photoelectrochemical cell. In our experimental study, we developed a synthesis method to produce high-quality, uniform $n$-type \ce{BiFeO3} photoanodes and examined their photoelectrochemical properties. A bandgap energy of $sim$2.1 eV was determined for the \ce{BiFeO3} photoanodes, and this value agreed well with the films’ orange color. The \ce{BiFeO3} photoanode showed a photocurrent onset potential of 0.3 V vs. RHE for sulfite oxidation, which is equivalent to its flatband potential. This value is significantly more negative than that of other ternary Fe-based oxide photoanodes. Upon annealing under a \ce{N2} environment to intentionally introduce more oxygen vacancies, the flatband potential was slightly shifted to the negative direction, and the photocurrent increased considerably. These results indicate that oxygen vacancies can contribute to an increase in carrier density, thus improving the charge transport properties of \ce{BiFeO3}. While the photocurrent reported in this study is one of the highest among those reported for \ce{BiFeO3} photoanodes, the observed value was still far below that expected for a photoanode having a bandgap of 2.1 eV, suggesting that bulk recombination is a major limitation of \ce{BiFeO3}. Considering that nanostructuring other Fe$^{3+}$-containing photoanodes such as \ce{Fe2O3} that suffer from short hole-diffusion lengths can significantly increase electron-hole separation, nanostructuring \ce{BiFeO3} is a logical next step to take to improve its photocurrent generation.

In our theoretical study, we showed for the first time that an extra electron in \ce{BiFeO3} spontaneously localizes on an Fe$^{3+}$ ion and forms a small polaron. The formation of the small polaron also perturbs valence states and creates additional localized states above the VBM of pristine \ce{BiFeO3}. When an oxygen vacancy is introduced into the \ce{BiFeO3} lattice, it forms two electron-polarons at the two Fe sites nearest to the $\rm V_O$ site. By accurately referencing the charge transition level to the free electron polaron level instead of to the CBM in our charge formation energy calculations, we showed that the first ionization energy of the $\rm V_O$ is 99 meV, meaning that the $\rm V_O$ is capable of serving as a donor to enhance the carrier concentration of \ce{BiFeO3}.
Overall, \ce{BiFeO3} has many attractive properties for use as a photoanode in a water-splitting PEC, and we expect that strategies such as nanostructuring and substitutional doping that can introduce shallow donors can continue to increase the photocurrent generation. Our combined investigation contributes to a fundamental understanding of the photoelectrochemical properties of \ce{BiFeO3} that can aid future systematic investigations of both $n$-type and $p$-type \ce{BiFeO3} photoelectrodes.




\subsection{\ce{LaFeO3}}

\textbf{Computational Methods} Density functional theory calculations were performed using the Quantum ESPRESSO package~\cite{QE1} with PBE+U exchange and correlation functional, with a Hubbard U parameter on O $2p$ of 2 eV and Fe $3d$ of 3 eV and norm-conserving pseudopotential~\cite{ONCV1} with a plane wave energy cutoff of 80 Ry. A total energy convergence of at least 10$^{-7}$ Ry was used for all self-consistent calculations while atomic forces were relaxed until a threshold of 10$^{-3}$ Ry/au was met. Final calculations implemented a $2\times 2\times 2$ supercell consisting of 160 atoms with a k-point grid of $2\times 2\times 1$ with the cell parameters attained via a variable cell relaxation. To investigate the effect of K doping, 1 out of 32 La atoms was replaced with K atom, which is equivalent to 3 at.\ \% K doping at the La site. This level of doping was chosen for computational studies because experimental studies showed that 3 at \% K doping at the La site is optimal for enhancing the photoelectrochemical properties of \ce{LaFeO3}.


\begin{figure}
    \centering
    \includegraphics[keepaspectratio=true,width=0.9\linewidth]{2-tmo/figures-lafeo3/lafeo3-1}
    \caption{
        (\textbf{a}) Total (black) and projected (red for O $2p$ and blue for Fe $3d$) density of states of pristine \ce{LaFeO3}. The reference zero is chosen to be at the valence band maximum. (\textbf{b}) Band structure of pristine (black) and K-doped \ce{LaFeO3} (red = spin up and orange = spin down). (\textbf{c}) Total (black) and projected (red for O $2p$ and blue for Fe $3d$) density of states of K-doped \ce{LaFeO3}. The inset image shows magnified hole-polaron states.
    }
    \label{lafeo3:fig:prist}
\end{figure}


\textbf{Effect of K Doping on Electronic Structure and Optical Absorption.}	\ce{LaFeO3} is a charge transfer insulator with valence states mainly composed of O $2p$ and conduction states mainly composed of Fe $3d$ orbitals (Figure~\ref{lafeo3:fig:prist}a). The electronic band structure of pristine \ce{LaFeO3} shows a direct $\Gamma-\Gamma$ bandgap of 2.13 eV. We used Hubbard U parameters of 2 eV for O $2p$ and 3 eV for Fe $3d$, which enables the formation of both electron- and hole-polarons and results in a bandgap value of pristine \ce{LaFeO3} close to the reported experimental value.~\cite{wheeler2017photoelectrochemical} We confirmed that the major effects of K doping on the band structure of \ce{LaFeO3} discussed here are not affected by the choice of U parameters as long as they are chosen to allow for the formation of both electron- and hole-polarons.~\cite{smart2018mechanistic,seo2018role,smart2017effect,wu2018combining} For example, the major results obtained with Hubbard U parameters of 4 eV for O $2p$ and 5 eV for Fe $3d$ (Figure S1) are the same as those shown in Figure~\ref{lafeo3:fig:prist} although the exact energy levels of the band structures may vary.

After doping the system with K, we observed the formation of an isolated hole-polaron state, which is a hybridized state consisting of O $2p$ and Fe $3d_{x^2-y^2}$ located at 0.3 eV above the VBM (Figure~\ref{lafeo3:fig:prist} b-c). This suggests that the two holes generated by the replacement of La$^{3+}$ with K$^+$ are localized on Fe atoms and their neighboring O atoms,~\cite{zhu2017electronic} forming Fe$^{4+}$. This result agrees well with experimental observations that the electrical conduction and dipolar relaxation in \ce{LaFeO3} are dominated by polaronic hole hopping between Fe$^{4+}$ and Fe$^{3+}$.~\cite{idrees2011origin,mizusaki1983electronic} Figure~\ref{lafeo3:fig:doped}a shows a structural unit of \ce{LaFeO3} where eight Fe atoms numbered through 1 and 8 are located at the corners of a cube with a K atom at the center. The most energetically preferred configurations of the two hole-polarons formed by K doping are those with the holes occupying Fe at the off-diagonal corners. Due to crystal symmetry, only two unique Fe-Fe diagonal positions are present as shown in Figure~\ref{lafeo3:fig:doped}b-c (Table S1). Between these two configurations, the one shown in Figure~\ref{lafeo3:fig:doped}b is more stable by 113 meV. Considering that our calculations were performed at 0 K and only included electronic energy without an entropy contribution, it is possible that a non-negligible concentration of hole-polarons formed in the sample at room temperature may exist in the configuration shown in Figure~\ref{lafeo3:fig:doped}c.


\begin{figure}
    \centering
    \includegraphics[keepaspectratio=true,width=0.9\linewidth]{2-tmo/figures-lafeo3/lafeo3-2}
    \caption{
        (\textbf{a}) A cube composed of eight Fe atoms at the corners numbered from 1 through 8 with a K atom at the center (K = purple, Fe = brown, O = red). (b-c) Hole-polaron wavefunction modulus with an isosurface at 10\% of the maximum value for the spin up (yellow) and spin down (turquoise); (\textbf{b}) hole-polarons formed at the (1, 2) Fe pair, which are symmetrically identical to those formed at the (5, 6) Fe pair; (c) hole-polarons formed at the (3, 4) Fe pair, which are symmetrically identical to those formed at the (7, 8) Fe pair.
    }
    \label{lafeo3:fig:doped}
\end{figure}


Our calculations also showed that K doping decreases the bandgap by lowering the conduction band minimum (CBM) from 2.13 eV to 2.06 eV and additionally forms multiple highly localized states below the CBM (Figure~\ref{lafeo3:fig:prist}b-c). We note that these states are formed not due to the presence of K but due to the free holes from K doping that form small polarons, which perturbed the system. (Even when free holes are introduced without K, the same states form.) To understand how these changes in electronic structure affect the optical properties of the system, we calculated the absorption spectra of pristine and K-doped \ce{LaFeO3} (Figure~\ref{lafeo3:fig:optical}). Because of the presence of optical anisotropy in the system, we averaged optical responses from light that was polarized along the a, b, c crystal axes. The results showed that K doping causes a negligible change in the high energy range ($\geq$ 2.4 eV) in terms of the position and intensity of the absorption peaks. However, K doping clearly enhances absorption in the low energy range ($<$ 2.4 eV) due to the lowering of the CBM and the introduction of perturbed states below the CBM. Since the solar spectrum contains a considerable number of photons near 2 eV, even a small increase in absorbance in this region may significantly increase the number of photons that can be utilized by \ce{LaFeO3} for photocurrent generation. We note that we did not include excitonic effects in this calculation because in most transition metal oxides, the exciton binding energy is small (~0.1 eV) and its change by doping is negligible.~\cite{kim2015simultaneous,ping2013,ping2014optimizing}


\begin{figure}
    \centering
    \includegraphics[keepaspectratio=true,width=0.9\linewidth]{2-tmo/figures-lafeo3/lafeo3-3}
    \caption{
        Calculated absorption spectra for pristine (black) and K-doped (red) \ce{LaFeO3}.
    }
    \label{lafeo3:fig:optical}
\end{figure}

\textbf{K as a Shallow Acceptor in \ce{LaFeO3}.} Although K doping introduces two holes into the system, the experimental carrier concentration will only increase when the dopant site does not trap the holes within the system. A dopant is generally considered to be a trap if the defect ionization energy or the energy that is required to free a defect-bound polaron is much higher than $kT$. We computed the defect ionization energies and formation energies from first principles.
In K-doped \ce{LaFeO3}, the $q = -1$ system (where one of the two holes from K substitution of La has been removed) is unstable leading to a direct $q = 0$ to $q = -2$ transition. Furthermore, the value of this charge transition level ($\varepsilon^{0|-2}$) is only 30 meV above the VBM. Since the energy required for this transition is comparable to $kT$ at room temperature (25 meV), K can serve as a shallow acceptor. We also note that the ionization energy obtained from defect formation energies is not significantly different from the energy difference between the VBM and the position of the hole state in the PDOS (Figure~\ref{lafeo3:fig:prist}c). This result indicates that the DFT+U level has largely corrected the delocalization errors that are known to occur at the semi-local DFT level. Additionally, it implies that the geometry relaxation at different charge states has minimum contribution.~\cite{smart2018fundamental,wu2017first} From these results we concluded that K doping can effectively increase the hole concentration in \ce{LaFeO3}.

\textbf{Effects of K Doping on Atomic Structure.} \ce{LaFeO3} has a perovskite structure in which the La atoms are located at the center of a cube composed of eight Fe atoms at the corners that are octahedrally coordinated with O. However, the atoms in \ce{LaFeO3} are slightly displaced from those of an ideal cubic perovskite structure, and \ce{LaFeO3} has an orthorhombic Pnma structure with a G-type antiferromagnetic ordering on the Fe atoms.~\cite{beausoleil2014thermal} Theoretically predicted cell parameters for pristine and K-doped \ce{LaFeO3} are shown in Table~\ref{lafeo3:table}. For K-doped \ce{LaFeO3}, the results for two configurations of hole-polaron formation shown in Figure~\ref{lafeo3:fig:doped}b-c are shown as Configuration 1 and Configuration 2, respectively.

\begin{table}[H]
    \footnotesize
    \centering
    \begin{tabular}{ccccc}
    \hline \hline
        System & $a$ (\AA{}) & $b$ (\AA{}) & $c$ (\AA{}) & Volume ($\rm \AA^3$) \\
    \hline
    Pristine    & 5.652 & 7.903 & 5.568 & 248.69 \\
    K-doped C1  & 5.644 & 7.887 & 5.569 & 247.91 \\
    K-doped C2  & 5.649 & 7.886 & 5.566 & 247.95 \\
    \hline \hline
    \end{tabular}
    \caption{Theoretically predicted cell parameters and band gap values for pristine \ce{LaFeO3} and K-doped \ce{LaFeO3}.}
    \label{lafeo3:table}
\end{table}

Despite the fact that K$^+$ (164 pm) has a larger ionic radius than La$^{3+}$ (136 pm),~\cite{shannon1976revised} the result shows that K doping (one K atom in the 160 atom cell) shrinks the unit cell volume by $\sim$0.3\% for both configurations. This is because a fraction of Fe$^{3+}$ (64.5 pm) is substituted by Fe$^{4+}$ (58.5 pm) owing to the hole-polaron formation. To confirm this, we considered four different systems: (1) pristine sample with a neutral charge; (2) pristine sample plus two holes without K doping, giving a $+2$ charge; (3) K-doped minus two holes, giving a $-2$ charge; and (4) K-doped sample with corresponding Fe$^{4+}$ formation with a neutral charge (Table S2). System 2 has the same number of Fe$^{4+}$ ions as System 4, but La is not replaced with K. System 3 has no Fe$^{4+}$ formation even though La is replaced with K. We found that System 2 (pristine sample plus two holes, $+2$ charge) showed a decreased cell volume due to Fe$^{4+}$ formation, whereas System 3 (K-doped sample minus two holes, $-2$ charge) showed an increased cell volume because K$^+$ has a larger radius than La$^{3+}$. System 4 (K-doped sample with corresponding Fe$^{4+}$ formation) showed consistency with the average cell volume of Systems 2 and 3 as expected. However, because the magnitude of cell shrinkage caused by the formation of Fe$^{4+}$ is greater than that of expansion caused by the replacement of La by K, the cell volume of System 4 is smaller than that of the pristine sample (System 1).


In summary, we performed a combined theoretical and experimental investigation to examine the effect of substitutional K doping at the La site (3 at.\ \%) of \ce{LaFeO3}. The theoretical study showed that K doping creates shallow acceptor levels above the VBM of \ce{LaFeO3} and can effectively increase the majority carrier density. Furthermore, K doping decreased the bandgap by lowering the CBM and additionally generated multiple highly localized defect states below the CBM. The theoretical study also showed that the two holes generated by the replacement of La$^{3+}$ with K$^+$ are localized on Fe atoms and their neighboring O atoms, forming Fe$^{4+}$. The most stable hole-polaron configurations around K were examined. In terms of structure, our theoretical study predicted that K doping results in a shrinkage of the unit cell owing to the conversion of Fe$^{3+}$ to smaller Fe$^{4+}$ even though K$^+$ has a large ionic radius than La$^{3+}$.

The results obtained from the theoretical studies were verified by experimentally preparing \ce{LaFeO3} and K-doped (3 at.\ \%) \ce{LaFeO3} as high-quality, high-surface-area electrodes having the same morphology. The XRD patterns indeed showed a decrease in all cell parameters, which confirms the substitution of La$^{3+}$ by K$^+$ and the resulting Fe$^{4+}$ formation. The increase in Fe$^{4+}$ concentration in K-doped \ce{LaFeO3} was also supported by XPS. The UV-vis-NIR absorption spectra showed that K doping decreased the bandgap by $\sim$0.2 eV and enhanced the absorption below the bandgap as predicted by the electronic band structure and simulated optical response of K-doped \ce{LaFeO3}. Furthermore, Mott-Schottky analysis confirmed an increase in carrier density caused by K doping. Due to the increase in photon absorption and charge transport properties, K-doped \ce{LaFeO3} generated a significantly increased photocurrent for \ce{O2} reduction, which is a good measure of the increase in the number of surface-reaching electrons under illumination. K-doped \ce{LaFeO3} also increased the photocurrent for water reduction. However, the increase in this case was not significant due the poor catalytic ability of both the pristine and K-doped \ce{LaFeO3} for water reduction, resulting in considerable surface recombination. The beneficial effect of the decreased bandgap of K-doped \ce{LaFeO3} on photocurrent generation was confirmed by IPCE, which showed photocurrent generation below the bandgap of the pristine \ce{LaFeO3}. The enhanced photocurrent caused by K doping was also confirmed to be stable. This study demonstrated that substitutional doping of La$^{3+}$ with K$^+$ offers an effective way to increase both photon absorption and charge transport properties that directly influence photocurrent generation by \ce{LaFeO3}. We plan to perform further investigations of doping at the Fe site as well as doping at the La site with other dopants that vary in nature (alkaline, alkaline earth, or transition metals) and valency to obtain a comprehensive understanding of the effect of composition tuning of $p$-type oxides with a perovskite structure.



% \subsection{Fe2TiO5}


% ------------------------------------------------------------
% 2d
\chapter{Designing Efficient Transition Metal Oxides}

% %%%%%%%%%%%%%%%%%%%%%%%%%%%%%%%%%%%%%%%%%%%%%%%%%%%%%%%%%%
%    Section
% %%%%%%%%%%%%%%%%%%%%%%%%%%%%%%%%%%%%%%%%%%%%%%%%%%%%%%%%%%
\section{Overview}
Solar water splitting, the process of utilizing the Sun's energy by converting water to hydrogen fuel represents the most promising and sustainable means of harvesting renewable energies \cite{yao2018photoelectrocatalytic,tachibana2012artificial,walter2010solar}. In particular, many transition metal oxide (TMO) based photoelectrochemcial (PEC) devices (such as \ce{Fe2O3} and \ce{BiVO4}) are capable of achieving the highest solar-to-hydrogen efficiencies and offer the greatest avenue of developing this field, while simultaneously utilizing cheap and abundant sources \cite{kim2019elaborately,tamirat2016using,lee2019progress}.
Unfortunately, progress within this field has been stunted by the formation of small polarons (hereinafter referred to simply as polarons) which conduct via a thermally activated hopping mechanism and yield very poor conduction when compared to typical band conduction \cite{bosman1970small}. Hence, polarons are the chief bottleneck in the development of TMO-based PEC devices, as poor conduction leads to low efficiency.
% Small polarons represent carriers which induce large lattice polarization result in carrier wavefunctions which are spatially small (on the order of a unit cell).


Therefore, my research focuses on the development of our understanding of polarons in TMOs and the discovery of methods for facilitating polaron conduction in TMOs for improved solar-to-hydrogen efficiency. In particular, atomic doping in TMOs has been highly successful for improving the solar-to-hydrogen efficiency of TMOs. For example, in my first publication we studied how several defects in \ce{Fe2O3} influence carrier conduction \cite{smart2017effect}. This work formulated methods for studying polaron transport in doped systems and showed how Sn doping in \ce{Fe2O3} can improve carrier concentration. Following this, we developed mechanistic insights into how Li doping facilitates novel spin polaron conduction in CuO, which was validated by experimental measurement of enhanced photocathode performance by Li-doped CuO \cite{smart2018mechanistic}. Recently, our work demonstrating how the intrinsic band gap of \ce{Co3O4} had been misidentified due to the formation of optically active hole polarons caught a lot of attention \cite{smart2019optical}. In particular, this work received a Lawrence Livermore National Laboratory (LLNL) summer student poster symposium award, was highlighted in LLNL notable news, and was pivotal in my reception of the Graduate Student Scholar Program fellowship from LLNL. Finally, I have collaborated in many experimental works where our abilities to simulate polarons \cite{wheeler2019combined,radmilovic2020combined} and chemical reactions \cite{lu2017,kou2018,kou2019,lu2019} have proven vital in our understanding of how atomic doping enhances the performance of these materials.

\def\cbns{C$_\text{B}$}
\def\cb{C$_\text{B}$ }
\def\vncbns{V$_\text{N}$C$_\text{B}$}
\def\vncb{V$_\text{N}$C$_\text{B}$ }
\def\vnns{C$_\text{N}$}
\def\cn{C$_\text{N}$ }
\def\gwns{$G_0W_0$}
\def\gw{$G_0W_0$ }
\def\gns{$\Gamma$}
\def\g{$\Gamma$ }
\def\bR{\mathbf{R}}
\def\veps{\varepsilon}
\def\pans{PBE0($\alpha$)}
\def\p0a{PBE0($\alpha$)\text{}}


\section{Charge Defect Formation and Ionization Energies}

In our 2018 work published in Physical Review Materials~\cite{smart2018fundamental}, we demonstrated methods of computing charge transition levels at various levels of theory and propose referencing to vacuum among other things as a way to achieve consistency at semi-local DFT, hybrid DFT, and GW. We also employ Koopmans’ condition to achieve hybrid functionals which reproduce GW results.
Two-dimensional (2D) materials provide the unique opportunity to scale future electronics smaller than ever believed physically possible, implying engineering 2D materials is a promising strategy that can meet the demands of future nanotechnologies~\cite{butler2013progress}.
As defects play a crucial role in the optical and electronic properties of these systems, the engineering of defects in 2D materials has sparked continuous interest~\cite{wang2017engineering,hong2017atomic,lin2016defect,dreyer2018first,weston2018native,tawfik2017first}. For example, defects in \textit{h}-BN have been found to be the source of stable polarized and ultra-bright single-photon emissions at room temperature~\cite{bourrellier2016bright,abdi2018color,aharonovich2017quantum,tran2016quantum}. Hence, the development of our understanding of defects in 2D materials will open up further possibilities for emerging applications in quantum information and nanotechnology with much better scalability than traditional defects in 3D materials.

Unlike in their 3D counterparts~\cite{freysoldt2009fully,freysoldt2014first,vinichenko2017accurate,kumagai2014electrostatics,komsa2013finite}, first-principles techniques for calculating defect properties in 2D materials still face significant challenges.  Specifically, eliminating the periodic charge interactions for charged defects in 2D materials requires a charge correction scheme that accounts for the weak and anisotropic dielectric screening of 2D systems~\cite{komsa2014charged,komsa2018erratum,wang2015determination}. Furthermore, several exchange-correlation functionals that provide accurate electronic structures for 3D bulk systems are no longer applicable to ultrathin 2D systems.
For example, the fraction of Fock exchange ($\alpha$) in hybrid functionals can be approximated as the inverse of dielectric constant ($\varepsilon_{\infty}$) of the material~\cite{alkauskas2011defect,skone2014self}, but this is problematic for ultrathin 2D materials where $\varepsilon_{\infty}$  decreases to unity in the limit of infinite vacuum sizes (complete separation between periodic images). Therefore, the determination of $\alpha$ in hybrid functionals for 2D materials remains an open question. On the other hand, many body perturbation theory techniques ($\textit{e.g.}$ GW approximation) give accurate quasiparticle energies such as band gaps and band positions; however, high computational cost and slow convergence with respect to empty states make the screening of many defects in 2D materials impractical with conventional implementations~\cite{qiu2016screening,rasmussen2016efficient,thygesen2017calculating,huser2013quasiparticle,attaccalite2011coupling,felipe2017nonuniform}.

In our previous work~\cite{wu2017first,PING2017JCP}, we developed an efficient and accurate method that can give reliable charge corrections for total energies and electronic states of charged defects in 2D materials \textit{without any supercell extrapolations}, and then provided accurate defect CTLs with the DFT+GW scheme~\cite{malashevich2014first,chen2013correspondence,chen2015first,chen2017accuracy}. Such implementation is built on top of the WEST-code~\cite{govoni2015large}, Quantum-Espresso~\cite{QE1} and JDFTx~\cite{JDFTx} packages. In our GW calculations, we avoided explicit inclusion of empty states and inversion of dielectric matrices~\cite{govoni2015large,ping20132electronic,pham2013gw}, while also speeding up vacuum size convergence with a 2D Coulomb truncation~\cite{ismail2000new}. In this letter, we propose to solve two important issues for 2D materials. First, we determine which level of theory and which electron chemical potential reference one should use to calculate a CTL. Second, we show how to define the fraction of Fock exchange in hybrid functionals for accurate band edges and band gaps. In the end, we combine these two findings to obtain accurate defect ionization energies for 2D materials.

%%%%%%%%%%%%%%%%%%%%%%%%%%%%%%%%%%%%%%%%%%%%%%%%%%%%%%%%%%%%%%%%%%%%%%%%%%%%%
%%%%%%%%%%%%%%%%%%%%%%%%%%%%%%%%%METHODS%%%%%%%%%%%%%%%%%%%%%%%%%%%%%%%%%%%%%
%%%%%%%%%%%%%%%%%%%%%%%%%%%%%%%%%%%%%%%%%%%%%%%%%%%%%%%%%%%%%%%%%%%%%%%%%%%%%

\subsection{Methodology}

\textbf{Computational Methods}
In this work, all structural relaxations and total energy calculations were performed using open source plane wave code Quantum-ESPRESSO~\cite{QE1} with Perdew-Burke-Ernzerhof (PBE)~\cite{perdew1996generalized} exchange-correlation functional, ONCV norm-conserving pseudopotentials~\cite{ONCV1,ONCV2}, a wavefunction cutoff of 70 Ry and a $k$-point mesh corresponding to $12\times12\times1$ or higher in the primitive cell. The vacuum between periodic images along non-periodic direction is at least 30 Bohr.

Once the structural parameters were determined, we performed a separate single-point calculation using a wavefunction cutoff of 45 Ry and hybrid functionals including HSE, B3PW91, PBE0 and PBE0($\alpha$) with a sufficient  $k$-point mesh as large as $36\times 36$.
The band gap is determined from the difference between valence band maximum (VBM) and conduction band minimum (CBM). If the k-point of VBM or CBM is not included in the $k$-point mesh, it is interpolated between eigenvalues of the same band of nearby $k$-points.

A single point calculation using a wavefunction cutoff of 45 Ry and PBE functional was performed as the starting point for GW calculations. The GW calculations were performed using the WEST code~\cite{govoni2015large}. We used the $G_0W_0$ approach with starting wavefunctions and eigenvalues at the PBE level of theory. We employed the contour deformation technique for frequency integration of the self energy. For the dielectric matrix calculation, the number of eigenpotentials ($N_{\textrm{PDEP}}$) was chosen to be  $3N_{\textrm{electron}}$, and we used $4N_{\textrm{electron}}$ to validate its convergence. The final GW correction values were extrapolated between $9\times 9$ and $12\times 12$ $k$-point meshes to infinite $k$-points similar to Ref.~\cite{wu2017first}.


The charge corrections~\cite{PING2017JCP} for the total energies and eigenvalues of charged defects at the DFT level employed the techniques developed in Ref.~\cite{PING2017JCP} and in the SI section IV, which were implemented in the JDFTx code~\cite{JDFTx,ismail2000new,arias1992ab} (computed dielectric profiles are shown in the SI). Dielectric profiles are computed by applying finite electric fields following the procedure discussed in Ref.~\cite{wu2017first}, with a smearing width of 1 Bohr (smearing widths of 0.5 to 4.0 Bohr yield  identical charge corrections).



\textbf{Thermodynamic Charge Transition Levels}

\begin{figure}[t]
\includegraphics[keepaspectratio=true,width=\linewidth]{3-2d/figures-prm/path.png}
\caption{Schematic plot of the two paths (distinguished with blue/red color) that transition from charge state $q$ to $q+1$. For each path, there is a corresponding vertical excitation, which can be computed either with EA$_{q+1}$ or IP$_q$ (noted with up/down arrowheads), as discussed in the main text.}  \label{fig:path}
\end{figure}


A thermodynamic CTL is the value of electron chemical potential $\veps_F$ at which the stable charge state of the system changes, \textit{e.g.} from $q$ to $q+1$. Therefore, CTLs are calculated through the equivalency of the formation energies $q$ and $q+1$, given by Eq. (\ref{eq:ctl0})~\cite{freysoldt2014first}.
\begin{align}
\veps_{q+1|q}&=E_q^f(\bR_q)-E_{q+1}^f(\bR_{q+1}) \nonumber\\
&=E_q(\bR_q)-E_{q+1}(\bR_{q+1})-\veps_F
\label{eq:ctl0}
\end{align}
Here $E_q^f(\bR)$ is the defect formation energy with charge $q$ and geometry $\bR$, and $\bR_q$ is the relaxed geometry of the system with charge $q$. $E_q(\bR)$ is the total energy that relates to $E_q^f(\bR)$ and $\veps_F$ following the definition of Eq. (1) in Ref.~\cite{wu2017first}.  Diagrammatically, Eq. (\ref{eq:ctl0}) is the energy difference between two potential  surface minimua in position space $\bR$, as shown in Fig. \ref{fig:path}.



% \section{Results}
\subsection{Implementing Quasiparticle Corrections in Defect Charge Transition Levels}


It is well-known that local and semi-local functionals do not give accurate total energy differences between two charge states, where an electron removing (IP)/adding process (EA) is involved. An alternative approach~\cite{wu2017first} is to separate Eq.~(\ref{eq:ctl0}) into two parts: the vertical excitation energy between two charge states ($q$ and $q+1$) at the same geometry ($\bR$) (denoted as quasiparticle energies $\veps^{QP}$) and the geometry relaxation energy at a fixed charge state (denoted with $E^{rlx}$). Since DFT is known to provide reliable geometry relaxation energies (if one corrects the fictitious charge interactions between periodic images as we did in Ref.~\cite{wu2017first}), this separation allows us to accurately calculate the vertical excitation energies with a higher level of theory appropriate for non-neutral excitations, such as the GW approximation.

One can separate Eq.~(\ref{eq:ctl0}) by two possible physical pathways from $E_q^f(\bR_q)$ to $E_{q+1}^f(\bR_{q+1})$ as shown in Fig. \ref{fig:path}. One pathway (red path) occurs with a vertical excitation at $\bR_{q}$ ($E_{q+1}^f(\bR_q)-E_q^f(\bR_q)$) followed by a geometry relaxation at the charge state $q+1$ ($E_{q+1}^f(\bR_{q+1})-E_{q+1}^f(\bR_q)$), shown in Eq.~(\ref{eq:ctl1}). The other pathway (blue path) occurs through the geometry relaxation at the charge state $q$ plus a vertical excitation at $\bR_{q+1}$, corresponding to Eq.~(\ref{eq:ctl2}).
\begin{align}
\veps_{q+1|q} &= \underbrace{E_{q}^f(\bR_q)-E_{q+1}^f(\bR_q)}_{\veps^{QP}}+\underbrace{E_{q+1}^f(\bR_q)-E_{q+1}^f(\bR_{q+1})}_{E^{rlx}}  \nonumber\\
&= \veps^{QP}_{q+1|q}(\bR_q) + E^{rlx}_{q+1}
\label{eq:ctl1}
\end{align}
\begin{align}
\veps_{q+1|q}&=\underbrace{E_q^f(\bR_q)-E_q^f(\bR_{q+1})}_{E^{rlx}}+\underbrace{E_q^f(\bR_{q+1})-E_{q+1}^f(\bR_{q+1})}_{\veps^{QP}} \nonumber\\
&= E^{rlx}_q  + \veps^{QP}_{q+1|q}(\bR_{q+1})
\label{eq:ctl2}
\end{align}
Note that all three equations (Eq. (\ref{eq:ctl0}), (\ref{eq:ctl1}), (\ref{eq:ctl2})) are exactly equivalent theoretically. Yet, in practice they may yield sizable differences, as discussed later.

Furthermore, the vertical excitation energies $\veps^{QP}_{q+1|q}$ in Eq.~(\ref{eq:ctl1}) and Eq.~(\ref{eq:ctl2}) can be determined from either the ionization potential of the charge state $q$ ($\text{IP}_q$) or
the electron affinity of the charge state $q+1$ ($\text{EA}_{q+1}$)
, as noted in Fig.~\ref{fig:path} with up/down arrowheads. Note that we obtained IP and EA through eigenvalues at different levels of theory based on the Janak's theorem~\cite{janak1978proof}.
%\begin{align}
%\veps^{QP}_{q|q+1}(\bR_q) &= \text{EA}_{q+1}(\bR_q) = %\text{IP}_q(\bR_q) \label{eq:kp1} \\
%\veps^{QP}_{q|q+1}(\bR_{q+1}) &= \text{EA}_{q+1}(\bR_{q+1}) = %\text{IP}_q(\bR_{q+1}) \label{eq:kp2}
%\end{align}
The difference between $\text{IP}_q$ and $\text{EA}_{q+1}$ is largely related to the delocalization or localization error at a particular level of theory, and serves as a stringent test for an exchange-correction scheme in electronic structure calculations~\cite{bruneval2009g}.


\begin{table}[H]
    \footnotesize
\centering
% \begin{tabular}{p{1.5cm}p{1.5cm}p{1.2cm}p{1.2cm}p{1.2cm}p{1.2cm}}
\begin{tabular}{cccccc}
\hline\hline
\multicolumn{2}{c}{Method} & \multicolumn{4}{c}{Defect} \\
%\hline
     &                 & \cb & \vncb & C$_\text{N}$ & \vncb \\
% &&&&&\\
     &    CTL             & (0/+1) & (0/+1) & (-1/0) & (-1/0) \\
\hline
 %   &&&&&\\
     & Eq\ref{eq:ctl0} & -3.63  & -4.22  & -3.54  & -1.57  \\
PBE  & Eq\ref{eq:ctl1} & -3.61  & -4.29  & -3.51  & -1.66  \\
     & Eq\ref{eq:ctl2} & -3.64  & -4.33  & -3.49  & -1.67  \\
     \hline
    % & IP-EA           & 2.68  & 2.60  & 2.75  & 2.50  \\
 % &&&&&\\
     & Eq\ref{eq:ctl0} & -3.65  & -4.19  & -3.50  & -1.87  \\
PBE0 & Eq\ref{eq:ctl1} & -3.60  & -4.17  & -3.50  & -1.87  \\
     & Eq\ref{eq:ctl2} & -3.62  & -4.21  & -3.50  & -1.21*  \\
\hline
    % & IP-EA           & 1.15  & 1.09  & 1.13  & 1.42  \\
% &&&&&\\
     & Eq\ref{eq:ctl1} & -3.40  & -4.29  & -3.74  & -1.74 \\
\gw  & Eq\ref{eq:ctl2} & -3.28  & -4.22  & -3.73  & -1.70  \\
\hline
    % & IP-EA           & 0.04  & 0.20  & 0.03  & 0.19  \\
    % &&&&&\\
    \multicolumn{6}{c}{IP$_{q}(\bR_{q})$-EA$_{q+1}(\bR_{q})$} \\
     % &&&&&\\
PBE && 2.68  & 2.60  & 2.75  & 2.50  \\
PBE0 && 1.15  & 1.09  & 1.13  & 1.42  \\
\gw && 0.04  & 0.20  & 0.03  & 0.19  \\
 \hline\hline
\end{tabular}
\caption{\label{table:path} Charge transition levels (CTLs) relative to vacuum (in eV) of multiple defects in monolayer \textit{h}-BN. These values are collected via three methods (Eq. (\ref{eq:ctl0}-\ref{eq:ctl2})) at various levels of theory (PBE, PBE0, \gwns$@$PBE ). The CTLs relative to vacuum are remarkably similar. The one exception, \vncb (-1/0) at PBE0 (marked with *) incidentally has a band inversion resulting in a defect level within the valence band, breaking the reliability of Eq. (\ref{eq:ctl2}).
We also show IP$_{q}(\bR_{q})-$EA$_{q+1}(\bR_{q})$ at different levels of theory. Note that at the \gw level, this difference is $<0.2$ eV.}
\end{table}

%%%%%%%%%%%%%%%%%%%%%%%%%%%%%%%%%%%%%%%%%%%%%%%%%%%%%%%%%%%%%%%%%%%%%%%%%%%%%
%%%%%%%%%%%%%%%%%%%%%%%%%%%%RESULTS & DISCUSSION%%%%%%%%%%%%%%%%%%%%%%%%%%%%%
%%%%%%%%%%%%%%%%%%%%%%%%%%%%%%%%%%%%%%%%%%%%%%%%%%%%%%%%%%%%%%%%%%%%%%%%%%%%%
Therefore, we firstly compared the CTL with PBE, PBE0 and $G_0W_0@$PBE for three different defects in monolayer BN as shown in Table \ref{table:path}, where $\veps^{QP}$ is obtained by taking the average of $\text{IP}_q$ and $\text{EA}_{q+1}$ as:
\begin{equation}
\veps^{QP}_{q+1|q}(\bR) = \frac{1}{2}(\text{EA}_{q+1}(\bR) + \text{IP}_q(\bR))
\end{equation}
Note that we propose to set $\veps_F$ equal to the vacuum level (determined by the electrostatic potential in the vacuum region of supercells) and use it as a reference for Eq. (\ref{eq:ctl0}).
We found this choice (opposed to band edges) is particularly advantageous for obtaining consistent CTLs among different methods as shown in Table~\ref{table:path}.
(More computational details for $G_0W_0@$PBE can be found in SI, with similar numerical techniques and parameters used in Ref.~\cite{wu2017first}). There are several interesting observations from Table \ref{table:path}, as follows. First, we found excellent agreement (within 0.1 eV) among Eq. (\ref{eq:ctl0}), (\ref{eq:ctl1}) and (\ref{eq:ctl2}) for each defect at each level of theory. Second, we found the results obtained among PBE, PBE0 and $G_0W_0@$PBE are also strikingly similar (within 0.2 eV) for each defect. This means the CTLs of 2D materials relative to vacuum are \textit{not affected by the level of theory one chooses}. Note that the difference between $\text{IP}_q$ and $\text{EA}_{q+1}$ is more than 2 eV for PBE, reduced to $~$1 eV at PBE0 level ($\alpha=0.25$), but less than 0.2 eV at $G_0W_0@$PBE, which indicates the delocalization error present with semi-local DFT has been mostly corrected at $G_0W_0$@PBE~\cite{bruneval2009g}.
%As we will discuss later, it is the linearity of the Kohn-Sham eigenvalues with $\alpha$ (Figure \ref{fig:ipea}) that allows the averaging of $\text{IP}_q$ and $\text{EA}_{q+1}$ to give accurate PBE, PBE0 results in Eq. \ref{eq:ctl1} and \ref{eq:ctl2}, which are in great agreement with each other, despite this error.


\subsection{Generalized Koopman's Condition for Exact Exchange of 2D Materials}
After we obtained reliable CTLs, in particular relative to vacuum, we focused on how to calculate accurate band edge positions and band gaps of 2D materials in order to determine defect ionization energies. Using the GW approximation, we obtained an accurate quasiparticle band gap (indirect at T$\rightarrow$M) 6.01 eV for bulk h-BN (Table \ref{table:gap}), in excellent agreement with the experimental fundamental electronic gap 6.08 $\pm$ 0.015~\cite{cassabois2016hexagonal}. Nonetheless, GW is still too computationally demanding for materials' screening and computing forces is non-trivial. Therefore, the development of computationally affordable methods such as accurate non-empirical hybrid functionals for 2D materials is strongly desired.

\begin{figure}[h]
\begin{center}
\includegraphics[keepaspectratio=true,width=0.5\linewidth]{3-2d/figures-prm/exx_eig.pdf}
\caption{The IP at $q=0$ and the EA at $q=+1$ for the defects \cbns , \cn and \vncb in monolayer h-BN as a function of the fraction of Fock exchange $\alpha$ for PBE0($\alpha$). The predicted exchange constant ($\alpha = 0.409$, 0.41 and 0.382, respectively) is the corresponding crossing point where $\text{EA}_{q+1}=\text{IP}_{q}$.} \label{fig:ipea}
\end{center}
\end{figure}

\begin{figure}[h]
\begin{center}
\includegraphics[keepaspectratio=true,width=0.7\linewidth]{3-2d/figures-prm/p0a_gap.pdf}
\caption{Comparing computed band gaps of \textit{h}-BN (monolayer, bilayer, trilayer, bulk) and graphane with \p0a versus those computed with \gwns$@$PBE . Overall we find that our \p0a results agree very well with \gwns , yielding a MAE of 0.14 eV. } \label{fig:p0a}
\end{center}
\end{figure}

\begin{table}[H]
    \footnotesize
    \centering
\begin{tabular}{ccccccc}
	 \hline\hline
     System & PBE & HSE & PBE0 & B3PW & \p0a & \gw \\
	 % \hspace{2mm} System & \hspace{4mm} PBE & \hspace{4mm} HSE & \hspace{3mm} PBE0 & \hspace{2mm} B3PW & \hspace{1mm} \p0a & \hspace{3mm} \gw \\
	 \hline
	% \hspace{0.1mm} Graphane
	Graphane
	& 3.57 $|$ \gns$\,\rightarrow\,$\g & 4.41 $|$ \gns$\,\rightarrow\,$\g
	& 5.06 $|$ \gns$\,\rightarrow\,$\g & 5.04 $|$ \gns$\,\rightarrow\,$\g
	& 6.54 $|$ \gns$\,\rightarrow\,$\g & 6.41 $|$ \gns$\,\rightarrow\,$\g   \\
	% \hspace{0.1mm} ML BN
	ML BN
	& 4.71 $|$ K$\,\rightarrow\,$K & 5.70 $|$ K$\,\rightarrow\,$\g
	& 6.33 $|$ K$\,\rightarrow\,$\g & 6.33 $|$ K$\,\rightarrow\,$\g
	& 7.34 $|$ K$\,\rightarrow\,$\g & 7.01 $|$ K$\,\rightarrow\,$\g   \\
	% \hspace{0.1mm} BL BN
	BL BN
	& 4.49 $|$ T$\,\rightarrow\,$M & 5.81 $|$ T$\,\rightarrow\,$M
	& 6.46 $|$ T$\,\rightarrow\,$\g & 6.17 $|$ T$\,\rightarrow\,$M
	& 7.08 $|$ T$\,\rightarrow\,$\g & 7.00 $|$ T$\,\rightarrow\,$\g   \\
	% \hspace{0.1mm} TL BN
	TL BN
	& 4.36 $|$ T$\,\rightarrow\,$M & 5.68 $|$ T$\,\rightarrow\,$M
	& 6.40 $|$ T$\,\rightarrow\,$M & 6.03 $|$ T$\,\rightarrow\,$M
	& 7.01 $|$ T$\,\rightarrow\,$\g & 6.92 $|$ T$\,\rightarrow\,$M  \\
	% \hspace{0.1mm} Bulk BN
	Bulk BN
	& 4.22 $|$ T$\,\rightarrow\,$M & 5.60 $|$ T$\,\rightarrow\,$M
	& 6.28 $|$ T$\,\rightarrow\,$M & 5.91 $|$ T$\,\rightarrow\,$M
	& 6.07 $|$ T$\,\rightarrow\,$M & 6.01 $|$ T$\,\rightarrow\,$M   \\
	 \hline
    % \hspace{0.5mm} Exp.(Bulk BN) & \multicolumn{2}{c}{6.08 $\pm$ 0.015}  \\
	Exp.(Bulk BN) & \multicolumn{2}{c}{6.08 $\pm$ 0.015}  \\
	 \hline
	 \hline
	\end{tabular}
      \captionof{table}{Band gaps for various pristine 2D materials. In general, PBE severely underestimates the gap. Hybrid functionals HSE, B3PW, and PBE0 ($\alpha=0.25$) generally enlarge the bulk band gap, but still underestimate the gaps of ultrathin 2D systems compared with experiments and GW approximation. Only PBE0($\alpha$) with $\alpha$ satisfying $\text{IP}_{q}=\text{EA}_{q+1}$ of localized defects (\cbns) yield gaps in good agreement with experiment~\cite{cassabois2016hexagonal} and $G_0W_0$@PBE.}
      \label{table:gap}
\end{table}



The generalized Koopmans' condition has been mostly used to determine the appropriate fraction of Fock exchange ($\alpha$) for molecules and molecular crystals~\cite{atalla2016enforcing,korzdorfer2014organic,refaely2012quasiparticle,sai2011hole,cohen2007development,pinheiro2015length,ma2016using}.
One recent work~\cite{miceli2018nonempirical} enforced this condition (\textit{i.e.} $\text{EA}_{q+1}=\text{IP}_{q}$) on defects in bulk semiconductors to obtain $\alpha$ and in turn predicted accurate electronic structure of the corresponding pristine bulk systems. The fundamental assumption is that the optimized $\alpha$ depends on the long range screening of the system and not on the nature of the probe defects. This condition is also valid for deep defects in 2D materials, where defect wavefunctions are well localized like molecule orbitals in the supercells, and their  contribution to dielectric screening is negligible compared to the crystal environment.
Another advantage of applying this condition to 2D systems is that both $\text{EA}_{q+1}$ and $\text{IP}_{q}$ can be exactly referenced to vacuum.
In order to validate the applicability of the generalized Koopmans' condition to 2D materials, we used the defect \cb as a probe to determine $\alpha$ for \textit{h}-BN (B$_\text{C}$ for graphane). This method gives $\alpha$ of
%we use it to predict $\alpha$ instead of 25$\%$ in PBE0 calculations (\pans ) as shown in Eq. \ref{eq:p0a}:
%\begin{equation}
%E_{X}^{PBE0(\alpha)} = \alpha E_{X}^{HF} + (1-\alpha) %E_{X}^{PBE}
%\label{eq:p0a}
%\end{equation}
%Namely, the $\alpha$ which gives an equivalent ionization %potential and electron affinity (IP$(q=0)=$EA$(q=+1)$) for a %given defect may determine the accurate hybrid functional for %the host 2D materials.
%In this paper, we will refer to this method simply as  \pans. %Using the defect \cb at geometry $R_{0}$ we can determine %$\alpha$ to be
0.409, 0.347, 0.324, 0.225 for monolayer, bilayer, trilayer and bulk \textit{h}-BN, respectively. Note that the $\alpha$ value 0.225 for bulk h-BN, agrees well with the predicted $\alpha$ from the inverse of high frequency dielectric constant ($\alpha = 1/\veps_{\infty} \approx 0.2$)~\cite{brar2014hybrid}, which supports the assumption that long-range screening determines $\alpha$. We also investigated other defects C$_{\text{N}}$ and {\vncbns} as probes of $\alpha$ (their corresponding electronic structure can be found in SI).
%while the band gaps of few layer \textit{h}-BN follow the trend of a monotonic decrease  in dielectric constant vs. number of layers~\cite{dielectric2}.

\begin{figure}
\begin{center}
\includegraphics[keepaspectratio=true,width=1.0\linewidth]{3-2d/figures-prm/cbctl.png}

% \begin{picture}(500,180)
% \put(0,10){\subfloat[\hspace{2mm}PBE\hspace{1mm}]{\includegraphics[keepaspectratio=true,scale=0.7]{3-2d/figures-prm/cbctl_pbe.pdf}}}
% \put(120,10){\subfloat[\hspace{2mm}HSE\hspace{1mm}]{\includegraphics[keepaspectratio=true,scale=0.7]{3-2d/figures-prm/cbctl_hse.pdf}}}
% \put(240,10){\subfloat[\hspace{0mm}PBE0($\alpha$)]{\includegraphics[keepaspectratio=true,scale=0.7]{3-2d/figures-prm/cbctl_pbe0ie.pdf}}}
% \put(360,10){\subfloat[\hspace{2mm}\gw\hspace{1mm}]{\includegraphics[keepaspectratio=true,scale=0.7]{3-2d/figures-prm/cbctl_gw.pdf}}}
% \put(305,10){\includegraphics[keepaspectratio=true,scale=0.3]{3-2d/figures-prm/white.png}}
% \put(240,10){\includegraphics[keepaspectratio=true,scale=0.7]{3-2d/figures-prm/cbctl_pbe0ie.pdf}}
% \put(185,10){\includegraphics[keepaspectratio=true,scale=0.3]{3-2d/figures-prm/white.png}}
% \put(120,10){\includegraphics[keepaspectratio=true,scale=0.7]{3-2d/figures-prm/cbctl_hse.pdf}}
% \put(65,10){\includegraphics[keepaspectratio=true,scale=0.3]{3-2d/figures-prm/white.png}}
% \put(0,10){\includegraphics[keepaspectratio=true,scale=0.7]{3-2d/figures-prm/cbctl_pbe.pdf}}
% \end{picture}

\caption{Charge transition level \cb (+1/0) in \textit{h}-BN with different  levels of theory. Defect charge transition levels gradually become shallower with lower ionization energies while increasing the number of layers (ionization energies are written adjacent to arrows from the CTL to CBM). Note that the defect CTLs are very similar relative to vacuum between different methods.}  \label{fig:ctl2}
\end{center}
\end{figure}


Interestingly, we found that IP$_q$ and EA$_{q+1}$ from Kohn-Sham eigenvalues varied linearly with $\alpha$.
%consistent with the case of bulk materials in Ref.~\cite{miceli}.As shown in Figure \ref{fig:ipea}, linear fits of IP at $q=0$ and EA at $q=+1$ of the \cb defect in monolayer \textit{h}-BN %yield R-Squared values of 0.9997 and 0.9999, respectively for %$\alpha$ from 0.25 to 0.51.
Fig. \ref{fig:ipea} shows this linearity for three defects in monolayer \textit{h}-BN, and three defects predict very similar $\alpha$, which justifies the insensitivity of $\alpha$ to the explicit defect. %Note that  defects with localized wavefunctions such as atomic substitutions (\cb) determine more accurate $\alpha$ because their contribution to dielectric screening is negligible. In this way, the screening is determined from the crystal.
It is also notable that the slopes of  IP$_q$ and EA$_{q+1}$ are opposite but nearly equal, explaining how the average of IP$_q$ and EA$_{q+1}$ as $\veps^{QP}$ for CTL in Eq. (\ref{eq:ctl1}) and (\ref{eq:ctl2}) works well (as shown in Table \ref{table:path}).


%%%%%%%%%%%%%%%%%%%%%%%%%%%%%%%%%%%%%%%%%%%%%%%%%%%%%%%%%%%%%%%%%%%%%%%%%%%%%
%%%%%%%%%%%%%%%%%%%%%%%%%%%%RESULTS & DISCUSSION%%%%%%%%%%%%%%%%%%%%%%%%%%%%%
%%%%%%%%%%%%%%%%%%%%%%%%%%%%%%%%%%%%%%%%%%%%%%%%%%%%%%%%%%%%%%%%%%%%%%%%%%%%%



Most commonly, two-dimensional systems are synthesized with a few layers of the material, therefore understanding the effect of increasing thickness is essential to connect with realistic experiments. As such, we have computed the band gaps of monolayer, bilayer, trilayer and bulk \textit{h}-BN, as well as graphane with several hybrid functionals including HSE, PBE0, B3PW and \p0a (with $\alpha$ predicted earlier), as well as with \gwns @PBE for a reliable comparison (see Table \ref{table:gap}). As anticipated, PBE strongly underestimated monolayer \textit{h}-BN band gap: 4.71 eV with a direct transition at the K point. With any level of theory beyond PBE, monolayer \textit{h}-BN is predicted to have a larger, indirect gap from K to \gns. In accordance with quantum confinement, we observed that the band gaps of \textit{h}-BN obtained at B3PW,  \pans , and \gw show a sharp increase at ultrathin BN (monolayer to trilayer) compared to bulk BN. However, HSE and PBE0 provide almost the same band gaps between ultrathin and bulk BN. This is because there is a severe change in the dielectric screening from monolayer to bulk, and a different portion of Fock exchange must be instilled.

Using \p0a we obtained results consistent with quantum confinement and in best agreement with our \gw calculations with a MAE of 0.14 eV (see SI Fig. 3). In addition, the B3PW functional~\cite{becke1993density,crowley2016resolution} provided a more accurate bulk BN band gap than PBE0 and HSE but still underestimated the band gaps of ultrathin BN. Therefore, the direct/indirect transitions and magnitude of the gaps from bulk to monolayer are provided accurately solely with \p0a and \gwns.
%The measured band gap of bulk \textit{h}-BN is 6.08 eV with valence and conduction band extrema near the symmetric points K and M~\cite{expgap} respectively, in agreement with our B3PW, \pans , and \gw results (within 0.1 eV).
In brief, the results shown in Table \ref{table:gap} validate our method for determining accurate fundamental band gaps for 2D materials from first-principles. We note that calculated band edge positions relative to vacuum are also similar at \p0a and \gw as shown in Fig. \ref{fig:ctl2} and SI.



\subsection{Defect Ionization Energies in 2D Materials}
Finally, CTLs and ionization energies for \cb in \textit{h}-BN computed at PBE, HSE, \p0a and $G_0W_0$ levels of theory as a function of number of layers are shown in Fig. \ref{fig:ctl2}. Consistent with the findings in Table \ref{table:path}, CTLs changed less than 0.1 eV across different theoretical methods relative to vacuum. Interestingly, no clear trend and only small difference have been found in the band edge positions of \textit{h}-BN from monolayer to triple layers. These results illustrate that one just needs to correct the band edge positions of pristine 2D materials with \p0a or \gwns , and use CTLs determined from DFT with semi-local functionals, then the difference of the two yields accurate defect ionization energies in 2D materials. On another note, we found there is a clear monotonic decrease in the ionization energies of defects in BN with increasing number of layers, mostly contributed by CTLs' shift towards vacuum (shown in Fig. \ref{fig:ctl2}; also see SI Fig. 5). This effect can be understood as a result of increased dielectric screening with more layers of \textit{h}-BN, and is consistent with the effect of dielectric environments on the ionization energies of MoS$_2$~\cite{noh2015deep}.

\begin{figure}[t]
\begin{center}
\includegraphics[keepaspectratio=true,width=0.5\linewidth]{3-2d/figures-prm/ie.pdf}
\caption{Ionization energies of \cb in \textit{h}-BN with varying number of layers. It is observed that ionization energies decrease monotonically with increasing number of layers. Note that \p0a and \gw give results in excellent agreement.} \label{fig:ie}
\end{center}
\end{figure}

%%%%%%%%%%%%%%%%%%%%%%%%%%%%%%%%%%%%%%%%%%%%%%%%%%%%%%%%%%%%%%%%%%%%%%%%%%%%%
%%%%%%%%%%%%%%%%%%%%%%%%%%%%%%%%%CONCLUSION%%%%%%%%%%%%%%%%%%%%%%%%%%%%%%%%%%
%%%%%%%%%%%%%%%%%%%%%%%%%%%%%%%%%%%%%%%%%%%%%%%%%%%%%%%%%%%%%%%%%%%%%%%%%%%%%

\subsection{Conclusions}

In summary, we established fundamental principles to reliably and efficiently compute ionization energies for defects in 2D materials.
Specifically, band edge positions of the pristine systems should be computed with our proposed \p0a hybrid functional or GW approximations, and the defect CTL can be obtained reliably by standard DFT with semi-local functional, if relative to vacuum.
We successfully applied the proposed methods for a variety of defects from monolayer to triple layer \textit{h}-BN, as well as  graphane.  We also demonstrated that defect ionization energies decreased with increasing number of layers in \textit{h}-BN, mainly due to enlarged dielectric screening. Our findings in this work suggest efficient and accurate methods to compute defect ionization energies and electronic structures in 2D materials, which can be applied to screening new promising defects for quantum information and optoelectronic applications.

%\newcommand{\vecauto}[1]{\ensuremath{\vectorarrow* {#1}}}
\newcommand{\vecauto}[1]{\ensuremath{\vectorbold*{#1}}}
\newcommand{\vR}{\vecauto{R}}

\newcommand{\CN}{\ensuremath{\textrm{C}_\textrm{N}}}
\newcommand{\NBVN}{\ensuremath{\textrm{N}_\textrm{B}\textrm{V}_\textrm{N}}}
\newcommand{\CBVN}{\ensuremath{\textrm{C}_\textrm{B}\textrm{V}_\textrm{N}}}
\newcommand{\XBVN}{\ensuremath{\textrm{X}_\textrm{B}\textrm{V}_\textrm{N}}}
\newcommand{\OBOBVN}{\ensuremath{\textrm{O}_\textrm{B}\textrm{O}_\textrm{B}\textrm{V}_\textrm{N}}}

\newcommand{\Bandname}[1]{\ensuremath{
  \StrChar{#1}{1}[\chara]
  \StrChar{#1}{2}[\charb]
  \StrChar{#1}{3}[\charc]
  \StrChar{#1}{4}[\chard]
\IfEqCase{\chard}{
   {U}{\chara\mathrm{\charb}_{\charc}\uparrow}
   {D}{\chara\mathrm{\charb}_{\charc}\downarrow}
}[\PackageError{Bandname}{Undefined option for spin: \StrChar{#1}{4}}]
}}


\section{Carrier Recombination Mechanisms}
% PRB


\subsection{Introduction}
The engineering of spin defects in wide-band semiconductors offers a promising avenue for the development of quantum spin devices.~\cite{aharonovich2011diamond,ladd2010quantum, seo2016design,dreyer2018first} They are among the few alternatives for quantum technologies that operate at room temperature. Deep defects in two-dimensional (2D) materials such as hexagonal boron nitride ($h$-BN)~\cite{aharonovich2014diamond,tran2016quantum,kim2018photonic,bourrellier2016bright,grosso2017tunable,li2017nonmagnetic,exarhos2017optical,exarhos2019magnetic,museur2008defect,sontheimer2017photodynamics,jungwirth2016temperature,jungwirth2017optical,noh2018stark} and transition metal dichalcogenides (TMD)~\cite{mak2016photonics,palacios2017large} have proven to be promising single photon sources with polarized and ultrabright single photon emission at room temperature. These materials exhibit unprecedented potential for several applications, including large-scale nanophotonics and quantum information processing,~\cite{beveratos2002single,obrien2009quantum,scarani2009the,aharonovich2016solid,vogl2018fabrication} which in turn provide a new platform for exploring quantum phenomena.~\cite{dreyer2018first}
In order for these defect centers to provide bright SPE,~\cite{aharonovich2011diamond,aharonovich2016solid} the radiative recombination rate (photon emitted) needs to be high, while the non-radiative recombination rate (no photon emitted) must be substantially lower to yield a high quantum efficiency.
Furthermore, a weak electron-phonon coupling is also required to ensure long spin relaxation time for the application of qubit and stable single photon emission at room temperature.

Despite the importance of maximizing radiative rates for quantum information, the factors which determine the recombination process at defects in 2D materials are not understood experimentally or theoretically. Past theoretical studies have either focused on radiative recombination in pristine 2D materials~\cite{palummo2015exciton,wu2017first} or phonon assisted non-radiative recombination for defects in 3D wide band-gap semiconductors.~\cite{alkauskas2014first1,shi2015comparative} Therefore, a fully comparative study of both recombination processes for defect centers in 2D materials is highly desired.

\begin{figure}[t]
    \centering
  \includegraphics[width=0.8\linewidth]{3-2d/figures-nonrad/toc3.png}
  \caption{Schematic diagram of carrier recombination at the \NBVN\ defect in monolayer $h$-BN. In order for a defect to be a robust single photon source, it is necessary to have the radiative recombination rates much higher than the non-radiative ones.}
  \label{fig:toc1}
\end{figure}

Furthermore, the high tunability of SPE allows them to be integrated within a vast array of applications.~\cite{tran2016quantum,aharonovich2016solid} Among these methods, strain modulation is one of the most effective strategies, especially for low-dimensional materials which can work under large distortion.~\cite{brawand2015surface}
For example, in 2D systems, some key electronic properties such as band gap, change by 1.5\% under 1\% uniaxial tension in TMD monolayers~\cite{chang2013orbital} or 6\% under 1\% uniaxial tension in phosphorene~\cite{peng2014strain}. Additionally, the non-radiative process, which is intrinsically sensitive to lattice deformation (as it is phonon mediated), may exhibit even more drastic changes under strain.

In this communication, we first introduce the formalism of computing the radiative and non-radiative lifetime of defect excited states from first-principles. We then focus on comparing radiative and non-radiative processes of different transitions in a series of important defects in monolayer $h$-BN, where we discuss the dominant recombination processes and their implication on SPE efficiency.
Finally, we show that applying strain to $h$-BN defects can effectively tune the non-radiative rates and quantum yield of SPE.

%\section{Computational Methods}
The radiative and non-radiative transition rates between two electronic states under perturbation can be computed via Fermi's golden rule:
\begin{align}
    r_{if}^{R} &=  \frac{2\pi}{\hbar}g\left|\mel{f}{H^R}{i}\right|^2\delta(E_i-E_f),  \label{eq:rate-radiative-fg} \\
    r_{if}^{NR} &= \frac{2\pi}{\hbar}g\sum_{n,m}p_{in}\left|\mel{fm}{H^{e-ph}}{in}\right|^2\delta(E_{in}-E_{fm}). \label{eq:rate-nonradiative-fg}
\end{align}
Here, $r_{if}^R$ and $r_{if}^{NR}$ denote the recombination rates between electronic states $i$ and $f$ via a radiative process ($r_{if}^R$) and non-radiative process ($r_{if}^{NR}$), respectively. %Photons carry negligible momentum, and so the radiative rate is considered at a fixed phonon state.
$g$ is the degeneracy factor of the final state, i.e. several equivalent energy-degenerate atomic configurations of the final state might exist.~\cite{alkauskas2014first2}
For defects in 2D materials studied in this work, $g$ factors are all equal to 1.
$H^R$ is the electron-photon coupling (electromagnetic) Hamiltonian and $H^{e-ph}$ is the electron-phonon coupling Hamiltonian. A sum over phonon states $n,m$ enters the non-radiative recombination process with an occupation number $p_{in}$ of the vibronic state $\ket{in}$. For ground state calculations, we employed open source plane-wave code Quantum ESPRESSO~\cite{QE1} with ONCV norm conserving pseudopotentials~\cite{ONCV1,ONCV2} and a supercell size of $6\times6$ or higher. Charge corrections for the total energies and eigenvalues of charged defects were applied by employing the techniques developed in Ref.~\cite{wu2017first,PING2017JCP}. The total energies, defect formation energies and geometry were computed at both PBE and hybrid functional levels (the results presented in the main text are computed at PBE, and detailed comparison between two levels can be found in the Supporting Information (SI) Table S4). The band gaps of pristine $h$-BN are computed at $GW@PBE$ as done in our previous work~\cite{wu2017first}, which are 6.01 eV for bulk and 7.01 eV for monolayer $h$-BN respectively.
The exciton dipole moments and exciton energies as input for radiative lifetime were computed at many body perturbation theory with GW approximation for quasiparticle energies~\cite{govoni2015large,ping20132electronic,nguyen2012improving} and then solving the Bethe-Salpeter equation with the Yambo-code~\cite{YAMBO}, as well as Random Phase approximation with DFT eigenvalues (detailed comparison can be found in the SI, Table S3; the results in the main text are computed at DFT-RPA). More computational details and formulation of radiative rates $r_{if}^R$ are discussed in the SI and Ref.~\cite{wu2019dimensionality}.

The non-radiative rate is simplified by the static coupling approximation with a one-dimensional (1D) effective phonon approximation\cite{alkauskas2014first1,alkauskas2014first2,barmparis2015theory,alkauskas2012first,alkauskas2016tutorial,brawand2015surface,giustino2017electron,henry1977nonradiative,howgate1969calculation,monserrat2018phonon,passler1974description,passler1976relationships,shi2012ab,shi2015comparative,wickramaratne2018comment,shi2018reply,yang2016non,Book_TheoryofDefects_Stoneham} (the validation of 1D effective phonon approximation in $h$-BN is based on the similarity of Huang-Rhys factors between 1D effective phonon and all phonon calculations, as discussed in the SI, section IV):
\begin{align}
r_{if}^{NR} =& \frac{2\pi}{\hbar}g|W_{if}|^2 X_{if}(T), \label{eq:rate-nonrad-allterms}\\
X_{if}(T)=& \sum_{n,m}p_{in}\left|\mel{\phi_{fm}(\vR)}{Q-Q_a}{\phi_{in}(\vR)}\right|^2 \nonumber \\
& \hspace{1.5cm} \cross \delta(m\hbar\omega_f - n\hbar\omega_i+\Delta E_{if}), \label{eq:rate-nonrad-X} \\
W_{if} =& \mel{\psi_i(\mathbf{r},\vR)}{\frac{\partial H}{\partial Q}}{\psi_f(\mathbf{r},\vR)}\bigg{|}_{\vR=\vR_a},
\end{align}
where $r_{if}^{NR}$ is naturally separated into an electronic term $W_{if}$ and a phonon term $X_{if}(T)$ with temperature dependence from thermal population ($p_{in}$). Here $\Delta E_{if}$ is the zero phonon line energy (ZPL), which can be measured experimentally by photoluminescence. We implemented the non-radiative recombination rates as postprocessing codes of Quantum ESPRESSO~\cite{QE1}.

%\section{Results and Discussions}
\subsection{Defect-band versus defect-defect transitions in $h$-BN}


A single defect may introduce several energy levels within the band gap of the host material. This yields the possibility for transitions to occur between defect states (``defect-defect'' transition), as well as from a defect state to a band edge (``defect-band'' transition). The computed non-radiative lifetimes and capture coefficients of the most probable defect-band transitions for hole captures in multiple defects in monolayer and bulk $h$-BN as well as bulk GaN are listed in Table~\ref{table:defect-band-lifetime} (where \XBVN\ (X=C, O, N) denotes X substitution of boron accompanied by a nitrogen vacancy). The capture coefficients are defined as a product of recombination rates $r_{ij}$ with surface area or volume for 2D or 3D systems, respectively. The corresponding defect formation energies and configuration coordinate diagrams are presented in the SI, Figure S1.
We find that all defect-band transitions in monolayer $h$-BN have very small rates (the corresponding lifetime exceeds milliseconds). This is in contrast to typical 3D bulk defects in other materials, such as GaN-\CN\ or GaN-$(\mathrm{Zn_{Ga}V_N})$, where the non-radiative lifetime is at the picosecond level with a similar defect concentration to $h$-BN.~\cite{shi2012ab,alkauskas2014first2,Reshchikov2014}
%\textcolor{red}{(YP: similar concentration between 6x6 (72 atom)hBN and 2x2x2 (8x8?) GaN?)}

\begin{table}
\centering
    \caption{Non-radiative lifetimes and capture coefficients of defects in $h$-BN and GaN through defect-band recombination(only for the hole capture processes $A^{-1}+h^{+}\rightarrow A^{0}$).  For comparison a dominant defect-defect recombination at \NBVN\ in monolayer $h$-BN is also listed. The capture coefficients $C_p$ (with a unit of cm$^2/$s for 2D and cm$^3/$s for 3D systems) and lifetimes are reported at $T=300$ K. Lifetimes are defined as the inverse of rates $\uptau^{NR}=1/r_{if}^{NR}$ and computed in $6\times 6$ $h$-BN supercell or $2\times 2\times 2$ GaN supercell.
    %Capture coefficients $C_p$ are defined as $C_p^{2D} = g\,A \, r_{if}$ with unit of cm$^2$/s for 2D systems and $C_p^{3D} = g\,V \, r_{if}$ with unit  cm$^3$/s for 3D systems, where A is the surface, V is the volume, and $g$ is the number of degenerate states per defect.
   \label{table:defect-band-lifetime}}
  \begin{threeparttable}
    \begin{tabular}{cccccc}
      \hline
        System & \thead{ZPL \\ (eV)} & \thead{$\Delta Q$ \\ (amu$^{1/2}$\AA) }  &
        \thead{$\hbar\omega_f$ \\ (meV) } &
        \thead{$C_p$ \\ (cm$^n$/s) } & $\uptau^{NR}$ \\
      \hline
      ML \CBVN (2D) & 5.78 & 0.58 & 86 & $10^{-32}$ & $>1$ ms \\
      ML \OBOBVN (2D) & 4.26 & 0.84 & 85 &
      $10^{-29}$ & $>1$ ms \\
      ML \NBVN (2D) & 5.46 & 0.51 & 95 & $10^{-33}$ & $>1$ ms \\
      ML \CN (2D) & 3.87 & 0.35 & 150 & $10^{-16}$ & $>1$ ms \\
      Bulk \CN (3D) & 2.69 & 0.35 & 149 & $10^{-16}$ & 6.6 $\mu$s \\
      GaN-\CN (3D) & 1.00 & 1.39 & 39 & $10^{-9}$* & 0.29 ps \\
      \hline
      ML \NBVN (2D) & 2.04 & 0.53 & 100 & $10^{-4}$ & 102 ps \\
      (defect-defect) &&&&&\\
      \hline
    \end{tabular}
    \begin{tablenotes}
      \item * $7\times10^{-10}$ in Ref.~\cite{alkauskas2014first2}.%\textcolor{red}{YP: why do we have 5 times difference from theirs? }
    \end{tablenotes}
\end{threeparttable}
\end{table}

One key reason that non-radiative defect-band recombination in monolayer $h$-BN is typically slow, is due to large energy differences between defect states and band edges ($\sim4-6$ eV) as the ZPL shown in Table~\ref{table:defect-band-lifetime}.~\cite{wu2017first,smart2018fundamental,rasmussen2016efficient} Nonetheless, other factors such as the effective phonon frequencies $\hbar\omega_f$ and the change of nuclear positions $\Delta Q$ can also affect the rates, as discussed later.
For example, comparing monolayer BN-\CN\ with bulk BN-\CN, only the ZPL changes significantly (over 1 eV) and other parameters ($\hbar\omega_f$ and $\Delta Q$) retain nearly constants, which ultimately leads to a two order of magnitude difference in their capture rates.
Physically, the phonon-assisted non-radiative rate is dominated by a charge transfer process between initial and final state potential energy surfaces, and can be approximated by a classical Marcus' theory picture (see Figure \ref{fig:toc1}). Given the form of the energy barrier for charge transfer,~\cite{brawand2015surface} a large energy difference between the two states (ZPL) results in an exponential drop in the transfer rate (although exceptions can be found~\cite{nan2009nuclear,jortner1976temperature,passler1982nonradiative,shi2015comparative}). Therefore, in monolayer $h$-BN, the large ZPLs of defect-band transitions result in extremely slow non-radiative recombination processes (over milliseconds).
On the other hand, several defects have allowed defect-defect transitions with viable non-radiative rates due to smaller energy differences, e.g. the \NBVN\ (nitrogen substitution of boron accompanied by a nitrogen vacancy) defect-defect transition in monolayer $h$-BN (Table \ref{table:defect-defect-NBVN}).

Hence, the remaining discussions are focused on defect-defect transitions in monolayer $h$-BN. Defect-defect non-radiative recombination is performed for neutral excited and ground state with constrained occupation number. The equilibrium geometry, ZPL and vibrational frequency can be also obtained at DFT with constrained occupation. More computational details for defect-defect nonradiative recombination can be found in SI, section III, Figure S2 and Table S4. %\textcolor{red}{end of defect-band section. probably dont want to completely remove it but perhaps integrate main points into other sections $\sim$471 words}

\subsection{Competition between radiative and non-radiative recombination lifetime at $h$-BN defects}

\begin{figure}%[b]
    \centering
  %\includegraphics[width=0.99\linewidth]{3-2d/figures-nonrad/tfig2.png}%bandstructure-NBNV-gs-sym.png}
  \includegraphics[width=0.6\linewidth]{3-2d/figures-nonrad/bandstructure-NBNV-gs-sym-2.png}
  \caption{Defect levels and possible defect-defect transitions of \NBVN\ in monolayer $h$-BN. Both up and down spin channels of the $\mathrm{2B_1}/\mathrm{1B_1}$ transitions are marked in red as they are optically allowed with light polarized along defect $C_{2}$ symmetry axis. The exact radiative  ($\uptau_R$) and non-radiative ($\uptau_{NR}$) lifetimes are given for the spin up transition with a 6$\times$6 supercell. The remaining transitions in gray are all  optically forbidden and have very long radiative and non-radiative recombination lifetimes (exceeding 1 ms).
  %\textcolor{red}{(YP:check the description here)}
  %\textcolor{red}{(YP: forbidden for both radiative and non-radiative or only non-radiative? this needs to be clear)}
  }
  \label{fig:NBVN-compare-R-NR-all}
\end{figure}


Considering a typical point defect such as \NBVN\, which has been proposed as a promising defect for SPE,~\cite{exarhos2019magnetic,tran2016quantum,abdi2018color}
we find it introduces several isolated energy levels that lead to multiple possible radiative and non-radiative defect-defect recombination pathways (as shown in Figure \ref{fig:NBVN-compare-R-NR-all}). However, we found only
the transition between \Bandname{1B1U} and \Bandname{2B1U} (HOMO-LUMO transition for the majority spin channel)
has a viably short radiative lifetime and non-radiative lifetime. All other processes have a non-radiative lifetime longer than ms, much slower than this transition which is at a picosecond level.


\begin{table}[htbp]
  \centering
    \caption{Properties of defect-defect non-radiative recombination of the \NBVN\ defect in monolayer $h$-BN. Non-radiative lifetimes are computed with a 6$\times$6 supercell at 300 K and $S_{f}$ denotes the ground-state Huang-Rhys factor.}
    %\textcolor{red}{(YP: add capture coefficient here and remove $h\omega_i$)}
    \label{table:defect-defect-NBVN}
  \begin{threeparttable}
    \begin{tabular}{ccccccccc}
      \hline
        Transition & \thead{ZPL \\ (eV)} &
        \thead{$\hbar\omega_f$\\ (meV)} &
        \thead{$k$*} &
        \thead{$S_f$} &
        \thead{$X_{if}$}  & % changed to 300K \thead{$X_{if}$ \\ (0K) }  &
        \thead{$W_{if}$} &
        \thead{$C_{p}$ \\ (cm$^2$/s)} &
        \thead{$\uptau^{NR}$ \\ (ps)}
        \\
      \hline
        \Bandname{2B1U} / \Bandname{1B1U} & 2.04 & 100 & 20 &  5.3
        & 1.3 & 0.38 & $10^{-4}$ & 102 \\
        \Bandname{1B1D} / \Bandname{1A1D} & 1.33 &  58 & 23 & 16.6
        & $10^{5}$ & $10^{-7}$ & $10^{-11}$ & $>10^{9}$ \\
        \Bandname{2B1D} / \Bandname{1A1D} & 2.94 &  65 & 46 &  7.8
        & $10^{-4}$ & $10^{-6}$ & $10^{-19}$ & $>10^{9}$ \\
        \Bandname{2B1D} / \Bandname{1B1D} & 1.61 &  57 & 28 &  3.2
        & $10^{-13}$ & 0.03 & $10^{-19}$ & $>10^{9}$ \\
      \hline
    \end{tabular}
    \begin{tablenotes}
    \item * $k=\Delta E_{if}/\hbar\omega_f$
    \end{tablenotes}
\end{threeparttable}
\end{table}


The non-radiative transition rate is determined by multiple factors based on Eq.~\ref{eq:rate-nonrad-allterms}.
The first factor is the phonon term $X_{if}$. As the ZPL for all defect-defect transitions are relatively small (less than 3 eV, unlike defect-band transitions), we analyze the subtle difference causing variation of $X_{if}$ among different transitions, based on the relation:
$X_{if}\propto e^{-S}\frac{S^k}{k!}$ where $k\approx \Delta E_{if}/\hbar\omega_f$ and $S$ is the HR factor.~\cite{alkauskas2014first2}. Specifically ($k>S$ for all defect-defect transitions we study here), a high $S$ implies a large electron-phonon coupling and generally will increase the phonon contribution $X_{if}$.
For example, the HR factor for the \Bandname{1B1D}/\Bandname{1A1D} transition (16.6) is several times larger than other transitions in Table \ref{table:defect-defect-NBVN} and therefore yields the largest  $X_{if}$ of $10^5$ at 300 K.
On the other hand, a high value of $k$ means a large energy difference (ZPL) relative to the phonon frequency and will reduce phonon contribution $X_{if}$, similar to earlier discussions on defect-band transitions.
The second factor is the electronic term $W_{if}$, which is proportional to the overlap between electronic wavefunctions $\braket{\psi_i}{\psi_f}$.
Ultimately, only the \Bandname{2B1U}/\Bandname{1B1U} transition has a reasonably large $X_{if}$ and the largest $W_{if}$, which leads to a  viable non-radiative recombination process with a lifetime of 102 ps at 300 K.


The radiative process is more straightforward to interpret as it is directly related to the symmetries of wavefunctions via the dipole transition matrix elements $\mel{\psi_i}{\textbf{r}}{\psi_f}$.
Computational details can be found in section VI and Table S3 in the SI. The corresponding transition section rules for radiative recombination of \NBVN\ defect in $h$-BN are listed in Table S5.
Both the \Bandname{2B1U}/\Bandname{1B1U} transition and the \Bandname{2B1D}/\Bandname{1B1D} transition are symmetry allowed,~\cite{abdi2018color} resulting in short radiative lifetimes of 1.4 ps and 2.5 ps, respectively. This lifetime can be considered to be a lower-bound compared to that of experimental results, because a much higher defect concentration is adopted in practical calculations (1 defect in a 72-atom supercell, i.e. 1 defect per 2 nm$^2$, compared to order of one SPE per $\mu$m$^2$ in experiments~\cite{tran2016quantum}) and both radiative and non-radiative lifetimes will increase linearly with decreasing defect concentrations or increasing supercell size (see Ref.~\cite{alkauskas2016tutorial} and section V Table S1 and S2 in the SI).
At the low concentration limit, we can consider the defect acts as an isolated molecule in the 2D plane~\cite{wu2019dimensionality},
which gives an upper-bound of the actual lifetime, i.e. 40 ns for \Bandname{2B1U}/\Bandname{1B1U} radiative lifetime at \NBVN.
This is in good agreement with the experimental radiative lifetime of monolayer $h$-BN SPE, which are measured to be on the order of ns.\cite{tran2016quantum,bourrellier2016bright,grosso2017tunable,exarhos2017optical} We note that different from the recombination rates, the capture coefficient is generally constant as a function of defect concentration or supercell sizes (see Table S1 and S2 in the SI).

The quantum yield of a SPE (excluding substrate effects) is defined as $\gamma_{if}=r^\mathrm{R}_{if}/(r^\mathrm{R}_{if}+r^{\mathrm{NR}}_{if})$.~\cite{Book_Optical_Pankove,chevallier2017method,hamanaka2014enhancement} By comparing the radiative lifetimes with the non-radiative ones shown in Table \ref{table:defect-defect-NBVN} for the \NBVN\ defect, we have $\gamma>98\%$ which shows it has the potential to be a highly efficient quantum emitter. In practice, several other external effects can cause the quantum yield to be substantially lower. In particular, substrate recombination~\cite{rosenberger2018electrical}, photobleaching~\cite{martinez2016efficient}, and strain (discussed in the next section) are known to play the role of limiting the quantum yield of defect SPE.


\subsection{Effect of strain on the non-radiative lifetime of defects in $h$-BN}

\begin{table}[htbp]
    \caption{Properties of $1\mathrm{B}_1\uparrow-2\mathrm{B}_1\uparrow$ defect-defect state transition for \NBVN\ defect in monolayer $h$-BN under strain. Strain directions are shown in Figure \ref{fig:strain-axis}. $X_{if}$ and lifetime are reported at 300 K.    \label{table:defect-defect-NBVN-strain}}
  \begin{threeparttable}
    \begin{tabular}{cccccccc}
      \hline
        Strain &
        \thead{ZPL \\ (eV)} &
        \thead{$\Delta Q$ \\ (\AA) } &
        \thead{$\hbar\omega_f$ \\ (meV)} &
        \thead{$S_f$} &
        \thead{$W_{if}$} & %\\ (eV/amu$^{1/2}$\AA) } &
        \thead{$X_{if}$} &
        \thead{$\uptau^{NR}$\\ (ps)}
        \\
      \hline
No strain &2.04&0.666&100&5.33&0.38&1.26&102 \\
Biaxial $-1\%$&2.08&0.613&105&4.69&0.39&0.28&429 \\
Biaxial $1\%$&2.01&0.732&96&6.18&0.36&9.20&16 \\
Uniaxial $\parallel -1\%$&2.02&0.637&102&4.96&0.39&0.95&127 \\
Uniaxial $\parallel 1\%$&2.07&0.703&98&5.80&0.36&1.98&70  \\
Uniaxial $\bot -1\%$&2.10&0.642&103&5.05&0.38&0.38&336 \\
Uniaxial $\bot 1\%$&1.98&0.697&98&5.69&0.37&5.39&25 \\
      \hline
    \end{tabular}
    %\begin{tablenotes}
    %\end{tablenotes}
\end{threeparttable}
\end{table}

\begin{figure}%[p]
    \centering
  \includegraphics[width=0.5\linewidth]{3-2d/figures-nonrad/tfig3.png}%wave-spin1-band143-144-view2-strain.png}
  \caption{Illustration of the directions of uniaxial strain based on the $C_{2v}$ symmetry of \NBVN\ in $h$-BN. Uniaxial strains applied parallel ($||$ blue arrows) or perpendicular ($\perp$ red arrows) to the $C_{2}$ axis are considered. The optimized atomic structure of \NBVN\ defect is also shown. The green balls denote B atoms and the grey balls denote N atoms.}
  \label{fig:strain-axis}
\end{figure}

\begin{figure}
    \centering
  \includegraphics[width=0.98\linewidth]{3-2d/figures-nonrad/strain.png}
  \caption{Strain induced properties related to non-radiative recombination lifetime of the $1\mathrm{B}_1-2\mathrm{B}_1$ defect-defect transition of \NBVN\ in monolayer $h$-BN. Strain directions are shown in Figure \ref{fig:strain-axis}.}
  \label{fig:NBVN-strain}
\end{figure}


In this work, we discuss the impact of strain on non-radiative recombination (and leave other external effects for future study) with \NBVN\ as an example.
Presumably, strain will change the radiative lifetime little (orbital overlaps between initial and final states are largely preserved) compared to that of the non-radiative lifetime which can be strongly affected by changes in local structures.
As shown in Figure \ref{fig:strain-axis}, strain may be applied along the $C_{2}$ symmetry axis (denoted as $||$ strain) or  orthogonal to the symmetry axis (denoted as $\perp$ strain). We consider effects of strain along both directions as well as the combinatory effects of biaxial strain for the \Bandname{2B1U}/\Bandname{1B1U} transition (shown in Table \ref{table:defect-defect-NBVN-strain}).

As discussed earlier, the non-radiative recombination rate is composed of an electronic term $W_{if}$ and a phonon term $X_{if}$. Because $W_{if}$ is proportional to the wavefunction overlap, the change in $W_{if}$ due to strain is found to be negligible, as shown in Figure \ref{fig:NBVN-strain}a.
However, there are significant changes of the phonon term $X_{if}$ due to strain.
We note that compressive strain indicates lattice shrinking ($-$); while tensile strain indicates lattice stretching ($+$) and induces opposite changes on non-radiative rates from the former. Therefore, we only discuss compressive strain here. First, compressive strain decreased interatomic distances, which in turn decreased the change in the atomic coordination between initial and final states ($\Delta Q$ in Table~\ref{table:defect-defect-NBVN-strain}). As such, under compressive strain, the HR factor $S=\omega_f\Delta Q^2/2\hbar$ decreased resulting in an exponential decrease of the phonon term $X_{if}$. Such trends occurred regardless of the direction of strain applied (i.e. $\perp$ or $||$ to the $C_2$ axis). Second, a change in the ZPL also occurred under strain~\cite{grosso2017tunable}. After the formation of the nitrogen vacancy, a weak B-B bond is formed perpendicular to the $C_2$ symmetric axis (see Figure \ref{fig:strain-axis}). When compressive strain is applied perpendicular to the $C_2$ axis ($\perp$ strain), the ZPL is increased, due to larger bonding-antibonding splitting of the B-B bond that shifts up the 2B$_1$ energy level (see Figure~\ref{fig:NBVN-compare-R-NR-all} for related wavefunctions and energy levels). As a result, for $\perp$ strain the change in ZPL and HR factor coincided and yielded an exponential decrease of $X_{if}$ under compressive strain (red curve Figure \ref{fig:NBVN-strain}b). In contrast, for $||$ strain, these changes counteracted each other resulting in a nearly constant value of $X_{if}$ (blue curve Figure \ref{fig:NBVN-strain}b). In addition, biaxial strain is a simple combinatory effect of $||$ and $\perp$ strain, mostly dominated by the trend of $\perp$ strain (black curve Figure \ref{fig:NBVN-strain}b).
All in all, the exponential change in $X_{if}$ for both biaxial and uniaxial $\perp$ strain resulted in exponential modification to the non-radiative lifetime of the defect, as black and red curves shown in Figure \ref{fig:NBVN-strain}c.
In particular, in the case of tensile biaxial strain, with $1\%$, the quantum yield decreased by $10\%$ due to an order of magnitude decrease in non-radiative lifetime. This highlights the significant impact strain can have on the efficiency of defect SPE.

\subsection{Conclusion}

In summary, in this work we compared the radiative and phonon-assisted non-radiative recombinations at defects in wide bandgap 2D materials, using monolayer $h$-BN as a prototypical example. We found the radiative recombination rates far surpass the non-radiative ones, highlighting the potential of point defects in wide bandgap 2D materials as single photon emitters. Defect-band non-radiative recombinations all have negligible rates possibly due to large energy differences between initial and final states, and only a small subset of defect-defect non-radiative transitions are possible. Transitions vary on several orders of magnitude due to wavefunction symmetry, HR factor, as well as zero-phonon line (ZPL). Finally, we show that compressive or tensile strain up to 1\% can alter the non-radiative lifetime by orders of magnitude.
Hence, strain largely impacts the quantum yield of single photon emitters and alters the photon energy of the emitter for use towards specific optoelectronic applications. Our study provides important insights on the critical factors of defects in 2D materials as single photon emitters for quantum information applications.

% shortcut for vacancy
\def\vac{\text{V}}
% shortcut for boron vacancy
\def\vb{\vac_\text{B}}
% shortcut for nitrogen vacancy
\def\vn{\vac_\text{N}}
% shortcut for nitrogen substitute boron nitrogen vacancy
\def\nbvn{\text{N}_\text{B}\vac_\text{N}}
% shortcut for GW@PBE0 ($\alpha=0.41$)
\def\gw{\text{G}_0\text{W}_0@\text{PBE0($\mathrm{\alpha}$)}}
% shortcut for GW+RPA@PBE0 ($\alpha=0.41$)
\def\gwrpa{\text{G}_0\text{W}_0+\text{RPA}@\text{PBE0($\mathrm{\alpha}$)}}
% shortcut for GW+BSE@PBE0 ($\alpha=0.41$)
\def\gwbse{\text{G}_0\text{W}_0+\text{BSE}@\text{PBE0($\mathrm{\alpha}$)}}
% shortcut for GW+RPA@PBE
\def\gwrpapbe{\text{G}_0\text{W}_0+\text{RPA}@\text{PBE}}
% shortcut for GW+BSE@PBE
\def\gwbsepbe{\text{G}_0\text{W}_0+\text{BSE}@\text{PBE}}
% shortcut for general X_VV defect
\newcommand{\vv}[1]{\text{#1}_{\vac\vac}}
% shortcut for Ti defect
\def\ti{\vv{Ti}}
% shortcut for Mo defect
\def\mo{\vv{Mo}}
% shortcut for singlet Si defect
\def\sis{\mathrm{Si_{VV}(S)}}
% shortcut for triplet Si defect
\def\sit{\mathrm{Si_{VV}(T)}}
% shortcut for general X_Y defect
\newcommand{\xy}[2]{\text{#1}_{\text{#2}}}
% shortcuts for ti states
\def\gs{\ket{\prescript{3}{0}{A''}}}    % triplet ground state (gs)
\def\es{\ket{\prescript{3}{1}{A''}}}    % triplet excited states (es)
\def\pjt{\ket{\prescript{3}{1}{A}}}     % triplet excited PJT state (pjt) -- this is C1
\def\sin{\ket{\prescript{1}{0}{A'}}}    % singlet state (sin)
% shortcuts for mo states
\def\mogs{\ket{\prescript{3}{0}{A}}}    % triplet ground state (gs)
\def\moes{\ket{\prescript{3}{1}{A}}}    % triplet excited states (es)
\def\mosin{\ket{\prescript{1}{0}{A}}}    % singlet state (sin)
% shortcut for NV center
\def\nv{\text{NV}}

\section{Spin Properties}

\subsection{Introduction}
Optically addressable defect-based qubits offer a distinct advantage in their ability to operate with high fidelity under room temperature conditions~\cite{koehl2011room,weber2010quantum}.
Despite tremendous progress made in years of research, systems which exist today remain inadequate for real-world applications. The identification of stable single photon emitters in 2D materials has opened up a new playground for novel quantum phenomena and quantum technology applications, with improved scalability in device fabrication and
a leverage in doping spatial control, qubit entanglement, and qubit tuning~\cite{liu20192d,aharonovich2017quantum}.
In particular, hexagonal boron nitride ($h$-BN) has demonstrated that it can host stable defect-based single photon emitters (SPEs)~\cite{mendelson2020strain,feldman2019phonon,yim2020polarization,mackoit2019carbon} and spin triplet defects~\cite{kianinia2020generation,turiansky2020spinning}.
However,
%whereas the hallmark nitrogen-vacancy ($\nv$) center in diamond has been investigated extensively~\cite{gali2019ab},
persistent challenges must be resolved before 2D quantum defects can become the most promising quantum information platform. These challenges include the undetermined chemical nature of
existing SPEs\cite{li2017nonmagnetic,yim2020polarization}, difficulties in controlled generation of desired spin defects, and scarcity of reliable theoretical methods which
can accurately predict critical physical parameters for defects in 2D materials due to their complex many-body interactions.

To circumvent these challenges, design of promising spin defects by high-integrity theoretical methods is urgently needed.
Introducing extrinsic defects can be unambiguously produced and controlled, which fundamentally solves the current issues of undetermined chemical nature of existing SPEs in 2D systems.
As highlighted by Ref.~\cite{weber2010quantum,ivady2018first}, promising spin qubit candidates should satisfy several essential criteria: deep defect levels, stable high spin states, large zero-field splitting, efficient radiative recombination, high intersystem crossing rates and long spin coherence and relaxation time.
Using these criteria for theoretical screening can effectively identify promising candidates but
requires theoretical development of first-principles methods, significantly beyond the static and mean-field level. For example, accurate defect charge transition levels in 2D materials necessitates careful treatment of defect charge corrections for removal of spurious charge interactions~\cite{komsa2014charged,komsa2018erratum,wang2015determination,wu2017first} and electron correlations for non-neutral excitation, e.g.\ from GW approximations~\cite{wu2017first,govoni2015large} or Koopmans-compliant hybrid functionals~\cite{smart2018fundamental,nguyen2018koopmans,weng2018wannier,miceli2018nonempirical}. Optical excitation and exciton radiative lifetime must account for defect-exciton interactions, e.g.\ by solving the Bethe-Salpeter equation, due to large exciton binding energies in 2D systems~\cite{refaely2018defect,gao2020radiative}.  Spin-phonon relaxation time calls for a general theoretical approach to treat complex symmetry and state degeneracy of defective systems, along the line of recent development based on
ab-initio density matrix approach~\cite{xu2020spin}.
Spin coherence time due to the nuclei spin and electron spin coupling
can be accurately predicted for defects in solids by combining first-principles and spin Hamiltonian approaches~\cite{seo2016quantum,ye2019spin}. In the end, nonradiative processes, such as phonon-assisted nonradiative recombination, have been recently computed with first-principles electron-phonon couplings for defects in $h$-BN~\cite{wu2019carrier}, and resulted in less competitive rates than corresponding radiative processes. However, the spin-orbit induced intersystem crossing as the key process for pure spin state initialization during qubit operation has not been investigated for spin defects in 2D materials from first-principles in-depth.

This work has developed a complete theoretical framework which enables the design of spin defects based on the critical physical parameters mentioned above and highlighted in Figure~\ref{fig:screen}a. We employed state-of-the-art first-principles methods, focusing on many-body interaction such as defect-exciton couplings and dynamical processes through radiative and nonradiative recombinations. We developed methodology to compute nonradiative intersystem crossing rates with explicit overlap of phonon wavefunctions beyond current implementations in the Huang-Rhys approximation\cite{thiering2017ab}. We showcase the discovery of transition metal complexes such as Ti and Mo with vacancy ($\ti$ and $\mo$) to be spin triplet defects in $h$-BN, and the discovery of $\vv{Si}$ to be a bright SPE in $h$-BN.
We predict $\ti$ and $\mo$ are stable triplet defects in $h$-BN (which is rare considering the only known such defect is $\vb^-$~\cite{gottscholl2020initialization}) with large zero-field splitting and spin-selective decay, which will set 2D quantum defects at a competitive stage with $\nv$ center in diamond for quantum technology applications.

%%%%%%%%%%%%%%%%%%%%%%%%
% Figure for screening
\begin{figure}[H]
    \centering
    \includegraphics[keepaspectratio=true,width=0.8\linewidth]{3-2d/figures-npj/tibn_screen.png}
    \caption{\textbf{Screening of spin defects in $h$-BN.} \textbf{a} Schematic of the screening criteria and workflow developed in this work, where we first search for defects with stable triplet ground state, followed by large zero-field splitting (ZFS), then ``bright'' optical transitions between defect states required for SPEs or qubit operation by photon, and at the end large intersystem crossing rate (ISC) critical for pure spin state initialization. \textbf{b} Divacancy site in $h$-BN corresponding to adjacent B and N vacancies (denoted by $\vb$ and $\vn$). \textbf{c} Top-view and \textbf{d} Side-view of a typical doping configuration when placed at the divacancy site, denoted by $\vv{X}$. Atoms are distinguished by color: grey=N, green=B, purple=Mo, blue=Ti, red=X (a generic dopant).}
    \label{fig:screen}
\end{figure}

%%%%%%%%%%%%%%%%%%%%%%%%%%%%%%%%%%%%%%%%%%%%%%%%%%%%%%%%%%%%%%%%%%%%%%%%%%%%%%%%%%%%%%%%%%%%%%%%%%%%%%%%%%%%%%%%%%

In the development of spin qubits in 3D systems (e.g.\ diamond, SiC, and AlN), defects beyond $sp$ dangling bonds from N or C have been explored. In particular, large metal ions plus anion vacancy in AlN and SiC were found to have potential as qubits due to triplet ground states and large zero-field splitting (ZFS)~\cite{seo2017designing}.
Similar defects may be explored in 2D materials~\cite{turiansky2019dangling}, such as the systems shown in Figure~\ref{fig:screen}b-d.
This opens up the possibility of overcoming the current limitations of uncontrolled and undetermined chemical nature of 2D defects, and unsatisfactory spin dependent properties of existing defects.
In the following, we will start the computational screening of spin defects with static properties of the ground state (spin state, defect formation energy and ZFS) and the excited state (optical spectra), then we will discuss dynamical properties including radiative and nonradiative (phonon-assisted spin conserving and spin flip) processes, as the flow chart shown in Figure~\ref{fig:screen}a. We will summarize the complete defect discovery procedure and discuss the outlook at the end.

\subsection{Screening Triplet Spin Defects in $h$-BN}
To identify stable qubits in $h$-BN, we start from screening neutral dopant-vacancy defects for a triplet ground state based on total energy calculations of different spin states at both semi-local PBE (Perdew–Burke-Ernzerhof) and hybrid functional levels. We considered the dopant substitution at a divacancy site in $h$-BN (Figure~\ref{fig:screen}b) for four different elemental groups. The results of this procedure are summarized in Supplementary Table 1 and Note 1. With additional supercell tests in Supplementary Table 2, our screening process finally yielded that only $\mo$ and $\ti$ have a stable triplet ground state.
%\textbf{Thermodynamic Charge Formation Energy}
We further confirmed the thermodynamic charge stability of these defect candidates via calculations of defect formation energy and charge transition levels.
As shown in Supplementary Figure 1,
both $\ti$ and $\mo$ defects have a stable neutral ($q=0$) region for a large range of Fermi level ($\varepsilon_F$), from 2.2 eV to 5.6 eV for $\mo$ and from 2.9 eV to 6.1 eV for $\ti$. These neutral states will be stable in intrinsic $h$-BN systems or with weak p-type or n-type doping (see Supplementary Note 2).

With a confirmed triplet ground state, we next computed the two defects' zero-field splitting. A large ZFS is necessary to isolate the $m_s = \pm 1$ and $m_s = 0$  levels even at zero magnetic field allowing for controllable preparation of the spin qubit.
Here we computed the contribution of spin-spin interaction to ZFS by implementing the plane-wave based method developed by Rayson et al.\ (see Methods section for details of implementation and benchmark on $\nv$ center in diamond)~\cite{rayson2008first}. Meanwhile, the spin-orbit contribution to ZFS was computed with the ORCA code.
We find that both defects have sizable ZFS including both spin-spin and spin-orbit contributions (axial $D$ parameter) of 19.4 GHz for $\ti$ and 5.5 GHz for $\mo$, highlighting the potential for the basis of a spin qubit with optically detected magnetic resonance (ODMR) (see Supplementary Note 3 and Figure 2).
They are notably larger than previously reported values for ZFS of other known spin defect in solids~\cite{seo2017designing}, although at a reasonable range considering large ZFS values (up to 1000 GHz) in transition-metal complex molecules~\cite{zolnhofer2020electronic}.

\subsection{Screening SPE Defects in $h$-BN}
To identify single photon emitters in $h$-BN, we considered a separate screening process of these dopant-vacancy defects, targeting those with desirable optical properties. Namely, an SPE efficiently emits a single photon at a time at room temperature.
Physically this corresponds to identifying defects which have a single bright intra-defect transition with high quantum efficiency (i.e.\ much faster radiative rates than nonradiative ones), for example current SPEs in $h$-BN have radiative lifetimes $\sim$1-10 ns and quantum efficiency over 50$\%$.~\cite{tran2016robust,schell2017coupling}

Using these criteria we screened the defects by computing their optical transitions and radiative lifetime at Random Phase Approximation (RPA) (see Supplementary Note 4, Figure 3 and Table 3). This offers a cost-efficient first-pass to identify defects with bright transition and short radiative lifetime as potential candidates for SPEs.
From this procedure, we found that $\vv{C}$(T), $\sis$, $\sit$, $\vv{S}$(S), $\vv{Ge}$(S) and $\vv{Sn}$(S) could be promising SPE defects ((T) denotes triplet; (S) denotes singlet), with a bright intra-defect transition and radiative lifetimes on the order of 10 ns, at the same order of magnitude of the SPEs' lifetime observed experimentally.
\cite{schell2017coupling}
Among these, $\sis$ has the shortest radiative lifetime, and in addition, Si has recently been experimentally detected in $h$-BN with samples grown in chemical vapor deposition (the ground state of $\vv{Si}$ is also singlet).\cite{ahmadpour2019substitutional}
Hence we will focus on $\vv{Si}$ as an SPE candidate in the following sections as we compute optical and electronic properties at higher level of theory from many-body perturbation theory including accurate electron correlation and electron-hole interactions.
Note that $\vv{C}$ (commonly denoted $\rm C_BV_N$) has also been suggested to be a SPE source in $h$-BN.~\cite{sajid2020vncb}


\subsection{Single-Particle Levels, Optical Spectra and Radiative Lifetime}

The single-particle energy levels of $\ti$, $\mo$ and $\vv{Si}$ are shown in Figure~\ref{fig:sg_ctl}.
These levels are computed by many-body perturbation theory ($\mathrm{G_0W_0}$) for accurate electron correlation,
with hybrid functional (PBE0($\alpha$), $\alpha=0.41$ based on the Koopmans' condition~\cite{smart2018fundamental}) as the starting point to address self-interaction errors for $3d$ transition metal defects.~\cite{fuchs2007quasiparticle,bechstedt2016many}
For example, we find that both the wavefunction distribution and ordering of defect states can differ between PBE and PBE0($\alpha$) (see Supplementary Figure 4-6). The convergence test of $\mathrm{G_0W_0}$ can been found in Supplementary Figure 7, Note 5, and Table 4.
Importantly, the single particle levels in Figure~\ref{fig:sg_ctl} show there are well localized occupied and unoccupied defect states in the $h$-BN band gap, which yield the potential for intra-defect transitions.

\begin{figure}[H]
    \centering
    \includegraphics[keepaspectratio=true,width=1.0\linewidth]{3-2d/figures-npj/singlet_particle_diagrams.png}
    \caption{
    \textbf{Single-particle levels and wavefunctions.} Single-particle defect levels (horizontal black lines) of the (a) $\ti$, (b) $\mo$, and (c) $\vv{Si}$ defects in $h$-BN, calculated at $\mathrm{G_0W_0}$ with PBE0($\alpha$) starting wavefunctions. The blue/red area corresponds to the valence/conduction band of $h$-BN.
    States are labelled by their ordering and representation within the $C_S$ group with up/down arrows indicating spin and filled/unfilled arrows indicating occupation.
    A red arrow is drawn to denote the intra-defect optical transition found in Figure~\ref{fig:bse}.
    Defect wavefunctions at PBE0($\alpha$) are shown with an isosurface value 10\% of the maximum. The blue and yellow color denotes different signs of wavefunctions.
}
    \label{fig:sg_ctl}
\end{figure}

Obtaining reliable optical properties of these two-dimensional materials necessitates solving the Bethe-Salpeter equation (BSE)
%, \textcolor{red}{make a note here: see numerical convergence in SI Figure S8-S10}
to include excitonic effects due to their strong defect-exciton coupling, which is not included in RPA calculations (see comparison in Supplementary Figure 8 and Table 5).~\cite{ping20132electronic,rocca2012solution,ping2012ab,ping2013}
The BSE optical spectra are shown for each defect in Figure~\ref{fig:bse}a-c (the related convergence tests can be found in Supplementary Figure 9-10).
In each case we find an allowed intra-defect optical transition (corresponding to the lowest energy peak as labeled in Figure~\ref{fig:bse}a-c, and red arrows in Figure~\ref{fig:sg_ctl}).
From the optical spectra we can compute their radiative lifetimes as detailed in the Methods section on Radiative Recombination.
We find the transition metal defects' radiative lifetimes (tabulated in Table~\ref{table:rad}) are long, exceeding $\mu$s.
% Therefore, they are not good candidates for SPEs, although they still have potentials for spin qubits with optically-allowed intra-defect transitions.
Therefore, they are not good candidates for SPE. In addition, while they still are potential spin qubits with optically-allowed intra-defect transitions, optical readout of these defects will be difficult.
Referring to Table~\ref{table:rad} and the expression of radiative lifetime in Eq.~\ref{eq:rate-radiative-0D} we can see this is due to their low excitation energies ($E_0$, in the infrared region) and small dipole moment strength ($\mu^2_{e-h}$). The latter is related to the tight localization of the excitonic wavefunction for $\ti$ and $\mo$ (shown in Figure~\ref{fig:bse}d-f), as strong localization of the defect-bound exciton leads to weaker oscillator strength.~\cite{hours2005exciton}

\begin{figure}[ht!]
    \centering
    \includegraphics[keepaspectratio=true,width=1.0\linewidth]{3-2d/figures-npj/bse_ti_mo_si.png}
    \caption{
    \textbf{BSE optical spectra and exciton wavefunctions.} Absorption spectra  of the (a) $\ti$, (b) $\mo$, and (c) $\vv{Si}$ defects in $h$-BN at the level of $\gwbse$. The left and right panels provide absorption spectra for two different energy ranges, where the former is magnified by a factor of 40 for $\ti$ and $\mo$ and a factor of 5 for $\vv{Si}$ for increasing visibility. A spectral broadening of 0.02 eV is applied. The exciton wavefunctions of (d) $\ti$, (e) $\mo$ and (f) $\vv{Si}$ are shown on the right for the first peak.
    }
    \label{fig:bse}
\end{figure}

On the other hand, the optical properties of the $\vv{Si}$ defect are quite promising for SPEs, as Figure~\ref{fig:bse}c shows it has a very bright optical transition in the ultraviolet region. As a consequence, we find that the radiative lifetime (Table~\ref{table:rad}) for $\vv{Si}$ is 22.8 ns at $\rm G_0W_0+BSE@PBE0(\alpha)$.
We note that although the lifetime of $\vv{Si}$ at the level of BSE is similar to that obtained at RPA (13.7 ns),
the optical properties of 2D defects at RPA are still unreliable, due to the lack of excitonic effects. For example, the excitation energy ($E_0$) can deviate by $\sim$1 eV and oscillator strengths ($\mu_{e-h}^2$) can deviate by an order of magnitude (more details can be found in Supplementary Table 5).
Above all, the radiative lifetime of $\vv{Si}$ is comparable to experimentally observed SPE defects in $h$-BN,\cite{schell2017coupling} showing that $\vv{Si}$ is a strong SPE defect candidate in $h$-BN.



\begin{table}[H]
    \footnotesize
    \centering

\begin{tabular}{c c c c c}
    \hline \hline
        Defect    &   $E_{0}$ (eV)    &   $\mu^2_{e-h}$ (bohr$^2$)   &    $\tau_R$ (ns)   &   $E_{b}$ (eV)\\
    \hline
         $\ti$    &       0.556       &        $2.81*10^{-2}$        &    $1.95*10^{5}$   &   4.018\\
         $\mo$    &       1.079       &        $2.29*10^{-2}$        &    $3.26*10^{4}$   &   3.965\\
         $\vv{Si}$   &       4.036       &        $6.28*10^{-1}$        &    $22.8$          &   2.189\\
         %$\sit$   &       2.883       &        $5.83*10^{-1}$        &    $67.3$          &   3.099\\
        $\nbvn$   &       2.408       &        $1.87$                &    $35.9$          &   2.428\\
    \hline \hline
\end{tabular}

    \caption{
Optical excitation energy ($E_{0}$), modulus square of the transition dipole moment ($\mu^2_{e-h}$), radiative lifetime ($\tau_R$) and exciton binding energy ($E_b$) of several defects in $h$-BN at the level of theory of $\gwbse$. The corresponding excitation transitions are $1a'_\uparrow \rightarrow 2a'_\uparrow$ for the $\ti$ defect, $1a''_\uparrow \rightarrow 3a'_\uparrow$ for the $\mo$ defect and $1a'_\uparrow \rightarrow 2a'_\uparrow$ for the $\vv{Si}$ defect. For comparison, we include the results of $\nbvn$ (in-plane structure) from Ref.~\cite{wu2019carrier}.
    }
    \label{table:rad}
\end{table}
%%%%%%%%%%


%%%%%%%%%%%%%%%%%%%%%%%%%%%%%%%%%%%%%%%%%%%%%%%%%%%%%%%%%%%%%%%%%%%%%%%%%%%%%%%%%%%%%%%%%%%%%%%%%%%%%%%%%%%%%%%%%%

\subsection{Multiplet Structure and Excited-State Dynamics}
Finally, we discuss the excited-state dynamics of the spin qubit candidates $\ti$ and $\mo$ defects in $h$-BN, where the possibility of intersystem crossing is crucial.
This can allow for polarization of the system to a particular spin state by optical pumping, required for realistic spin qubit operation.

\begin{figure}[H]
    \centering
    \includegraphics[keepaspectratio=true,width=1\linewidth]{3-2d/figures-npj/tibn_multi.png}
    \caption{
         \textbf{Multiplet structure of triplet defects.}
        Multiplet structure and related radiative and nonradiative recombination rates of the (\textbf{a}) $\ti$ defect and the (\textbf{b}) $\mo$ defect in $h$-BN, computed at $T=10K$.
    The radiative process is shown in red with zero-phonon line (ZPL) and radiative lifetime ($\tau_R$); the ground state nonradiative recombination ($\tau_{NR}$) is denoted with a dashed line in dark blue; and finally the intersystem crossing (ISC) to the singlet state from the triplet excited state is shown in light blue. The zero-field splitting ($D$) is denoted by the orange line. For the $\ti$ defect, the pseudo Jahn-Teller (PJT) process is shown with a solid line in dark blue.
   }
    \label{fig:multi}
\end{figure}

An overview of the multiplet structure and excited-state dynamics is given in Figure~\ref{fig:multi} for the $\ti$ and $\mo$ defects.
For both defects, the system will begin from a spin-conserved optical excitation from the triplet ground state to the triplet excited state, where next the excited state relaxation and recombination can go through several pathways. The excited state can directly return to the ground state via a radiative (red lines) or nonradiative process (dashed dark blue lines). For the $\ti$ defect shown in Figure~\ref{fig:multi}a, we find the system may relax to another excited state with lower symmetry through a pseudo Jahn-Teller distortion (PJT; solid dark blue lines), and ultimately recombine back to the ground state nonradiatively. Most importantly, a third pathway is to nonradiatively relax to an intermediate singlet state through a spin-flip intersystem crossing (ISC), and then again recombine back to the ground state (dashed light-blue lines).
This ISC pathway is critical for the preparation of a pure spin state, similar to $\nv$ center in diamond.
Below, we will discuss our results for the lifetime of each radiative or nonradiative process, in order to determine the most competitive pathway under the operation condition.

\subsection{Direct Radiative and Nonradiative Recombination}

First, we will consider the direct ground state recombination processes.
Figure~\ref{fig:config} shows the configuration diagram of the $\ti$ and $\mo$ defects.
The zero-phonon line (ZPL) for direct recombination can be accurately computed by subtracting its vertical excitation energy computed at BSE (0.56 eV for $\ti$ and 1.08 eV for $\mo$) by its relaxation energy in the excited state (i.e.\ Franck-Condon shift~\cite{van2004first}, $\Delta E_{FC}$ in Figure~\ref{fig:config}). This yields ZPLs of 0.53 eV and 0.91 eV for $\ti$ and $\mo$, respectively. Although this method accurately includes both many-body effects and Franck-Condon shifts, it is difficult to evaluate ZPLs for the triplet to singlet-state transition currently.
Therefore, we compared with the ZPLs computed by constrained occupation DFT (CDFT) method at PBE. This yields ZPLs of 0.49 eV and 0.92 eV for $\ti$ and $\mo$, respectively, which are in great agreement with the ones obtained from BSE excitation energies subtracting $\Delta E_{FC}$ above.
Lastly, the radiative lifetimes for these transitions are presented in Table~\ref{table:rad} as discussed in the earlier section, which shows $\ti$ and $\mo$ have radiative lifetimes of 195 $\mu$s and 33 $\mu$s, respectively (red lines in Figure~\ref{fig:multi}).

In terms of nonradiative properties, the small Huang-Rhys ($S_f$) for the $\es$ to $\gs$ transition of the $\ti$ defect (0.91) implies extremely small electron-phonon coupling and potentially an even slower nonradiative process.
On the other hand, $S_f$ for the $\moes$ to $\mogs$ transition of the $\mo$ defect is modest (3.53) and may indicate a possible nonradiative decay.
%%%%%%%%%%%%%%%%%%%%%%
%%%%%%%%%%%%%%%%%%%%%%
Following the formalism presented in Ref.~\cite{wu2019carrier}, we computed the nonradiative lifetime of the ground state direct recombination ($T=10$ K is chosen to compare with measurement at cryogenic temperatures~\cite{goldman2015phonon}). %for the $\es$ triplet excited state to the $\gs$ triplet ground state.
Consistent with their Huang-Rhys factors, the nonradiative lifetime of $\ti$ is found to be 10 s, while the nonradiative lifetime of the $\mo$ defect is found to be 0.02 $\mu$s.
The former lifetime is indicative of a forbidden transition; however, the $\ti$ defect also possesses a pseudo Jahn-Teller (PJT) effect in the triplet excited state (red curve in Figure~\ref{fig:config}a).
Due to the PJT effect, the excited state ($C_S$, $\es$) can relax to lower symmetry ($C_1$, $\pjt$) with a nonradiative lifetime of 394 ps (solid dark blue line in Figure~\ref{fig:multi}a, additional details see Supplementary Note 9 and Figure 11). Afterward, nonradiative decay from $\pjt$ to the ground state ($\gs$) (dashed dark blue line in Figure~\ref{fig:multi}a) exhibits a lifetime of 0.044 ps due to a large Huang-Rhys factor (14.95).



\begin{figure}[H]
    \centering
    \includegraphics[keepaspectratio=true,width=0.8\linewidth]{3-2d/figures-npj/tibn_configuration.png}
    \caption{
            \textbf{Configuration coordinate diagrams.}
    Configuration diagram of the (\textbf{a}) $\ti$ defect and (\textbf{b}) $\mo$ defect in $h$-BN. The potential energy surfaces for each state are as follows: the triplet ground state in black, triplet excited state in blue, and for the $\ti$ defect the pseudo Jahn-Teller triplet excited state in red. The zero-phonon lines (ZPL) are given as the energetic separation between the minima of the respective potential energy surfaces, along with the corresponding Huang-Rhys factors ($S$). The dashed black line represents the vertical excitation energy between triplet ground and excited states, and $\Delta E_{FC}$ represents relaxation energy to equilibrium geometry at the excited state.
}
    \label{fig:config}
\end{figure}


%%%%%%%%%%%%%%%%%%%%%%%
\subsection{Spin-Orbit Coupling and Nonradiative Intersystem Crossing Rate}
%($\es \rightarrow \sin$)}
Lastly, we considered the possibility of an ISC between the triplet excited state and the singlet ground state for each defect, which is critical for spin qubit application.
In order for a triplet to singlet transition to occur, a spin-flip process must take place. For ISC, typically spin-orbit coupling (SOC) can entangle triplet and singlet states yielding the possibility for a spin-flip transition. To validate our methods for computing SOC (see methods section), we first computed the SOC strengths for the $\nv$ center in diamond. We obtained SOC values of 4.0 GHz for  the axial $\lambda_z$ and 45 GHz for non-axial  $\lambda_\perp$ in fair agreement with previously computed values and experimentally measured values~\cite{thiering2017ab,bassett2014ultrafast}.
We then computed the SOC strength for the $\ti$ defect ($\lambda_z = 149$ GHz, $\lambda_\perp = 312$ GHz) and the $\mo$ defect ($\lambda_z = 16$ GHz, $\lambda_\perp = 257$ GHz).
The value of $\lambda_\perp$ in particular leads to the potential for a spin-selective pathway for both defects, analogous to $\nv$ center in diamond.

% To compute the ISC rate, we considered a new approach which is a derivative of the nonradiative recombination formalism presented in Eq.~\ref{eq:rate-nonrad-allterms}:
To compute the ISC rate, we developed an approach which is a derivative of the nonradiative recombination formalism presented in Eq.~\ref{eq:rate-nonrad-allterms}:
\begin{align}
    \Gamma_{ISC} &= 4 \pi \hbar \lambda_\perp^2 \widetilde{X}_{if}(T) \label{eq:isc-full}\\
    \widetilde{X}_{if}(T) &= \sum_{n,m}p_{in} \left|
        \braket{\phi_{fm}(\textbf{R})}{\phi_{in}(\mathbf{R})}
    \right|^2 \delta(m\hbar\omega_f - n\hbar\omega_i+\Delta E_{if}) \label{eq:isc-F}
\end{align}
% Opposed to Eq.~\ref{eq:rate-nonrad-X}, the phonon overlap in Eq.~\ref{eq:isc-F} does not have an expectation on $\Delta Q$, however it behaves similarly nonetheless and has an exponential dependence on the Huang-Rhys factor $S$.
Compared with previous formalism,~\cite{thiering2017ab} this method
%has the advantages of a temperature dependence (previous approximate $T \rightarrow 0$) and
allows different values for initial state vibrational frequency ($\omega_i$) and final state one ($\omega_f$) through explicit calculations of phonon wavefunction overlap. Again to validate our methods we first computed the intersystem crossing rate for $\nv$ center in diamond. Using the experimental value for $\lambda_\perp$ we obtain an intersystem crossing rate for $\nv$ center in diamond of 2.3 MHz which is in excellent agreement with the experimental value of 8 and 16 MHz~\cite{goldman2015phonon}.
In final, we obtain an intersystem crossing time of 83 ps for $\ti$ and 2.7 $\mu$s for $\mo$ as shown in Table~\ref{table:nonrad} and light blue lines in Figure~\ref{fig:multi}.

% Table for nonradiative lifetimes
\begin{table}
    \footnotesize
    \centering

% \resizebox{1\columnwidth}{!}{
\begin{tabular}{cccccc}
\hline \hline
    % \multirow{2}{*}{GSR}
    $\ti$ & GSR & ZPL (eV) & $S_f$ & $W_{if}$ (eV/(amu$^{1/2}\si{\angstrom}$)) & $\tau_{NR}$ (ps) \\
    % & & (eV) & & eV/(amu$^{1/2}\si{\angstrom}$) & (ps) \\

    & $\es \rightarrow \gs$  & 0.494 & 0.91 & $1.02\times 10^{-1}$ & $8.80\times 10^{12}$ \\
    & $\pjt \rightarrow \gs$ & 0.482 & 14.95 & $1.91\times 10^{-2}$ & $4.41\times 10^{-2}$ \\

    & PJT & $E_{JT}$ (eV) & $S_f$  & $\delta_{JT}$ (eV) & $\tau^C_{NR}$ (ps) \\
    % &  & (eV) & & (eV) & (ps)\\

    % $\es \rightarrow \pjt$ & 0.012 & 13.30 & -- & 10.75 & 0.006 & -- & $3.94\times 10^{2}$ \\
    & $\es \rightarrow \pjt$ & 0.012 & 10.75 & 0.006 & $3.94\times 10^{2}$ \\

    & ISC & ZPL (eV) & $S_f$ & $\lambda_{\perp}$ (GHz) & $\tau_{ISC}$ (ps) \\
    % & & (eV) & & (GHz) & (ps) \\

    & $\es \rightarrow \sin$ & 0.189 & 17.48 & 312 & $8.30\times 10^{1}$ \\

\hline
    $\mo$ & GSR & ZPL (eV) & $S_f$ & $W_{if}$ (eV/(amu$^{1/2}\si{\angstrom}$)) & $\tau_{NR}$ ($\rm \mu s$) \\
    % & & (eV) & & eV/(amu$^{1/2}\si{\angstrom}$) & ($\rm \mu s$)\\

    & $\moes \rightarrow \mogs$  & 0.915 & 22.05 & $1.5\times 10^{-2}$ & $0.02$ \\

    & ISC & ZPL (eV) & $S_f$ & $\lambda_{\perp}$ (GHz) & $\tau_{ISC}$ ($\rm \mu s$) \\
    % & & (eV) & & (GHz) & ($\rm \mu s$) \\

    & $\moes \rightarrow \mosin$ & 0.682 & 7.22 & 257 & $2.7$ \\

\hline \hline
\end{tabular}
% }

    \caption{
Various nonradiative recombination lifetimes along with relevant quantities for the $\ti$ and $\mo$ defects in $h$-BN, including ground state recombination (GSR), pseudo Jahn-Teller (PJT), and intersystem crossing (ISC).
    }
    \label{table:nonrad}
\end{table}

The results of all the nonradiative  pathways for two spin defects are summarized in Table~\ref{table:nonrad} and are displayed in Figure~\ref{fig:multi} along with the radiative pathway.
We begin by summarizing the results for $\ti$ first and then discuss $\mo$ below. In short, for $\ti$ the spin conserved optical excitation from the triplet ground state $\gs$ to the triplet excited state $\es$ cannot directly recombine nonradiatively due to a weak electron-phonon coupling between these states.
In contrast, a nonradiative decay is possible via its PJT state ($\pjt$) with a lifetime of 394 ps.
Finally, the process of intersystem crossing from the triplet excited state $\es$ to the singlet state ($\sin$) is an order of magnitude faster (i.e.\ 83 ps) and is in-turn a dominant relaxation pathway. Therefore the $\ti$ defect in $h$-BN is predicted to have an expedient spin purification process due to a fast intersystem crossing with a rate of 12 GHz.
We note that while the defect has a low optical quantum yield and is predicted to not be a good SPE candidate, it is still noteworthy, as to date the only discovered triplet defect in $h$-BN is the negatively charged boron vacancy, which also does not exhibit SPE and has similarly low quantum efficiency.~\cite{kianinia2020generation} Meanwhile, the leveraged control of an extrinsic dopant can offer advantages in spatial and chemical nature of defects.

For the $\mo$ defect, its direct nonradiative recombination lifetime from the triplet excited state $\moes$ to the ground state $\mogs$ is 0.02 $\mu$s.
While the comparison with its radiative lifetime (33 $\mu$s) is improved compared to the $\ti$ defect, it still is predicted to have low quantum efficiency.
However, again the intersystem crossing between $\moes$ and $\mosin$ is competitive with a lifetime of 2.7 $\mu$s. This rate (around MHz) is similar to diamond and implies a feasible intersystem crossing. Owing to its more ideal ZPL position ($\sim$1eV) and improved quantum efficiency, optical control of the $\mo$ defect is seen as more likely and may be further improved by other methods such as coupling to optical cavities~\cite{kim2018photonic,zhong2018optically} and applying strain~\cite{wu2019carrier,mendelson2020strain}.

%%%%%%%%%%%%%%%%%%%%%%%%%%%%%%%%%%%%%%%%%%%%%%%%%%%%%%%%%%%%%%%%%%%%%%%%%%%%%%%%%%%%%%%%%%%%%%%%%%%%%%%%%%%%%%%%%%
\subsection{Conclusion}
In summary, we proposed a general theoretical framework for identifying and designing optically-addressable spin defects for the future development of quantum emitter and quantum qubit systems.
We started from searching for defects with triplet ground state by DFT total energy calculations which allow for rapid identification of possible candidates. Here we found that the $\ti$ and $\mo$ defects in $h$-BN have a neutral triplet ground state. We then
computed zero-field splitting of secondary spin quantum sublevels and found they are sizable for both defects, larger than that of $\nv$ center in diamond, enabling possible control of these levels for qubit operation.
In addition, we screened for potential single photon emitters (SPEs) in $h$-BN based on allowed intra-defect transitions and radiative lifetimes, leading to the discovery of $\vv{Si}$.
Next the electronic structure and optical spectra of each defect were computed from many-body perturbation theory.
Specifically, the $\vv{Si}$ defect is shown to possess an exciton radiative lifetime similar to experimentally observed SPEs in $h$-BN and is a potential SPE candidate.
Finally, we analyzed all possible radiative and nonradiative dynamical processes with first-principles rate calculations. In particular, we identified a dominant spin-selective decay pathway via intersystem crossing at the $\ti$ defect which gives a key advantage for initial pure spin state preparation and qubit operation.
Meanwhile, for the $\mo$ defect we found that it has the benefit of improved quantum efficiency for more realistic optical control.

%%%%%%%%%%%

This work emphasizes that the theoretical discovery of spin defects requires careful treatment of many-body interactions and various radiative and nonradiative dynamical processes such as intersystem crossing.
We demonstrate high potential of extrinsic spin defects in 2D host materials as qubits for quantum information science.
Future work will involve further examination of spin coherence time and its dominant decoherence mechanism,
as well as other spectroscopic fingerprints from first-principles calculations to facilitate experimental validation of these defects.

%%%%%%%%%%%%%%%%%%%%%%%%%%%%%%%%%%%%%%%%%%%%%%%%%%%%%%%%%%%%%%%%%%%%%%%%%%%%%%%%%%%%%%%%%%%%%%%%%%%%%%%%%%%%%%%%%%

\subsection{First-Principles Calculations} In this study, we used the open source plane-wave code Quantum ESPRESSO~\cite{QE1} to perform calculations on all structural relaxations and total energies with optimized norm-conserving Vanderbilt (ONCV) pseudopotentials~\cite{ONCV1} and a wavefunction cutoff of 50 Ry. A supercell size of $6\times 6$ or higher was used in our calculations with a $3\times 3\times 1$ k-point mesh. Charged cell total energies were corrected to remove spurious charge interactions by employing the techniques developed in Refs.~\cite{PING2017JCP,wu2017first,wang2020layer} and implemented in the JDFTx code~\cite{JDFTx}.
The total energies, charged defect formation energies and geometry were evaluated at the Perdew-Burke-Ernzerhof (PBE) level~\cite{perdew1996generalized}.
Single-point calculations with k-point meshes of $2\times 2\times 1$ and $3\times 3\times 1$ were performed using hybrid exchange-correlation functional PBE0($\alpha$), where the mixing parameter $\alpha=0.41$ was determined by the generalized Koopmans’ condition as discussed in Ref.~\cite{smart2018fundamental,miceli2018nonempirical}. Moreover, we used the YAMBO code~\cite{YAMBO} to perform many-body perturbation theory with the GW approximation to compute the quasi-particle correction
using PBE0($\alpha$) eigenvalues and wavefunctions as the starting point.
The random phase approximation (RPA) and Bethe-Salpeter Equation (BSE) calculations were further solved on top of the GW approximation for the electron-hole interaction to investigate the optical properties of the defects, including absorption spectra and radiative lifetime.
%%%%%%%%%

\subsection{Thermodynamic Charge Transition Levels and Defect Formation Energy}
The defect formation energy ($FE_q$) was computed for the $\ti$ and $\mo$ defects following:
\begin{align}
    FE_q(\varepsilon_F) = E_q - E_{pst} + \sum_i \mu_i \Delta N_i + q \varepsilon_F + \Delta_q
    \label{eq:cfe}
\end{align}
where $E_q$ is the total energy of the defect system with charge $q$, $E_{pst}$ is the total energy of the pristine system, $\mu_i$ and $\Delta N_i$ are the chemical potential and change in the number of atomic species $i$, and $\varepsilon_F$ is the Fermi energy. A charged defect correction $\Delta_q$ was computed for charged cell calculations by employing the techniques developed in Ref.~\cite{wu2017first,PING2017JCP}. The chemical potential references are computed as $\mu_{Ti} = E_{Ti}^{bulk}$ (total energy of bulk Ti), $\mu_{Mo} = E_{Mo}^{bulk}$ (total energy of bulk Mo),  $\mu_{BN} = E_{BN}^{ML}$ (total energy of monolayer $h$-BN). Meanwhile the corresponding charge transition levels of defects can be obtained from the value of $\varepsilon_F$ where the stable charge state transitions from $q$ to $q'$.
\begin{align}
    \epsilon_{q|q'} = \frac{FE_q - FE_{q'}}{q' - q}
    \label{eq:ctl}
\end{align}

\subsection{Zero-Field Splitting}
The first-order ZFS due to spin-spin interactions was computed for the dipole-dipole interactions of the electron spin:
\begin{align}
    H_{ss} = \frac{\mu_0}{4\pi} \frac{(g_e\hbar)^2}{r^5} \left[
        3(\textbf{s}_1 \cdot \textbf{r})(\textbf{s}_2 \cdot \textbf{r})
        -(\textbf{s}_1 \cdot \textbf{s}_2)r^2
    \right].
\end{align}
Here, $\mu_0$ is the magnetic permeability of vacuum, $g_e$ is the electron gyromagnetic ratio, $\hbar$ is the Planck's constant, $\textbf{s}_1$, $\textbf{s}_2$ is the spin of first and second electron, respectively, and $\textbf{r}$ is the displacement vector between these two electron.
The spatial and spin dependence can be separated by introducing the effective total spin $\textbf{S}=\sum_i \textbf{s}_i$. This yields a Hamiltonian of the form $H_{ss} =  \textbf{S}^T\hat{\textbf{D}}\textbf{S}$, which introduces the traceless zero-field splitting tensor $\hat{\textbf{D}}$. It is common to consider the axial and rhombic ZFS parameters $D$ and $E$ which can be acquired from the $\hat{\textbf{D}}$ tensor:
\begin{align}
    D = \frac{3}{2} D_{zz} \quad \text{and} \quad E = (D_{yy} - D_{xx})/ 2\ .
\end{align}
Following the formalism of Rayson et al.,~\cite{rayson2008first} the ZFS tensor $\hat{\textbf{D}}$ can be computed with periodic boundary conditions as:
\begin{align}
    D_{ab} = \frac{1}{2}\frac{\mu_0}{4\pi} (g_e\hbar)^2 \sum_{i>j} \chi_{ij}
        \expval{
            \frac{\textbf{r}^2\delta_{ab}-3\textbf{r}_a\textbf{r}_b}{r^5}
        }{
            \Psi_{ij}(\textbf{r}_1, \textbf{r}_2)
        }.
\end{align}
Here the summation on pairs of $i,j$ runs over all occupied spin-up and spin-down states, with $\chi_{ij}$ taking the value $+1$ for parallel spin and $-1$ for anti-parallel spin, and $\Psi_{ij}(\textbf{r}_1,\textbf{r}_2)$ is a two-particle Slater determinant constructed from the Kohn-Sham wavefunctions of the $i$th and $j$th states. This procedure was implemented as a post-processing code interfaced with Quantum ESPRESSO.
To verify our implementation is accurate, we computed the ZFS of the $\nv$ center in diamond which has a well-established result. Using ONCV pseudopotentials, we obtained a ZFS of 3.0 GHz for $\nv$ center, in perfect agreement with previous reported results~\cite{seo2017designing}.
For heavy elements such as transition metals, spin-orbit (SO) coupling can have substantial contribution to zero-field splitting. Here, we also computed the SO contribution of the ZFS as implemented in the ORCA code~\cite{neese2012orca,neese2007calculation} (additional details can be found in Supplementary Note 10, Figure 12, and Table 6).

\subsection{Radiative Recombination}
In order to quantitatively study radiative processes, we computed the radiative rate $\Gamma_R$ from Fermi's Golden Rule and considered the excitonic effects by solving BSE~\cite{wu2019dimensionality}:
\begin{align}
    \Gamma_R (\textbf{Q}_{ex}) &=
    \frac{2\pi}{\hbar}
    \sum_{q_L, \lambda}
    \left|
        \mel{G,1_{q_L,\lambda}}
        {H^R}
        {S(\textbf{Q}_{ex}),0}
    \right|^2
    \delta(E(\textbf{Q}_{ex}) -\hbar c q_L).
    \label{eq:rate-radiative-full}
\end{align}
% \begin{align}
%     \Gamma_R (\textbf{Q}) &= \frac{4\pi^2e^2}{\hbar c^2V} \Omega(\textbf{Q})^2 \sum_{q, \lambda} \frac{1}{q} \left|
%         \epsilon_{q\lambda} \cdot \mel{G}{\textbf{r}}{S}
%     \right|^2 \delta(\frac{\Omega(\textbf{Q})}{c} - q)
%     \label{eq: rate-radiative-full}
% \end{align}
Here, the radiative recombination rate is computed between the ground state $G$ and the two-particle excited state $S(\textbf{Q}_{ex})$, $1_{q_L,\lambda}$ and 0 denote the presence and absence of a photon, $H^R$ is the electron-photon coupling (electromagnetic) Hamiltonian,  $E(\textbf{Q}_{ex})$ is the exciton energy, and $c$ is the speed of light.
The summation indices in Eq.~\ref{eq:rate-radiative-full} run over all possible wavevector ($q_L$) and polarization ($\lambda$) of the photon.
Following the approach described in Ref.~\cite{wu2019dimensionality}, the radiative rate (inverse of radiative lifetime $\tau_R$) in SI unit at zero temperature can be computed for isolated defect-defect transitions as:
 %\Gamma_R  &= \frac{4e^2}{3\hbar c^3} \Omega_0^3 |\mu|_{e-h}^2,\\
\begin{equation}
     \Gamma_R = \frac{n_D e^2}{3\pi\epsilon_0\hbar^4 c^3} E_0^3 \mu_{e-h}^2,
    \label{eq:rate-radiative-0D}
\end{equation}
where $e$ is the charge of an electron, $\epsilon_0$ is vacuum permittivity, $E_0$ is the exciton energy at $\textbf{Q}_{ex}=0$, $n_D$ is the reflective index of the host material and $\mu_{e-h}^2$ is the modulus square of exciton dipole moment with length$^2$ unit. Note that Eq.~\ref{eq:rate-radiative-0D} considers defect-defect transitions in the dilute limit; therefore the lifetime formula for zero-dimensional systems embedded in a host material is used~\cite{gupta2018two,mackoit2019carbon} (also considering $n_D$ is unity in isolated 2D systems at the long-wavelength limit). We did not consider the radiative lifetime of $\ti$ defect at a finite temperature because the first and second excitation energy separation is much larger than $kT$. Therefore a thermal average of the first and higher excited states is not necessary and the first excited state radiative lifetime is nearly the same at 10 K as zero temperature.


\subsection{Phonon-Assisted Nonradiative Recombination} In this work, we compute the phonon-assisted nonradiative recombination rate via a Fermi's golden rule approach:
\begin{align}
    \Gamma_{NR} &= \frac{2\pi}{\hbar}g\sum_{n,m}p_{in}\left|\mel{fm}{H^{e-ph}}{in}\right|^2\delta(E_{in}-E_{fm}) \label{eq:rate-nonradiative-fg}
\end{align}
Here, $\Gamma_{NR}$ is the nonradiative recombination rate between electron state $i$ in phonon state $n$ and electron state $f$ in phonon state $m$, $p_{in}$ is the thermal probability distribution of the initial state $\ket{in}$, $H^{e-ph}$ is the electron-phonon coupling Hamiltonian, $g$ is the degeneracy factor and $E_{in}$ is the energy of vibronic state $\ket{in}$. Within the static coupling and one-dimensional (1D) effective phonon approximations, the nonradiative recombination can be reduced to:
\begin{align}
    \Gamma_{NR} &= \frac{2\pi}{\hbar}g|W_{if}|^2 X_{if}(T), \label{eq:rate-nonrad-allterms}\\
    X_{if}(T) &= \sum_{n,m}p_{in}\left|\mel{\phi_{fm}(\textbf{R})}{Q-Q_a}{\phi_{in}(\mathbf{R})}\right|^2 \delta(m\hbar\omega_f - n\hbar\omega_i+\Delta E_{if}), \label{eq:rate-nonrad-X} \\
    W_{if} &= \mel{\psi_i(\mathbf{r}, \textbf{R})}{\frac{\partial H}{\partial Q}}{\psi_f(\mathbf{r},\textbf{R})}\bigg{|}_{\textbf{R}=\textbf{R}_a}. \label{eq:rate-nonrad-W}
\end{align}
Here, the static coupling approximation naturally separates the nonradiative recombination rate into phonon and electronic terms, $X_{if}$ and $W_{if}$, respectively. The 1D phonon approximation introduces a generalized coordinate $Q$, with effective frequency $\omega_i$ and $\omega_f$. The phonon overlap in Eq.~\ref{eq:rate-nonrad-X} can be computed using the quantum harmonic oscillator wavefunctions with $Q-Q_a$ from the configuration diagram (Figure~\ref{fig:config}). Meanwhile the electronic overlap in Eq.~\ref{eq:rate-nonrad-W} is computed by finite difference using the Kohn-Sham orbitals from DFT at the $\Gamma$ point. The nonradiative lifetime $\tau_{NR}$ is given by taking the inverse of rate $\Gamma_{NR}$. Supercell convergence of phonon-assisted nonradiative lifetime is shown in Supplementary Note 11 and Table 7. We validated the 1D effective phonon approximation by comparing the Huang-Rhys factor with the full phonon calculations in Supplementary Table 8.

\subsection{Spin-Orbit Coupling Constant} Spin-orbit coupling (SOC) can entangle triplet and singlet states yielding the possibility for a spin-flip transition. The SOC operator is given to zero-order by~\cite{maze2011properties}:
\begin{align}
    H_{so} = \frac{1}{2} \frac{1}{c^2m_e^2} \sum_i \left(
        \nabla_i V \cross \textbf{p}_i \right) \textbf{S}_i \label{eq:soc-full}
\end{align}
where $c$ is the speed of light, $m_e$ is the mass of an electron, $\textbf{p}$ and $\textbf{S}$ are the momentum and spin of electron $i$ and $V$ is the nuclear potential energy. The spin-orbit interaction can be rewritten in terms of the angular momentum $L$ and the SOC strength $\lambda$ as~\cite{maze2011properties},
\begin{align}
    H_{so} = \sum_i \lambda_{\perp} (L_{x,i}S_{x,i} + L_{y,i}S_{y,i}) + \lambda_z L_{z,i}S_{z,i}.
\end{align}
where $\lambda_{\perp}$ and $\lambda_z$ denote the non-axial and axial SOC strength, respectively. The SOC strength was computed for the $\ti$ defect in $h$-BN using the ORCA code by TD-DFT~\cite{neese2012orca,de2019predicting}.
More computational details can be found in Supplementary Note 10.


% ------------------------------------------------------------
% other
\chapter{Selected Topics}
\section{Improving Catalytic}
% Ni3S2 and Fe doped graphene

\section{Stablity of hBN Defect}
% 2DEnv calculations

\section{NV Center in Diamond}
% NV Center in Diamond

\section{Optimizing Workflow as a Computational Scientist}
% Non-scientific skills worth learning and sharing
% Why develop these non-scientific skills?
%     - improve productivity
%     - minimize work stress both mental and physical
%     - learning and taking advantage of these strategies makes work more fun!
%
% [1] VIM
%     - basic and advanced vim usage
%     - use hjkl
%     - use /, f, t
%     - v, shift-v, ctrl-v
%     - macros
%     - vim bindings command line and vimium
% [2] command line
%     - zsh - autocomplete and syntax highlighting
%     - fzf - better history search
%     - tmux - save and organize your ssh sessions
% [3] git
%     - for organizing projects (example hbn)
%     - backup!!
%     - same exact configuration on every machine
% [4] python
%     - pandas = easy way to organize and process data
%     - pw2py = atomgeo, qeinp, qeout, qesave examples...
%     - pymatgen
%     - how to use numpy (numpy dot product v. manual dot product v. c++ dot product)
% [5] other (useful commands)
%     - convert
%     - tree
%     - tail -f (perfect for watching output of a calculation)
%     - tee (another way of watching output of a calculation)
%     - watch (great for monitoring squeue)
%     - tldr (quick check how to use command, tar is a good example)

\section{Open Questions}
% What I think is interesting for future investigation


% ------------------------------------------------------------
\chapter{Conclusions}
blank

\appendix

\def\bS{\mathbf{\hat{S}}}
\def\bD{\mathbf{D}}
\def\H{\mathcal{\hat{H}}_D}
\def\br{\mathbf{r}}
\def\bR{\mathbf{R}}
\def\bG{\mathbf{G}}
\def\bC{\mathbf{C}}
\def\intinf{\int_{-\infty}^{\infty}}

\newcommand{\svar}[1]{\hat{S}_{#1}}
\newcommand{\ovar}[1]{\hat{\sigma}_{#1}}

\chapter{Zero-Field Splitting}

\section{Dipole-Dipole Spin Hamiltonian}

This section follows chapter 6 in Ref.~\cite{weil2007electron} closely. Here is a method for computing spin-spin contribution of the zero-field splitting from Kohn-Sham orbitals at defect centers with high spin ($S > 1$). Originally, we consider the spin-spin interaction of two dipole moments $\bS_1$ and $\bS_2$, separated by $\br$:
\begin{align}
    \H = \frac{\mu_0}{4\pi}g^2\mu_B^2\left[\frac{ \bS_1\cdot\bS_2}{r^3}-3\frac{(\bS_1\cdot\br)(\bS_2\cdot\br)}{r^5} \right]
    \label{zfs:eq:1}
\end{align}
where $g$ is the electron g-factor and $\mu_B$ is the Bohr magneton. Expanding scalar products gives:
\begin{align}
    \H =& \frac{\mu_0}{4\pi}\frac{g^2\mu_B^2}{r^5}\bigg[ (r^2-3x^2)\svar{1x}\svar{2x}
    + (r^2-3y^2)\svar{1y}\svar{2y} + (r^2-3z^2)\svar{1z}\svar{2z} \label{zfs:eq:2} \\
    & - 3xy(\svar{1x}\svar{2y}+\svar{1y}\svar{2x}) - 3xz(\svar{1x}\svar{2z}+\svar{1z}\svar{2x})
    - 3yz(\svar{1y}\svar{2z}+\svar{1z}\svar{2y}) \bigg] \nonumber \, .
\end{align}
Now we seek to convert from the operators $\bS_1$ and $\bS_2$ to the combinatory $\bS = \bS_1+\bS_2$. It easily shown (see appendix) that:
\begin{align}
    \svar{1x}\svar{2x}=\frac{1}{2}\hat{S}_x^2-\frac{1}{4}\mathbb{1}
    \label{zfs:eq:3}
\end{align}
and
\begin{align}
    \svar{1x}\svar{2y}+\svar{1y}\svar{2x}=\frac{1}{2}(\hat{S}_x\hat{S}_y+\hat{S}_y\hat{S}_x)\, .
    \label{zfs:eq:4}
\end{align}
Inserting Eq.~\ref{zfs:eq:3} and Eq.~\ref{zfs:eq:4} into Eq.~\ref{zfs:eq:2} gives:
\begin{align}
    \H =& \frac{\mu_0}{4\pi}\frac{g^2\mu_B^2}{r^5}\frac{1}{2}\bigg[ (r^2-3x^2)\svar{x}^2
    + (r^2-3y^2)\svar{y}^2 + (r^2-3z^2)\svar{z}^2 \label{zfs:eq:5} \\
    & - 3xy(\svar{x}\svar{y}+\svar{y}\svar{x}) - 3xz(\svar{x}\svar{z}+\svar{z}\svar{x})
    - 3yz(\svar{y}\svar{z}+\svar{z}\svar{y}) \bigg] \nonumber \, .
\end{align}
which is analogous to Eq.~\ref{zfs:eq:2} but is in terms of a single spin operator $\bS$. Therefore Eq.~\ref{zfs:eq:5} can be converted into a typical spin-hamiltonian form where $\braket{\,}$ represents integration over real space wavefunctions.
\begin{align}
    \H = \bS^{T}\cdot\bD\cdot\bS
    \label{zfs:eq:6}
\end{align}
\begin{align}
    \bD = \frac{\mu_0}{8\pi}g^2\mu_B^2
        \begin{bmatrix}
        \braket{\frac{r^2-3x^2}{r^5}} &
        \braket{\frac{-3xy}{r^5}}     &
        \braket{\frac{-3xz}{r^5}}     \\
        &
        \braket{\frac{r^2-3y^2}{r^5}} &
        \braket{\frac{-3yz}{r^5}}     \\
        &
        &
        \braket{\frac{r^2-3z^2}{r^5}} &
        \end{bmatrix}
    \quad \quad
    \bS =
    \begin{bmatrix}
    \svar{x} \\
    \svar{y} \\
    \svar{z}
    \end{bmatrix}
    \label{zfs:eq:7}
\end{align}


\section{Two Particle Wavefunctions from DFT Calculations}

This section follows Ref.~\cite{rayson2008first} exactly. Here we are interested in the evaluation of integrals of the form:
\begin{align}
    I_{ab}=\iint\rho(\br_1,\br_2)f_{ab}(\br_1-\br_2)\, d\br_1\, d\br_2 \, ,
    \label{zfs:eq:8}
\end{align}
with
\begin{align}
    f_{ab}(\br_1-\br_2)=\frac{r^2\delta_{ab}-3r_ar_b}{r^5} \, ,
    \label{zfs:eq:9}
\end{align}
$\br=\br_1-\br_2$ and $a,b \in \{x,y,z\}$. Here $\rho(\br_1,\br_2)$ is the two-particle charge density which is constructed by the anti-symmetric product of single particle orbitals:
\begin{align}
    \Psi(\br_1,\br_2)=\frac{1}{\sqrt{2}}\left[ \psi_i(\br_1)\psi_j(\br_2) - \psi_i(\br_2)\psi_j(\br_1) \right] \, .
    \label{zfs:eq:10}
\end{align}
And thus the two particle charge density is given by (note $f_{ab}$ is symmetric under particle exchange):
\begin{align}
    \rho(\br_1,\br_2)=\left[
    \psi_i^*(\br_1)\psi_j^*(\br_2)\psi_i(\br_1)\psi_j(\br_2) - \psi_i^*(\br_2)\psi_j^*(\br_1)\psi_i(\br_1)\psi_j(\br_2) \right] \, .
    \label{zfs:eq:11}
\end{align}

Here we alleviate the expense in this calculation by expanding $\rho(\br_1,\br_2)$ in terms of plane waves and thus circumventing the cost from direct integration. This expansion gives
\begin{align}
    \rho(\br_1,\br_2) = \sum_{\bG_1,\bG_2} \rho(\bG_1,\bG_2) e^{i\bG_1\cdot\br_1} e^{i\bG_2\cdot\br_2}
    \label{zfs:eq:12}
\end{align}
and thus
\begin{align}
    I_{ab} = \frac{1}{8} \sum_{\bG_1,\bG_2} \rho(\bG_1,\bG_2)\iint
    e^{i[\br\cdot(\bG_1-\bG_2)]/2} e^{i[\bR\cdot(\bG_1+\bG_2)]/2} f_{ab}(\br)\, d\br\, d\bR \, ,
    \label{zfs:eq:13}
\end{align}
where $\br=\frac{1}{2}(\br_1-\br_2)$ and $\bR=\frac{1}{2}(\br_1+\br_2)$, with $d\br_1\,d\br_2=1/2\,d\br\, d\bR$. The integration over $\bR$ can be done directly and yields $8\Omega\delta_{\bG_1,-\bG_2}$. Therefore $\bG_1=-\bG_2=\bG$ and Eq.~\ref{zfs:eq:13} simplifies to
\begin{align}
    I_{ab} = \Omega \sum_{\bG} \rho(\bG,-\bG)\int
    e^{i\bG\cdot\br} f_{ab}(\br)\, d\br\, ,
    \label{zfs:eq:14}
\end{align}
where $\Omega$ is the volume of the unit cell. The integration over $r$ is slightly more tricky but can still be evaluated to give the final result (see appendix for integration over $\bR$ and $\br$)
\begin{align}
    I_{ab} = 4\pi\Omega \sum_\bG \rho(\bG,-\bG) \left( \frac{G_aG_b}{G^2} - \frac{\delta_{ab}}{3} \right)\, ,
    \label{zfs:eq:15}
\end{align}
where
\begin{align}
D_{ab}= \frac{\mu_0}{4\pi}g^2\mu_{B}^2\, \left[\frac{1}{2}I_{ab}\right] \, .
    \label{zfs:eq:16}
\end{align}
Therefore the spin-spin Hamiltonian is given by (Ref.~\cite{ivady2014pressure})
\begin{align}
    \H = D \left(\svar{z}^2-\frac{S(S+1)}{3} \right) + E \left(\frac{\svar{+}^2+\svar{-}^2}{2}\right) \, ,
    \label{zfs:eq:17}
\end{align}
with
\begin{align}
    D=\frac{3}{2}D_{zz} \, , \quad
    E=\frac{D_{yy}-D_{xx}}{2}  \, , \quad
    \svar{\pm} = \svar{x} \pm i \, \svar{y}
    \label{zfs:eq:18}
\end{align}


\section{Beyond two-band result}

Consider a two particle wavefunction built of parallel spin electrons with $i\neq j$:
\begin{align}
    \Psi^\alpha(1,2)&=\frac{1}{\sqrt{2}} \begin{bmatrix}
    \psi_i^\alpha(1) &
    \psi_j^\alpha(1) \label{zfs:eq:19} \\
    \psi_i^\alpha(2) &
    \psi_j^\alpha(2)
   \end{bmatrix} \\
   &= \frac{1}{\sqrt{2}}\left[ \psi_i^\alpha(1)\psi_j^\alpha(2) - \psi_i^\alpha(2)\psi_j^\alpha(1) \right]
    \label{zfs:eq:20}
\end{align}
The corresponding two particle density is:
\begin{align}
    \rho^{\alpha} &= \Psi^\alpha(1,2)^* \Psi^\alpha(1,2) \label{zfs:eq:21} \\
    &= \frac{1}{2} \bigg[
    |\psi_i^\alpha(1)|^2|\psi_j^\alpha(2)|^2
    + |\psi_i^\alpha(2)|^2|\psi_j^\alpha(1)|^2 \label{zfs:eq:22} \\
    &\hspace{1cm}- \psi_i^\alpha(1)^*\psi_j^\alpha(2)^*\psi_i^\alpha(2)\psi_j^\alpha(1)
                 - \psi_i^\alpha(2)^*\psi_j^\alpha(1)^*\psi_i^\alpha(1)\psi_j^\alpha(2)
    \bigg]
    \nonumber
\end{align}
Consider a two particle wavefunction built of anti-parallel spin electrons with $i\neq j$:
\begin{align}
    \Psi^{\alpha\beta}(1,2)&=\frac{1}{\sqrt{2}} \begin{bmatrix}
    \psi_i^\alpha(1) &
    \psi_j^\beta(1) \\
    \psi_i^\alpha(2) &
    \psi_j^\beta(2)
   \end{bmatrix} \label{zfs:eq:23} \\
   &= \frac{1}{\sqrt{2}}\left[ \psi_i^\alpha(1)\psi_j^\beta(2) - \psi_i^\alpha(2)\psi_j^\beta(1) \right]
    \label{zfs:eq:24}
\end{align}
The corresponding two particle density is:
\begin{align}
    \rho^{\alpha\beta} &= \Psi^{\alpha\beta}(1,2)^* \Psi^{\alpha\beta}(1,2) \label{zfs:eq:25} \\
    &= \frac{1}{2} \bigg[
    |\psi_i^\alpha(1)|^2|\psi_j^\beta(2)|^2
    + |\psi_i^\alpha(2)|^2|\psi_j^\beta(1)|^2 \label{zfs:eq:26} \\
    &\hspace{1cm} - \psi_i^\alpha(1)^*\psi_j^\beta(2)^*\psi_i^\alpha(2)\psi_j^\beta(1)
    - \psi_i^\alpha(2)^*\psi_j^\beta(1)^*\psi_i^\alpha(1)\psi_j^\beta(2)
    \bigg] \nonumber \\
    &=\frac{1}{2} \bigg[
    |\psi_i^\alpha(1)|^2|\psi_j^\beta(2)|^2
    + |\psi_i^\alpha(2)|^2|\psi_j^\beta(1)|^2 \bigg]
    \label{zfs:eq:27}
\end{align}


The full many-particle D tensor including spin operator terms is given by (see Eq. 18 of Ref.~\cite{rayson2008first} and Eq. 31 of Ref.~\cite{ganyushin2006first}):
\begin{align}
    D_{ab}=&\frac{1}{S(2S-1)} \left[
        \sum_{i>j}\braket{\Psi|f_{ab}(\br_i,\br_j)
        (2\svar{i_z}\svar{j_z}-\svar{i_x}\svar{j_x}-\svar{i_y}\svar{j_y})|\Psi}
        \right] \label{zfs:eq:28} \\
    =&\frac{1}{S(2S-1)} \bigg[
        \frac{1}{2}\sum_{i>j}\braket{\Psi|f_{ab}(\br_i,\br_j)
        \ovar{i_z}\ovar{j_z}|\Psi} \label{zfs:eq:29} \\
        &\hspace{1cm}-\frac{1}{4}\sum_{i>j}\braket{\Psi|f_{ab}(\br_i,\br_j)
        \ovar{i_x}\ovar{j_x}|\Psi}
        -\frac{1}{4}\sum_{i>j}\braket{\Psi|f_{ab}(\br_i,\br_j)
        \ovar{i_y}\ovar{j_y}|\Psi}
        \bigg]
    \nonumber
\end{align}
Recalling that $\ovar{z}\ket{\pm}=\pm\ket{\pm}$, $\ovar{x}\ket{\pm}=\ket{\mp}$, and $\ovar{y}\ket{\pm}=\pm i\ket{\mp}$. We have:
\begin{align}
    &\text{for parallel spin states: }&
        \ovar{i_z}\ovar{j_z} = 1 \text{,  }
        \ovar{i_x}\ovar{j_x} = 1 \text{,  }
        \ovar{i_y}\ovar{j_y} = -1 \text{,  } \label{zfs:eq:30} \\
    &\text{for anti-parallel spin states: }&
        \ovar{i_z}\ovar{j_z} = -1 \text{,  }
        \ovar{i_x}\ovar{j_x} = 1 \text{,  }
        \ovar{i_y}\ovar{j_y} = 1 \text{. }
        \label{zfs:eq:31}
\end{align}
Now we consider each term in Eq.~\ref{zfs:eq:29}, first for the case of parallel spins then in the case of anti-parallel spins.\\
Parallel spins ($\ovar{i_z}\ovar{j_z}$ term):
\begin{align}
    (\ovar{i_z}\ovar{j_z})\rho^{\alpha} &= \Psi^\alpha(1,2)^*(\ovar{i_z}\ovar{j_z})\Psi^\alpha(1,2) \label{zfs:eq:32} \\
     &= \Psi^\alpha(1,2)^*\Psi^\alpha(1,2) \label{zfs:eq:33} \\
    &= \frac{1}{2} \bigg[
    |\psi_i^\alpha(1)|^2|\psi_j^\alpha(2)|^2
    + |\psi_i^\alpha(2)|^2|\psi_j^\alpha(1)|^2 \label{zfs:eq:34} \\
    &\hspace{2cm}- \psi_i^\alpha(1)^*\psi_j^\alpha(2)^*\psi_i^\alpha(2)\psi_j^\alpha(1) \notag \\
    &\hspace{2cm}- \psi_i^\alpha(2)^*\psi_j^\alpha(1)^*\psi_i^\alpha(1)\psi_j^\alpha(2) \bigg] \notag \\
    f_{ab}(1,2)(\ovar{i_z}\ovar{j_z})\rho^{\alpha}&= J_{ij}^{\alpha} - K_{ij}^{\alpha} \label{zfs:eq:35}
\end{align}
Parallel spins ($\ovar{i_x}\ovar{j_x} + \ovar{i_y}\ovar{j_y}$ term):
% , since these terms both flip the spin of the operand, they will be identical up to their sign which will be opposite as noted above, for parallel spins $\ovar{i_x}\ovar{j_x}=1$ and $\ovar{i_y}\ovar{j_y}=-1$. Therefore:
\begin{align}
    (\ovar{i_x}\ovar{j_x} + \ovar{i_y}\ovar{j_y})\rho^{\alpha} &= \Psi^\alpha(1,2)^*(\ovar{i_x}\ovar{j_x} + \ovar{i_y}\ovar{j_y})\Psi^\alpha(1,2) \label{zfs:eq:36} \\
    &= \Psi^\alpha(1,2)^*\Psi^\beta(1,2) - \Psi^\alpha(1,2)^*\Psi^\beta(1,2) \label{zfs:eq:37} \\
    f_{ab}(1,2)(\ovar{i_x}\ovar{j_x} + \ovar{i_y}\ovar{j_y})\rho^{\alpha}&= 0
    \label{zfs:eq:38}
\end{align}
Anti-parallel spins ($\ovar{i_z}\ovar{j_z}$ term):
\begin{align}
    (\ovar{i_z}\ovar{j_z})\rho^{\alpha\beta} &= \Psi^{\alpha\beta}(1,2)^*(\ovar{i_z}\ovar{j_z})\Psi^{\alpha\beta}(1,2) \label{zfs:eq:39} \\
    &= - \Psi^{\alpha\beta}(1,2)^*\Psi^{\alpha\beta}(1,2) \label{zfs:eq:40} \\
    &= - \frac{1}{2} \bigg[
    |\psi_i^\alpha(1)|^2|\psi_j^\beta(2)|^2
    + |\psi_i^\alpha(2)|^2|\psi_j^\beta(1)|^2
    \bigg] \label{zfs:eq:41} \\
    f_{ab}(1,2)(\ovar{i_z}\ovar{j_z})\rho^{\alpha\beta}&= - J_{ij}^{\alpha\beta}  \label{zfs:eq:42}
\end{align}
Anti-parallel spins ($\ovar{i_x}\ovar{j_x} + \ovar{i_y}\ovar{j_y}$ term):
\begin{align}
    (\ovar{i_x}\ovar{j_x} + \ovar{i_y}\ovar{j_y})\rho^{\alpha\beta} &= \Psi^{\alpha\beta}(1,2)^*(\ovar{i_x}\ovar{j_x} + \ovar{i_y}\ovar{j_y})\Psi^{\alpha\beta}(1,2)  \label{zfs:eq:43} \\
    &= 2\Psi^{\alpha\beta}(1,2)^*\Psi^{\beta\alpha}(1,2)  \label{zfs:eq:44} \\
    &= \bigg[
    \psi_i^\alpha(1)^*\psi_i^\beta(1)\psi_j^\beta(2)^*\psi_j^\alpha(2)  \label{zfs:eq:45} \\
    &\hspace{2cm} + \psi_i^\alpha(2)^*\psi_i^\beta(2)\psi_j^\beta(1)^*\psi_j^\alpha(1) \notag \\
    &\hspace{2cm} - \psi_i^\alpha(1)^*\psi_j^\beta(2)^*\psi_i^\beta(2)\psi_j^\alpha(1) \notag \\
    &\hspace{2cm} - \psi_i^\alpha(2)^*\psi_j^\beta(1)^*\psi_i^\beta(1)\psi_j^\alpha(2) \bigg]  \notag \\
    f_{ab}(1,2)(\ovar{i_x}\ovar{j_x} + \ovar{i_y}\ovar{j_y})\rho^{\alpha\beta}&= - 2K_{ij}^{\alpha\beta}
    \label{zfs:eq:46}
\end{align}
Plugging Eq.~\ref{zfs:eq:35} (both up and down), Eq.~\ref{zfs:eq:38}, Eq.~\ref{zfs:eq:42}, and Eq.~\ref{zfs:eq:46} into Eq.~\ref{zfs:eq:29} gives:
\begin{align}
    D_{ab}
    =&\frac{1}{2S(2S-1)} \left[
        (\sum_{i>j}^{\alpha}J_{ij}^{\alpha} - K_{ij}^{\alpha})
        +(\sum_{i>j}^{\beta}J_{ij}^{\beta} - K_{ij}^{\beta})
        +(\sum_{i>j}^{\alpha\beta}-J_{ij}^{\alpha\beta})
        -\frac{1}{2}(\sum_{i>j}^{\alpha\beta}-2K_{ij}^{\alpha\beta})
        \right]
\end{align}
\textit{Origin of each term:} $\hspace{0.5cm}\text{Eq.~\ref{zfs:eq:35}}
\hspace{0.5cm}\text{Eq.~\ref{zfs:eq:35}}
\hspace{0.5cm}\text{Eq.~\ref{zfs:eq:42}}
\hspace{0.5cm}\text{Eq.~\ref{zfs:eq:46}}$\\
\\
Therefore in final we have:
\begin{align}
    D_{ab}
    =&\frac{1}{2S(2S-1)}
        \sum_{i>j}\chi_{ij}(J_{ij} - K_{ij})
    \label{zfs:eq:48}
\end{align}
Note that the inclusion of an anti-parallel $i=j$ term is theoretically possible but resolves to zero. This is because in this case the $J$ and $K$ term will cancel:
\begin{align}
    K_{ii}^{\alpha\beta} &= f_{ab}(1,2) \bigg[
    \psi_i^\alpha(1)^*\psi_i^\beta(2)^*\psi_i^\beta(2)\psi_i^\alpha(1)
    + \psi_i^\alpha(2)^*\psi_i^\beta(1)^*\psi_i^\beta(1)\psi_i^\alpha(2)
    \bigg]  \\
    &=f_{ab}(1,2) \bigg[
    |\psi_i^\alpha(1)|^2|\psi_i^\beta(2)|^2
    + |\psi_i^\alpha(2)|^2|\psi_i^\beta(1)|^2
    \bigg] \\
    &= J_{ii}^{\alpha\beta}
\end{align}


\section{Important notes}

\textbf{Spin Contamination} One very important drawback when approximating the multielectron wavefunction with Kohn-Sham wavefunctions as in Eq.~\ref{zfs:eq:10} is the affect of spin-contamination. In principle, the sum over multiple states above may not be necessary and terms where $i=j$ for anti-parallel should cancel to zero. However, due to the construction of a Slater determinant from spin-unrestricted DFT calculations, the wavefunctions are no longer eigenfunctions of the total spin operator $S^2$, thereby introducing error into the calculation of ZFS (see Ref.~\cite{biktagirov2018calculation} for more detail). Hence a more appropriate form of Eq.~\ref{zfs:eq:48} should be written as:
\begin{align}
    D_{ab}
    =&\frac{1}{S(2S-1)} \left[
        \frac{1}{2}(\sum_{i>j}^{\alpha}J_{ij}^{\alpha} - K_{ij}^{\alpha})
        +\frac{1}{2}(\sum_{i>j}^{\beta}J_{ij}^{\beta} - K_{ij}^{\beta})
        -\frac{1}{2}(\sum_{i}^{\alpha}\sum_{j}^{\beta}J_{ij}^{\alpha\beta}-K_{ij}^{\alpha\beta})
        \right]
\end{align}
Therefore there are $\binom{\alpha+\beta}{2}$ number of $J_{ij}-K_{ij}$ terms to be computed. This form is more explicit to what is actually implemented in the code.
\medskip

\textbf{Convert $G$ vectors to Cartesian} A second important note, is that in most plane-wave basis codes the $G$-vectors will be defined in 3D on a crystal basis (so the components are simple integers). However, when computing the sum in Eq.~\ref{zfs:eq:15} it is essential (for non-simple cubic systems) to convert the $G$-vectors to the Cartesian basis. For example if $g = (n,l,m)$ in the crystal basis ($n,l,m\in \mathbb{Z}$), we can convert it to it's Cartesian form as:
\begin{align}
    \begin{bmatrix}
        b_{1x} & b_{2x} & b_{3x} \\
        b_{1y} & b_{2y} & b_{3y} \\
        b_{1z} & b_{2z} & b_{3z}
    \end{bmatrix}
    \begin{bmatrix}
        n \\ l \\ m
    \end{bmatrix}
    =
    \begin{bmatrix}
        n b_{1x} + l b_{2x} + m b_{3x} \\
        n b_{1y} + l b_{2y} + m b_{3y} \\
        n b_{1z} + l b_{2z} + m b_{3z}
    \end{bmatrix}
\end{align}

\textbf{Diagonalize $D_{ab}$} The resulting matrix from the approach detailed above will be in the Cartesian basis and therefore in order to obtain the ZFS parameters $D_x, D_y, D_z $, one must compute the eigenvalues and eigenvectors of the matrix $D_{ab}$. The eigenvector corresponding to $D_z$ is known as the principal axis and can be identified as the eigenvector with the largest eigenvalue (and should also be fairly recognizable given the symmetry of the defect and the orientation of the defect orbitals). In many cases, such as cases with $C_{3v}$, or $D_{3h}$ symmetry, etc. the eigenvalues $D_x$ and $D_y$ will be equivalent. Distinguishing the two can be done in terms of considering the cross product of their eigenvectors ($\hat{v}_{D_x}$ and $\hat{v}_{D_y}$, respectively) should satisfy $\hat{v}_{D_x}\times\hat{v}_{D_y}=\hat{v}_{D_z}$.
\medskip


% \subsubsection*{References}
% \begin{enumerate}
%     \item ``Electronic paramagnetic resonance: Elementary theory and practical applications'', J. A. Weil, J. R. Bolton and J. E. Wertz, (John Wiley \& Sons, New York, 1994).
%     \item ``First principles method for the calculation of zero-field splitting tensors in periodic systems'', M. J. Rayson and P. R. Briddon, \textit{Physical Review B} \textbf{77}, 035119 (2008).
%     \item ``Designing defect-based qubit candidates in wide-gap binary semiconductors for solid-state quantum technologies'', H. Seo, H. Ma, M. Govoni and G. Galli, \textit{Physical Review Materials} \textbf{1}, 075002 (2017).
%     \item ``Pressure and temperature dependence of the zero-field splitting in the ground state of NV centers in diamond: A first-principles study'', V. Ivády, T. Simon, J. R. Maze, I. A. Abrikosov and A. Gali, \textit{Physical Review B} \textbf{90}, 235205 (2014).
%     \item ``First principles calculation of spin-related quantities for point defect qubit research'', V. Ivády, I. A. Abrikosov and A. Gali, \textit{npj Computational Materials} \textbf{4}, 76 (2018).
%     \item ``First-principles calculations of zero-field splitting parameters'', D. Ganyushin and F. Neese, \textit{The Journal of Chemical Physics} \textbf{125}, 024103 (2006).
%     \item ``Calculation of spin-spin zero-field splitting within periodic boundary conditions: Towards all-electron accuracy'', T. Biktagirov, W. G. Schmidt, and U. Gerstmann, \textit{Physical Review B} \textbf{97}, 115135 (2018).
% \end{enumerate}
% \medskip
%
% \hline

\section{Supplemental Derivation}
% \setcounter{equation}{0}
% \renewcommand{\theequation}{A\arabic{equation}}
% \renewcommand{\eqref}[1]{equation \ref{#1}}

\textbf{Proof of Eq.~\ref{zfs:eq:3}}
Recall $\svar{x}=\svar{1x}+\svar{2x}$ and $[\svar{1x},\svar{2x}]=0$.
\begin{align}
    \svar{x}^2 &= \svar{1x}^2+\svar{2x}^2+2\svar{1x}\svar{2x}
\end{align}
Therefore,
\begin{align}
    \svar{1x}\svar{2x} &= \frac{1}{2} \svar{x}^2 - (\svar{1x}^2+\svar{2x}^2)\, ,
\end{align}
and since $\svar{1x}$ and $\svar{2x}$ represent individual pauli matrices, it can be easily shown that $\svar{1x}^2=\svar{2x}^2=(1/2)^2$. This gives:
\begin{align}
    \svar{1x}\svar{2x} &= \frac{1}{2} \svar{x}^2 - \frac{1}{4}\mathbb{1}\, .
\end{align}
The same is true for $(y,z)$ and components.


\textbf{Proof of Eq.~\ref{zfs:eq:4}}
The anti-commutator (denoted as $\{,\}$) of $\svar{x}$ and $\svar{y}$ is given by
\begin{align}
    \{\svar{x},\svar{y}\} &= \svar{x}\svar{y}+\svar{y}\svar{x} \nonumber \\
    &= (\svar{1x}+\svar{2x})(\svar{1y}+\svar{2y})+(\svar{1y}+\svar{2y})(\svar{1x}+\svar{2x}) \nonumber \\
    &=\svar{1x}\svar{1y}+\svar{1y}\svar{1x}+\svar{2x}\svar{2y}+\svar{2y}\svar{2x}
    +\svar{1x}\svar{2y}+\svar{2x}\svar{1y}+\svar{1y}\svar{2x}+\svar{2y}\svar{1x} \nonumber \\
    &= \{\svar{1x},\svar{1y}\} + \{\svar{2x},\svar{2y}\} +
    2(\svar{1x}\svar{2y}+\svar{1y}\svar{2x}) \nonumber \\
    &= 2(\svar{1x}\svar{2y}+\svar{1y}\svar{2x}) \, .
\end{align}
Where the anti-commutator relation $\{\svar{1a},\svar{1b}\}=0$ for $a\neq b$ was used and we have shown
\begin{align}
    \svar{1x}\svar{2y}+\svar{1y}\svar{2x} = \frac{1}{2}(\svar{x}\svar{y}+\svar{y}\svar{x}) \, .
\end{align}
The same is true for other combinations of $(x,y,z)$.


\textbf{Substitution between Eq.~\ref{zfs:eq:8} and Eq.~\ref{zfs:eq:13}}
The transformation of coordinates requires calculating the Jacobian for integration.
\begin{align}
    dr_{1x}\,dr_{2x} &=
    \begin{vmatrix}
        \partial r_{1x} /\partial r_x & \partial r_{1x} /\partial R_x \\
        \partial r_{2x} /\partial r_x & \partial r_{2x} /\partial R_x \\
    \end{vmatrix}
    dr_x\,dR_x \nonumber \\
    &=
    \begin{vmatrix}
         1/2 &  1/2 \\
        -1/2 &  1/2 \\
    \end{vmatrix}
    dr_x\,dR_x \nonumber \\
    &= \frac{1}{2}\, dr_x\,dR_x
\end{align}
Repeating for $y$ and $z$ components gives a total factor of $1/8$.


\textbf{Integration over $\bR$ in Eq.~\ref{zfs:eq:13}}
Consider the part of $I_{ab}$ from Eq.~\ref{zfs:eq:13} which depends on $R$:
\begin{align}
    I_{\bR} = \int_{\Omega} e^{i[\bR\cdot(\bG_1+\bG_2)]/2}\, d\bR
\end{align}
For simplicity consider the $x$ component which integrates over lattice vector ($\mathbf{a}$):
\begin{align}
    I_{R_x} &= \int_{0}^{a} e^{i[R_x(G_{1x}+G_{2x})]/2}\, dR_x \nonumber \\
    &= \frac{-2i}{G_{1x}+G_{2x}}\left[ e^{i[a(G_{1x}+G_{2x})]/2} - 1 \right]
\end{align}
As $\bG_1$ and $\bG_2$ represent reciprocal lattice vectors they both have the form:
\begin{align}
    \bG = v_1\mathbf{b}_1 + v_2\mathbf{b}_2 + v_3\mathbf{b}_3
\end{align}
where the reciprocal lattice vectors $\mathbf{b}_i$ obey the property $\mathbf{b}_i\cdot\mathbf{a}_j=2\pi\delta_{ij}$. This will result in
\begin{align}
    I_{R_x} &= \frac{-2i}{G_{1x}+G_{2x}}\left[ e^{2\pi i k} - 1 \right]
\end{align}
where $k \in \mathbb{Z}$. Therefore the integral equates to zero except under the condition that $G_{1x}=G_{2x}$. In this limit we have:
\begin{align}
    \lim_{\eta\to 0} \frac{-2i}{\eta}\left[ e^{ia \eta} -1 \right]
    &\overset{LH}{=} \lim_{\eta\to 0} -2i (ia) e^{ia \eta} \nonumber \\
    &= 2a
\end{align}
Carrying out the integration over $y$ and $z$ will result in the volume of the unit cell and force $G_{1}=G_{2}$. This gives the final result:
\begin{align}
    I_{\bR} = \int_{\Omega} e^{i[\bR\cdot(\bG_1+\bG_2)]/2}\, d\bR = 8\Omega \delta_{\bG_1,\bG_2}
\end{align}


\textbf{Integration over $\br$ in Eq.~\ref{zfs:eq:14}}
Consider the solution to the following integral:
\begin{align}
    \int e^{i\bG\cdot\br} \frac{1}{|\br-\bC|} \, d\br = \frac{4\pi}{G^2}e^{i\bG\cdot\bC}\, .
\end{align}
Taking the second derivative ($a,b\in \{x,y,z\}$):
\begin{align}
    -\frac{\partial^2}{\partial C_a \partial C_b} \int e^{i\bG\cdot\br} \frac{1}{|\br-\bC|} \, d\br
    = \frac{4\pi G_a G_b}{G^2}e^{i\bG\cdot\bC}\, .
\end{align}
Meanwhile,
\begin{align}
    -\frac{\partial^2}{\partial C_a \partial C_b} \frac{1}{|\br-\bC|}
    = \frac{\delta_{ab}}{r^3} - \frac{3r_a'r_b'}{r^5} + \frac{4\pi\delta_{ab}}{3}\delta(\br)\, .
\end{align}
Therefore,
\begin{align}
    \int e^{i\bG\cdot\br} \frac{r^2\delta_{ab} - 3r_a'r_b'}{r^5}\, d\br =
    \frac{4\pi G_a G_b}{G^2}e^{i\bG\cdot\bC} + \frac{4\pi}{3}\delta_{ab} \, ,
\end{align}
and the final form fo $I_{ab}$ is obtained:
\begin{align}
    I_{ab} = 4\pi\Omega \sum_\bG \rho(\bG,-\bG) \left( \frac{G_aG_b}{G^2} - \frac{\delta_{ab}}{3} \right)
\end{align}


\textbf{Convolution Theorem}

Consider the Fourier transform of $\psi$.
\begin{align}
    \psi(\br) = \frac{1}{\Omega}\sum_{\bG} \psi(\bG)\, e^{i\bG\cdot\br} \quad \quad \text{and} \quad \quad
    \psi(\bG) = \int_\Omega \psi(\br)\, e^{-i\bG\cdot\br} \, d\br
\end{align}
The product $\psi_j^*(\br)\psi_i(\br)$ therefore has the form:
\begin{align}
    \psi_j^*(\br)\psi_i(\br) = \frac{1}{\Omega^2}\left( \sum_{\bG'} \psi_i(\bG') e^{i\bG'\cdot\br} \right)
    \left( \sum_{\bG''} \psi_j^*(\bG'') e^{-i\bG''\cdot\br} \right) \, .
\end{align}
The Fourier transform of the product of these wavefunctions is thus given by:
\begin{align}
    \mathcal{F}\left\{ \psi_j^*(\br)\psi_i(\br) \right\} &= \frac{1}{\Omega^2}
    \int_\Omega \left[ \left( \sum_{\bG'} \psi_i(\bG') e^{i\bG'\cdot\br} \right)
    \left( \sum_{\bG''} \psi_j^*(\bG'') e^{-i\bG''\cdot\br} \right) \right]
    e^{-i\bG\cdot\br} \, d\br \nonumber \\
    &= \frac{1}{\Omega^2} \sum_{\bG'} \psi_i(\bG') \left[ \sum_{\bG''} \psi_j^*(\bG'') \left(
    \int_\Omega e^{i(\bG'-\bG''-\bG)\cdot\br} \, d\br \right) \right] \nonumber \\
    &= \frac{1}{\Omega} \sum_{\bG'} \psi_i(\bG') \left[ \sum_{\bG''} \psi_j^*(\bG'')
    \, \delta(\bG'-\bG''-\bG) \right] \nonumber \\
    &= \frac{1}{\Omega} \sum_{\bG'} \psi_i(\bG') \psi_j^*(\bG-\bG') \, .
\end{align}
Therefore we can compute $f_1(\bG)$, $f_2(\bG)$, $f_3(\bG)$ as:
\begin{align}
    f_1(\bG) = \frac{1}{\Omega} \sum_{\bG'} \psi_1^* (\bG-\bG') \psi_1(\bG') \, , \quad
    f_2(\bG) = \frac{1}{\Omega} \sum_{\bG'} \psi_2^* (\bG-\bG') \psi_2(\bG') \, , \quad \nonumber\\
    f_3(\bG) = \frac{1}{\Omega} \sum_{\bG'} \psi_1^* (\bG-\bG') \psi_2(\bG') \, .
    \hspace{3cm} \label{eq:fG}
\end{align}


% molecular orbital
% jahn teller
% codes

% \chapter{PW2PY}


\nocite{*}
% \bibliographystyle{plain}
\bibliographystyle{ieeetr}
\singlespacing
\bibliography{ref}

\doublespacing

\end{document}
