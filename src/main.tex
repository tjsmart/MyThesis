\documentclass[12pt]{../include/ucthesis}
\input{../include/macros}

\begin{document}

\title{Computational Design of Materials from First-Principles: From energy conversion to quantum information}
\author{Tyler J. Smart}
\degreeyear{2021}
\degreemonth{June}
\degree{DOCTOR OF PHILOSOPHY}
\chair{Prof.\ Frank Bridges}
\committeememberone{Prof.\ Yuan Ping}
\committeemembertwo{Prof.\ Yat Li}
%\committeememberthree{Professor 4} % Uncomment if you have 4 committee members. Also change the next line to `\numberofmembers{4}`
\numberofmembers{3}
\deanlineone{Quentin Williams}
\deanlinetwo{Acting Vice Provost and Dean of Graduate Studies}
\deanlinethree{}
\field{Physics}
\campus{Santa Cruz}

\begin{frontmatter}
\maketitle

\copyrightpage

\tableofcontents

\listoffigures

\listoftables

\begin{abstract}
    State-of-the-art first-principles calculations are implemented and utilized in order to optimize or give insight into material properties for various techonologies and applications. These applications, and therefore the research, have two primary focuses: energy conversion and storage using transition metal oxides and quantum defects in two-dimensional materials. Regarding the former, the limitation of transition metal oxides is in their poor carrier conduction due to the formation of small polarons. Here, detail and insight are given into the formation of small polarons, especially under atomic doping wherein doping can help to improve carrier conduction in these systems and offer better energy conversion. The latter application, has an emphasis in the design of single photon emitters and spin-based qubits in hexagonal boron nitride. In particular, defect ionization energies, optoelectronic properties, and nonradiative propreties, including intersystem crossing, is discussed. Finally, selected topics are presented which does not necessarily fall into these two categories but is otherwise interesting to discuss.
\end{abstract}

\begin{dedication}
\vspace*{\fill}
\begin{center}
Enter dedication here.
\end{center}
\vspace*{\fill}
\end{dedication}

\begin{acknowledgements}
Enter acknowledgements here.
\end{acknowledgements}

\end{frontmatter}

% ------------------------------------------------------------
% intro
% //////////////////////////////////////////////////////////|
% ++++++++++++++++++++++++++++++++++++++++++++++++++++++++++|
% %%%%%%%%%%%%%%%%%%%%%%%%%%%%%%%%%%%%%%%%%%%%%%%%%%%%%%%%%%|
%    CHAPTER                                                |
% %%%%%%%%%%%%%%%%%%%%%%%%%%%%%%%%%%%%%%%%%%%%%%%%%%%%%%%%%%|
% ++++++++++++++++++++++++++++++++++++++++++++++++++++++++++|
% \\\\\\\\\\\\\\\\\\\\\\\\\\\\\\\\\\\\\\\\\\\\\\\\\\\\\\\\\\|
\chapter{Introduction}

% general "big picture" motivation, will introduce briefly both renewable energy and quantum information science motivations
Understanding and developing material properties is essential to solving some of societies greatest concerns. One such concern of particular interest is the desperate need for renewable energy sources.
Advancing novel technologies for energy harvesting, conversion, and storage is critical to ensure the economic viability of U.S. energy and chemical industries.~\cite{eia,renewable} For many of these technologies, a detailed understanding of chemical processes at electrochemical interfaces is essential. For instance, optimizing water splitting reactions at the semiconductor-water interface in photoelectrochemical cells is key for improving the device efficiency and stability for generating hydrogen fuel from water and sunlight.~\cite{mccrory2015} Alternatively, the causes of oxidation and corrosion at the interface can be illuminated via chemical degradation processes.~\cite{huang2019} Last but not least, understanding the relationship between reactivity and electronic properties of liquid electrolytes at the interface with electrode materials is also one of the prerequisites for manipulating the electrochemical stability of electrode-electrolyte interfaces in ion batteries and supercapacitors.~\cite{wang2018}

However, the development of renewable energy resources is only a single example of the type of problem in which material design is crucial. As another example, material science is unsurprisingly at the very heart of developing brand new technoligies, particularly those in the realm of quantum information science. The development of innovative quantum technologies is immanent and will make broad impacts on our national technology sector.~\cite{quantum} For example, point defects in two-dimensional materials are hosts for emerging quantum phenomena such as single-photon emitters and defect-based spin qubits. Both of these technologies neccesitate the development of material design.

With the intention to expand these fields, the role of computational material science has grown immensely alongside ever-growing supercomputing facilities. These facilities enable calculations of large-scale simulations which provide improved theoretical understanding of these aforementioned fields. In particular, first-principles simulations allow us to better understand the quantum mechanical nature of materials, which is an essential part of their application. These simulations have proven to be pivotal in the evolution of many fields all the way from renewable energy to quantum technology.

As such, my research in modeling materials from first-principles calculations is bifurcated into two branches of motivation: 1.\ renewable energy and energy conversion with transition metal oxides (TMOs) and 2.\ quantum information sciences in two-dimensional (2D) materials. In this dissertation I will discuss my research within both of these fields by covering the motivation behind my research, discussing progress achieved thus far and finally how these efforts have culiminated or are being extended.



% %%%%%%%%%%%%%%%%%%%%%%%%%%%%%%%%%%%%%%%%%%%%%%%%%%%%%%%%%%
%    section
% %%%%%%%%%%%%%%%%%%%%%%%%%%%%%%%%%%%%%%%%%%%%%%%%%%%%%%%%%%
\section{Density Functional Theory}

\subsection{Background}

In the interrelated field of physics, chemistry, and material science, there is no greater problem than that of the electron. The electron can determine so many of the properties of material, from its ability to absorb light, conduct electrical currents, thermally or electrically insulate, and so much more. Hence, in order to have a grasp on fundamental material properties we must understand the electron and the quantum mechancial nature by which it lives by solving the multi-electron Schr{\"o}dinger equation (SE). The multi-electron problem essentially refers to any problem involving interactions between more than one electron, often in an external field. In principal, the problem is well understood in the formalism of Schr{\"o}dinger quantum mechanics, the Hamiltonian of the multi-electron system in a material with ion potential within the Born-Oppenheimer approximation (rigid ion approximation) is given by:
\begin{align}
    \left[ -\frac{\hbar^2}{2m} \sum_i \nabla_i^2 + \sum_i V_{ext} (\textbf{r}_i) + \frac{1}{2} \sum_{i\neq j} \frac{e^2}{|\textbf{r}_i-\textbf{r}_j|} \right] \Psi(\textbf{r}_1,\textbf{r}_2,\ldots\textbf{r}_N) = E\ \Psi(\textbf{r}_1,\textbf{r}_2,\ldots\textbf{r}_N)
    \label{eq:multi-se}
\end{align}
However, this has its immediate challenges. Namely, even the classical (non-quantum) three body problem has no general solution! Clearly a non-analytical approach is needed. Such a non-analytical approach is discussed below and is the foundation of all density functional theory (DFT) calculations.

\begin{figure}[h]
\begin{center}
\includegraphics[keepaspectratio=true,width=\linewidth]{1-intro/figures/mean-field.png}
\caption{Density functional theory is a mean field approach which replaces the many-electron problem with one of a single electron interacting with a mean-field electron density at a significantly reduced complexity while maintaining remarkable predictive power.}  \label{intro:fig:mean}
\end{center}
\end{figure}

\subsection{Hohenberg Kohn and Sham}

Typically, one may envision that given an external potential $V_{ext}(\textbf{r})$, one may solve the multi-electron SE as in Eq.~\ref{eq:multi-se}, determining all of the eigenstates of the SE, $\Psi_i({\textbf{r}})$. This would include a ground state $\Psi_0({\textbf{r}})$ wavefunction and corresponding ground-state density $n_0(\textbf{r})$. This process logically demonstrates that given an external potential, a unique ground state density can be found ($V_{ext}(\textbf{r})\Rightarrow n_0(\textbf{r})$). Hohenberg and Kohn's first theorem,~\cite{hohenberg1964inhomogeneous} demonstrates that the reverse is also true, namely that given a ground state denisty, one can find (up to a constant), a unique external potential, \textit{e.g.}\ $n_0(\textbf{r}) \Rightarrow V_{ext}(\textbf{r})$. In other words, all properties of the system are completely determined by the ground state density. Secondly, Hohenberg and Kohn defined that a universal functional of the energy $E[n]$ can be constructed for any external potential. And the ground state density is a global minimum of this functional, $E[n]\geq E[n_0]$.


\begin{figure}[h]
\begin{center}
\includegraphics[keepaspectratio=true,width=\linewidth]{1-intro/figures/hks.png}
    \caption{Top left, the true multi-particle potential and wavefunctions are replaced by an auxiliary single-particle system, as in the top right. At the end the Kohn-Sham method involves solving the Schr{\"o}dinger equation for the auxiliary Hamiltonian ($H_{KS}$) as defined on the bottom panel. Adapted from Ref.~\cite{martin2020electronic}.}  \label{intro:fig:hks}
\end{center}
\end{figure}

Following these theorems by Hohenberg and Kohn, Kohn and Sham were able to demonstrate the very basis of density functional theory.~\cite{kohn1965self} Namely, they proved that there exists an auxiliary single-particle Hamiltonian ($H_{KS}$) with the exact same electron density ($n_0(\textbf{r})$) as would be obtained by solving the multi-particle system. Schematically the theorems of Hohenberg and Kohn (HK) as well as Kohn and Sham (KS) are shown in Figure~\ref{intro:fig:hks}.

\subsection{Self-Consistent Approach}

Today, there are many density functional theory (DFT) codes which are built upon the theory presented by Hohenberg, Kohn, and Sham. The exact procedure is shown in Figure~\ref{intro:fig:scf}. Specifically, an initial guess of the electron density is constructed. Here a basis set of gaussians (for molecular systems) or plane waves (for crystal systems) is constructed to represent the density and make the computation efficient and cheap. The typical guess of the electron density may involve a completely random guess or one utilizing a linear combination of atomic orbitals (or both). From this initial guess, the effective potential can be constructed, which is then solved by typical diagonalization methods such as Davidson or conjugate gradient. This will yield eigenfunctions ($\psi_i(\textbf{r})$) and a new electron density $n(\textbf{r})$. These electron density can be checked for self consistency (for example do they produce a similar total energy) and if not then they are mixed to create a new guess. Once self-consistency is reached we can obtain energy, forces, stress, eigenvalues, and so much more from the DFT approach.

\begin{figure}[h]
\begin{center}
\includegraphics[keepaspectratio=true,width=0.5\linewidth]{1-intro/figures/scf-loop.png}
    \caption{The procedure of the self-consistent loop implemented in various density functional theory codes existing today.}  \label{intro:fig:scf}
\end{center}
\end{figure}


\subsection{Electron-Electron interactions}

In the auxiliary approach, there is a single functional component which is expected to capture the full many-body interacting electron problem, the exchange-correlation energy $E_{xc}[n]$. And much of the success of density functional theory must be paid to the success in finding an approximate exchange-correlation functional which yields reliable results.
To cover all the many methods of exchange and correlation would be an unbearably difficult task. In practice, the most popular method by far is that of the local density approximation (LDA) which uses the approximate form of exchange-correlation one can obtain for an electron gas. Alternatively, the PBE functional is an immensely popular GGA (generalized gradient approximation) which is widely used today.~\cite{perdew1996generalized}

The shortcoming of LDA and GGA, comes for systems where the electrons in the system deviate significantly from that of an electron gas. For example, transition metal oxides, which possess $3d$ orbitals exhibit strong correlation, and have been more successfully treated by including a Hubbard correction.~\cite{dudarev1998electron}
\begin{align}
    E_{\text{DFT}+\text{U}}=E_{\text{DFT}}+\frac{U_{\text{eff}}}{2}\sum_{I,\sigma}\sum_{i}\lambda_{i}^{I\sigma}(1-\lambda_{i}^{I\sigma})
    \label{intro:eq:U}
\end{align}
Here, $E_\text{DFT}$ is the energy obtained from standard DFT methods which is corrected by the following term which includes the occupation matrix $\lambda_i^{I\sigma}$ ($I$ ranges over all ions, $i$ ranges over $3d$ orbitals and $\sigma$ is for spin up or down). For example, we have shown that applying a $U$ of 4.3 eV on Fe $3d$ orbitals in \ce{Fe2O3} yields bandgap, electron localization, hopping barrier, and ionization energies which agree with experiment.~\cite{smart2017effect}

An alternative method, is built upon mixture of the semi-local PBE exchange-correlation functional and that of the non-local exact Hartree-Fock exchange:
\begin{align}
    E_x^{HF} = -\frac{1}{2} \sum_{i,j} \int\int \psi_i^*(\textbf{r}_1)\psi_j^*(\textbf{r}_2) \frac{1}{r_{12}} \psi_j(\textbf{r}_1)\psi_i(\textbf{r}_2) d\textbf{r}_1 d\textbf{r}_2
    \label{intro:eq:HF}
\end{align}
A popular and most simple hybrid functional method (PBE0($\alpha$)) mixes $E_x^{HF}$ with that obtained from PBE:
\begin{align}
    E_{x}^{PBE0}(\alpha) = (1-\alpha) E_x^{PBE} + \alpha E_x^{HF},
    \label{intro:eq:PBE0a}
\end{align}
where most commonly $alpha=0.25$ (denoted simply PBE0). Other variations of hybrid functionals, including HSE, B3LYP, and B3PW, will be discussed in necessary detail as they pertain to the research in later sections.


More exact methods of including many-body interactions include the GW approximation. In the one shot $G_0W_0$ (here all cases will be oneshot, so I will simply write GW instead of explicitly $G_0W_0$) approach the self energy $\Sigma$ pertubatively replaces the XC functional obtained in DFT. First the single particl Green's function is constructed:
\begin{align}
    G_0 = \sum_i \frac{\varphi_i(r)\varphi_i^*(r')}{\omega-\varepsilon_i \pm i\eta },
    \label{intro:eq:g0}
\end{align}
where $\varphi_i$ and $\varepsilon_i$ are eigenfunctions and eigenvalues obtained from DFT. Then the screened Coulomb interaction can be obtained where the vertex $\Gamma = 1$, $W = v/(1+\chi_0 v) = \epsilon^{-1} v$. Here the polarizability is obtained directly from the Green's function $P_0=G_0G_0$. The new new self energy is then obtained as $\Sigma = iG_0W_0$, and quasi-particle corrections are obtained from 1$^\text{st}$ order perturbation theory with $V_p = \Sigma - V_{xc}$.



% %%%%%%%%%%%%%%%%%%%%%%%%%%%%%%%%%%%%%%%%%%%%%%%%%%%%%%%%%%
%    section
% %%%%%%%%%%%%%%%%%%%%%%%%%%%%%%%%%%%%%%%%%%%%%%%%%%%%%%%%%%
\section{Formalism of Charged Defect Formation}

The below research takes special interst into the formation of impurities into crystal lattices. These impurities could include native vacancies, interstitials, and antisites but also extrinsic dopants and localized carriers (small polarons will be discussed later).
The most fundamental properties of these impurities are their formation energy (how easily the impurity can form in the lattice) and their ionization energy (how easily the impurity changes charge state and contributes electrons or holes).
Below an overview is provided on computing formation energies from First-Principles.

\subsection{Elemental Chemical Potentials}
In order to evalute the formation of atomic impurities the source of the impurities need to be evaluated in the form of a chemical potential. Rather than introduce this notion abstractly, below I present the procedure for obtaining chemical potential energies for the \ce{CsPbBr3} compound.
For, \ce{CsPbBr3} the chemical potential of atomic Cs, Pb, and Br can be evaluated by determining the stability of the parent compound \ce{CsPbBr3} against its byproducts.
Namely in thermodynamic equilibrium growth conditions, the chemical potentials $\mu_{\rm Cs}$, $\mu_{\rm Pb}$ and $\mu_{\rm Br}$ must satisfy Eq.~\ref{intro:eq:chem_parent}$-$\ref{intro:eq:chem_byproduct}:
\begin{align}
    & \Delta \mu_{\rm Cs} + \Delta \mu_{\rm Pb} + 3\Delta \mu_{\rm Br} = \Delta H_{\ce{CsPbBr3}} \label{intro:eq:chem_parent} \\
    & i\Delta \mu_{\rm Cs} + j\Delta \mu_{\rm Pb} + k\Delta \mu_{\rm Br} \leq \Delta H_{ {\rm Cs}_i {\rm Pb}_j {\rm Br}_k}, \hspace{0.5cm} (i, j, k) \in \mathbb{N}. \label{intro:eq:chem_byproduct}
\end{align}
Here $\Delta \mu_{\rm X}$ is the chemical potential of species $X$ referenced to its most stable phase.
In Eq.~\ref{intro:eq:chem_byproduct}, ${\rm Cs}_i {\rm Pb}_j {\rm Br}_k$ refers to possible byproducts of \ce{CsPbBr3}, e.g.\ \ce{Cs}, \ce{Pb}, \ce{Br2}, \ce{CsBr}, \ce{CsBr3}, \ce{PbBr2}, \ce{Cs4Pb9}, and \ce{Cs4PbBr6}.
From Eq.~\ref{intro:eq:chem_parent}$-$\ref{intro:eq:chem_byproduct} and considering each of these byproducts we obtained a phase diagram as shown in the main text Figure 2a. The results are in good agreement with previous reported diagrams for \ce{CsPbBr3}.~\cite{kang2017high,li2019sodium}
More details can be found at \url{https://github.com/Ping-Group-UCSC/PhaseDiagram} and in particular see \href{https://github.com/Ping-Group-UCSC/PhaseDiagram/blob/main/Examples/CsPbBr3-PhaseDiagram/diagram.ipynb}{the \ce{CsPbBr3} tutorial} (NOTE: at the time of this writing, this link is only available internally within the Ping Group).


\subsection{Defect Formation Energy and Ionization Energy}
The charge defect formation energy provides insight into the charge states of dopants providing some insight into the influence on carrier concentration.
We computed the charge defect formation energy ($E^f_q$) of each defect system according to:
\begin{align}
    E^f_q(X; \varepsilon_F) = E_q(X) - E_{prist} + \sum_i \mu_i \Delta N_i + q \varepsilon_F + \Delta_q,
    \label{intro:eq:cfe}
\end{align}
where $E_q(X)$ is the total energy of the defect system ($X$) with charge $q$, $E_{prist}$ is the total energy of the pristine system, $\mu_i$ and $\Delta N_i$ are the chemical potential and change in the number of atomic species $i$, and $\varepsilon_F$ is the electron chemical potential. A charged defect correction $\Delta_q$ must be computed for charged cell calculations. For the cases presented here, with the JDFTx code~\cite{JDFTx} by employing the techniques developed in Ref.~\cite{wu2017first,ping2013}. Meanwhile, chemical potentials can be carefully evaluated against the stability of byproduct compounds as detailed above.
Finally, the corresponding charge transition levels of the defects can be obtained from the value of $\varepsilon_F$ where the stable charge state transitions from $q$ to $q'$.
\begin{align}
    \epsilon_{q|q'} = \frac{E^f_q - E^f_{q'}}{q' - q}
    \label{eq:ctl}
\end{align}
Typically, for a semiconductor or insulator the ionization energy of a p-type/n-type dopant is given by the value(s) of its charge transition level(s) referenced to the valence/conduction band edge of the host materials. However, in systems which form small polarons the ionization energy should be referenced to the free polaron state.~\cite{smart2017effect,seo2018role} For example, the free electron small electron polaron level is defined as the $\epsilon_{0|-1}$ transition level in the pristine system.



\subsection{Defect Concentration}
In order to simultaneously consider defect, dopant, and carrier fomrations, I implemented the procedure of defect concentrations via a self-consistent approach based on charge neutrality. Following the formalism presented in Ref.~\cite{freysoldt2014first}, the charged defect concentration ($c_q$) is computed as:
\begin{align}
    c_q(X; \varepsilon_F) = g \exp [- E^f_q(X; \varepsilon_F) / k_B T],
\end{align}
where $g$ is the degeneracy factor accounting for the internal degrees of freedom of the point defect, $k_B$ is the Boltzmann factor, and $T$ is temperature. In order to maintain neutrality, the introduction of defect $X$ with charge $q$ into the lattice must be compensated by defects of opposing charge or through the generation of free carriers. Specifically charge neutrality must be held:
\begin{align}
    \sum_{X,q} c_q(X; \varepsilon_F) + n_h - n_e = 0, \label{intro:eq:neutrality}
\end{align}
where the concentration of free delocalized holes ($n_h$) and free delocalized electrons ($n_e$) can be evaluated via:
\begin{align}
    n_e - n_h = \int_{-\infty}^{\infty} dE \frac{D(E)}{1+\exp[(E-\varepsilon_F)/k_B T]}.
\end{align}
Here $D(E)$ is the electronic density of states of the pristine system. Eq.~\ref{intro:eq:neutrality} can be evaluated by standard root-finding algorithms to obtain $\varepsilon_F$ where charge neutrality is held.
Note, for systems where free carriers will localize into small polarons, the formation of free electron small polarons is entered in a similar way to a defect, i.e.\ with a formation energy.
Finally, in order to relate to experimental measurements, concentrations are first computed at a synthesis temperature ($T_S$) and then charge neutrality is recomputed at room temperature ($T_O=300$ K) while fixing the total defect concentration to that obtained at synthesis condition as employed by Ref.~\cite{lee2013thermodynamics}.
The software computing defect concentrations cane be found (for Ping Group members) at \url{https://github.com/Ping-Group-UCSC/DefectConcentration}.


% %%%%%%%%%%%%%%%%%%%%%%%%%%%%%%%%%%%%%%%%%%%%%%%%%%%%%%%%%%
%    section
% %%%%%%%%%%%%%%%%%%%%%%%%%%%%%%%%%%%%%%%%%%%%%%%%%%%%%%%%%%
\section{Defect Mediated Carrier Recombination}

\subsection{Radiative Recombination}


\subsection{Nonradiative Recombination}


\subsection{Intersystem Crossing}


\subsection{Photodynamics}

% Zero field splitting mentioned here, will go into appendix



% ------------------------------------------------------------
% tmo
\chapter{Designing Quantum Defects in Two Dimensional Materials}

% %%%%%%%%%%%%%%%%%%%%%%%%%%%%%%%%%%%%%%%%%%%%%%%%%%%%%%%%%%
%    Section
% %%%%%%%%%%%%%%%%%%%%%%%%%%%%%%%%%%%%%%%%%%%%%%%%%%%%%%%%%%
\section{Overview}
Quantum technologies offer exotic and impressive capabilities in computation, sensing, and information \cite{malik2019science}. While several systems of quantum computation exist, defect based qubits offer a distinct advantage in their ability to operate optically and under room temperature conditions \cite{koehl2011room,weber2010quantum,falk2013polytype}. Furthermore defects in two-dimensional (2D) materials yield a higher-ceiling for defect based quantum technologies where spatially controlling doping, entangling qubits, and qubit tuning are all more attainable \cite{aharonovich2017quantum,sajid2020single}. In particular, two-dimensional hexagonal boron nitride ($h$-BN) has demonstrated that it can host defect-based single photon emitters (SPEs) \cite{grosso2017tunable} and qubits \cite{gottscholl2020initialization}.

As such, my work has focused on the prediction of defects in $h$-BN for quantum applications. From a computational perspective, studying defects in 2D materials offers several technical challenges. In 2018, I was awarded an NSF scholarship for studying quantum information science through a program known as QISE-NET. This program provides supplemental funding to perform ongoing research in collaboration with Argonne National Laboratory, so that we may study defects in $h$-BN (this was later highlighted in the UC Santa Cruz newsletter). In particular, these efforts culminated in our work from 2018, wherein we demonstrated how to compute the single-particle band gap of $h$-BN via a Koopmans' compliant hybrid functional approach which incorporates improved screening effects in our calculation but mitigates the expense of many-body theory based methods i.e. the GW approximation \cite{smart2018fundamental}. Additionally, this work demonstrated the layer dependence on defect ionization energies. Following this up, we then studied radiative and nonradiatiave recombination of defects in $h$-BN. One particularly interesting facet of this research in the demonstration of how significantly the nonradiative recombination of defects can be effected by strain, wherein we predicted the strain fingerprints of N$_\text{B}$V$_\text{N}$ which matches closely with subsequent experimental measurements.~\cite{mendelson2020strain}
Recently, I also implmented computing the zero-field splitting of $S \ge 1$ systems, an essential quantity in defect based-qubit systems like NV center in diamond. In addition, I have implemented computing intersystem crossing (necessary for spin initialization and readout) with spin-orbit coupling and electron-phonon interaction. With these computed static and dynamical properties,  we are able to predict spin qubits read-out efficiency and new quantum spin defect systems in hexagonal boron nitride which can be potential candidates for spin-based quantum technologies.

\section{Polaron Formation and Transport in \ce{Fe2O3}}

% shortcut for vacancy
\def\vacancy{\text{V}}
% shortcut for general vacancy defect (V_X)
\newcommand{\vac}[1]{\vacancy_\text{#1}}
% shortcut for substitution (A_B) where A substitutes B
\newcommand{\sub}[2]{\text{#1}_\text{#2}}
% shortcut for interstitial (X_i)
\newcommand{\itl}[1]{\text{#1}_i}

% shortcut for all defects
\def\vo{\vac{O}}
\def\vfe{\vac{Fe}}
\def\oi{\itl{O}}
\def\fei{\itl{Fe}}

% shortcut for delta mu
\newcommand{\dmu}[1]{\Delta\mu_\text{#1}}
% shortcut for log O2
\def\po{p_{\rm O_2}}
\def\logp{\log (\po)}
% shortcut for EP (in case we want to change notation)
\def\ep{EP}


\section{Carrier Concentrations in \ce{Fe2O3}}

\def\be{\Delta_{quad}}

\section{Dopant Clustering in \ce{Fe2O3}}
To overcome these limitations, several efforts have been made to dope hematite by tetravalent ions that yield improved photoelectrochemical performance of hematite photoelectrodes
\cite{lohaus2018limitation,biswas2020tuning,ling2011sn,li2017morphology,kumar2011electrodeposited,fu2014highly,malviya2017influence,yang2013new,tian2020electronic,liu2013ge}.
While the performance can be moderately improved via group IV and XIV dopants, the optimal doping concentration strongly varies with each individual dopant,\cite{kumar2011electrodeposited,fu2014highly,malviya2017influence,yang2013new,tian2020electronic,liu2013ge} thereby requiring extensive experimental testing each time \cite{walsh2017instilling}.
%exact optimization of doping remains obscure due to physical limitations of doping \cite{walsh2017instilling}.
%Specifically, the doping of tetravalent ions in hematite shows consistently non-monotonic behavior \cite{kumar2011electrodeposited,fu2014highly,malviya2017influence,yang2013new,tian2020electronic,liu2013ge}, although the exact doping concentration  optimal for improving performance, is seemingly random amongst dopants.

For example, researchers\cite{fu2014highly,malviya2017influence} have found that Ti-doped hematite photoanodes had the highest carrier density and photocurrent at a doping concentration of around 0.1\%. Meanwhile, several works\cite{yang2013new,tian2020electronic,bindu2018electrical} have found that optimal PEC performance with Sn-doped hematite photoanodes was achieved at 3\% Sn doping concentration. In all of these cases there is a direct correlation between optimizing carrier density and PEC performance; however, the mystery of extremely low optimal doping concentration for certain dopants remains elusive.
% There are two logical mechanisms for the stifling of the dopant effects: compensation by opposite-charged impurities or the clustering of dopants.
Two possible mechanisms could be responsible for the doping bottleneck: compensation by oppositely-charged defects or the clustering of dopants.
However, the concentration of intrinsic p-type defects is expected to be negligible in \ce{Fe2O3}\cite{lee2013thermodynamics}, which leaves a strong rationale for the clustering of dopants being the cause of low optimal doping concentration.
 % TS: I shortened this part and combined with the previous paragraph
On the theoretical side, group IV and XIV dopants in \ce{Fe2O3} have been previously investigated from first-principles, focusing on their
electronic structure, formation energy, and polaron hopping barrier\cite{smart2017effect,zhou2019silicon,zhou2015understanding,liao2011electron}.
% Although certain dopants were suggested to improve carrier concentration based on their reasonably low ionization energies\cite{smart2017effect,zhou2015understanding}, their equilibrium carrier concentrations were not computed and the interplay between dopants and intrinsic defects were not taken into account.
However, these studies cannot explain the low optimal doping concentration observed experimentally.

In this work, we will reveal the origin of the extremely low optimal doping concentration in \ce{Fe2O3} through a joint theoretical and experimental study. We suggest a novel form of dopant clustering in polaronic oxides and conclude its critical role on determining carrier concentration.
We begin by detailing our computational methodology, including our proposed model for disentangling the effects of dopant clustering.
Next, the electronic structure of isolated and clustered Sn dopant formation is provided, which resembles an electric dipole and quadrupole, respectively. The binding energy of the clustered dopants as quadrupoles is computed to validate their thermodynamic stability. Then, the formation of the theoretically predicted Sn-Sn pairs are confirmed by experimental EXAFS, and
their mechanistic origin is unraveled theoretically in terms of electrostatic, magnetostatic and strain effects.
Finally, carrier concentrations of \ce{Fe2O3} with and without dopant clustering are computed to elucidate the underlying mechanism of the doping bottleneck. At the end, essential design principles are provided to yield higher conductivity in polaronic oxides for the advancement of energy conversion applications.

\subsection{First-Principles Calculations}
All Density Functional Theory (DFT) calculations were carried out using the open source plane-wave code QuantumESPRESSO \cite{QE1} with ultrasoft pseudopotentials \cite{gbrv} and an effective Hubbard $U$ \cite{dudarev1998electron} value of 4.3 eV for Fe $3d$ orbitals \cite{smart2017effect,adelstein2014density}.
This $U$ value is chosen for its ability to reproduce the bandgap of hematite ($\sim2.21$ eV) but also has shown to capture physics of small polarons such as the polaron hopping barrier~\cite{adelstein2014density,smart2017effect}.
Plane-wave cutoff energies of 40 Ry and 240 Ry were used for wavefunctions and charge density, respectively. All calculations were performed with a $2\times 2\times 1$ supercell (120 atoms) of the hexagonal unit cell with a $2\times 2\times 2$ $k$-point mesh for integration over the Brillouin zone. A $3\times 3\times 1$ supercell was also tested to ensure convergence with supercell sizes (see supercell convergence in SI Table S1).
The consistency between these supercell sizes also validates that the present calculations are in the dilute limit and while the Sn at Fe concentrations of the aforementioned supercells are 2.08\% and 0.93\%, respectively, these concentrations do not yield interactions between dopant periodic images (even for systems with two dopants, see SI Figure S1). The actual concentrations of dopants are determined by evaluating charge neutrality directly from dopant formation energies at a synthesis condition as discussed later in this section.
Finally, we note that we use the same $U$ value for systems with dopants. While it is true that changing $U$ will vary the bandgap (in this case the conduction band shifts due to $U$ correction on Fe $3d$), it has been shown that the formation energies computed with different $U$ values were very similar and the ionization energies changed little when referenced to the free polaron level instead of the CBM for \ce{Fe2O3}.~\cite{smart2017effect}
%Any non-neutral calculations (such as charged formation energy calculations) included a charged defect correction was computed using the JDFTx code \cite{JDFTx} following the scheme developed in Ref.~\cite{ping2013}, which is necessary in order to remove the spurious interactions of the polarons with their periodic images and with the uniform compensating background charge \cite{kokott2018first}.


\subsection{Electrostatic, Magnetostatic, and Strain Model of Binding Energy}
We will later demonstrate that single dopants resemble dipoles, while dopant-pairs resemble quadrupoles. Here, we demonstrate our analysis on the physical contributions to the quadrupole binding energies computed from first-principles as we will discuss later, by using an electrostatic, magnetostatic, and strain (EMS) model.
In this model, the binding energy is obtained by separately computing electrostatic ($\be^{elec}$), magnetostatic ($\be^{mag}$), and strain ($\be^{strain}$) contributions.
For the electrostatic effect, we compute the electrostatic potential contribution to binding energy ($\be^{elec}$) by taking the difference between the quadrupole
(quad) and twice the dipole configuration (dipole):

\begin{align}
    \be^{elec} = \frac{k\alpha}{2 \epsilon_r} \left( \sum_{ij}^{quad} \frac{q_i q_j}{r_{ij}} - 2\sum_{ij}^{dipole} \frac{q_i q_j}{r_{ij}} \right).
    \label{eq:be_elec}
\end{align}
% k = 14.4   (eV*Å/e)
% eps_r = 22.9 for hematite
% alpha = 4.1719 for corundum
Here $k$ is the Coulomb constant, $\alpha$ is the Madelung constant, and $\epsilon_r$ is the relative permeability (22.9 for hematite). \cite{onari1977infrared} The summation goes over all polarons and dopants $i$ and $j$, with relative charges $q_{i}$ and $q_{j}$, and physical separation $r_{ij}$.
%\textcolor{red}{not sure this is necessary:}For example, in the dipole system there will be one term for the electrostatic interaction between the single Sn ($q=+1$) and the single $EP$ ($q=-1$). Meanwhile in the quadrupole system there will be six terms corresponding to four Sn-$EP$ interactions, one $EP$-$EP$ interaction, and one Sn-Sn interaction.

Magnetic effects were computed using the Heisenberg Hamiltonian $H_{spin}=-\frac{1}{2}\sum_{ij}J_{ij}\hat{S}_i\cdot\hat{S}_j$, where $J_{ij}$ is the magnetic coupling between the spins of the $i^{\rm th}$ and $j^{\rm th}$ ion, and $\hat{S}_i$ is the spin of the $i^{\rm th}$ ion. Here we use the magnetic exchange coupling constants computed from Ref.~\cite{nabi2010magnetic}, which provided magnetic couplings for both the superexchange between two $\rm Fe (3+)$ or between two $\rm Fe (2+)$, as well as the double-exchange between $\rm Fe (3+)$ and $\rm Fe (2+)$. For high-spin $\rm Fe (3+)$ and $\rm Fe (2+)$, the value of $\hat{S}_i$ is $5/2$ and $2$, respectively, while the spin of the tetravalent dopants is zero (hence the magnetic interaction with these dopants is always zero). In this way, we can compute the magnetic contribution to the binding energy from the magnetic energy of the quadrupole system subtracted by two times of the dipole system:
\begin{align}
    \be^{mag} = -\frac{1}{2}\left( \sum_{ij}^{quad} J_{ij} \textbf{S}_i \cdot \textbf{S}_j - 2\sum_{ij}^{dipole} J_{ij} \textbf{S}_i \cdot \textbf{S}_j\right).
    \label{eq:be_mag}
\end{align}
In this work, we assume $J_{ij}$ between two Fe ions before and after doping are the same, as lattice distortions are generally small compared to Fe distances.
%and make the approximation that changes in $J_{ij}$ due to lattice distortions are likely negligible.
With the above consideration and the non-magnetic nature of dopants, our computed magnetic energy is identical for all the dopants.
% Note the magnetic energy does not depend on the dopant when the spin of the dopant is zero. Also, we make the approximation that changes in the coupling parameters $J$ between Fe ions due to lattice distortion is negligible., our computed magnetic energy is identical for all non-magnetic dopants.

Finally, in order to compute the strain contribution to the binding energy, we evaluated the change in energy induced by lattice distortions before and after doping. As usual, taking this energy for the quadrupole system and subtracting twice the dipole system:
\begin{align}
    % \be^{strain} = \left[E_{prist} - E_{prist}(R_{quad})\right] - 2\left[E_{prist} - E_{prist}(R_{dipole})\right].
    \be^{strain} = E_{quad}^{strain} - 2 E_{dipole}^{strain}.
    \label{eq:be_strain}
\end{align}
Here $E_{X}^{strain}$ is the strain energy of system $X$, computed as a difference of total energy of the pristine system with its equilibrium geometry and with relaxed geometry from the doped system (first relax with dopants then substitute back Fe atoms to keep the same composition as pristine \ce{Fe2O3}). We note a similar approach was used in Ref.~\cite{bao2017first} to evaluate strain energies.
% Here $E_{prist}$ is the total energy of the pristine system with a relaxed structure and $E_{prist}(R_{X})$ is the total energy of the pristine system with a fixed structure corresponding to the atomic positions in the doped system. We note a similar approach was used in Ref~\cite{bao2017first}.

\subsection{Charged Defect Formation Energy and Concentration}
We computed the formation energy ($E^f_q$) of each defect at a charge state $q$ according to:
\begin{align}
    E^f_q(X; \varepsilon_F) = E_q(X) - E_{prist} + \sum_i \mu_i \Delta N_i + q \varepsilon_F + \Delta_q,
    \label{eq:cfe}
\end{align}
where $E_q(X)$ is the total energy of the defect system ($X$) with charge $q$, $E_{prist}$ is the total energy of the pristine system, $\mu_i$ and $\Delta N_i$ are the chemical potential and change in the number of atomic species $i$, and $\varepsilon_F$ is the electron chemical potential. A charged defect correction $\Delta_q$ was computed with techniques developed in Refs. \cite{PING2017JCP, wu2017first} and implemented in the JDFTx code \cite{JDFTx}. The chemical potentials were carefully evaluated against the stability of byproduct compounds as detailed in the elemental chemical potential section of the SI.
The corresponding charge transition levels of the defects were obtained from the value of $\varepsilon_F$ where the stable charge state transitions from $q$ to $q'$.
\begin{align}
    \epsilon_{q|q'} = \frac{E^f_q - E^f_{q'}}{q' - q}
    \label{eq:ctl}
\end{align}
The ionization energies are computed by referencing the CTLs to the free polaron state.
\cite{seo2018role,radmilovic2020combined,zhou2020interstitial,lee2020electrochemical}
Namely in \ce{Fe2O3}, it has been experimentally observed that photoexcited carriers relax on picosecond timescale to form small polarons~\cite{carneiro2017excitation}, which have been measured to form at energies $\sim$0.5 eV below the conduction band minimum (CBM)~\cite{lohaus2018limitation,pastor2019situ}. Theoretically this free polaron level is computed as the charge transition level from $q=0$ to $q=-1$ in the pristine system, $\varepsilon_{FP} = \varepsilon_{-1|0}^{prist} = E^f_{-1}(prist) - E^f_0(prist)$. By this method we obtain that the free polaron level is positioned at 0.497 eV below the CBM in excellent agreement with experimental observation.~\cite{lohaus2018limitation,pastor2019situ}

From charge defect formation energies, charged defect concentration ($c_q$) can be computed as:
\begin{align}
    c_q(X; \varepsilon_F) = g \exp [- E^f_q(X; \varepsilon_F) / k_B T],
\end{align}
where $g$ is the degeneracy factor accounting for the internal degrees of freedom of the point defect, $k_B$ is the Boltzmann factor, and $T$ is temperature.
Concentrations including intrinsic defects, extrinsic dopants and free electron polarons were computed by determine their charge neutrality condition~\cite{lee2013thermodynamics,freysoldt2014first}
(additional details are provided in the defect concentration section of the SI).
To best relate to experimental measurements of \ce{Fe2O3} photoanodes, we computed the concentration first at a synthesis temperature of $T_S=1073$ K (800 $^\circ$C is a common synthesis temperature \cite{ling2011sn,tian2020electronic}), and then recomputed charge neutrality at normal operation (room) temperature $T_O=300$ K.
The partial pressure of oxygen gas ($p_{\rm O_2}$) corresponding to atmospheric condition (1 atm) is used.

% In order to maintain charge neutrality, the introduction of defect $X$ with charge $q$ into the lattice must be compensated by defects of opposite charge or through the generation of free carriers. Specifically charge neutrality must be held:
% \begin{align}
%     \sum_{X,q} c_q(X; \varepsilon_F) + n_h - n_e = 0, \label{eq:neutrality}
% \end{align}
% where $n_h$ denotes the concentration of free delocalized holes, and $n_e$ denotes free delocalized electrons. However, in hematite free electrons will nearly entirely localize into small polarons, so the concentration of free small electron polarons is entered in Eq.~\ref{eq:neutrality} in a similar way to that of a defect, i.e.\ with a specific formation energy.\cite{lee2013thermodynamics}
% Furthermore, in order to relate to experimental measurements of \ce{Fe2O3} photoanodes, we computed the concentration first at a synthesis temperature of $T_S=1073$ K (800 $^\circ$C is a common synthesis temperature \cite{ling2011sn,tian2020electronic}), and then recomputed charge neutrality at normal operation (room) temperature $T_O=300$ K, considering thermodynamic equilibrium at each temperature (additional details can be found in the sections of defect formation energy and concentration in SI). In the end, the only variable of this approach is the partial pressure of oxygen gas ($p_\ce{O2}$).



% \section{Results and Discussion}

% \subsection{Formation of dopant-polaron quadrupoles}
% Electronic and local structure
% Binding energy (scatter plot)
% EMS model

% prelude to the section

\begin{figure}
    \centering
    \includegraphics[keepaspectratio=true,width=0.8\linewidth]{2-tmo/figures-fe2o3/electronic_500.png} % high resolution
    \caption{Electronic structure of Sn-doped \ce{Fe2O3}. (\textbf{a}) Wavefunction of the small electron polaron ($EP$) in the single Sn-dopant system where the $EP$ and Sn form a dipole. (\textbf{b}) Band structure and projected density of states (PDOS) of the dipole Sn system. (\textbf{c}) Wavefunctions of the two $EP$ in the two Sn-dopant system where the two $EP$ and two Sn form a quadrupole.
    The Sn-Sn separation is 3.784 {\AA}.
    (\textbf{d}) Band structure and PDOS of the quadrupole Sn system. For the atomistic plots, gold=Fe, red=O, grey=Sn, and the yellow/blue ($+$/$-$) cloud is the isosurface of the polaron wavefunction (the isosurface level is 1\% of the maximum). In the band structures, dark/light blue is spin up/down and $\varepsilon_F$ is the Fermi energy.
    }
    \label{fig:electronic}
\end{figure}

% \subsection{Stable Multipole Formation by Dopant Clustering}
\subsection{Dopant Clustering by Multipole Formation}
% . experimentally there is strong indication of clustering from XAS; showing certain high Sn-Sn distance at very low concentration.
% . We found quadrupole can be formed with stable binding energy and agree with experimental XAS.
%
% -> Propose this interpretation of clustering
% -> Figure 1 goes here ;  describe the results of one dopant and two dopants; and binding energy (mention negative values)


Substitutional doping by group IV and XIV elements was investigated theoretically by replacing a single Fe site by the dopant ($X$). Consistent with previous studies \cite{smart2017effect,zhou2015understanding} and experimental observation,\cite{ling2011sn,li2017morphology} we found this process yields the formation of small electron polarons corresponding to the identification of $\rm Fe (2+)$ after replacing $\rm Fe (3+)$ by the tetravalent dopant $X (4+)$. (Note in this paper we use the notation $X (i)$ to denote an ion $X$ with valency $i$). The electronic structure of the single Sn-doped system is shown in Figure~\ref{fig:electronic}b right panel, wherein the band structure exhibits a flat isolated occupied state in the gap corresponding to the small electron polaron ($EP$) with tight spatial localization similar in size to the Fe$-$O bond lengths. Likewise, the projected density of states (PDOS) in Figure~\ref{fig:electronic}b shows a sharp isolated peak composed mostly by Fe $3d$. The wavefunction of the $EP$ is shown in Figure~\ref{fig:electronic}a with a clear $d_{z^2}$ character. The $EP$ forms at the Fe site nearest to the Sn dopant with a Sn$-EP$ distance of $2.981$ {\AA} ($d_0$ in Figure~\ref{fig:electronic}a).


To investigate dopant-dopant interactions, we placed a second Sn dopant in the lattice. All possible Sn-Sn pair configurations were tested, and the lowest energy configuration was clearly identified (see SI Figure S2, S3 and Table S2). The electronic structure of this corresponding configuration is shown in Figure~\ref{fig:electronic}c-d, which exhibit the formation of two $EP$ states.
% with a noticeable energetic difference of 0.12 eV.
We find the asymmetry of the local structure, which %differing distances of the polaron to the two Sn atoms
is a natural consequence of the corundum crystalline form, causes a noticeable energetic difference of 0.12 eV between $EP_1$ and $EP_2$.
Specifically, in Figure~\ref{fig:electronic}c, $EP_1$ has distances to the two adjacent Sn of $d_1=3.011$ and $d_2=3.129$ {\AA}, whereas $EP_2$ has distances of $d_3=3.510$ and $d_4=4.112$ {\AA}. The proximity of $EP_1$ to the Sn yields a lower energy state relative to $EP_2$.
Lastly, the theoretically predicted Sn-Sn distance of $3.784$ {\AA} closely matches experimentally observed Sn-Sn peak in EXAFS data of Sn-doped hematite samples (as discussed in next section).
% closely matches experimentally observed Sn-Sn peak in EXAFS data of Sn-doped hematite samples when the dopant concentration exceeds 3\%, which might imply a similar dopant clustering. \cite{annamalai2015activation}
% TS: I modified the above sentence since we have our own EXAFS data in the next section. That section also refers back to this section

The remaining group IV (Ti, Zr, Hf) and XIV (Si, Ge, Sn, Pb) dopants were also simulated in both single and pair dopant configurations with negligible differences in their electronic structure
and polaron configurations from Sn (all electronic structures are presented in SI Figure S4-S9).
Note for the present study, the configuration of two dopants is chosen to be the same for all dopants for the purpose of discussing chemical trends, as predicted by the case of Sn. It is possible that dopants may vary in their exact pair dopant configuration, for example see SI Table S3. This variation does not affect the main implications on carrier concentration we conclude later.
Most importantly, the stable configuration of a single tetravalent dopant (such as Sn) resembles an electric dipole where the Sn and $EP$ represent positive and negative charge centers, respectively. In this way the system with two dopants resembles an electric quadrupole (two positive Sn centers and two negative $EP$ centers).
Therefore, we will denote the single doped system as a first-order multipole (dipole) system and the pair doped systems as a second-order multipole (quadrupole) system.
%Expanding on this idea, one may consider that larger order multipoles in \ce{Fe2O3} may form, wherein the negative charge of the polarons and the positive charge of the dopants aggregate together analogous to an electric multipole expansion.
To examine the thermodynamic stability of dopant-polaron quadrupole, we studied their binding energy ($\be$) from two separate dipoles:
%by the difference in the formation energy of the quadrupole and that of two separate dipoles:
\begin{align}
    \be = E^f(quad) - 2 E^f(dipole).
    \label{eq:be1}
\end{align}
Here, $E^f(X)$ is the formation energy of the system with neutral dopants in a configuration $X$ (e.g.\ quadrupole or dipole dopant system) following Eq.~\ref{eq:cfe} at charge state $q=0$.
%and is defined as $E^f(X) = E(X) - E_{prist} + \sum_i \mu_i \Delta N_i$. Here $E(X)$ and $E_{prist}$ is the DFT total energy of the doped and pristine systems, respectively, and $\mu_i$ is the chemical potential of atomic species $i$ with change $\Delta N_i$.
%Substituting this definition into Eq.~\ref{eq:be1} simplifies the binding energy into a simple expression of DFT total energy,
%\begin{align}
%    \be = E(quad) - 2 E(dipole) + E_{prist}.
%    \label{eq:be2}
%\end{align}
The quadrupole binding energy was evaluated for all group IV and XIV dopants considered in this study. We observed that the binding energy for all dopants is negative ($\sim-$0.1 to $-$0.2 eV), as shown in Table~\ref{table:ems} ($\be^{DFT}$), indicative of a strong tendency for dopants and polarons to aggregate.
We note that we expect dopant clustering occurs during the cooling process from synthesis temperature (over one thousand K here) down to room temperature. At a synthesis condition, dopants will be all ionized and the binding of dopants into quadrupoles will not occur.
Since all the binding energies are lower than $kT$ at room temperature, it is expected that the quadrupoles are stable at room temperature.
% All in all, the negative binding energy is indicative of a strong tendency for dopants and polarons to aggregate.


% \subsection {Experimental evidence for Sn pair formation: EXAFS results}
\subsection{Experimental Evidence for Dopant-Pair Formations}

Extended x-ray absorption fine structure (EXAFS) data at the Sn edge were collected at SSRL for two Sn doped \ce{Fe2O3} samples, with Sn nominal concentrations of 0.1\% and 1.0\%. (Sn concentration of 1\% corresponds to replacing 1 out of 100 Fe with Sn; measured concentrations are 10-20\% lower, see Table S4 in SI for details). Synthesis methods are detailed in the SI. A standard fluorescence set-up (32 element Ge fluorescent detector) was used with the sample set at 45$^{\circ}$ to the beam, and an Oxford helium cryostat maintained the temperature at 10 K. Details about the data collection and reduction are in the SI section on EXAFS characterization. The $r$-space data are plotted in Figure~\ref{exafs} for the 0.1\% Sn and 1.0\% Sn samples. For the 0.1\% Sn sample (Figure~\ref{exafs}a), the amplitudes of the further neighbor peaks are quite large and the data can be well fit (solid orange line) to the hematite structure, with a small expansion for the Sn-Fe pairs compared to hematite;  roughly 0.1 {\AA} for closer pairs but only 0.02 {\AA} for Fe neighbors near 3.7 {\AA}. This is the expected behavior around a substitutional dopant site when the dopant valence $\rm Sn(4+)$ is higher that the host valence $\rm Fe(3+)$, and this behavior has been observed in other similar situations.\cite{mackeen2018substitution}. The further neighbor Sn-O peaks are expected to contract very slightly, but because these small peaks overlap the larger Sn-Fe peaks, the pair-distances fluctuate too much. The first O shell, although split in hematite, collapses to a single peak with an average Sn-O distance of 2.05 {\AA}, very close to the averaged first neighbor distance in hematite, 2.03 {\AA}; this is a competition between a larger ionic radius for $\rm Sn(4+)$, and larger electrostatic force between $\rm Sn(4+)$ and $\rm O(2-)$.

On the other hand, the EXAFS $r$-space plot for 1\% Sn sample is quite different (Figure~\ref{exafs}b). The data up to 3 {\AA} are very similar to that for 0.1\% Sn - i.e.\ the phase of the real part of the Fourier transform, R($r$), is the same. However in the range 3-3.8 {\AA}, the phase changes dramatically and a dip develops in the amplitude near 3.3 {\AA} which has the  shape of an interference dip. It occurs close to the expected position for the Sn-Fe peak in an EXAFS plot (actual distance $\sim$3.7 {\AA}: note that there is a calculable phase shift of peaks in $r$-space plots to lower $r$). These data can't be fit to a simple distorted hematite model and the shape of R($r$) suggests that another peak is present (see SI Figure S11). Consequently, an additional peak, corresponding to one Sn-Sn pair with a distance close to 3.7 {\AA}, was included in the fit. The number of Sn-Fe pairs was correspondingly reduced from 6 to 5. This yielded the good fit (details in SI) shown in Figure~\ref{exafs}b, and is a clear evidence that Sn-Sn pairs have formed.
Remarkably, this Sn-Sn pair distance ($\sim$3.7 {\AA}) matches the theoretically predicted distance of Sn-Sn pair (3.784 {\AA}) shown above.
%which forms due to quadrupole formation.
% Again the Sn-O peaks are slightly contracted while the Sn-Fe pairs are slightly expanded; the long Sn-Sn peaks is at 3.67 {\AA}.

%***************************** Fig. X ******************************
\begin{figure}[H]
    \centering
    \includegraphics[width=0.47\linewidth]{2-tmo/figures-fe2o3/exafs_500.png}
    \caption{EXAFS $r$-space data at the Sn K edge, for (\textbf{a}) 0.1\% and (\textbf{b}) 1\% Sn in Fe$_2$O$_3$. The plot for 0.1\% Sn also shows a fit to the hematite structure; good agreement is obtained with a slight contraction of the Sn-O pairs and a slight expansion of the Sn-Fe pairs. At 1\% Sn, the EXAFS changes significantly. Although the first two peaks are very similar, the region from 3-3.8 \AA{} is quite different, particularly the shape of the phase (fast oscillating function), and a dip develops near 3.3 \AA{}. These data cannot be fit to the hematite structure. The data suggest that there is another peak present; in the fit shown in part (\textbf{b}), one of the Fe neighbors at $\sim$3.7 \AA{} is replaced with a Sn atom, forming a Sn-Sn pair. This leads to the excellent fit shown in (\textbf{b}).
    Fourier transform range, 3.5-13 \AA$^{-1}$; fit range in $r$-space, 1.1-4.2 \AA{} for both plots.
    In both figures, the blue and gold bars at the bottom indicate the position of Sn-O and Sn-Fe peaks, respectively, in undistorted hematite. The bar positions include the known shifts in $r$.
    }
    \label{exafs}
\end{figure}
%***************************** Fig. X ******************************


\subsection{Mechanisms of Dopant-Polaron Binding into Quadrupoles}
% . There are three parts:  electrostatic, strain,  and  magnetic. You need to explain briefly why physically important and how you compute them.
% . Then show results agree with DFT.
%
% One figure a,b,c
% -> first dipole can be explained by electrostatic
% -> however quadrupole we notice it has a correlation with strain (ionic radius)
% -> so we consider a model with three parts (ems)

% Considering the electron polaron carry a net negative charge, while the tetravalent dopants carry a net positive charge (with respect to $\rm Fe (3+)$), this system closely resembles the formation of an electronic dipole where the charge of the electron polaron and dopant are attracted electrostatically. % This was already said above
After confirming the existence of Sn-Sn clustering both theoretically and experimentally, we turn to investigate the mechanisms of their formation.
Above the analogy was made between the single Sn-doped system and electric dipoles,
so in order to probe this electrostatic interaction, we plotted the total energy of the single Sn-doped system as a function of Sn-polaron distance, as shown in Figure~\ref{fig:mechanisms}a.
The computed total energies were fitted to a Coulomb potential ($-a/r+b$) with the fitted values of $a=0.658\ {\rm eV\cdot\text{AA}}$ and $b=0.230$ eV, and a coefficient of determination ($R^2$) of 0.85.
This validates a clear electrostatic attraction between the two bodies with opposite charges like a dipole.
%where the interaction is analogous to two point charges of opposite charge and the denotation of this system as a first-order multipole (dipole).
Furthermore, we find that $b$ is close to the value of ionization energy of Sn (0.25 eV), as expected.

%We note this Coulombic behavior is not universal, as non-Coulombic behavior was observed in,
However, this Coulombic interaction competes with other factors. For example
in Mo doped \ce{BiVO4}\cite{wu2018combining}, strain causes short-range repulsion between the dopant and polaron, and dominates over the Coulombic attraction.\cite{wu2018combining,zhang2018unconventional}
We find this was also reflected in the computed quadrupole binding energies ($\be$), as shown in Figure~\ref{fig:mechanisms}b, where we plotted them versus the ionic radius ($R_I$) of each dopant.
Specifically, there is a roughly positive correlation between the ionic radius and the quadrupole binding energy. We attribute this to the compensatory size effects of the dopant and the polarons. Namely, the replacement of Fe(3+) with ionic radius of 64.5 pm by $\rm Fe (2+)$ with ionic radius of 78.0 pm, yields an expansion strain at the lattice site. This strain can be reduced by smaller radii dopants (e.g.\ Ti, Ge, Si) that will increase the magnitude of $\be$ as shown in Figure~\ref{fig:mechanisms}b, or enhanced by larger radii dopants (e.g.\ Sn, Hf, Zr, Pb) that will decrease the magnitude of $\be$ towards zero.
We note that the case of Ti, an outlier in Figure~\ref{fig:mechanisms}b, possess stronger correlated $3d$ orbitals, which in turn exhibit stronger electron localization, may compensate local expansion from small polarons  and lower its binding energy (similar to Si which intrinsically is smaller than the rest).

\begin{figure}[H]
    \centering
    \includegraphics[keepaspectratio=true,width=1.0\linewidth]{2-tmo/figures-fe2o3/mechanisms_500.png} % high quality
    \caption{
    Mechanisms of dopant-polaron binding in \ce{Fe2O3}.
    (\textbf{a}) Total energy of Sn-doped hematite system as a function of the Sn$-EP$ distance.
    % The black curve represents a simple Coulomb potential fit with fit parameters $a=0.658\ {\rm eV/\AA}$ and $b=0.230$ eV and an $R^2$ value of 0.85. The accurate fit of the Coulomb potential strongly reflects the electrostatic attraction between the Sn (positive charge) and $EP$ (negative charge).
    The black curve represents a simple Coulomb potential fit which has an $R^2$ value 0.85, supporting the intuition of an electrostatic interaction between the Sn (positive charge) and $EP$ (negative charge) centers.
    (\textbf{b}) Quadrupole binding energy ($\be$) of group IV and XIV dopants in hematite computed by Eq.~\ref{eq:be1}, plotted against the ionic radius of the dopant~\cite{shannon1976revised} ($R_I$; valency 4+, coordination VI).
    % Additionally, the ionization energy ($IE$) is shown by the color of each point.
    % TS: IE is discussed later so I decided to remove it here
    (\textbf{c}) Computed $\be$ by Eq.~\ref{eq:be1} plotted against those computed with the EMS model in Eq.~\ref{eq:ems}. The linear fit relation ($f(x)$) between these models is shown in the inset box.
    }
    \label{fig:mechanisms}
\end{figure}


Therefore, it is necessary to consider electrostatic and strain effects simultaneously, and also include magnetic effects (binding energies may also be modified by the antiferromagnetism of \ce{Fe2O3}).
%In order to put both electrostatic and strain effects together and additionally consider magnetic effects which can be considerable for \ce{Fe2O3},
Hence, we propose a model of the quadrupole binding energy based on electrostatic, magnetostatic, and strain effects (abbreviated to EMS), in order to analyze the importance of each contribution:
% In order to disentangle the various competing effects which dictate the formation of dopant-polaron quadrupoles, we propose a model of the binding energy based on electrostatic, magnetostatics, and strain (abbreviated to EMS):
\begin{align}
    \be^{EMS} = \be^{elec} + \be^{mag} + \be^{strain}.
    \label{eq:ems}
\end{align}
Here $\be^{elec}$, $\be^{mag}$, $\be^{strain}$, correspond to electrostatic, magnetostatic, and strain contributions to the quadrupole binding energy, respectively.
The exact formulation for each component of the EMS model is detailed in the methods section (Eq.~\ref{eq:be_elec}$-$\ref{eq:be_strain}) and the results are summarized in Table~\ref{table:ems}.
We evaluate how well this model reproduces DFT calculations by plotting them against each other in Figure~\ref{fig:mechanisms}c.
%In order to gauge the EMS model, we first consider its ability to reproduce the results from our DFT calculations which are plotted against each other in Figure~\ref{fig:mechanisms}c.
The linear fitting shows adequate agreement between the simple EMS model and our exact DFT calculations (with a slope near unity and an $R^2$ value of 0.864), which justifies its use for interpreting the DFT binding energies.
%Thus, the EMS model is able to reproduce the DFT results rather well and allows us to disentangle them into the three different components and interpret them separately.

As shown in Table~\ref{table:ems}, each of the three components contributes significantly to the overall quadrupole binding energy. The electrostatic interaction ($\be^{elec}$) is typically the most dominant factor, and intuitively it is chiefly responsible for the attraction of dopants into the quadrupole configuration.
Interestingly, we also found a non-negligible contribution from magnetostatic interactions ($\be^{mag}$, 42 meV) which further participates in the binding of quadrupoles.
This effect is non-trivial but is an indirect consequence of placing non-magnetic dopants next to each other, which in-turn breaks fewer antiferromagnetic interactions and yields a lower energy configuration when dopant-pairs form. % \textcolor{red}{dopant clustering} occurs.
In contrast, the effect of strain ($\be^{strain}$) typically mitigates the formation of dopant-pairs % \textcolor{red}{dopant clustering}
(increases the system's energy with clustering) due to the accumulation of lattice distortion. However, as aforementioned, this effect can be compensatory in the dopant cases with smaller ionic radii than Fe(3+), which can pack more efficiently next to the polarons with larger ionic radius as specified in Figure~\ref{fig:mechanisms}b.
% TS: I suppose the best solution is to just remove this sentence: \textcolor{red}{This is particularly evident in the case of Si which was found to be an outlier with a large negative strain component due to its significantly smaller ionic radius. -

\begin{table}[H]
    \footnotesize
    \centering
    \begin{tabular}{c|ccc|c|c}
    \hline \hline
    Dopant & $\be^{elec}$ & $\be^{mag}$ & $\be^{strain}$ & $\be^{EMS}$ & $\be^{DFT}$ \\
    \hline
    Si & -0.184 & -0.042 & -0.33 & -0.553 & -0.227 \\
    Ge & -0.188 & -0.042 &  0.07 & -0.160 & -0.117 \\
    Ti & -0.175 & -0.042 & -0.04 & -0.256 & -0.233 \\
    Sn & -0.188 & -0.042 &  0.18 & -0.054 & -0.029 \\
    Hf & -0.188 & -0.042 &  0.13 & -0.098 & -0.055 \\
    Zr & -0.192 & -0.042 &  0.13 & -0.103 & -0.080 \\
    Pb & -0.204 & -0.042 &  0.11 & -0.135 & -0.039 \\
    \hline \hline
    \end{tabular}
    \caption{Collected values of the binding energy for group IV and XIV dopants in \ce{Fe2O3} computed by DFT ($\be^{DFT}$) or with the EMS model ($\be^{EMS}$) as in Eq.~\ref{eq:ems}. The various components of the EMS model are tabulated as well including the electronic ($\be^{elec}$), magnetic ($\be^{mag}$), and strain ($\be^{strain}$). All values are given in eV.}
    \label{table:ems}
\end{table}


\subsection{Effects of Dopant Clustering on Polaron Concentrations}
% \subsection{Dopant Clustering Negatively Impacts Free Polaron Concentration}

\begin{figure}
    \centering
    \includegraphics[keepaspectratio=true,width=0.9\linewidth]{2-tmo/figures-fe2o3/quad_500.png}
    \caption{
    Effects of quadrupole binding on carrier concentration in \ce{Fe2O3}.
    (\textbf{a}) Band diagram of various doped systems either single-doped (dipole) or with two dopants (quadrupole), the solid horizontal lines represent the thermodynamic charge transition levels (CTLs), and the ionization energy corresponds to the separation of the CTLs to the free polaron line (dashed grey line).
    (\textbf{b}) Change in the first ionization energy when quadrupoles are formed vs.\ ionic radius ($R_I$). Notably, when the ionic radius of dopants is below that of $\rm Fe (3+)$, the ionization energy is increased (as shown in blue bars), whereas it is decreased when the dopant radius is larger (as shown in orange bars).
    (\textbf{c}) Computed free electron polaron concentration as a function of dopant concentration for Sn, Ge, and Ti, with and without the effect of clustering (i.e.\ quadrupole formation).
    }
    \label{fig:quad}
\end{figure}

Lastly, we discuss the effects of dopant clustering, which we will show to be responsible for the low optimal doping concentrations of \ce{Fe2O3} observed experimentally.
The computed thermodynamic charge transition levels (CTLs) are displayed in Figure~\ref{fig:quad}a for both the dipole (single dopant) and quadrupole (pair dopants) systems. In quadrupole systems, there are two charge transition levels (gold and orange lines) which correspond to the ionization energies of two electron polarons (as shown in Figure~\ref{fig:electronic}c).
Because quadrupoles yield consistently lower second CTLs, they possess very high second ionization energy ranging from 0.34 eV to 0.48 eV (the orange lines in Figure~\ref{fig:quad}a, nearly doubled the ionization energies of corresponding single-doped systems denoted by the blue lines). Therefore, the ionization of both electrons after quadrupole formation is nearly impossible. On the other hand, the first ionization energies of quadrupoles compared to the ones of dipoles shift in a manner correlated with the ionic radius as shown in Figure~\ref{fig:quad}b.
In particular, dopants with a radius larger than the host site ($\rm Fe (3+)$) have a lower first ionization after quadrupole formation (orange bars in Figure~\ref{fig:quad}b) and vice versa.


To show how dopant clustering affects carrier concentration quantitatively, we compute the polaron concentrations with and without the effect of dopant clustering for three representative cases: Sn, Ge, and Ti in Figure~\ref{fig:quad}c.
Equilibrium polaron concentrations are computed following the defect formation energy and charge neutrality approach as detailed in the method section.\cite{lee2013thermodynamics,freysoldt2014first}.
We then introduce clustering in the theoretical synthesis process, by considering the probability at which two dopants form into a quadrupole following a thermal Boltzmann distribution (details in the SI section on probability of quadrupole formation).
As dopant-dopant pairs form during cooling process from synthesis temperature to room temperature, an intermediate temperature of $T=500$ K is chosen as an approximate temperature where pairs start to form. We show the trends among different dopants remain the same for different temperatures (see SI Figure S13).
%% Defect and polaron concentrations of Sn-doped, Ge-doped, and Ti-doped \ce{Fe2O3} are shown in \textcolor{red}{Figure~\ref{fig:dc}b-d} as representative examples (other simulated doped systems in group IV and XIV are shown in SI Figure S11). Although in undoped \ce{Fe2O3} free electron polarons are mostly generated by $\rm Fe_{i}^{+}$, in doped \ce{Fe2O3} they are mostly generated  from tetravalent group IV and XIV dopants (more than six orders of magnitude higher than the ones generated by $\rm Fe_{i}^{+}$).%%
In Figure~\ref{fig:quad}c, we show that
polaron concentration is reduced due to clustering (solid lines) compared to the case without clustering(dashed lines), and the magnitude of this reduction is closely related to their binding energy ($\be$) in Table \ref{table:ems}. Specifically, dopants such as Ge and Ti with large binding energies show significant trapping of polarons due to clustering in the second and third panels of Figure~\ref{fig:quad}c. %(the large difference between solid and dashed lines in the second and third panels of Figure~\ref{fig:quad}c).
% TS -- I added a paragraph break here

In order to avoid the adverse effects of clustering, Ti will be better suited to be doped at very low concentrations into hematite.
% On the other hand, Ti will be better suited to be doped at very low concentrations into hematite in order to avoid the adverse effects of clustering.
This explains the experimentally observed very low optimal doping concentration of Ti in hematite (about 0.1\%).\cite{fu2014highly,malviya2017influence} Furthermore, co-doping Ti with another dopant less prone to clustering may offer higher performance in hematite, which explains the success of recent co-doping strategies.\cite{zhang2010improved,mirbagheri2014visible,pan2015ti}
On the other hand, the small quadrupole binding energy of Sn makes it easier to dope Sn into hematite to higher concentrations without experiencing an immediate bottleneck.
This explains why the optimal doping of Sn into hematite (3\%) is an order of magnitude larger than Ti.\cite{yang2013new,tian2020electronic,bindu2018electrical}
If neglecting the effect of clustering, Ti would be a better dopant than both Sn and Ge (see dashed lines in Figure 4c, which shows that Ti has predicted to contribute the highest carrier concentration without clustering). Therefore, to unlock the potential of Ti, it is necessary to mitigate its strong tendency of clustering.
A strategy of co-doping Sn and Ti in hematite may offer the most effective strategy for maximizing performance of hematite photoanodes.
Alternatively, co-doping of divalent dopants, such as Mg, has been shown to relieve lattice distortion and can also offer further improvement to PEC performance~\cite{cai2020engineered}.
Ultimately, Ge is seen as the best dopant in group IV and XIV with the highest polaron concentration, which outperforms Sn and Ti in Figure~\ref{fig:quad}c.
%due to its ideal solubility and ionization energy, along with a lower tendency for clustering (compared to Ti).
%not the best for any of the property here

Finally, we remark that here we have focused on formations of dopant clustering at an early stage, which dominate at relatively low concentrations of doping in hematite. We also tested higher-order multipole clustering such as `hexapole' formation in hematite (e.g.\ three Sn dopants with the three introduced $EP$, details in SI section on higher-order multipoles and SI Figure S14). We found that hexapoles also have a negative binding energy, and thus
it is entirely possible that dopant clustering may grow even larger than the second-order multipoles we considered here.
However, there will also be larger and more complicated strain effects and configurational entropy which can compensate binding energies of larger sized aggregations.
Eventually, at even larger doping concentrations, the precipitation of different phases may occur, for example at 6\% Sn-doping in hematite XRD shows \ce{SnO2} secondary phases form \cite{tian2020electronic}.
% In addition,  in experiment, carrier conductivity is often measured instead of carrier concentration alone, where the carrier mobility (specifically polaron mobility here) will also play a significant role. We have studied doping effects on polaron mobility in transition metal oxides in our past work\cite{wu2018combining,zhang2018unconventional,smart2017effect,smart2018mechanistic}, and we will investigate the effects of dopant clustering on polaron mobility in our future study.
Future theoretical work should provide further insights to PEC experiments by investigating the role of dopants and dopant clustering on polaron mobility via small polaron hopping~\cite{wu2018combining,zhang2018unconventional,smart2017effect,smart2018mechanistic}
and optical absorption\cite{smart2019optical,wheeler2019combined,zhou2020interstitial}, which together, along with carrier concentrations studied here, directly impact the photoconductivity of transition metal oxide based photoelectrodes.
% Our future theoretical work will provide further insights to PEC experiments by investigating the role of dopants and dopant clustering on polaron mobility\cite{wu2018combining,zhang2018unconventional,smart2017effect,smart2018mechanistic} and optical absorption\cite{smart2019optical,wheeler2019combined,zhou2020interstitial}, which together, along with carrier concentrations studied here, directly impact the photoconductivity of transition metal oxide based photoelectrodes.


\subsection{Conclusions}
This work discovers a new mechanism of dopant clustering via the aggregation of n-type dopants and electron polarons into dopant-pairs % \textcolor{red}{bound dopant-polaron pairs}
which resemble electric multipoles.
% We find doping in \ce{Fe2O3} is limited by the formation of bound dopant-polaron pairs which resemble electric multipoles and trap electron polarons.
These pairs are thermodynamically stable due to several contributions % and they severely slow down the carrier concentration increase with doping concentration.
which we disentangle by a simple model involving three components: electronic, magnetic, and strain. Our model illuminates that binding occurs predominantly through electrostatic interactions but surprisingly is also mediated by magnetic interactions which together overcome strain to yield the consistently negative binding energies of tetravalent dopants in \ce{Fe2O3}. EXAFS experiments confirm the existence of these Sn-Sn pairs which formed at 1\% doping and have an identical interatomic distance compared to those predicted theoretically  ($\sim3.7$ {\AA}).

The effect of doping with and without clustering on carrier (small electron polaron) concentration is carefully examined.
We find doping in \ce{Fe2O3} is limited by dopant clustering which traps electron polarons and severely lowers the carrier concentration with respect to doping concentration. This clustering is shown to be responsible for the doping bottleneck in hematite, where dopants such as Ti exhibit extremely low optimal doping concentration (i.e.\ 0.1\%) for PEC application.
% We exclude the possibility of intrinsic compensating defects such as iron vacancy as the cause of doping clustering through our first-principles prediction of intrinsic and extrinsic dopant concentrations.
%and extrinsic dopants based on charge neutrality condition at thermodynamic equilibrium.
% Meanwhile, the binding energy of dopants into quadrupoles was shown to be responsible for the low optimal doping in hematite, where dopants such as Ti with a large binding energy exhibit extremely low optimal doping for PEC application.
Strategies to overcome this doping bottleneck are proposed; specifically codoping with dopants that exhibit low binding energies for clustering (e.g.\ Sn-Ti codoping) is seen as an ultimate solution.
Lastly, for single-type doping in group IV and XIV, we found Ge is the best dopant which can contribute the highest polaron concentrations even at presence of dopant clustering.
%Lastly, we demonstrated that dopants of slightly smaller ionic radii than the host site possess high solubility as well as low ionization energy making them better choices than other single dopants for optimal carrier concentrations.}
These findings provide a cohesive picture of the doping bottleneck in hematite and help to establish an improved rationale for further development of hematite photoanodes usage in renewable energy applications.


\section{Spin Polaron Conduction in CuO}

In our 2018 work published in npj Computational Materials~\cite{smart2018mechanistic}, we investigated hole transport in cupric oxide (CuO), a p-type semiconductor. Due to its relatively small bandgap (1.2-1.8 eV) and a conduction band minimum located at a more negative potential than that of water reduction,
\cite{jang2015tree,koffyberg1982photoelectrochemical,chiang2012copper,izaki2011electrodeposition,sagu2014rapid,masudy2016nanocrystal,lee2016scalable,guo2014cuo,emin2014novel,septina2017stabilized}
it has the potential to serve as an inexpensive and environmentally benign photocathode for a water splitting PEC.
However, like other TMO's, CuO suffers from poor carrier conductivity, which limits the effectiveness of CuO-based devices.\cite{guo2014cuo,emin2014novel,wong2016current,masudy2016nanocrystal}
Additionally, cathodic photocorrosion of CuO can also limit the use of CuO for photoelectrochemical applications. Fortunately, a recent study demonstrated that the photocorrosion of CuO can be effectively suppressed by depositing a thin protection layer that prevents direct contact of CuO and the electrolyte,\cite{septina2017stabilized} which encourages studies on further improving charge transport and photoelectrochemical properties of CuO. Facilitated charge transport in CuO can also be advantageous for the use of CuO in other electrochemical devices such as gas sensors.\cite{poloju2018improved,rydosz2013nano}

The development of charge transport in CuO depends on the understanding and optimization of the small polaron hopping process. Strong electron-phonon coupling in many transition metal oxides (Fe2O3, BiVO4, TiO2) leads to the localization of carriers into polarons, a quasi-particle representing the carrier and local lattice distortion.\cite{rettie2016unravelling,rettie2013combined,carneiro2017excitation}
Due to this localization, carriers are no longer transported through the system via typical band mechanisms. Rather, carriers must be thermally activated in order to “hop” between sites, a process known as polaron hopping conduction.\cite{devreese1996polarons,mott1968conduction}
This type of conduction leads to an extremely low carrier mobility (e.g. 0.1 $\rm cm^2/V/s$ for CuO)~\cite{rettie2016unravelling} several orders of magnitude lower than band-like semiconductors such as Si (~1000 $\rm cm^2/V/s$). Previous experimental studies have indicated that polaron formation also occurs in CuO.~\cite{samokhvalov1993low,wu2014charge,zheng2001evidence,jeong1996nonstoichiometry,zheng2004fast}
Interestingly, more exotic properties such as ``one-dimensional charge stripes'' and ``spin polarons'' have been found in CuO due to strong spin-charge-lattice interactions,\cite{zheng2001evidence} which distinguishes its conduction mechanism from the common electron polaron hopping conduction in non-magnetic oxides such as BiVO4.  However, there has yet to be a theoretical investigation on the existence and transport of polarons in CuO, which would provide deeper understanding of carrier transport and therefore offer effective doping strategies to improve the carrier transport properties in CuO and other magnetic oxides in general.
Finally, although there have been a few experimental doping studies of CuO to date,\cite{masudy2015titanium,zheng2004fast,choi2017p,gao2007ferromagnetic,masudy2015optical,zheng2003effect,chiang2016dopant,chiang2011li,chand2014structural} the role of dopants in improving hole conduction in CuO has not been clearly understood.

In this study, we address these fundamental questions by comparatively investigating hole conduction in pristine and Li-doped CuO. Our focus is on the elucidation of the mechanisms by which Li doping improves hole concentration and mobility through a combined theoretical and experimental effort. Our work is organized as follows, first we provide theoretical background on CuO and discuss the mechanism of hole conduction which involves a unique spin-flip hopping process of spin polarons. Second, we show how Li doping enhances hole concentrations and hole mobility in CuO. Finally, we confirm our theoretical results by preparing CuO and Li-doped CuO electrodes and experimentally compare their photoelectrochemical properties.

% Polaron formation & hole conduction in CuO from first-principles
\subsection{Polaron formation and hole conduction in CuO from first-principles }
Several experimental studies have shown that CuO has an Arrhenius dependence of conductivity to temperature.~\cite{samokhvalov1993low,wu2014charge,zheng2001evidence,jeong1996nonstoichiometry,zheng2004fast}
This dependence is expected for materials which form small polarons (a trapped electron or hole due to local lattice distortion) which must be thermally activated in order to hop between lattice sites in the material, a process known as polaron hopping.~\cite{mott1968conduction}
The mobility of small polaron hopping follows the relationship shown in Eq.~\ref{cuo:eq:mobility},
\begin{align}
    \mu \propto e^{-E_a / kT}
    \label{cuo:eq:mobility}
\end{align}
where $E_a$ is the activation energy, $k$ is the Boltzmann constant, $T$ is the temperature, and $\mu$ is the carrier mobility which is related to the conductivity $\sigma$, by $\sigma = en\mu$ ($n$ is majority carrier concentration, which are holes in this case). To confirm the presence of polarons in CuO, we computed the electronic structure of pristine CuO with a single electron removed from a 96 atom system ($2\times 3\times 2$ supercell), corresponding to a hole concentration of 2\% (at.\ \% of hole = 100 $\times$ [mol of hole] / [mol of O]). Detailed descriptions of our electronic structure calculations can be found below under the Computational Methods section and SI. From the density of states and wave function of the hole state, we determine that holes form localized polaron states. These polaron states are predominantly O $2p$ mixed with Cu $3d$ as seen from the partial density of states (Figure S2) in agreement with previous studies on the electronic structure of CuO.~\cite{ghijsen1988electronic}

An intriguing consequence of hole localization around a single Cu $3d^9$ ion is that the Cu magnetic moment will flip, forming a ``spin polaron'' (SP).\cite{de1962interactions,vigren1973mobility,mott1993polaron,chernyshev2003models,lu1990spin,yonemitsu1992sensitivity}
Such states are common in copper oxides as the combination of two SPs will create a spinless Cooper pair state which obeys Bose-Einstein statistics and is the basis of superconducting.\cite{mott1993polaron,alexandrov1994bipolarons} In general, a SP forms in polaronic materials where the kinetic energy of the state can be lowered substantially from the increased delocalization of the electron or hole wavefunction after the spin-flip occurs.~\cite{wood1992d} For example, after an electron at a spin-up state is removed, a hole is created at the same spin state. As a fermion, the hole obeys the Pauli exclusion principle like electrons and can only be added to a state which is already occupied (i.e. a state must be occupied by an electron of the same spin for a hole to form). Therefore, in an anti-ferromagnetic system, the delocalization of a spin-up hole is limited by the availability of neighboring atoms’ spin-up occupied states. As shown in Figure 1B, with anti-ferromagnetic ordering, the hole polaron may form into a highly localized state (limited to forming over a single Cu atom and its bonding O atoms that have up spins, as neighboring Cu atoms do not have an available spin-up state at which the hole can form). But after a neighboring Cu ion's moment flips (Figure 1C) an extra channel is created, and the spin-up hole can redistribute over several sites that all have an up spin, lowering the kinetic energy of the hole polaron. The resulting flipped Cu ion with a distributed polaron state over several Cu and O atoms is shown in Figure 1D. As discussed in Ref.~\cite{wood1992d}, this lowering of kinetic energy through wavefunction delocalization dominates over the energy cost of the spin flip and facilitates the formation of  spin polarons in CuO. Additional explanations can be found in the SI (Figures S3-S4).

Considering that the magnetic couplings between Cu ions in CuO are significantly large ($J\sim 100$ meV), spin-spin interactions will have an important effect on the conduction of holes in CuO. To address this point, we consider the total kinetic rate $\kappa$ of the hopping process in Eq.~\ref{cuo:eq:kinetic},
\begin{align}
    \kappa = \left( \sum_i e^{-E_i/kT} \kappa_i \right) / \left( \sum_i e^{-E_i/kT} \right)
    \label{cuo:eq:kinetic}
\end{align}
where $E_i$ is the energy of the $i^{\rm th}$ configuration and $\kappa_i$ is the hopping rate between configurations (for full details see SI). For the case of CuO, we found that the formation of a polaron at a site without a flipped spin was not possible, and, presumably, the total energy of such a state is very high, which reduces the possible configurations entering Eq.~\ref{cuo:eq:kinetic}. Then we define hopping that does not involve a spin-flip process to have a rate given by $\kappa_i=e^{-E_a^{e-ph}/kT}$, with $E_a^{e-ph}$ being the usual hopping activation energy barrier due to electron-phonon interactions. We have shown that the Boltzmann factors that are related to the energies of different spin configurations will be dominated by the most probable hopping path (Figure S9). As a result, the full hopping rate $\kappa$ then reduces to Eq.~\cite{cuo:eq:eph_spin},
\begin{align}
    \kappa \sim e^{-(E_a^{e-ph}+E_a^{spin})/kT}
    \label{cuo:eq:eph_spin}
\end{align}
Namely as holes are conducted through the system they will invoke a spin-flip process which will cost energy equal to the cost of flipping a spin of a Cu ion (illustrated in Figure 2). In final, we see that the energy of this spin-flip process $E_a^{spin}$ can be simply added to the electron-phonon process $E_a^{e-ph}$ to give the full activation energy $E_a$, given in Eq.~\ref{cuo:eq:eph_spin2}.
\begin{align}
    E_a = E_a^{e-ph}+E_a^{spin}
    \label{cuo:eq:eph_spin2}
\end{align}
Intuitively, a spin-flip hopping process will not have a well-defined transition state; if there was a transition state, it would be a spin delocalized state which is not favored to form in a polaronic oxide. To confirm this point, we employed the newly-developed constrained density functional theory (CDFT) technique for solids in which an external potential is added to the Kohn-Sham potentials, and its strength is varied self-consistently in order to localize a desired number of charges on a specific site.~\cite{goldey2017charge,kaduk2012constrained} This allows for a direct calculation of the electronic coupling constant between initial and final states $|H_{ab}|$ in CuO, which we obtained to be 1.01 meV (the numerical accuracy is 0.01 meV). This is two orders of magnitude smaller than the computed activation energy (shown later), implying that transport in CuO is indeed non-adiabatic, which cannot be described by a semi-classical transition state theory.

The energy of this spin-flip ($E_a^{spin}$) can be obtained directly using the Heisenberg Hamiltonian $H_{spin}=-\sum_{i<j}J_{ij}\hat{S}_i\dot \hat{S}_j$, where $J_{ij}$ is the magnetic coupling between the spin of the $i^{\rm th}$ and $j^{\rm th}$ Cu ion and $\hat{S}_i$ is the spin of the $i^{\rm th}$ Cu ion (taken to be $1/2$ as Cu is in a $3d^9$ configuration with one unpaired electron). The use of this model is well-established in accurate modeling of the magnetic couplings of CuO,~\cite{rocquefelte2010short,rocquefelte2012theoretical,rocquefelte2013room} and our calculations show that fitting the total energy of different magnetic configurations of CuO with this Hamiltonian yields an R-squared of 0.999 (Figure S7). Following a previous work,~\cite{rocquefelte2010short} we considered five magnetic couplings in CuO: $J_z$, $J_x$, $J_a$, $J_b$, $J_2$. Of these five couplings, the coupling $J_z$ is dominant over the rest with a value of $-111$ meV and is responsible for the long-range antiferromagnetic transition of CuO at 230 K. Note that in CuO the magnetic correlation length remains large at temperatures above the transition temperature $T_N$, so magnetic coupling is still relevant to our discussion of hole conduction in CuO at room temperature.~\cite{zheng2001evidence,yang1989magnetic}
Second is the super-superexchange $J_2$ which is $-39$ meV yet is still three times smaller than $J_z$. The remaining values are $-17.4$ meV for $J_a$, $+3.0$ meV for $J_x$, and $+2.6$ meV for $J_b$. From this we can directly compute $E_a^{spin}$ according to Eq.~\ref{cuo:eq:easpin}.
\begin{align}
    E_a^{spin} = -\sum_{i<j}J_{ij} \Delta \left(\hat{S}_i\dot \hat{S}_j\right)
    \label{cuo:eq:easpin}
\end{align}
In CuO, $-\sum_{i<j}J_{ij} \Delta \left(\hat{S}_i\dot \hat{S}_j\right) = -(J_z+J_x+J_a+2J_b+J_2)$, which gives $E_a^{spin}$ to be 160 meV, a similar magnitude to $E_a^{e-ph}$ (99 meV) as shown in Table 1. This result validates that $E_a^{spin}$ contributes significantly to the overall activation energy. Therefore, this result suggests that dopants that can reduce the magnetic coupling contribution $E_a^{spin}$ as well as the electron-phonon contribution $E_a^{e-ph}$ to the activation energy can more effectively improve hole mobility in CuO.
Note that here we consider hopping along the ferromagnetic (FM) [101] direction (see Figure 2) due to shorter Cu-Cu distances and superior orbital overlap between initial and final states. Meanwhile, we find that hopping along the anti-ferromagnetic (AFM) [10$\bar{1}$] direction is energetically unlikely to occur (Figure S10-S12).

% Spin polaron conduction in Li-doped CuO from first-principles
\subsection{Spin polaron conduction in Li-doped CuO from first-principles}
Our experimental work (discussed later) shows Li-doped CuO electrodes have significantly increased photocurrent and show a positive shift of onset potential, while also retaining a similar crystallinity and photon absorption to the pristine CuO electrodes. Thus, it is anticipated that Li doping in CuO improves electron-hole separation and/or carrier conduction (concentration and/or mobility). To confirm this postulation, we applied our theoretical techniques discussed above to clarify how Li doping improves hole conduction in CuO.

The enhancement of carrier concentration after Li doping can be confirmed by the low hole ionization energy in Li-doped CuO, which is comparable to $kT$. Specifically, the ionization energy for a $p$-type dopant is defined by the difference between its charge transition level (CTL) and the valence band maximum. The CTL ($\varepsilon_{q|q'}$) is defined as the value of electron chemical potential at which the stable charge state of the defect changes from $q$ to $q'$, given by Eq.~\ref{cuo:eq:ctl}.
\begin{align}
    \varepsilon_{q|q'} = \left( E_q^f - E_{q'}^f \right) / (q' - q)
    \label{cuo:eq:ctl}
\end{align}
And the formation energy $E_q^f$ of the charge state $q$ is defined by Eq.~\ref{cuo:eq:cfe},
\begin{align}
    E_q^f [\varepsilon_F ]=E_q-E_{prist}+\sum_i \mu_i \Delta N_i +q\varepsilon_F
    \label{cuo:eq:cfe}
\end{align}
where $E_q$ is the total energy of the system with the charged defect, $E_prist$ is the total energy of the pristine system, and the third term on the right side accounts for the change in number of atoms of each species $i$ between these two configurations ($\Delta N_i$), with $\mu_i$ being the atomic chemical potential of that element in its stable form. From Eq.~\ref{cuo:eq:ctl}-\ref{cuo:eq:cfe} we computed the hole ionization energy of Li-doped CuO to be 55 meV (corresponding to the $-1|0$ transition level). Since this energy is small and comparable to $kT$ at room temperature, it indicates that Li introduces shallow hole states which can be ionized at room temperature to increase the hole concentration. This indicates a shift of the Fermi level towards the valence band maximum as has been experimentally shown with a positive shift of the onset potential by $\sim 210$ mV. The introduction of shallow states from Li doping is also in agreement with previous theoretical and experimental works.\cite{zheng2004fast,choi2017p}

To investigate the effects of Li doping on the transport of holes in CuO (\textit{i.e.}\ the effect of Li on hole hopping mobility) we first focused on the electron-phonon contribution to the activation energy, $E_a^{e-ph}$. For this part, we computed the electron-phonon activation energy of pristine and Li-doped CuO via Eq.~\ref{cuo:eq:eaeph}. This method represents an averaged doping effect in a continuum polarization medium and avoids the sampling of all possible doping configurations and hopping paths.~\cite{austin1969polarons}
\begin{align}
    E_q^{e-ph} = \frac{e^2}{4\varepsilon_p} \left( 1/r_p - 1/R \right)
    \label{cuo:eq:cfe}
\end{align}
Here $r_p$ is the polaron radius which is approximated as $r_p=1/2 (\pi/6)^{1⁄3} V^{1⁄3}$, $R$ is the average hopping distance, and $1/\varepsilon_p =1/varepsilon_\infty -1/\varepsilon_0$ where $\varepsilon_\infty$ is the high frequency dielectric constant and $\varepsilon_0$ is the static dielectric constant. The results of this calculation (Table 1) show that Li doping increased the high frequency dielectric constant ($\varepsilon_\infty$) due to increased carrier concentrations after Li doping. Although the static dielectric constant ($\varepsilon_0$) is also increased due to weaker Li-O bonds ($\varepsilon_0$ is inversely proportional to the bonding energy squared)\cite{gonze1997dynamical}, an increased $\varepsilon_\infty$ dominated and resulted in an overall lower barrier ($E_a^{e-ph}$). For example, the barrier decreases by 11 meV after 12.5\% Li doping, which corresponds to 1.5 times improvement on hopping mobility based on the $\mu \propto e^{-E_a/kT}$ relation between $E_a$ and mobility $\mu$. Therefore, Li doping assists the electron-phonon kinetics of carriers in CuO.

To consider the effect of Li on the magnetic contribution to the activation energy $E_a^{spin}$, we first recalculated the magnetic couplings in CuO after a significant amount of Li doping (12.5\%) using the same methods as before (Figure S8, Table S3-S4). We find that the predominate magnetic coupling $J_z$ is nearly the same after Li doping, although overall Li suppresses the anti-ferromagnetism of CuO due to the spinless character of Li, which has also been seen experimentally.~\cite{zheng2004fast} The resulting energy of the spin-flip process in Li-doped CuO from Eq.~\ref{cuo:eq:easpin} (assuming that there are no Li near the polarons) would be 137 meV, which is smaller than 160 meV in pristine CuO (mentioned above). We note that the largest benefit of Li doping is seen when we consider the interaction of neighboring SPs and Li. An analogue of the spin-flip hopping process after Li doping in CuO is shown in Figure 3. Since Li is non-magnetic, it does not interact with a SP when it passes by, and a single Li site can reduce the local hopping barrier of the SP by up to 55 meV which corresponds to approximately 9 times improvement of hopping mobility based on the $\mu \propto e^{-E_a/kT}$ relation (the case of Li breaking $J_z$ coupling). This larger effect of Li doping on the activation energy describes how Li doping significantly enhances the hole mobility in CuO, in agreement with previous experimental measurements of the activation energy of Li-doped CuO.\cite{zheng2004fast,gao2007ferromagnetic,zheng2003effect,chiang2016dopant} For example in Ref.~\cite{zheng2004fast}, a monotonic decrease of activation energy has been observed as a function of Li doping concentration (up to 16\%), accompanied by strong suppression of the anti-ferromagnetism of CuO. The activation energy decreases from 0.23 eV for pristine CuO to 0.035 eV for $\rm Cu_{0.92}Li_{0.08}O$, which leads to a three order of magnitude decrease of resistivity in experiments.~\cite{zheng2004fast} Since what we have discussed is relevant for isolated Li doping (Li-Li interaction is neglected in our magnetic interaction models), even a small amount of Li doping will have a significant impact on the conduction of holes in CuO and can dramatically increase the photocurrent density of CuO as we have seen in our experimental investigation.~\cite{chiang2016dopant}


% Experimental comparison of photoelectrochemical properties of CuO and Li-doped CuO electrodes
\subsection{Experimental comparison of CuO and Li-doped CuO electrodes}
Since the direct measurement of charge transport properties of our high surface area, nanofibrous polycrystalline electrodes was not possible, the effect of Li doping on charge transport properties was evaluated by comparing photocurrent generation of CuO and Li-doped CuO electrodes.  Since these electrodes have the same absorbance (Figure S15B), the number of electron-hole pairs generated in these electrodes under illumination must be identical.  Then, if an interfacial charge transfer reaction that can quickly consume almost all the surface reaching electrons is chosen for photocurrent measurement, any change in photocurrent generation caused by Li doping must be due to a change in the number of surface-reaching electrons caused by a change in the charge transport properties, which affects electron-hole separation.

For this purpose, comparing photocurrent for water reduction may not be proper because the surface of CuO is not catalytic for water reduction, and a significant portion of the surface reaching electrons can be lost to surface recombination, making it difficult to accurately evaluate the change in the number of surface-reaching electrons. In this study, we used oxygen reduction as the photoelectrochemical reduction reaction that occurs on the CuO surface as the kinetics of this reaction is typically much faster than water reduction on oxide-based photocathodes.~\cite{cardiel2017electrochemical,kang2016photoelectrochemical,read2012electrochemical,wheeler2017photoelectrochemical} The J-V plots of CuO and Li-doped CuO for oxygen reduction obtained in a 0.1 M KOH (pH 13) solution purged with \ce{O2} under standard illumination conditions (AM1.5G, 100 mW/cm2) are shown in Figure 5.

The pristine CuO electrode already shows efficient photocurrent generation for ce{O2} reduction as its bandgap allows for the utilization of a great portion of the visible solar spectrum, and its nanostructure reduces bulk electron-hole recombination.  For example, it achieved a photocurrent density of $\sim 1.2\ {\rm mA/cm^2}$ at a potential as positive as 0.8 V vs. RHE, and it increased up to $\sim 4.0\ {\rm mA/cm^2}$ when the potential was swept to 0.6 V. (The dark current initiating around 0.65 V vs. RHE is due to electrochemical reduction of ce{O2} which was subtracted from the photocurrent to determine overall photocurrent.) The photocurrent observed for \ce{O2} reduction can be considered the upper limit of photocurrent that can be observed for water reduction when an efficient hydrogen evolution catalyst is placed on the CuO surface to improve the water reduction kinetics.

The Li-doped CuO electrode significantly enhanced photocurrent generation. For example, the Li-doped CuO electrode achieved a photocurrent density of $\sim 2.0\ {\rm mA/cm^2}$ at a potential as positive as 0.8 V vs. RHE, and it increased up to $\sim 5.6\ {\rm mA/cm^2}$ when the potential was swept to 0.6 V. In addition to the evident increase in magnitude of photocurrent density, Li-doped CuO electrodes demonstrated a considerable shift in photocurrent onset to the positive direction by $\sim 210$ mV. The photocurrent onset potential for a reaction that has high interfacial charge transfer kinetics, such as \ce{O2} reduction on an oxide photocathode, can be considered the flatband potential. This is because for such reactions the loss of the surface-reaching minority carriers to surface recombination is negligible.  In this case, it can be assumed that photocurrent disappears when the applied potential is the same as the flatband potential, where electron-hole separation is no longer possible.  The fact that the J-V plots of CuO and Li-doped CuO electrodes measured with chopped illumination do not show any transient photocurrent even when the applied potential is near the photocurrent onset potential is a good indication that recombination on the CuO surface during \ce{O2} reduction is negligible. This confirms that the photocurrent onset potentials of these electrodes can be regarded as their flatband potentials. Since the flatband potential is the same as the Fermi level after accounting for the Helmholtz layer potential drop at the semiconductor/electrolyte interface, and the Helmholtz layer potential drop should not be altered by 0.1 at. \% Li doping, the shift of the onset potential of Li-doped CuO directly indicates that Li doping shifted the Fermi level of CuO to the positive direction, closer to the valance band maximum.\cite{nozik1978photoelectrochemistry}

These experimentally obtained results agree well with computational results that Li doping generates shallow acceptors that effectively increase the hole concentration. The increase in hole concentration, which improves the hole conductivity, can reduce electron-hole recombination in the bulk or in the space charge region, increasing the number of minority carriers reaching the surface to perform oxygen reduction. Also, the increase in hole density that changed the Fermi level was confirmed by the shift of the flat band potential to the positive direction. Finally, according to our computational results, a simultaneous decrease in $E_a$ by Li doping also contributed to photocurrent enhancement by improving the hole mobility of CuO.

While the impact of Li doping on the activation energy $E_a$ and the impact on carrier density cannot be easily separated in our photocurrent measurements, the impact of Li doping on the activation energy has been discussed explicitly in dark resistivity measurements of Li doped CuO between a few K to 300 K.~\cite{zheng2004fast,gao2007ferromagnetic} An order of magnitude decrease of the hopping activation energy with 16 at.\ \% Li doping clearly confirmed the combined effect of $E_a^{spin}$ and $E_a^{e-ph}$ being lowered by Li doping, as the effect of $E_a^{e-ph}$ alone cannot explain the observed order of magnitude decrease in the hopping activation energy based on our calculations.~\cite{zheng2004fast} This study clearly demonstrated that the effect of Li-doping on $E_a^{spin}$ is still considerable at 300 K (because of the short-range magnetic couplings remaining above N\'eel temperature)\cite{zheng2001evidence,yang1989magnetic} and that Li-doping can play a critical role in improving the mobility of CuO at room temperature, which is relevant for its PEC applications.

\subsection{Conclusions}
In conclusion, we have studied in-depth hole conduction in pristine and Li-doped CuO by first-principles calculations accompanied by the PEC performance of experimentally prepared CuO and Li-doped CuO electrodes. In pristine CuO, we have verified the existence of spin polarons (SP), which occur via the flip of a single Cu ion’s spin so that the polaron may redistribute over several atoms, and this delocalization effect lowers the energy of the polaron state. We then showed how transport of SPs in CuO will involve a spin-flip hopping process and developed a theoretical framework of computing the activation energy which involves both electron-phonon and magnetic coupling contributions. Next, we displayed how Li doping in CuO generates shallow states above the valence band which pushes the Fermi level closer to the valence band maximum and improves hole concentrations in CuO. Then, we showed how Li doping improves hole hopping mobility in CuO by lowering the electron-phonon coupling contribution to the activation energy due to higher electronic screening. More importantly, we demonstrated that Li doping lowers the magnetic coupling contribution to the activation energy due to the destruction of magnetic interactions through the replacement of Cu ions with non-magnetic Li ions, culminating in a significantly lowered hopping barrier and increased hole mobility in Li-doped CuO. Finally, we prepared CuO and Li-doped CuO electrodes and compared their photoelectrochemical properties for \ce{O2} reduction, where the changes in photocurrent and the onset of photocurrent can be directly related to changes in charge transport properties and the Fermi level, respectively.  The experimental results show that Li doping enhances charge transport properties and shifted the Fermi level toward the valence band maximum while not affecting photon absorption, which agrees well with computational results. This work provides important insights on the mechanisms of the formation and transport of SPs and their effect on the charge transport properties of CuO and Li-doped CuO.  Similar to Li doping, doping with other non-magnetic shallow acceptors may also simultaneously improve carrier concentration and hopping mobility of magnetic oxides. In this case, shallow dopants can be ionized easily to increase carrier concentrations and increase dielectric screening, which weakens the charge-lattice interactions. Most importantly, non-magnetic dopants can break the magnetic couplings and lower the hopping barrier for SPs significantly, which is critical for improvement of hopping mobility.  These insights offer effective strategies for the improvement of hopping conduction in magnetic oxides through atomic doping, which provides important guidance for materials design.

% COMPUTATIONAL METHODS
\subsection{Computational Methods}
Cupric oxide (CuO) assembles in a monoclinic structure with C2/c symmetry and a geometric unit cell consisting of only 8 atoms. To consider the correct magnetic interactions prevalent in CuO, a $\sqrt{2}\times 1\times \sqrt{2}$ unit cell containing 16 atoms needs to be implemented (Figure S1).\cite{rocquefelte2012theoretical,yang1989magnetic,forsyth1988magnetism} Meanwhile, a $2\times 3\times 2$ supercell of 96 atoms was used for calculations considering polaron formation and doping. A final supercell of $2\sqrt{2}\times 3\times 2\sqrt{2}$ with 192 atoms was used to confirm spin polaron formation size and charged cell correction.

It is well known that both local and semi-local exchange and correlation functionals in DFT cannot accurately describe the electron correlation in magnetic insulators, which results in a qualitatively incorrect electronic structure. To account for this issue, we applied the Hubbard $U$ correction\cite{dudarev1998electron} with $U$ = 7.5 eV, a well-established model for this material.~\cite{peng2012density,peng2014ab,heinemann2013band,debbichi2012vibrational,wu2006lsda} All calculations were carried out in the open-source plane wave code Quantum ESPRESSO64 with ultrasoft LDA pseudopotentials,\cite{gbrv} unless otherwise noted. The choice of LSDA+U instead of GGA+U was made because GGA+U was unable to give the correct monoclinic structure of CuO, while LSDA+U yielded a geometry of CuO within 5\% of experimental values. Nonetheless, LSDA+U and GGA+U provide similar electronic structures (with the experimental geometry) and overall gave results in agreement with experimental expectations.  Our calculations yielded that Cu$^{2+}$ ions have a magnetic moment of 0.57 $\mu_B$ with a magnetic ordering according to Figure S1. Notably, the O atoms in this system share a non-negligible magnetic moment of 0.14 $\mu_B$ (in agreement with previous experiments\cite{forsyth1988magnetism} and theory\cite{peng2012density}). We were also able to replicate the computed magnetic couplings at higher levels of theory with the LSDA+U method (see Table S1-S2).

\def\co{Co$_\text{3}$O$_\text{4}$}
\def\vo{V$_\text{O}$}
\def\vco{V$_\text{Co(O)}$}
\def\vct{V$_\text{Co(T)}$}

\section{Co3O4}

Polarons, conduction electrons or holes with self-induced lattice polarization, are known to exist in most transition metal oxides (TMO) and deeply affect their optical and carrier transport properties \cite{reticcioli2019small}. In these materials, much of the interest has been related to the role of small polarons (SPs) that form when the induced lattice polarization is localized in a volume on the order of the unit cell. In particular, for many important TMOs, including \ce{Fe2O3}~\cite{sivula2011solar,ling2011sn,smart2017effect}, \ce{NiO}~\cite{gong2014nanoscale,hu2014efficient}, \ce{Co3O4}~\cite{wang2018phosphorus,aijaz2016}, \ce{MnO}~\cite{jin2015partially}, \ce{BiVO4}~\cite{wu2018combining,zhang2018unconventional,seo2018role,kim2015simultaneous}, \ce{CuO}~\cite{cardiel2017electrochemical,smart2018mechanistic}, it has been found that the formation of SPs is responsible for the low carrier mobility and conductivity, which hinders their practical application as electrochemical catalysts and photoelectrochemical (PEC) electrodes \cite{lee2019progress,tachibana2012artificial,roger2017earth,yan2016review, liao2013new}. It is also well-established that the transport of SPs in these TMOs can be characterized through the thermally activated hopping conduction mechanism and a logarithmic temperature dependence of the materials carrier mobility \cite{mott1968conduction}.

Unlike the distinct signature of SPs on the carrier conduction discussed above, the effect of polarons on electronic structure and optical transitions in TMOs is rather complex and difficult to elucidate. For example, in several TMOs, such as WO$_3$~\cite{gerosa2018role,ping2013}, TiO$_2$~\cite{TiO22018,Moser2013,Kang2010} and SrTiO$_3$~\cite{verdi2017origin,Wang2016}, the presence of large polarons that are delocalized over several unit cells may lead to a strong band gap renormalization through electron-phonon coupling. By contrast, the formation of SPs may introduce isolated gap states away from the band edges due to their spatially localized nature, which could be easily misinterpreted as band edges that define the fundamental band gap. A prime example is \ce{Fe2O3}, where recent time-resolved spectroscopy experiments have shown that its mid-gap states are indeed associated with optically active polarons, which in turn lead to transition energies that are significantly lower than the fundamental band gap \cite{carneiro2017excitation,biswas2018ultrafast}. This conclusion is also consistent with Lohaus \textit{et al.} \cite{lohaus2018limitation}, where the authors showed that while the band gap of \ce{Fe2O3} is 2.2~eV, an effective gap of 1.75~eV is observed due to the formation of small polarons. Nonetheless, distinguishing mid-gap states due to SP formation from other sources such as defect-bound states \cite{huda2010electronic,neufeld2015platinum,sanson2015polaronic} and surface states \cite{yatom2015toward} is still not well understood in literature. Despite extensive experiments on these TMOs, theoretical studies of SP effects on electronic structure and optical properties of TMOs are limited. More importantly, challenges remain in first-principles methods that accurately describe polarons and electronic structure of TMOs.

\begin{figure}[t]
\begin{center}
\includegraphics[keepaspectratio=true,width=6cm]{2-tmo/figures-co3o4/diag_poly_label.png}
\caption{Normal spinel atomic structure of \co. Octahedral Co are shown in green, tetrahedral Co are shown in blue/light blue (distinguishing spin polarization direction), and O are shown in red.}  \label{co3o4:fig:struct}
\end{center}
\end{figure}

In this paper, we discuss the role of SPs (hereinafter referred to as ``polarons'' for simplicity) in tricobalt tetraoxide (\co), an anti-ferromagnetic oxide with a normal spinel structure. Despite that \co\ has been extensively investigated for a wide range of technologies \cite{wu2010,li2005,ma2014,wang2018,hamdani2010,aijaz2016,meher2011,xia2011}, a fundamental understanding of the optical properties of this material remains largely lacking, and conflicting results have been reported, e.g., for the band gap of bulk \co. For instance, a value of 1.5-1.7~eV has been commonly reported for the optical gap of \co~\cite{lohaus2016,jiang2014,waegele2014,shinde2006,belova1983,lima2014,cheng1998}. On the other hand, several experimental studies conclude that, despite a transition being observed around 1.5-1.7~eV, the true band gap of \co\ is significantly smaller, yielding a value of around 0.7-0.9~eV~\cite{qiao2013,singh2014,sousa2019,martens1985}. This conclusion, however, is not supported by time-resolved optical spectroscopy measurements, which suggest the state at $\sim$0.8~eV above the valence band maximum is a localized polaron state \cite{waegele2014,jiang2014,wheeler2012}. Along this direction, other experiments have indicated that the intrinsic carriers in \co\ are hole polarons that are characterized by a nearest-neighbor hopping conduction mechanism; and such a signature implies that hole polarons may affect the optical properties of \co\ in a similar way as in \ce{Fe2O3} \cite{ngamou2010,tronel2006,shinde2006,lohaus2016,sahoo2013,sparks2018,cheng1998,koumoto1981}. Collectively, the existing results indicate that much is left to be understood regarding the nature of the optical transitions near the band edge of \co, and how it is related to polaron formation.

The aim of this work is to resolve the conflicting results in the literature on \co, and provide a coherent description of its electronic structure, carrier conduction, and optical properties through first-principles calculations. In particular, we discuss the level of theory needed for a proper description of the electronic structure of the material. In addition, we elucidate the role of polaron formation on the electronic band gap and optical spectra of $p$-doped \co, and we discuss how uniaxial strain can be used to distinguish polaron related transitions in the optical spectra. This work provides a straightforward method for considering SP effects in the optical absorption, alongside unambiguous SP peak assignment in agreement with several previous experimental studies. This work will help to distinguish SP formation from other optical effects often considered instead (e.g. exciton formation, thermal broadening of optical spectra and electron-phonon renormalization of band edges) for TMOs in general. Our study presents a roadmap for first-principles calculations in the investigation of SP effects on optical absorption.

\subsection{Computational Methods}

\begin{figure}[t]
\begin{center}
\includegraphics[keepaspectratio=true,width=9cm]{2-tmo/figures-co3o4/koop.png}
\caption{\textbf{a.} Generalized Koopmans' condition for electron polaron (EP) and hole polaron (HP) in \co. The exact exchange $\alpha$ for the PBE0($\alpha$) method is varied until the condition $\text{HOMO}_q=\text{LUMO}_{q+1}$ (at fixed geometry where polaron has formed) is met. In both cases, we find that at an exact exchange of 0.12 Koopmans' condition is satisfied. The corresponding pristine gap is computed to be 1.70 eV. \textbf{b.} Localized and \textbf{c.} delocalized hole wavefunction, subject to the value of the exact exchange. Isosurface plots use a cutoff of 10\% the maximum.}
\label{co3o4:fig:IPEA}
\end{center}
\end{figure}

We begin by discussing our computational strategy for addressing the electronic properties of \co{} (spinel structure shown in Figure \ref{co3o4:fig:struct}). It is well known that density functional theory (DFT), with conventional local or semi-local exchange-correlation functionals, is not sufficient to provide a proper description of polarons in transition metal oxides and often severely underestimates the band gap of these materials \cite{dudarev1998electron}. In order to mitigate this issue, several approaches have been proposed, including DFT+$U$ with an orbital specific Hubbard $U$ correction, and hybrid functional that includes a fraction of Hartree-Fock exchange ($\alpha$), hereinafter denoted as PBE0($\alpha$). However, these calculations are known to highly depend on the choice of the Hartree-Fock exchange or Hubbard $U$ correction. For instance, a wide range between 0.8 and 2.0~eV has been reported in the literature for the band gap of \co, depending on the level of theory employed \cite{singh2014}. In this context, it is also necessary to emphasize that, despite significant development has been made, establishing a first-principles approach that allows for an accurate prediction of the electronic properties of TMOs remains a significant challenge \cite{kent2018toward}.

Here, we implement both hybrid functional and DFT+$U$ calculations to provide an unbiased description of the electronic properties of \co. Notably, in variation with previous hybrid functional calculations, we invoked the generalized Koopmans' condition to determine the value of $\alpha$ from first-principles. We stress that this strategy has been shown to successfully predict the band gap of materials with band gaps up to 14 eV and has been particularly successful for polaronic systems \cite{miceli2018nonempirical,smart2018fundamental,liu2018electron,lany2011predicting}. Specifically, the generalized Koopmans' condition enforces the condition $\text{IP}_q=\text{EA}_{q+1}$ at a fixed geometry for an isolated state in the materials, where $\text{IP}_q$ is the ionization potential of an occupied state at the charge state $q$, whereas $\text{EA}_{q+1}$ is the electron affinity of the same state when it is unoccupied at the charge state $q+1$.  Here, we determine the value of $\alpha$ by enforcing the condition $\text{IP}_q=\text{EA}_{q+1}$ for both the hole and electron polaron (see Figure~S1 for more details of the electron polaron), and we obtained a value of 0.12 for the Hartree-Fock exchange $\alpha$ in both cases, as shown in Figure \ref{co3o4:fig:IPEA}. We note that hole polarons do not form for $\alpha$ below 10\%, as illustrated in Figure \ref{co3o4:fig:IPEA}b-c). The value of $\alpha$ determined in this manner is an intrinsic property of the bulk system, and the choice of defect used to enforce the Koopmans’ condition is proper as long as it has minimum hybridization with the bulk Bloch states \cite{bischoff2019adjustable,miceli2018nonempirical,smart2018fundamental}.

We then determined $U$ parameters based on the hybrid functional results, and the stable formation of electron and hole polarons (see Table~S1 for more details). In particular, we find that the use of $U$ values of $U_\text{Co(O)}=4$ eV and $U_\text{Co(T)}=3$ eV provides consistent results with the Koopmans' compliant hybrid functional (band gap agrees within 0.1~eV), as well as a proper description of the electronic properties of the system, as discussed later in this communication. We also note that hole polarons do not form for $U$ values below 2.5 eV (see Figure~S2).

\begin{figure}[t]
\begin{center}
\includegraphics[keepaspectratio=true,width=8cm]{2-tmo/figures-co3o4/crystal.png}
\caption{\textbf{a.} Projected density of states (PDOS) of Co(O) on $t_{2g}$ and $e_g$ orbitals. Schematic representation of the electronic configuration of \textbf{b.} octahedral Co$^{3+}$ and \textbf{c.} tetrahedral Co$^{2+}$ in \co\ due to a crystal field splitting ($\Delta_{oct}$ and $\Delta_{tet}$, respectively). \textbf{d.} PDOS of Co(T) on $t_2$ and $e$ orbitals. All the PDOS was computed with DFT+$U$.}
\label{co3o4:fig:split}
\end{center}
\end{figure}

All the calculations were then carried out using the plane-wave code Quantum ESPRESSO \cite{QE1} with norm-conserving pseudopotentials \cite{ONCV1}. A plane  wave cutoff of 100~Ry was used in all PBE+$U$ calculations, while a reduced cutoff of 50~Ry was implemented for the more demanding hybrid functional calculations (geometry and electronic structure are converged at 50 Ry). The calculations were generally performed within the 56 atom cubic cell with a $2\times 2\times 2$ k-point mesh for integration over the Brillouin zone. In addition, a $\sqrt{2}\times\sqrt{2}\times 2$ supercell with 224 atoms was utilized for comparison, particularly to understand the finite-size effects on polaron formation. Nevertheless, we find that the wavefunction character, energy level splitting, and band structure look largely unchanged between the 56 atom cubic cell and the 224 atom supercell (see Figure~S3 for further details). In all calculations with electron or hole polarons, the charged cell correction scheme as developed in Ref.~\cite{PING2017JCP} was employed, which is necessary in order to remove the spurious interactions of the polarons with their periodic images and with the uniform compensating background charge \cite{kokott2018first}. For the rest of the manuscript, unless otherwise noted, the results presented here were obtained at the DFT+$U$ level of theory that is computationally less demanding compared to hybrid functional calculations.

%\section{Results and Discussion}

\subsection{Electronic Structure of Pristine \co{}}

To set a baseline for the discussion of polaron effects in \co, we briefly summarize the electronic structure of the pristine system. As shown in Figure \ref{co3o4:fig:struct}, \co\ assembles in a normal spinel structure, where two thirds of Co occupy octahedral sites (denoted by Co(O)) and the remaining third occupy tetrahedral sites (denoted by Co(T)). We find that Co(O) exhibits a low-spin $3d^6$ orbital configuration with the $O_h$ symmetry, leading to filled $t_{2g}$ and empty $e_g$ bands, as shown in the calculated projected density of states (PDOS) and band diagram presented in Figure~\ref{co3o4:fig:split} a-b. On the other hand, Co(T) with the $T_d$ symmetry forms a high-spin $3d^7$ configuration with a half-filled $e_g$ band, yielding an overall magnetic moment of $\sim$3.2 $\mu_B$ as already reported in experiments \cite{roth1964}. These Co(T) sites experience an anti-ferromagnetic interaction mediated through a super-exchange of mutually bonded oxygen; in addition, the presence of a large Hund's spin exchange results in a splitting of the $t_{2}$ band. This is shown in Figure \ref{co3o4:fig:split} c-d, where we find that the $t_{2}$ minority spin states are formed at a higher energy level, whereas all majority spin states occur at lower and similar energies \cite{lima2014,chen2011} (also see Figure~S4 for further details). Overall, the behavior of the spin states and band splitting presented here are consistent with results reported in existing theoretical studies \cite{chen2011,wu2016}.

\subsection{Formation of Hole Polarons with Mid-Gap States}

Next, we discuss the nature of polaron formation in \co. Our calculations show that, among the Co(O) and Co(T) sites where hole polarons can form, the total energy of a hole polaron forming at Co(O) is lower than that of Co(T) by at least 70~meV. A more stable formation of the polaron on Co(O) is also reflected in the calculated density of states of \co\, where a larger contribution of Co(O) $d$ states is found at the valence band edge compared to the Co(T) $d$ states (see Figure~S4). Finally, our conclusion is consistent with the experimental study reported by Ngamou \textit{et al.} \cite{ngamou2010}, where the authors show that the hopping of polarons takes place in the octahedral sites and are responsible for driving the electrical transport in the oxide. Collectively, these results indicate that the computational approach employed here provides a proper description of polaron formation in \co.

\begin{figure*}%[t]
\begin{center}
\includegraphics[keepaspectratio=true,width=0.98\linewidth]{2-tmo/figures-co3o4/hole_diag_224.png}
\caption{\textbf{a.} Pristine band structure of \co\ with a 224 atom supercell (primitive cell band structure shown in SI Figure~S5). \textbf{b.} Hole polarons create a low-spin (LS) $d^5$ configuration at Co(O) along with a Jahn-Teller (JT) distortion which results in a $D_{4h}$ configuration and the creation of several mid-gap states. \textbf{c-e.} Wavefunction isosurface plots (yellow cloud) of the three polaron induced states under hole formation of $a_1(\uparrow)(d_{x^2-y^2})$, $b_1(\uparrow)(d_{z^2})$, and $b_2(\downarrow)(d_{xy})$ character, respectively. Isosurface plots use a cutoff value of 10\% the maximum. \textbf{f.} Band structure of \co\ with a hole polaron which shows several induced gap states (blue = spin up, black = spin down).}  \label{co3o4:fig:hole}
\end{center}
\end{figure*}

Beyond the findings on the thermodynamical stability of hole polaron formation in \co, our calculations show that the hole polaron at Co(O) leads to several mid-gap states. As shown in Figure \ref{co3o4:fig:hole}, we find that upon the hole polaron formation, Jahn-Teller (JT) distortion occurs at the Co(O) site due to an uneven occupation of the $t_{2g}$ band, and splits the degeneracy of the $O_h$ states. In addition, the uneven occupation of up/down states splits the spin degeneracy due to the on-site Coulomb repulsion of the $d$ orbitals. Specifically, we find that the majority spin states (e.g. $b_2(\uparrow), a_1(\uparrow), b_1(\uparrow)$) are located at a lower energy, whereas the minority spin states (e.g. $b_2(\downarrow), a_1(\downarrow), b_1(\downarrow)$) are pushed higher in the energy. Such splitting and ordering is consistent with the previous time-resolved spectroscopy measurements of Co(O) \cite{wu2016,belova1983}.
In addition, these mid-gap states are consistent with the experiment reported in Ref.~\cite{jiang2014}, where it was found that mid-gap excitations are associated with $a_1(\uparrow)$ and $b_1(\uparrow)$ states of Co(O). As a result, our analyses point to significant effects of hole polarons on the electronic structure of \co, most notably in the formation of mid-gap states in a similar way as found in \ce{Fe2O3}.

\subsection{Small Polaron Induced Optical Transitions}

We now turn to a more quantitative discussion of polaron effects on the electronic structure of \co. In particular, we obtain a band gap of 1.6~eV and 1.7~eV with the current choice of $U$ and $\alpha$ (respectively), which is in excellent agreement with the value of 1.5-1.7~eV reported in Refs.~\cite{lohaus2016,jiang2014,waegele2014,shinde2006,belova1983,lima2014,cheng1998}. However, this is significantly larger than the result of 0.7-0.9~eV claimed by other experiments. \cite{qiao2013,singh2014,sousa2019,martens1985}
We note that exciton binding energies are usually less than 150 meV in many TMOs \cite{ping20132electronic,le2014,ping2013}, and we show later that the calculated optical spectra, by including excitonic effects, cannot explain the low energy transition at 0.7-0.9~eV.
Interestingly, as shown in Figure~\ref{co3o4:fig:hole}b, at the current level of theory, we find that the mid-gap states are located at 0.8~eV away from the valence band maximum, and are associated with the hole polaron formation. This observation suggests that a scenario similar to the one observed for \ce{Fe2O3} may also occur in \co, i.e., the true gap of \co\ is $\sim$1.6~eV, and the mid-gap states are responsible for the transitions found at $\sim$0.8 eV in the experimental optical absorption spectra \cite{qiao2013,singh2014,sousa2019,martens1985}.
In order to verify our hypothesis and to further elucidate the role of polaron formation, we calculated the optical absorption spectra of \co\ with and without a hole polaron, and we compared the results with available experimental data. Absorption spectrum with a hole polaron was computed with a 56-atom supercell \cite{concentrationNote}, which better represents experimental hole concentrations of $p$-type \co\ \cite{waegele2014,lohaus2016} due to abundant cation vacancies \cite{tronel2006,godillot2013}.

Therefore, we computed the imaginary part of the dielectric function in the random phase approximation (RPA) with local field effects (as shown in Figure 5) and solving the Bethe-Salpeter equation that includes excitonic effects (as shown in SI Figures S8 and S9), as implemented in the YAMBO-code \cite{YAMBO}, using the single particle eigenvalues and wavefunctions derived from DFT+$U$. We note that this choice of starting point considers the balance between accuracy and computational cost, similar to this work in Ref.~\cite{Claudia2012}. In addition, for a direct comparison with experimental measurements, we calculated the absorption coefficient from the dielectric function \cite{jackson1999}.
%
\begin{align}
    A(\omega)=\frac{\omega}{c}\frac{\epsilon_2(\omega)}{\sqrt{\frac{\epsilon_1(\omega)+\sqrt{\epsilon_1(\omega)^2+\epsilon_2(\omega)^2}}{2}}}
\end{align}
%
\begin{figure}[t]
\begin{center}
\includegraphics[keepaspectratio=true,scale=1]{2-tmo/figures-co3o4/abs.png}
\caption{Optical absorption of \co\ in the pristine system (black) and the $p$-doped system (blue). Notably, only $p$-type doping i.e. the formation of holes, will cause mid-gap transitions below 1.6 eV, in agreement with experimental optical spectrum of \co\ shown in green \cite{qiao2013}. The inset image displays the mid-gap transitions which are labeled according to the states formed from hole polaron formation as in Figure \ref{co3o4:fig:hole}.
(Theoretical spectrum is an average of spectra with light polarized in the [100], [010], and [001] directions.)}  \label{co3o4:fig:abs}
\end{center}
\end{figure}

The calculated optical absorption spectra of \co\ are shown in Figure~\ref{co3o4:fig:abs}, together with the experimental spectrum. We find that the introduction of hole polarons leads to the formation of several lower energy optical transition peaks between 0.6 and 1.2 eV in \co. More importantly, we find that, in sharp contrast to the result obtained for the pristine system, the spectrum computed for \co\ with a hole polaron is in very good agreement with experimental data, where three lower lying transitions were also found between 0.7 and 1.1~eV \cite{qiao2013,lohaus2016,jiang2014,waegele2014}. Our analysis indicates that these transitions can be associated with those occur between the $p$-$d$ hybridized dispersive valence states and the localized $d$ states formed at the hole polaron site, for which the wavefunctions are illustrated in Figure \ref{co3o4:fig:hole} c-e. Collectively, these results clearly support the interpretation that the true optical gap of \co\ is $\sim$1.6 eV and that the optical transitions observed at $\sim$0.8 eV are due to hole polaron formation at Co(O) sites.
We note that electron polarons do not lead to the formation of low lying transitions (see Figure~S6), indicative of the $p$-doped nature of the experimental system.

In order to rule out the possibility that the low energy transitions $\sim$0.8 eV are caused by large excitonic effects~\cite{Laskowski2009,Bruneval2006,Wiktor2018,Rodl2012}, we also computed absorption spectra of pristine \co\ including excitonic effects by solving the Bethe-Salpeter Equation, as shown in SI Figure~S7, S8. More detailed discussions can be found in SI. Overall, no extra peaks in the BSE spectra show up at the energy range below 1 eV for pristine \co\ (no hole polaron included), which confirms the excitonic effects do not explain the low energy transitions in the absence of SPs.

In addition, we note that we neglect electron-phonon coupling and thermal expansion effects on the band edge positions and absorption spectra at finite temperature, as discussed in Refs.~\cite{monserrat2018phonon,bravic2019finite}. For example, the absorption edge may be subject to a red-shift, in addition to an overall broadening of the spectra. This offers additional directions for future works. In any case, these effects will not lead to an additional peak well separated from the main absorption in a direct band-gap semiconductor like \co; therefore, our conclusion on the small polaron contribution to the low energy optical transitions still holds.

\subsection{Detecting Hole Polaron Transitions via Strain}

Finally, we propose an experimentally viable method for distinguishing optical transitions involving localized polaron states from traditional band-band bulk state transitions. For \co, the JT distortion upon the introduction of the hole polaron extends the Co-O bonds along the $C_4$ axis, and this distortion may occur along any of the bond axis that aligns with the [100], [010], [001] directions of the cubic unit cell. Such a three-fold degeneracy can be broken if uniaxial strain is applied to the system along one of the crystal lattice directions, which in turn may affect the optical absorption spectrum. In this regard, monitoring the change in the optical spectrum of \co\ in the presence of an uniaxial strain could potentially provide signatures of hole polarons associated with a specific JT distortion.

For demonstration, we considered a 1\% tensile strain applied along the [100] direction. We find that the three-fold degeneracy in the polaron states is broken upon the introduction of the strain; in particular, polaron formation with the JT elongation along the [100] direction is  lowered  in  the  energy  by  5  meV  compared  to  those  associated  with  the  [010]  or  [001] directions.   Here,  to  investigate  the  collective  and  individual  effects  of  these  polarons  on the absorption spectrum, we computed a thermally averaged ensemble spectrum by using a Boltzmann probability distribution of the optical absorption obtained for each case:
\begin{align}
    P_i = \frac{e^{E_i/kT}}{\sum_i e^{E_i/kT}} \quad , \quad A(\omega)=\sum_{i}P_i\,A_i(\omega)\label{co3o4:eq:bolt},
\end{align}
where $E_{i}$ and $A_{i}$ are the energy and absorption spectrum of the system containing a polaron in the state $i$ ($i$ denotes different JT elongation direction), respectively.

\begin{figure}%[t]
\begin{center}
\includegraphics[keepaspectratio=true,width=7cm]{2-tmo/figures-co3o4/strain.png}
\caption{Optical absorption of $p$-doped \co\ under 1\% uniaxial tensile strain along the [100] direction. Temperature dependence determines the probability for which direction the Jahn-Teller elongation will occur (as the degeneracy is removed under strain) and results in a red-shift of optical peaks related to the hole polaron.}  \label{co3o4:fig:strain}
\end{center}
\end{figure}

The calculated optical absorption spectrum presented in Figure \ref{co3o4:fig:strain} clearly shows a red-shift in the first peak that is associated with polarons. In particular, we find that at lower temperatures where $kT$ is on the order of 5 meV or less, the resulting optical spectra follow that of the lowest energy polaron associated with JT elongation along the [100] direction. At higher temperatures, the clear red-shift remains, although a high temperature of 300~K is sufficient to quench the 5~meV energy difference between polaron states. In contrast to the transition associated with hole polarons, we find that bulk band-band transitions at higher energy (above 1.5 eV) remain mostly unchanged upon uniaxial strain (see Figure~S9). Accordingly, this allows one to clearly distinguish the local polaron state involved in optical transitions, whose JT distortion renders them quite sensitive to strain, from that of the band-band bulk state transitions which are insensitive to strain.

\subsection{Conclusions}

To summarize, we present a detailed investigation of the electronic structure and polaronic induced optical transitions in \co\ based on first-principles calculations. We resolved several contradicting findings in the literature related to the character of the charge carrier and band gap of the material. In particular, we show that the optical gap of pristine \co\ is 1.6 eV, whereas the lower lying transition around $\sim$0.8~eV is associated with the hole polaron, which was misinterpreted as the band edge of the material. We also demonstrated the important effects of uniaxial strain on the optical spectra of \co, which in turn can be used to reveal the localized character of polaron-induced electronic states.

Our study also suggests a strategy for establishing a potential first-principles approach that can simultaneously achieve an accurate description of polaron states, electronic band structure and optical properties in polaronic magnetic oxides. Specifically, the generalized Koopmans' condition can be utilized to derive the fraction of exact exchange from first-principles, which in turn can be used in hybrid functional for investigating the electronic structure of the oxide. These hybrid functionals can also be used for benchmarking DFT-$U$ calculations, which offer a much lower computational cost, or to provide inputs for higher level electronic structure methods, such as many-body perturbation theory within the \textit{GW} approximation.


\section{Combining Theory and Experiment}

\subsection{\ce{BiFeO3}}

Bismuth iron oxide, \ce{BiFeO3}, is a semiconductor with a rhombohedrally distorted perovskite structure that yields a large ferroelectric effect.~\cite{choi2009switchable} For this reason, it has been investigated as one of the most promising candidates for ferroelectric diode devices and ferroelectric photovoltaics.~\cite{choi2009switchable,yi2011mechanism,you2018enhancing,spanier2016power} Recently, \ce{BiFeO3} was also reported as a photoelectrode in a solar water-splitting cell.~\cite{shen2017dual,yilmaz2016perovskite,moniz2015visible,song2018domain,liu2016enhanced} In these reports, \ce{BiFeO3} was demonstrated to have a relatively narrow bandgap of $sim$2.2 eV and conduction band minimum (CBM) and valence band maximum (VBM) positions that straddle the water reduction and oxidation potentials,~\cite{shen2017dual,yilmaz2016perovskite} all of which are very attractive features for a photoelectrode in a water-splitting photoelectrochemical cell (PEC). Considering that \ce{Fe2O3}, an extensively studied photoanode with a similar bandgap, has a CBM that is $sim$200 mV more positive than the water reduction potential,~\cite{sivula2016semiconducting} the shifts in the CBM and VBM to the negative direction constitute an important advantage of \ce{BiFeO3} over \ce{Fe2O3}.~\cite{lee2019progress}

To date, both n-type and p-type \ce{BiFeO3} photoelectrodes have been reported,~\cite{shen2017dual,yilmaz2016perovskite,moniz2015visible,song2018domain,liu2016enhanced} meaning that \ce{BiFeO3} can serve as a photoanode or a photocathode, respectively. In these studies, the doping type varied without the introduction of external dopants, suggesting that the defects responsible for n-type and p-type \ce{BiFeO3} can both readily form.~\cite{lee2019progress} The defects that cause n-type behavior include oxygen vacancies,~\cite{paudel2012intrinsic} and the defects that cause p-type behavior include Bi vacancies.~\cite{rojac2017domain}

Despite the many interesting and advantageous features shown for \ce{BiFeO3} as a photoelectrode, there have been very few systematic investigations on the photoelectrochemical properties of \ce{BiFeO3}. Because of its popularity as a ferroelectric material, most studies on \ce{BiFeO3} photoelectrodes have focused on how the application of an external electric field on \ce{BiFeO3} affects its photocurrent generation or on the conversion between n-type and p-type photocurrent.~\cite{song2018domain,liu2016enhanced,cao2014switchable,huang2016tunable} From a careful analysis of these papers, it appears that the \ce{BiFeO3} electrodes used in these studies were very lightly doped.~\cite{lee2019progress} For the purpose of accurately evaluating the potential of \ce{BiFeO3} as a photoanode or a photocathode, optimally doped n-type and p-type \ce{BiFeO3} electrodes need to be prepared and examined individually.

Considering that charge transport in many oxide-based photoanodes involves small polaron hopping,~\cite{wheeler2019combined,smart2019optical,smart2018mechanistic,smart2017effect,seo2018role,kim2015simultaneous} understanding the formation and transport of small polarons in \ce{BiFeO3} is also critical. Unfortunately, while numerous theoretical studies on \ce{BiFeO3} have been published to date, they have focused on its bulk polarization,~\cite{kan2011chemical,neaton2005first} photovoltaic effects,~\cite{young2012first,liu2013development} and multiferroic effects.~\cite{wang2012atomistic,chang2016prediction} Small electron polaron formation and its effects on dopant ionization energies and concentration of free carriers in \ce{BiFeO3} have not yet been investigated theoretically.

In this work, we conducted combined experimental and theoretical studies on n-type \ce{BiFeO3} photoanodes. For the experimental investigation, we prepared highly uniform \ce{BiFeO3} photoanodes by electrodeposition and examined their photoelectrochemical properties and stability for use in a PEC. We then intentionally introduced oxygen vacancies into the pristine \ce{BiFeO3} lattice and examined the effect on carrier concentration and photocurrent generation. The experimental results were compared with those of a computational study, which examined the formation of small polarons in \ce{BiFeO3} and the effect of oxygen vacancies on small polaron formation and free carrier generation in \ce{BiFeO3} for the first time. The new experimental and computational results discussed in this study will significantly increase our fundamental understanding of \ce{BiFeO3} for use as a photoanode material.

\begin{figure}
    \centering
    \includegraphics[keepaspectratio=true,width=0.9\linewidth]{2-tmo/figures-bifeo3/exp}
    \caption{
    (\textbf{a}) J-V plots and (\textbf{b}) J-t plots at 0.8 V vs. RHE for pristine \ce{BiFeO3} (red) and \ce{N2}-treated \ce{BiFeO3} (blue) for sulfite oxidation. All measurements were obtained in pH 9.2 borate buffer containing 0.7 M sulfite under 1 sun illumination (100 mW/cm$^2$, AM 1.5 G).
    }
    \label{bifeo3:fig:exp}
\end{figure}

Density functional theory calculations were performed using the Quantum ESPRESSO package~\cite{QE1} with PBE+U exchange correlation functional, ultrasoft pseudopotentials,~\cite{gbrv} and Hubbard $U$ parameters of 2 eV on O $2p$ and 3 eV on Fe $3d$. All calculations were done using the hexagonal \ce{BiFeO3} cell (space group: R3c), which we expanded to a $2\times 2\times 1$ supercell to avoid spurious interactions during defect calculations. A $2\times 2\times 2$ $k$-point grid was used to calculate the charge density, and a $4\times 4\times 4$ $k$-point grid was used for density of states. We applied a newly developed charge correction scheme~\cite{PING2017JCP} to calculations containing excess charge as implemented in JDFTx.~\cite{JDFTx} For the calculations used to investigate the effect of oxygen vacancies, a single oxygen atom was removed from a 120-atom supercell. Because 72 oxygen atoms are present in this supercell, this is equivalent to removing 1.39 atomic \% oxygen (1.39 oxygen atoms out of 100 oxygen atoms), leading to the empirical formula of BiFeO$_{2.96}$.

Theoretical optical absorption spectra of \ce{BiFeO3} with and without $\rm V_O$ were obtained by computing the imaginary part of the dielectric function in the random phase approximation with local field effects as implemented in the YAMBO code.~\cite{YAMBO} The input of this calculation came directly from our single particle eigenvalues and wavefunctions from DFT+$U$ computed in Quantum ESPRESSO. The absorption spectrum ($\alpha$) is related to the real and imaginary parts of dielectric function ($\epsilon_1$ and $\epsilon_2$, respectively) as shown in the equation below.~\cite{jackson1999}
\begin{align}
    \alpha(\omega) = \frac{\omega}{c} \frac{\epsilon_2(\omega)}{\sqrt{\frac{\epsilon_1(\omega) +\sqrt{\epsilon_1(\omega)^2+\epsilon_2(\omega)^2}}{2}}}
    \label{bifeo3:eq:alpha}
\end{align}

To gain additional insight into the photoelectrochemical properties of \ce{BiFeO3}, we conducted density functional theory (DFT) calculations to investigate the small polaron formation and its effect on defect ionization energy and free carrier concentration in \ce{BiFeO3}. First-principles calculations were carried out on \ce{BiFeO3} using the DFT+$U$ method (see Computational Methods for more information). The computed bandgap of \ce{BiFeO3} was 2.2 eV, which is in great agreement with the experimentally measured value of the bandgap.

Due to strong electron-phonon interactions, carriers in many transition metal oxides are trapped by their self-induced lattice distortions, forming small polarons.~\cite{geneste2019polarons} Small polarons conduct through the system via a thermally-activated hopping mechanism unlike carriers in covalent semiconductors, which conduct through conventional band mechanisms.~\cite{lu2010room,gheorghiu2013preparation} Therefore, understanding and facilitating small polaron hopping are critical for the development of oxide-based photoelectrodes.~\cite{wheeler2019combined,smart2019optical,smart2018mechanistic,seo2018role,wu2018combining,zhang2018unconventional}

% Small Polaron Formation.

\begin{figure}
    \centering
    \includegraphics[keepaspectratio=true,width=0.9\linewidth]{2-tmo/figures-bifeo3/prist}
    \caption{
        (\textbf{a}) Norm-squared wavefunction of the electron polaron (yellow cloud) shown as an isosurface in the \ce{BiFeO3} lattice (purple = Bi, gold = Fe, red = O). Isosurface value is 1\% of the maximum amplitude of the wavefunction; (\textbf{b}) projected density of states (PDOS) for \ce{BiFeO3} before (left) and after (right) a single electron-polaron is introduced in a 120-atom supercell.
    }
    \label{bifeo3:fig:prist}
\end{figure}

The formation of an electron polaron in pristine \ce{BiFeO3} was simulated by adding one extra electron into the pristine \ce{BiFeO3} system and allowing the system to relax. We observed that the extra electron spontaneously localizes on a single Fe site, forming a small electron polaron as shown in Figure~\ref{bifeo3:fig:prist}a. This creates a deep, localized state that lies 1 eV below the CBM of the pristine \ce{BiFeO3} as shown in Figure~\ref{bifeo3:fig:prist}b. The formation of similar localized electron polaron states has been observed in other Fe$^{3+}$-based oxides such as \ce{Fe2O3}.~\cite{smart2017effect}

The Fe$^{3+}$ ions in \ce{BiFeO3} have $O_h$ crystal field splitting with $3d^5$ high-spin electron configuration.~\cite{baettig2005first} Thus, the extra electron occupies an Fe $t_{2g}$ state, and the small polaron state has mainly $t_{2g}$ character. Furthermore, the presence of a small polaron on the Fe ion, which lowers its valency from +3 to +2, perturbs valence states and creates additional localized states above the VBM of the pristine \ce{BiFeO3} (Figure~\ref{bifeo3:fig:prist}b). These localized states have character of $e_g$ orbitals of Fe$^{2+}$ and $2p$ orbitals of oxygen. (The corresponding wavefunctions are shown in Figure S6.)


% Electronic structure of $\rm V_O$ in \ce{BiFeO3}.
Previous theoretical studies on defect formation in \ce{BiFeO3} reported that the oxygen vacancy is a very deep donor with an ionization energy greater than 1 eV,~\cite{paudel2012intrinsic,kay2006new,zhang2010density,shimada2016multiferroic} meaning that the oxygen vacancy cannot contribute to the generation of n-type carriers at room temperature. This disagrees with our experimental observation that the \ce{N2}-treated \ce{BiFeO3} that contains more oxygen vacancies has a higher carrier density and generates significantly more photocurrent at room temperature. We note that previous theoretical studies did not consider the formation of small polarons in \ce{BiFeO3} and their effects on defect ionization energies and carrier concentration. Employing recently developed methods,~\cite{smart2017effect,seo2018role} we revisited the formation of oxygen vacancies in \ce{BiFeO3} to investigate their ionization energies with respect to the free polaron level in pristine \ce{BiFeO3}.

When an oxygen vacancy ($\rm V_O$) is formed in the lattice of \ce{BiFeO3}, it introduces two electrons that spontaneously generate two small electron polaron states. These two states correspond to the two peaks shown $sim$0.8 eV below the CBM in the PDOS of \ce{BiFeO3} in Figure~\ref{bifeo3:fig:vo}a. As in the case of introducing a free electron-polaron, these polaron states have mainly $t_{2g}$ character of Fe$^{2+}$. Due to attractive electrostatic interactions between the electron polarons and the $\rm V_O$ site, the most thermodynamically stable configuration is the one with the two electron-polarons located at the Fe sites nearest to the $\rm V_O$, as shown in Figure~\ref{bifeo3:fig:vo}b. The difference in energy of the $\rm V_O$ polarons seen in the PDOS is a result of their differing distances from the $\rm V_O$ (Figure S7). The introduction of $\rm V_O$ and the resulting small electron polarons also generate perturbed valence states above the VBM. These perturbed states are mainly composed of the $e_g$ orbitals (\textit{i.e.}\ $d_{x^2-y^2}$) of Fe$^{2+}$ and $2p$ orbitals of oxygen (Figure S8). When absorption spectra of \ce{BiFeO3} with and without $\rm V_O$ were simulated and compared (Figure S9), we found that the presence of the perturbed states above the VBM did not affect the absorption of \ce{BiFeO3}. This agrees with our experimental results.


\begin{figure}
    \centering
    \includegraphics[keepaspectratio=true,width=0.9\linewidth]{2-tmo/figures-bifeo3/vo}
    \caption{
    (\textbf{a}) Projected density of states (PDOS) for \ce{BiFeO3} with a single oxygen vacancy ($\rm V_O$) introduced into a 120-atom supercell; (\textbf{b}) Norm-squared wavefunction of the two electron-polarons (yellow regions surrounding Fe). Isosurface value is 1\% of the maximum amplitude of the wavefunction. The $\rm V_O$ is indicated by a single empty red circle between the two electron-polarons; (\textbf{c}) Charge formation energy (FE) diagram of $\rm V_O$ in \ce{BiFeO3}.
    }
    \label{bifeo3:fig:vo}
\end{figure}



% $\rm V_O$ as a shallow donor in \ce{BiFeO3}.
In order to consider the effects of oxygen vacancies on the carrier concentration in \ce{BiFeO3}, it is necessary to compute the formation energy of the defect in each of its charge states q.
\begin{align}
    E^f_q = [\varepsilon_F] = E_q - E_pst + \sum_i \mu_i \Delta N_i + q\varepsilon_F + \Delta_q
\end{align}
Here, $E^f_q$ is the formation energy of a defect with charge $q$, $E_q$ is the total energy of the defect system with charge $q$, $E_{pst}$ is the total energy of the pristine system, $\Delta N_i$ is the change in the number of atomic species $i$ with chemical potential $\mu_i$, $\varepsilon_F$ is the Fermi energy, and $\Delta_q$ is the defect charge correction consistent with recent developments~\cite{PING2017JCP,JDFTx} that serves to remove spurious interactions of the charged defect with its periodic images and with the uniform compensating background charge. The value of the Fermi level ($\varepsilon_F$) in which the system undergoes a transition of charge state $q$ to $q'$ defines the charge transition level $\varepsilon^{q|q'}$.
\begin{align}
    \varepsilon^{q|q'} = \frac{E^f_q-E^f_{q'}}{q'-q}
\end{align}
Typically, the charge transition level of an electron donor from one charge state to a more positive charge state referenced to the CBM defines the ionization energy of the defect. However, in polaronic oxides, the feasibility of polaron hopping is determined not by the ionization energy of the defect with respect to the CBM, but by the ionization energy of the defect with respect to a free polaron state where the polaron is not bound to a defect.~\cite{smart2017effect,seo2018role} Therefore, the true ionization energy of small polarons is equal to the energy difference between the charge transition levels of the defects (solid red dots in Figure~\ref{bifeo3:fig:vo}c) and the free polaron level (grey dashed line in Figure~\ref{bifeo3:fig:vo}c). The free polaron energy level can be obtained from the formation energy of the pristine system with ($q=-1$) and without ($q=0$) an extra electron. The Fermi level corresponding to the $\varepsilon^{0|-1}$ transition in the pristine system defines the free polaron level.~\cite{smart2017effect,seo2018role} The energy difference between the free polaron level and the CBM is the polaron binding energy.

Under this model, we computed the charge formation energy diagram of the $\rm V_O$ and its corresponding ionization energies. In an oxygen-rich environment, the formation energy of a neutral $\rm V_O$ is 3.4 eV, which agrees with previous calculations.~\cite{paudel2012intrinsic,kay2006new,zhang2010density,shimada2016multiferroic} We find that the $\rm V_O$ has two distinct charge transition levels corresponding to the (0/+1) and (+1/+2) transitions. The energies of these charge transition levels relative to the VBM (1.57 eV and 1.22 eV, respectively) are in excellent agreement with recent DFT+$U$ calculations.~\cite{geneste2019polarons} On the other hand, the positions of charge transition levels relative to the CBM of \ce{BiFeO3} vary drastically depending on the choice of $U$,~\cite{paudel2012intrinsic,geneste2019polarons,zhang2010density} which is a known effect. However, their positions with respect to the free polaron level are relatively insensitive to the choice of $U$, which is similar to what was reported for the case of Sn-doped \ce{Fe2O3}.~\cite{smart2017effect} Comparing the charge transition levels of $\rm V_O$ to the free polaron level, we found energy differences of 99 and 438 meV for the first and second charge transition levels, respectively. The energy of the first charge transition level (0/+1) relative to the free polaron level is comparable to kT at room temperature (26 meV) and indicates that a fraction of the oxygen vacancies in \ce{BiFeO3} can ionize at room temperature and contribute to an increased carrier concentration. A simple thermodynamic calculation (assuming a Boltzmann-like distribution) suggests that $sim$2.05\% of oxygen vacancies will be ionized to their +1 state at room temperature at thermal equilibrium. This result differs from previous reports that $\rm V_O$ in \ce{BiFeO3} is a deep donor and cannot increase the carrier concentration.~\cite{paudel2012intrinsic,kay2006new,zhang2010density,shimada2016multiferroic}

Our theoretical result has clarified the role of oxygen vacancies in enhancing carrier concentrations in \ce{BiFeO3}, and it is consistent with our experimental findings. This study emphasizes that in polaronic oxides, the defect ionization energies need to be considered with respect to the free polaron level and not to the CBM to more accurately understand the role of defects in the charge transport properties.


% Conclusions

To summarize, we performed combined experimental and theoretical investigations on n-type \ce{BiFeO3} to evaluate its properties relevant to its use as a photoanode in a photoelectrochemical cell. In our experimental study, we developed a synthesis method to produce high-quality, uniform n-type \ce{BiFeO3} photoanodes and examined their photoelectrochemical properties. A bandgap energy of $sim$2.1 eV was determined for the \ce{BiFeO3} photoanodes, and this value agreed well with the films’ orange color. The \ce{BiFeO3} photoanode showed a photocurrent onset potential of 0.3 V vs. RHE for sulfite oxidation, which is equivalent to its flatband potential. This value is significantly more negative than that of other ternary Fe-based oxide photoanodes. Upon annealing under a \ce{N2} environment to intentionally introduce more oxygen vacancies, the flatband potential was slightly shifted to the negative direction, and the photocurrent increased considerably. These results indicate that oxygen vacancies can contribute to an increase in carrier density, thus improving the charge transport properties of \ce{BiFeO3}. While the photocurrent reported in this study is one of the highest among those reported for \ce{BiFeO3} photoanodes, the observed value was still far below that expected for a photoanode having a bandgap of 2.1 eV, suggesting that bulk recombination is a major limitation of \ce{BiFeO3}. Considering that nanostructuring other Fe$^{3+}$-containing photoanodes such as \ce{Fe2O3} that suffer from short hole-diffusion lengths can significantly increase electron-hole separation, nanostructuring \ce{BiFeO3} is a logical next step to take to improve its photocurrent generation.

In our theoretical study, we showed for the first time that an extra electron in \ce{BiFeO3} spontaneously localizes on an Fe$^{3+}$ ion and forms a small polaron. The formation of the small polaron also perturbs valence states and creates additional localized states above the VBM of pristine \ce{BiFeO3}. When an oxygen vacancy is introduced into the \ce{BiFeO3} lattice, it forms two electron-polarons at the two Fe sites nearest to the $\rm V_O$ site. By accurately referencing the charge transition level to the free electron polaron level instead of to the CBM in our charge formation energy calculations, we showed that the first ionization energy of the $\rm V_O$ is 99 meV, meaning that the $\rm V_O$ is capable of serving as a donor to enhance the carrier concentration of \ce{BiFeO3}.
Overall, \ce{BiFeO3} has many attractive properties for use as a photoanode in a water-splitting PEC, and we expect that strategies such as nanostructuring and substitutional doping that can introduce shallow donors can continue to increase the photocurrent generation. Our combined investigation contributes to a fundamental understanding of the photoelectrochemical properties of \ce{BiFeO3} that can aid future systematic investigations of both n-type and p-type \ce{BiFeO3} photoelectrodes.




\subsection{LaFeO3}

\subsection{Fe2TiO5}


% ------------------------------------------------------------
% 2d
\chapter{Designing Quantum Defects in Two Dimensional Materials}

% %%%%%%%%%%%%%%%%%%%%%%%%%%%%%%%%%%%%%%%%%%%%%%%%%%%%%%%%%%
%    Section
% %%%%%%%%%%%%%%%%%%%%%%%%%%%%%%%%%%%%%%%%%%%%%%%%%%%%%%%%%%
\section{Overview}
Quantum technologies offer exotic and impressive capabilities in computation, sensing, and information \cite{malik2019science}. While several systems of quantum computation exist, defect based qubits offer a distinct advantage in their ability to operate optically and under room temperature conditions \cite{koehl2011room,weber2010quantum,falk2013polytype}. Furthermore defects in two-dimensional (2D) materials yield a higher-ceiling for defect based quantum technologies where spatially controlling doping, entangling qubits, and qubit tuning are all more attainable \cite{aharonovich2017quantum,sajid2020single}. In particular, two-dimensional hexagonal boron nitride ($h$-BN) has demonstrated that it can host defect-based single photon emitters (SPEs) \cite{grosso2017tunable} and qubits \cite{gottscholl2020initialization}.

As such, my work has focused on the prediction of defects in $h$-BN for quantum applications. From a computational perspective, studying defects in 2D materials offers several technical challenges. In 2018, I was awarded an NSF scholarship for studying quantum information science through a program known as QISE-NET. This program provides supplemental funding to perform ongoing research in collaboration with Argonne National Laboratory, so that we may study defects in $h$-BN (this was later highlighted in the UC Santa Cruz newsletter). In particular, these efforts culminated in our work from 2018, wherein we demonstrated how to compute the single-particle band gap of $h$-BN via a Koopmans' compliant hybrid functional approach which incorporates improved screening effects in our calculation but mitigates the expense of many-body theory based methods i.e. the GW approximation \cite{smart2018fundamental}. Additionally, this work demonstrated the layer dependence on defect ionization energies. Following this up, we then studied radiative and nonradiatiave recombination of defects in $h$-BN. One particularly interesting facet of this research in the demonstration of how significantly the nonradiative recombination of defects can be effected by strain, wherein we predicted the strain fingerprints of N$_\text{B}$V$_\text{N}$ which matches closely with subsequent experimental measurements.~\cite{mendelson2020strain}
Recently, I also implmented computing the zero-field splitting of $S \ge 1$ systems, an essential quantity in defect based-qubit systems like NV center in diamond. In addition, I have implemented computing intersystem crossing (necessary for spin initialization and readout) with spin-orbit coupling and electron-phonon interaction. With these computed static and dynamical properties,  we are able to predict spin qubits read-out efficiency and new quantum spin defect systems in hexagonal boron nitride which can be potential candidates for spin-based quantum technologies.

\def\cbns{C$_\text{B}$}
\def\cb{C$_\text{B}$ }
\def\vncbns{V$_\text{N}$C$_\text{B}$}
\def\vncb{V$_\text{N}$C$_\text{B}$ }
\def\vnns{C$_\text{N}$}
\def\cn{C$_\text{N}$ }
\def\gwns{$G_0W_0$}
\def\gw{$G_0W_0$ }
\def\gns{$\Gamma$}
\def\g{$\Gamma$ }
\def\bR{\mathbf{R}}
\def\veps{\varepsilon}
\def\pans{PBE0($\alpha$)}
\def\p0a{PBE0($\alpha$)\text{}}


\section{Charge Defect Formation and Ionization Energies}

In our 2018 work published in Physical Review Materials~\cite{smart2018fundamental}, we demonstrated methods of computing charge transition levels at various levels of theory and propose referencing to vacuum among other things as a way to achieve consistency at semi-local DFT, hybrid DFT, and GW. We also employ Koopmans’ condition to achieve hybrid functionals which reproduce GW results.
Two-dimensional (2D) materials provide the unique opportunity to scale future electronics smaller than ever believed physically possible, implying engineering 2D materials is a promising strategy that can meet the demands of future nanotechnologies~\cite{butler2013progress}.
As defects play a crucial role in the optical and electronic properties of these systems, the engineering of defects in 2D materials has sparked continuous interest~\cite{wang2017engineering,hong2017atomic,lin2016defect,dreyer2018first,weston2018native,tawfik2017first}. For example, defects in \textit{h}-BN have been found to be the source of stable polarized and ultra-bright single-photon emissions at room temperature~\cite{bourrellier2016bright,abdi2018color,aharonovich2017quantum,tran2016quantum}. Hence, the development of our understanding of defects in 2D materials will open up further possibilities for emerging applications in quantum information and nanotechnology with much better scalability than traditional defects in 3D materials.

Unlike in their 3D counterparts~\cite{freysoldt2009fully,freysoldt2014first,vinichenko2017accurate,kumagai2014electrostatics,komsa2013finite}, first-principles techniques for calculating defect properties in 2D materials still face significant challenges.  Specifically, eliminating the periodic charge interactions for charged defects in 2D materials requires a charge correction scheme that accounts for the weak and anisotropic dielectric screening of 2D systems~\cite{komsa2014charged,komsa2018erratum,wang2015determination}. Furthermore, several exchange-correlation functionals that provide accurate electronic structures for 3D bulk systems are no longer applicable to ultrathin 2D systems.
For example, the fraction of Fock exchange ($\alpha$) in hybrid functionals can be approximated as the inverse of dielectric constant ($\varepsilon_{\infty}$) of the material~\cite{alkauskas2011defect,skone2014self}, but this is problematic for ultrathin 2D materials where $\varepsilon_{\infty}$  decreases to unity in the limit of infinite vacuum sizes (complete separation between periodic images). Therefore, the determination of $\alpha$ in hybrid functionals for 2D materials remains an open question. On the other hand, many body perturbation theory techniques ($\textit{e.g.}$ GW approximation) give accurate quasiparticle energies such as band gaps and band positions; however, high computational cost and slow convergence with respect to empty states make the screening of many defects in 2D materials impractical with conventional implementations~\cite{qiu2016screening,rasmussen2016efficient,thygesen2017calculating,huser2013quasiparticle,attaccalite2011coupling,felipe2017nonuniform}.

In our previous work~\cite{wu2017first,PING2017JCP}, we developed an efficient and accurate method that can give reliable charge corrections for total energies and electronic states of charged defects in 2D materials \textit{without any supercell extrapolations}, and then provided accurate defect CTLs with the DFT+GW scheme~\cite{malashevich2014first,chen2013correspondence,chen2015first,chen2017accuracy}. Such implementation is built on top of the WEST-code~\cite{govoni2015large}, Quantum-Espresso~\cite{QE1} and JDFTx~\cite{JDFTx} packages. In our GW calculations, we avoided explicit inclusion of empty states and inversion of dielectric matrices~\cite{govoni2015large,ping20132electronic,pham2013gw}, while also speeding up vacuum size convergence with a 2D Coulomb truncation~\cite{ismail2000new}. In this letter, we propose to solve two important issues for 2D materials. First, we determine which level of theory and which electron chemical potential reference one should use to calculate a CTL. Second, we show how to define the fraction of Fock exchange in hybrid functionals for accurate band edges and band gaps. In the end, we combine these two findings to obtain accurate defect ionization energies for 2D materials.

%%%%%%%%%%%%%%%%%%%%%%%%%%%%%%%%%%%%%%%%%%%%%%%%%%%%%%%%%%%%%%%%%%%%%%%%%%%%%
%%%%%%%%%%%%%%%%%%%%%%%%%%%%%%%%%METHODS%%%%%%%%%%%%%%%%%%%%%%%%%%%%%%%%%%%%%
%%%%%%%%%%%%%%%%%%%%%%%%%%%%%%%%%%%%%%%%%%%%%%%%%%%%%%%%%%%%%%%%%%%%%%%%%%%%%

\subsection{Methodology}

\textbf{Computational Methods}
In this work, all structural relaxations and total energy calculations were performed using open source plane wave code Quantum-ESPRESSO~\cite{QE1} with Perdew-Burke-Ernzerhof (PBE)~\cite{perdew1996generalized} exchange-correlation functional, ONCV norm-conserving pseudopotentials~\cite{ONCV1,ONCV2}, a wavefunction cutoff of 70 Ry and a $k$-point mesh corresponding to $12\times12\times1$ or higher in the primitive cell. The vacuum between periodic images along non-periodic direction is at least 30 Bohr.

Once the structural parameters were determined, we performed a separate single-point calculation using a wavefunction cutoff of 45 Ry and hybrid functionals including HSE, B3PW91, PBE0 and PBE0($\alpha$) with a sufficient  $k$-point mesh as large as $36\times 36$.
The band gap is determined from the difference between valence band maximum (VBM) and conduction band minimum (CBM). If the k-point of VBM or CBM is not included in the $k$-point mesh, it is interpolated between eigenvalues of the same band of nearby $k$-points.

A single point calculation using a wavefunction cutoff of 45 Ry and PBE functional was performed as the starting point for GW calculations. The GW calculations were performed using the WEST code~\cite{govoni2015large}. We used the $G_0W_0$ approach with starting wavefunctions and eigenvalues at the PBE level of theory. We employed the contour deformation technique for frequency integration of the self energy. For the dielectric matrix calculation, the number of eigenpotentials ($N_{\textrm{PDEP}}$) was chosen to be  $3N_{\textrm{electron}}$, and we used $4N_{\textrm{electron}}$ to validate its convergence. The final GW correction values were extrapolated between $9\times 9$ and $12\times 12$ $k$-point meshes to infinite $k$-points similar to Ref.~\cite{wu2017first}.


The charge corrections~\cite{PING2017JCP} for the total energies and eigenvalues of charged defects at the DFT level employed the techniques developed in Ref.~\cite{PING2017JCP} and in the SI section IV, which were implemented in the JDFTx code~\cite{JDFTx,ismail2000new,arias1992ab} (computed dielectric profiles are shown in the SI). Dielectric profiles are computed by applying finite electric fields following the procedure discussed in Ref.~\cite{wu2017first}, with a smearing width of 1 Bohr (smearing widths of 0.5 to 4.0 Bohr yield  identical charge corrections).



\textbf{Thermodynamic Charge Transition Levels}

\begin{figure}[t]
\includegraphics[keepaspectratio=true,width=\linewidth]{3-2d/figures-prm/path.png}
\caption{Schematic plot of the two paths (distinguished with blue/red color) that transition from charge state $q$ to $q+1$. For each path, there is a corresponding vertical excitation, which can be computed either with EA$_{q+1}$ or IP$_q$ (noted with up/down arrowheads), as discussed in the main text.}  \label{fig:path}
\end{figure}


A thermodynamic CTL is the value of electron chemical potential $\veps_F$ at which the stable charge state of the system changes, \textit{e.g.} from $q$ to $q+1$. Therefore, CTLs are calculated through the equivalency of the formation energies $q$ and $q+1$, given by Eq. (\ref{eq:ctl0})~\cite{freysoldt2014first}.
\begin{align}
\veps_{q+1|q}&=E_q^f(\bR_q)-E_{q+1}^f(\bR_{q+1}) \nonumber\\
&=E_q(\bR_q)-E_{q+1}(\bR_{q+1})-\veps_F
\label{eq:ctl0}
\end{align}
Here $E_q^f(\bR)$ is the defect formation energy with charge $q$ and geometry $\bR$, and $\bR_q$ is the relaxed geometry of the system with charge $q$. $E_q(\bR)$ is the total energy that relates to $E_q^f(\bR)$ and $\veps_F$ following the definition of Eq. (1) in Ref.~\cite{wu2017first}.  Diagrammatically, Eq. (\ref{eq:ctl0}) is the energy difference between two potential  surface minimua in position space $\bR$, as shown in Fig. \ref{fig:path}.



% \section{Results}
\subsection{Implementing Quasiparticle Corrections in Defect Charge Transition Levels}


It is well-known that local and semi-local functionals do not give accurate total energy differences between two charge states, where an electron removing (IP)/adding process (EA) is involved. An alternative approach~\cite{wu2017first} is to separate Eq.~(\ref{eq:ctl0}) into two parts: the vertical excitation energy between two charge states ($q$ and $q+1$) at the same geometry ($\bR$) (denoted as quasiparticle energies $\veps^{QP}$) and the geometry relaxation energy at a fixed charge state (denoted with $E^{rlx}$). Since DFT is known to provide reliable geometry relaxation energies (if one corrects the fictitious charge interactions between periodic images as we did in Ref.~\cite{wu2017first}), this separation allows us to accurately calculate the vertical excitation energies with a higher level of theory appropriate for non-neutral excitations, such as the GW approximation.

One can separate Eq.~(\ref{eq:ctl0}) by two possible physical pathways from $E_q^f(\bR_q)$ to $E_{q+1}^f(\bR_{q+1})$ as shown in Fig. \ref{fig:path}. One pathway (red path) occurs with a vertical excitation at $\bR_{q}$ ($E_{q+1}^f(\bR_q)-E_q^f(\bR_q)$) followed by a geometry relaxation at the charge state $q+1$ ($E_{q+1}^f(\bR_{q+1})-E_{q+1}^f(\bR_q)$), shown in Eq.~(\ref{eq:ctl1}). The other pathway (blue path) occurs through the geometry relaxation at the charge state $q$ plus a vertical excitation at $\bR_{q+1}$, corresponding to Eq.~(\ref{eq:ctl2}).
\begin{align}
\veps_{q+1|q} &= \underbrace{E_{q}^f(\bR_q)-E_{q+1}^f(\bR_q)}_{\veps^{QP}}+\underbrace{E_{q+1}^f(\bR_q)-E_{q+1}^f(\bR_{q+1})}_{E^{rlx}}  \nonumber\\
&= \veps^{QP}_{q+1|q}(\bR_q) + E^{rlx}_{q+1}
\label{eq:ctl1}
\end{align}
\begin{align}
\veps_{q+1|q}&=\underbrace{E_q^f(\bR_q)-E_q^f(\bR_{q+1})}_{E^{rlx}}+\underbrace{E_q^f(\bR_{q+1})-E_{q+1}^f(\bR_{q+1})}_{\veps^{QP}} \nonumber\\
&= E^{rlx}_q  + \veps^{QP}_{q+1|q}(\bR_{q+1})
\label{eq:ctl2}
\end{align}
Note that all three equations (Eq. (\ref{eq:ctl0}), (\ref{eq:ctl1}), (\ref{eq:ctl2})) are exactly equivalent theoretically. Yet, in practice they may yield sizable differences, as discussed later.

Furthermore, the vertical excitation energies $\veps^{QP}_{q+1|q}$ in Eq.~(\ref{eq:ctl1}) and Eq.~(\ref{eq:ctl2}) can be determined from either the ionization potential of the charge state $q$ ($\text{IP}_q$) or
the electron affinity of the charge state $q+1$ ($\text{EA}_{q+1}$)
, as noted in Fig.~\ref{fig:path} with up/down arrowheads. Note that we obtained IP and EA through eigenvalues at different levels of theory based on the Janak's theorem~\cite{janak1978proof}.
%\begin{align}
%\veps^{QP}_{q|q+1}(\bR_q) &= \text{EA}_{q+1}(\bR_q) = %\text{IP}_q(\bR_q) \label{eq:kp1} \\
%\veps^{QP}_{q|q+1}(\bR_{q+1}) &= \text{EA}_{q+1}(\bR_{q+1}) = %\text{IP}_q(\bR_{q+1}) \label{eq:kp2}
%\end{align}
The difference between $\text{IP}_q$ and $\text{EA}_{q+1}$ is largely related to the delocalization or localization error at a particular level of theory, and serves as a stringent test for an exchange-correction scheme in electronic structure calculations~\cite{bruneval2009g}.


\begin{table}[H]
    \footnotesize
\centering
% \begin{tabular}{p{1.5cm}p{1.5cm}p{1.2cm}p{1.2cm}p{1.2cm}p{1.2cm}}
\begin{tabular}{cccccc}
\hline\hline
\multicolumn{2}{c}{Method} & \multicolumn{4}{c}{Defect} \\
%\hline
     &                 & \cb & \vncb & C$_\text{N}$ & \vncb \\
% &&&&&\\
     &    CTL             & (0/+1) & (0/+1) & (-1/0) & (-1/0) \\
\hline
 %   &&&&&\\
     & Eq\ref{eq:ctl0} & -3.63  & -4.22  & -3.54  & -1.57  \\
PBE  & Eq\ref{eq:ctl1} & -3.61  & -4.29  & -3.51  & -1.66  \\
     & Eq\ref{eq:ctl2} & -3.64  & -4.33  & -3.49  & -1.67  \\
     \hline
    % & IP-EA           & 2.68  & 2.60  & 2.75  & 2.50  \\
 % &&&&&\\
     & Eq\ref{eq:ctl0} & -3.65  & -4.19  & -3.50  & -1.87  \\
PBE0 & Eq\ref{eq:ctl1} & -3.60  & -4.17  & -3.50  & -1.87  \\
     & Eq\ref{eq:ctl2} & -3.62  & -4.21  & -3.50  & -1.21*  \\
\hline
    % & IP-EA           & 1.15  & 1.09  & 1.13  & 1.42  \\
% &&&&&\\
     & Eq\ref{eq:ctl1} & -3.40  & -4.29  & -3.74  & -1.74 \\
\gw  & Eq\ref{eq:ctl2} & -3.28  & -4.22  & -3.73  & -1.70  \\
\hline
    % & IP-EA           & 0.04  & 0.20  & 0.03  & 0.19  \\
    % &&&&&\\
    \multicolumn{6}{c}{IP$_{q}(\bR_{q})$-EA$_{q+1}(\bR_{q})$} \\
     % &&&&&\\
PBE && 2.68  & 2.60  & 2.75  & 2.50  \\
PBE0 && 1.15  & 1.09  & 1.13  & 1.42  \\
\gw && 0.04  & 0.20  & 0.03  & 0.19  \\
 \hline\hline
\end{tabular}
\caption{\label{table:path} Charge transition levels (CTLs) relative to vacuum (in eV) of multiple defects in monolayer \textit{h}-BN. These values are collected via three methods (Eq. (\ref{eq:ctl0}-\ref{eq:ctl2})) at various levels of theory (PBE, PBE0, \gwns$@$PBE ). The CTLs relative to vacuum are remarkably similar. The one exception, \vncb (-1/0) at PBE0 (marked with *) incidentally has a band inversion resulting in a defect level within the valence band, breaking the reliability of Eq. (\ref{eq:ctl2}).
We also show IP$_{q}(\bR_{q})-$EA$_{q+1}(\bR_{q})$ at different levels of theory. Note that at the \gw level, this difference is $<0.2$ eV.}
\end{table}

%%%%%%%%%%%%%%%%%%%%%%%%%%%%%%%%%%%%%%%%%%%%%%%%%%%%%%%%%%%%%%%%%%%%%%%%%%%%%
%%%%%%%%%%%%%%%%%%%%%%%%%%%%RESULTS & DISCUSSION%%%%%%%%%%%%%%%%%%%%%%%%%%%%%
%%%%%%%%%%%%%%%%%%%%%%%%%%%%%%%%%%%%%%%%%%%%%%%%%%%%%%%%%%%%%%%%%%%%%%%%%%%%%
Therefore, we firstly compared the CTL with PBE, PBE0 and $G_0W_0@$PBE for three different defects in monolayer BN as shown in Table \ref{table:path}, where $\veps^{QP}$ is obtained by taking the average of $\text{IP}_q$ and $\text{EA}_{q+1}$ as:
\begin{equation}
\veps^{QP}_{q+1|q}(\bR) = \frac{1}{2}(\text{EA}_{q+1}(\bR) + \text{IP}_q(\bR))
\end{equation}
Note that we propose to set $\veps_F$ equal to the vacuum level (determined by the electrostatic potential in the vacuum region of supercells) and use it as a reference for Eq. (\ref{eq:ctl0}).
We found this choice (opposed to band edges) is particularly advantageous for obtaining consistent CTLs among different methods as shown in Table~\ref{table:path}.
(More computational details for $G_0W_0@$PBE can be found in SI, with similar numerical techniques and parameters used in Ref.~\cite{wu2017first}). There are several interesting observations from Table \ref{table:path}, as follows. First, we found excellent agreement (within 0.1 eV) among Eq. (\ref{eq:ctl0}), (\ref{eq:ctl1}) and (\ref{eq:ctl2}) for each defect at each level of theory. Second, we found the results obtained among PBE, PBE0 and $G_0W_0@$PBE are also strikingly similar (within 0.2 eV) for each defect. This means the CTLs of 2D materials relative to vacuum are \textit{not affected by the level of theory one chooses}. Note that the difference between $\text{IP}_q$ and $\text{EA}_{q+1}$ is more than 2 eV for PBE, reduced to $~$1 eV at PBE0 level ($\alpha=0.25$), but less than 0.2 eV at $G_0W_0@$PBE, which indicates the delocalization error present with semi-local DFT has been mostly corrected at $G_0W_0$@PBE~\cite{bruneval2009g}.
%As we will discuss later, it is the linearity of the Kohn-Sham eigenvalues with $\alpha$ (Figure \ref{fig:ipea}) that allows the averaging of $\text{IP}_q$ and $\text{EA}_{q+1}$ to give accurate PBE, PBE0 results in Eq. \ref{eq:ctl1} and \ref{eq:ctl2}, which are in great agreement with each other, despite this error.


\subsection{Generalized Koopman's Condition for Exact Exchange of 2D Materials}
After we obtained reliable CTLs, in particular relative to vacuum, we focused on how to calculate accurate band edge positions and band gaps of 2D materials in order to determine defect ionization energies. Using the GW approximation, we obtained an accurate quasiparticle band gap (indirect at T$\rightarrow$M) 6.01 eV for bulk h-BN (Table \ref{table:gap}), in excellent agreement with the experimental fundamental electronic gap 6.08 $\pm$ 0.015~\cite{cassabois2016hexagonal}. Nonetheless, GW is still too computationally demanding for materials' screening and computing forces is non-trivial. Therefore, the development of computationally affordable methods such as accurate non-empirical hybrid functionals for 2D materials is strongly desired.

\begin{figure}[h]
\begin{center}
\includegraphics[keepaspectratio=true,width=0.5\linewidth]{3-2d/figures-prm/exx_eig.pdf}
\caption{The IP at $q=0$ and the EA at $q=+1$ for the defects \cbns , \cn and \vncb in monolayer h-BN as a function of the fraction of Fock exchange $\alpha$ for PBE0($\alpha$). The predicted exchange constant ($\alpha = 0.409$, 0.41 and 0.382, respectively) is the corresponding crossing point where $\text{EA}_{q+1}=\text{IP}_{q}$.} \label{fig:ipea}
\end{center}
\end{figure}

\begin{figure}[h]
\begin{center}
\includegraphics[keepaspectratio=true,width=0.7\linewidth]{3-2d/figures-prm/p0a_gap.pdf}
\caption{Comparing computed band gaps of \textit{h}-BN (monolayer, bilayer, trilayer, bulk) and graphane with \p0a versus those computed with \gwns$@$PBE . Overall we find that our \p0a results agree very well with \gwns , yielding a MAE of 0.14 eV. } \label{fig:p0a}
\end{center}
\end{figure}

\newpage

\begin{landscape}
\begin{table}
    \footnotesize
    \centering
\begin{tabular}{ccccccc}
	 \hline\hline
     System & PBE & HSE & PBE0 & B3PW & \p0a & \gw \\
	 % \hspace{2mm} System & \hspace{4mm} PBE & \hspace{4mm} HSE & \hspace{3mm} PBE0 & \hspace{2mm} B3PW & \hspace{1mm} \p0a & \hspace{3mm} \gw \\
	 \hline
	% \hspace{0.1mm} Graphane
	Graphane
	& 3.57 $|$ \gns$\,\rightarrow\,$\g & 4.41 $|$ \gns$\,\rightarrow\,$\g
	& 5.06 $|$ \gns$\,\rightarrow\,$\g & 5.04 $|$ \gns$\,\rightarrow\,$\g
	& 6.54 $|$ \gns$\,\rightarrow\,$\g & 6.41 $|$ \gns$\,\rightarrow\,$\g   \\
	% \hspace{0.1mm} ML BN
	ML BN
	& 4.71 $|$ K$\,\rightarrow\,$K & 5.70 $|$ K$\,\rightarrow\,$\g
	& 6.33 $|$ K$\,\rightarrow\,$\g & 6.33 $|$ K$\,\rightarrow\,$\g
	& 7.34 $|$ K$\,\rightarrow\,$\g & 7.01 $|$ K$\,\rightarrow\,$\g   \\
	% \hspace{0.1mm} BL BN
	BL BN
	& 4.49 $|$ T$\,\rightarrow\,$M & 5.81 $|$ T$\,\rightarrow\,$M
	& 6.46 $|$ T$\,\rightarrow\,$\g & 6.17 $|$ T$\,\rightarrow\,$M
	& 7.08 $|$ T$\,\rightarrow\,$\g & 7.00 $|$ T$\,\rightarrow\,$\g   \\
	% \hspace{0.1mm} TL BN
	TL BN
	& 4.36 $|$ T$\,\rightarrow\,$M & 5.68 $|$ T$\,\rightarrow\,$M
	& 6.40 $|$ T$\,\rightarrow\,$M & 6.03 $|$ T$\,\rightarrow\,$M
	& 7.01 $|$ T$\,\rightarrow\,$\g & 6.92 $|$ T$\,\rightarrow\,$M  \\
	% \hspace{0.1mm} Bulk BN
	Bulk BN
	& 4.22 $|$ T$\,\rightarrow\,$M & 5.60 $|$ T$\,\rightarrow\,$M
	& 6.28 $|$ T$\,\rightarrow\,$M & 5.91 $|$ T$\,\rightarrow\,$M
	& 6.07 $|$ T$\,\rightarrow\,$M & 6.01 $|$ T$\,\rightarrow\,$M   \\
	 \hline
    % \hspace{0.5mm} Exp.(Bulk BN) & \multicolumn{2}{c}{6.08 $\pm$ 0.015}  \\
	Exp.(Bulk BN) & \multicolumn{2}{c}{6.08 $\pm$ 0.015}  \\
	 \hline
	 \hline
	\end{tabular}
      \captionof{table}{Band gaps for various pristine 2D materials. In general, PBE severely underestimates the gap. Hybrid functionals HSE, B3PW, and PBE0 ($\alpha=0.25$) generally enlarge the bulk band gap, but still underestimate the gaps of ultrathin 2D systems compared with experiments and GW approximation. Only PBE0($\alpha$) with $\alpha$ satisfying $\text{IP}_{q}=\text{EA}_{q+1}$ of localized defects (\cbns) yield gaps in good agreement with experiment~\cite{cassabois2016hexagonal} and $G_0W_0$@PBE.}
      \label{table:gap}
\end{table}
\end{landscape}



The generalized Koopmans' condition has been mostly used to determine the appropriate fraction of Fock exchange ($\alpha$) for molecules and molecular crystals~\cite{atalla2016enforcing,korzdorfer2014organic,refaely2012quasiparticle,sai2011hole,cohen2007development,pinheiro2015length,ma2016using}.
One recent work~\cite{miceli2018nonempirical} enforced this condition (\textit{i.e.} $\text{EA}_{q+1}=\text{IP}_{q}$) on defects in bulk semiconductors to obtain $\alpha$ and in turn predicted accurate electronic structure of the corresponding pristine bulk systems. The fundamental assumption is that the optimized $\alpha$ depends on the long range screening of the system and not on the nature of the probe defects. This condition is also valid for deep defects in 2D materials, where defect wavefunctions are well localized like molecule orbitals in the supercells, and their  contribution to dielectric screening is negligible compared to the crystal environment.
Another advantage of applying this condition to 2D systems is that both $\text{EA}_{q+1}$ and $\text{IP}_{q}$ can be exactly referenced to vacuum.
In order to validate the applicability of the generalized Koopmans' condition to 2D materials, we used the defect \cb as a probe to determine $\alpha$ for \textit{h}-BN (B$_\text{C}$ for graphane). This method gives $\alpha$ of
%we use it to predict $\alpha$ instead of 25$\%$ in PBE0 calculations (\pans ) as shown in Eq. \ref{eq:p0a}:
%\begin{equation}
%E_{X}^{PBE0(\alpha)} = \alpha E_{X}^{HF} + (1-\alpha) %E_{X}^{PBE}
%\label{eq:p0a}
%\end{equation}
%Namely, the $\alpha$ which gives an equivalent ionization %potential and electron affinity (IP$(q=0)=$EA$(q=+1)$) for a %given defect may determine the accurate hybrid functional for %the host 2D materials.
%In this paper, we will refer to this method simply as  \pans. %Using the defect \cb at geometry $R_{0}$ we can determine %$\alpha$ to be
0.409, 0.347, 0.324, 0.225 for monolayer, bilayer, trilayer and bulk \textit{h}-BN, respectively. Note that the $\alpha$ value 0.225 for bulk h-BN, agrees well with the predicted $\alpha$ from the inverse of high frequency dielectric constant ($\alpha = 1/\veps_{\infty} \approx 0.2$)~\cite{brar2014hybrid}, which supports the assumption that long-range screening determines $\alpha$. We also investigated other defects C$_{\text{N}}$ and {\vncbns} as probes of $\alpha$ (their corresponding electronic structure can be found in SI).
%while the band gaps of few layer \textit{h}-BN follow the trend of a monotonic decrease  in dielectric constant vs. number of layers~\cite{dielectric2}.

\begin{figure}
\begin{center}
\includegraphics[keepaspectratio=true,width=1.0\linewidth]{3-2d/figures-prm/cbctl.png}

% \begin{picture}(500,180)
% \put(0,10){\subfloat[\hspace{2mm}PBE\hspace{1mm}]{\includegraphics[keepaspectratio=true,scale=0.7]{3-2d/figures-prm/cbctl_pbe.pdf}}}
% \put(120,10){\subfloat[\hspace{2mm}HSE\hspace{1mm}]{\includegraphics[keepaspectratio=true,scale=0.7]{3-2d/figures-prm/cbctl_hse.pdf}}}
% \put(240,10){\subfloat[\hspace{0mm}PBE0($\alpha$)]{\includegraphics[keepaspectratio=true,scale=0.7]{3-2d/figures-prm/cbctl_pbe0ie.pdf}}}
% \put(360,10){\subfloat[\hspace{2mm}\gw\hspace{1mm}]{\includegraphics[keepaspectratio=true,scale=0.7]{3-2d/figures-prm/cbctl_gw.pdf}}}
% \put(305,10){\includegraphics[keepaspectratio=true,scale=0.3]{3-2d/figures-prm/white.png}}
% \put(240,10){\includegraphics[keepaspectratio=true,scale=0.7]{3-2d/figures-prm/cbctl_pbe0ie.pdf}}
% \put(185,10){\includegraphics[keepaspectratio=true,scale=0.3]{3-2d/figures-prm/white.png}}
% \put(120,10){\includegraphics[keepaspectratio=true,scale=0.7]{3-2d/figures-prm/cbctl_hse.pdf}}
% \put(65,10){\includegraphics[keepaspectratio=true,scale=0.3]{3-2d/figures-prm/white.png}}
% \put(0,10){\includegraphics[keepaspectratio=true,scale=0.7]{3-2d/figures-prm/cbctl_pbe.pdf}}
% \end{picture}

\caption{Charge transition level \cb (+1/0) in \textit{h}-BN with different  levels of theory. Defect charge transition levels gradually become shallower with lower ionization energies while increasing the number of layers (ionization energies are written adjacent to arrows from the CTL to CBM). Note that the defect CTLs are very similar relative to vacuum between different methods.}  \label{fig:ctl2}
\end{center}
\end{figure}


Interestingly, we found that IP$_q$ and EA$_{q+1}$ from Kohn-Sham eigenvalues varied linearly with $\alpha$.
%consistent with the case of bulk materials in Ref.~\cite{miceli}.As shown in Figure \ref{fig:ipea}, linear fits of IP at $q=0$ and EA at $q=+1$ of the \cb defect in monolayer \textit{h}-BN %yield R-Squared values of 0.9997 and 0.9999, respectively for %$\alpha$ from 0.25 to 0.51.
Fig. \ref{fig:ipea} shows this linearity for three defects in monolayer \textit{h}-BN, and three defects predict very similar $\alpha$, which justifies the insensitivity of $\alpha$ to the explicit defect. %Note that  defects with localized wavefunctions such as atomic substitutions (\cb) determine more accurate $\alpha$ because their contribution to dielectric screening is negligible. In this way, the screening is determined from the crystal.
It is also notable that the slopes of  IP$_q$ and EA$_{q+1}$ are opposite but nearly equal, explaining how the average of IP$_q$ and EA$_{q+1}$ as $\veps^{QP}$ for CTL in Eq. (\ref{eq:ctl1}) and (\ref{eq:ctl2}) works well (as shown in Table \ref{table:path}).


%%%%%%%%%%%%%%%%%%%%%%%%%%%%%%%%%%%%%%%%%%%%%%%%%%%%%%%%%%%%%%%%%%%%%%%%%%%%%
%%%%%%%%%%%%%%%%%%%%%%%%%%%%RESULTS & DISCUSSION%%%%%%%%%%%%%%%%%%%%%%%%%%%%%
%%%%%%%%%%%%%%%%%%%%%%%%%%%%%%%%%%%%%%%%%%%%%%%%%%%%%%%%%%%%%%%%%%%%%%%%%%%%%



Most commonly, two-dimensional systems are synthesized with a few layers of the material, therefore understanding the effect of increasing thickness is essential to connect with realistic experiments. As such, we have computed the band gaps of monolayer, bilayer, trilayer and bulk \textit{h}-BN, as well as graphane with several hybrid functionals including HSE, PBE0, B3PW and \p0a (with $\alpha$ predicted earlier), as well as with \gwns @PBE for a reliable comparison (see Table \ref{table:gap}). As anticipated, PBE strongly underestimated monolayer \textit{h}-BN band gap: 4.71 eV with a direct transition at the K point. With any level of theory beyond PBE, monolayer \textit{h}-BN is predicted to have a larger, indirect gap from K to \gns. In accordance with quantum confinement, we observed that the band gaps of \textit{h}-BN obtained at B3PW,  \pans , and \gw show a sharp increase at ultrathin BN (monolayer to trilayer) compared to bulk BN. However, HSE and PBE0 provide almost the same band gaps between ultrathin and bulk BN. This is because there is a severe change in the dielectric screening from monolayer to bulk, and a different portion of Fock exchange must be instilled.

Using \p0a we obtained results consistent with quantum confinement and in best agreement with our \gw calculations with a MAE of 0.14 eV (see SI Fig. 3). In addition, the B3PW functional~\cite{becke1993density,crowley2016resolution} provided a more accurate bulk BN band gap than PBE0 and HSE but still underestimated the band gaps of ultrathin BN. Therefore, the direct/indirect transitions and magnitude of the gaps from bulk to monolayer are provided accurately solely with \p0a and \gwns.
%The measured band gap of bulk \textit{h}-BN is 6.08 eV with valence and conduction band extrema near the symmetric points K and M~\cite{expgap} respectively, in agreement with our B3PW, \pans , and \gw results (within 0.1 eV).
In brief, the results shown in Table \ref{table:gap} validate our method for determining accurate fundamental band gaps for 2D materials from first-principles. We note that calculated band edge positions relative to vacuum are also similar at \p0a and \gw as shown in Fig. \ref{fig:ctl2} and SI.



\subsection{Defect Ionization Energies in 2D Materials}
Finally, CTLs and ionization energies for \cb in \textit{h}-BN computed at PBE, HSE, \p0a and $G_0W_0$ levels of theory as a function of number of layers are shown in Fig. \ref{fig:ctl2}. Consistent with the findings in Table \ref{table:path}, CTLs changed less than 0.1 eV across different theoretical methods relative to vacuum. Interestingly, no clear trend and only small difference have been found in the band edge positions of \textit{h}-BN from monolayer to triple layers. These results illustrate that one just needs to correct the band edge positions of pristine 2D materials with \p0a or \gwns , and use CTLs determined from DFT with semi-local functionals, then the difference of the two yields accurate defect ionization energies in 2D materials. On another note, we found there is a clear monotonic decrease in the ionization energies of defects in BN with increasing number of layers, mostly contributed by CTLs' shift towards vacuum (shown in Fig. \ref{fig:ctl2}; also see SI Fig. 5). This effect can be understood as a result of increased dielectric screening with more layers of \textit{h}-BN, and is consistent with the effect of dielectric environments on the ionization energies of MoS$_2$~\cite{noh2015deep}.

\begin{figure}[t]
\begin{center}
\includegraphics[keepaspectratio=true,width=0.5\linewidth]{3-2d/figures-prm/ie.pdf}
\caption{Ionization energies of \cb in \textit{h}-BN with varying number of layers. It is observed that ionization energies decrease monotonically with increasing number of layers. Note that \p0a and \gw give results in excellent agreement.} \label{fig:ie}
\end{center}
\end{figure}

%%%%%%%%%%%%%%%%%%%%%%%%%%%%%%%%%%%%%%%%%%%%%%%%%%%%%%%%%%%%%%%%%%%%%%%%%%%%%
%%%%%%%%%%%%%%%%%%%%%%%%%%%%%%%%%CONCLUSION%%%%%%%%%%%%%%%%%%%%%%%%%%%%%%%%%%
%%%%%%%%%%%%%%%%%%%%%%%%%%%%%%%%%%%%%%%%%%%%%%%%%%%%%%%%%%%%%%%%%%%%%%%%%%%%%

\subsection{Conclusions}

In summary, we established fundamental principles to reliably and efficiently compute ionization energies for defects in 2D materials.
Specifically, band edge positions of the pristine systems should be computed with our proposed \p0a hybrid functional or GW approximations, and the defect CTL can be obtained reliably by standard DFT with semi-local functional, if relative to vacuum.
We successfully applied the proposed methods for a variety of defects from monolayer to triple layer \textit{h}-BN, as well as  graphane.  We also demonstrated that defect ionization energies decreased with increasing number of layers in \textit{h}-BN, mainly due to enlarged dielectric screening. Our findings in this work suggest efficient and accurate methods to compute defect ionization energies and electronic structures in 2D materials, which can be applied to screening new promising defects for quantum information and optoelectronic applications.

% shortcut for vacancy
\def\vac{\text{V}}
% shortcut for boron vacancy
\def\vb{\vac_\text{B}}
% shortcut for nitrogen vacancy
\def\vn{\vac_\text{N}}
% shortcut for nitrogen substitute boron nitrogen vacancy
\def\nbvn{\text{N}_\text{B}\vac_\text{N}}
% shortcut for GW@PBE0 ($\alpha=0.41$)
\def\gw{\text{G}_0\text{W}_0@\text{PBE0($\mathrm{\alpha}$)}}
% shortcut for GW+RPA@PBE0 ($\alpha=0.41$)
\def\gwrpa{\text{G}_0\text{W}_0+\text{RPA}@\text{PBE0($\mathrm{\alpha}$)}}
% shortcut for GW+BSE@PBE0 ($\alpha=0.41$)
\def\gwbse{\text{G}_0\text{W}_0+\text{BSE}@\text{PBE0($\mathrm{\alpha}$)}}
% shortcut for GW+RPA@PBE
\def\gwrpapbe{\text{G}_0\text{W}_0+\text{RPA}@\text{PBE}}
% shortcut for GW+BSE@PBE
\def\gwbsepbe{\text{G}_0\text{W}_0+\text{BSE}@\text{PBE}}
% shortcut for general X_VV defect
\newcommand{\vv}[1]{\text{#1}_{\vac\vac}}
% shortcut for Ti defect
\def\ti{\vv{Ti}}
% shortcut for Mo defect
\def\mo{\vv{Mo}}
% shortcut for singlet Si defect
\def\sis{\mathrm{Si_{VV}(S)}}
% shortcut for triplet Si defect
\def\sit{\mathrm{Si_{VV}(T)}}
% shortcut for general X_Y defect
\newcommand{\xy}[2]{\text{#1}_{\text{#2}}}
% shortcuts for ti states
\def\gs{\ket{\prescript{3}{0}{A''}}}    % triplet ground state (gs)
\def\es{\ket{\prescript{3}{1}{A''}}}    % triplet excited states (es)
\def\pjt{\ket{\prescript{3}{1}{A}}}     % triplet excited PJT state (pjt) -- this is C1
\def\sin{\ket{\prescript{1}{0}{A'}}}    % singlet state (sin)
% shortcuts for mo states
\def\mogs{\ket{\prescript{3}{0}{A}}}    % triplet ground state (gs)
\def\moes{\ket{\prescript{3}{1}{A}}}    % triplet excited states (es)
\def\mosin{\ket{\prescript{1}{0}{A}}}    % singlet state (sin)
% shortcut for NV center
\def\nv{\text{NV}}

\section{Intersystem Crossing and Exciton-Defect Coupling of Spin Defects}

In our 2021 work published in npj Computational Materials~\cite{smart2021intersystem}, we implemented and employed methods for computing static and dynamic properties of spin defects in h-BN, in particular zero-field splitting and intersystem crossing rates were computed. Using a vast array of methods we are able to identify new extrinsic dopants in h-BN as single photon emitters and spin defect qubits.
Optically addressable defect-based qubits offer a distinct advantage in their ability to operate with high fidelity under room temperature conditions~\cite{koehl2011room,weber2010quantum}.
Despite tremendous progress made in years of research, systems which exist today remain inadequate for real-world applications. The identification of stable single photon emitters in 2D materials has opened up a new playground for novel quantum phenomena and quantum technology applications, with improved scalability in device fabrication and
a leverage in doping spatial control, qubit entanglement, and qubit tuning~\cite{liu20192d,aharonovich2017quantum}.
In particular, hexagonal boron nitride ($h$-BN) has demonstrated that it can host stable defect-based single photon emitters (SPEs)~\cite{mendelson2020strain,feldman2019phonon,yim2020polarization,mackoit2019carbon} and spin triplet defects~\cite{kianinia2020generation,turiansky2020spinning}.
However,
%whereas the hallmark nitrogen-vacancy ($\nv$) center in diamond has been investigated extensively~\cite{gali2019ab},
persistent challenges must be resolved before 2D quantum defects can become the most promising quantum information platform. These challenges include the undetermined chemical nature of
existing SPEs\cite{li2017nonmagnetic,yim2020polarization}, difficulties in controlled generation of desired spin defects, and scarcity of reliable theoretical methods which
can accurately predict critical physical parameters for defects in 2D materials due to their complex many-body interactions.

To circumvent these challenges, design of promising spin defects by high-integrity theoretical methods is urgently needed.
Introducing extrinsic defects can be unambiguously produced and controlled, which fundamentally solves the current issues of undetermined chemical nature of existing SPEs in 2D systems.
As highlighted by Ref.~\cite{weber2010quantum,ivady2018first}, promising spin qubit candidates should satisfy several essential criteria: deep defect levels, stable high spin states, large zero-field splitting, efficient radiative recombination, high intersystem crossing rates and long spin coherence and relaxation time.
Using these criteria for theoretical screening can effectively identify promising candidates but
requires theoretical development of first-principles methods, significantly beyond the static and mean-field level. For example, accurate defect charge transition levels in 2D materials necessitates careful treatment of defect charge corrections for removal of spurious charge interactions~\cite{komsa2014charged,komsa2018erratum,wang2015determination,wu2017first} and electron correlations for non-neutral excitation, e.g.\ from GW approximations~\cite{wu2017first,govoni2015large} or Koopmans-compliant hybrid functionals~\cite{smart2018fundamental,nguyen2018koopmans,weng2018wannier,miceli2018nonempirical}. Optical excitation and exciton radiative lifetime must account for defect-exciton interactions, e.g.\ by solving the Bethe-Salpeter equation, due to large exciton binding energies in 2D systems~\cite{refaely2018defect,gao2020radiative}.  Spin-phonon relaxation time calls for a general theoretical approach to treat complex symmetry and state degeneracy of defective systems, along the line of recent development based on
ab-initio density matrix approach~\cite{xu2020spin}.
Spin coherence time due to the nuclei spin and electron spin coupling
can be accurately predicted for defects in solids by combining first-principles and spin Hamiltonian approaches~\cite{seo2016quantum,ye2019spin}. In the end, nonradiative processes, such as phonon-assisted nonradiative recombination, have been recently computed with first-principles electron-phonon couplings for defects in $h$-BN~\cite{wu2019carrier}, and resulted in less competitive rates than corresponding radiative processes. However, the spin-orbit induced intersystem crossing as the key process for pure spin state initialization during qubit operation has not been investigated for spin defects in 2D materials from first-principles in-depth.

This work has developed a complete theoretical framework which enables the design of spin defects based on the critical physical parameters mentioned above and highlighted in Figure~\ref{fig:screen}a. We employed state-of-the-art first-principles methods, focusing on many-body interaction such as defect-exciton couplings and dynamical processes through radiative and nonradiative recombinations. We developed methodology to compute nonradiative intersystem crossing rates with explicit overlap of phonon wavefunctions beyond current implementations in the Huang-Rhys approximation\cite{thiering2017ab}. We showcase the discovery of transition metal complexes such as Ti and Mo with vacancy ($\ti$ and $\mo$) to be spin triplet defects in $h$-BN, and the discovery of $\vv{Si}$ to be a bright SPE in $h$-BN.
We predict $\ti$ and $\mo$ are stable triplet defects in $h$-BN (which is rare considering the only known such defect is $\vb^-$~\cite{gottscholl2020initialization}) with large zero-field splitting and spin-selective decay, which will set 2D quantum defects at a competitive stage with $\nv$ center in diamond for quantum technology applications.

%%%%%%%%%%%%%%%%%%%%%%%%
% Figure for screening
\begin{figure}[H]
    \centering
    \includegraphics[keepaspectratio=true,width=0.8\linewidth]{3-2d/figures-npj/tibn_screen.png}
    \caption{\textbf{Screening of spin defects in $h$-BN.} \textbf{a} Schematic of the screening criteria and workflow developed in this work, where we first search for defects with stable triplet ground state, followed by large zero-field splitting (ZFS), then ``bright'' optical transitions between defect states required for SPEs or qubit operation by photon, and at the end large intersystem crossing rate (ISC) critical for pure spin state initialization. \textbf{b} Divacancy site in $h$-BN corresponding to adjacent B and N vacancies (denoted by $\vb$ and $\vn$). \textbf{c} Top-view and \textbf{d} Side-view of a typical doping configuration when placed at the divacancy site, denoted by $\vv{X}$. Atoms are distinguished by color: grey=N, green=B, purple=Mo, blue=Ti, red=X (a generic dopant).}
    \label{fig:screen}
\end{figure}

%%%%%%%%%%%%%%%%%%%%%%%%%%%%%%%%%%%%%%%%%%%%%%%%%%%%%%%%%%%%%%%%%%%%%%%%%%%%%%%%%%%%%%%%%%%%%%%%%%%%%%%%%%%%%%%%%%

In the development of spin qubits in 3D systems (e.g.\ diamond, SiC, and AlN), defects beyond $sp$ dangling bonds from N or C have been explored. In particular, large metal ions plus anion vacancy in AlN and SiC were found to have potential as qubits due to triplet ground states and large zero-field splitting (ZFS)~\cite{seo2017designing}.
Similar defects may be explored in 2D materials~\cite{turiansky2019dangling}, such as the systems shown in Figure~\ref{fig:screen}b-d.
This opens up the possibility of overcoming the current limitations of uncontrolled and undetermined chemical nature of 2D defects, and unsatisfactory spin dependent properties of existing defects.
In the following, we will start the computational screening of spin defects with static properties of the ground state (spin state, defect formation energy and ZFS) and the excited state (optical spectra), then we will discuss dynamical properties including radiative and nonradiative (phonon-assisted spin conserving and spin flip) processes, as the flow chart shown in Figure~\ref{fig:screen}a. We will summarize the complete defect discovery procedure and discuss the outlook at the end.

\subsection{Screening Triplet Spin Defects in $h$-BN}
To identify stable qubits in $h$-BN, we start from screening neutral dopant-vacancy defects for a triplet ground state based on total energy calculations of different spin states at both semi-local PBE (Perdew–Burke-Ernzerhof) and hybrid functional levels. We considered the dopant substitution at a divacancy site in $h$-BN (Figure~\ref{fig:screen}b) for four different elemental groups. The results of this procedure are summarized in Supplementary Table 1 and Note 1. With additional supercell tests in Supplementary Table 2, our screening process finally yielded that only $\mo$ and $\ti$ have a stable triplet ground state.
%\textbf{Thermodynamic Charge Formation Energy}
We further confirmed the thermodynamic charge stability of these defect candidates via calculations of defect formation energy and charge transition levels.
As shown in Supplementary Figure 1,
both $\ti$ and $\mo$ defects have a stable neutral ($q=0$) region for a large range of Fermi level ($\varepsilon_F$), from 2.2 eV to 5.6 eV for $\mo$ and from 2.9 eV to 6.1 eV for $\ti$. These neutral states will be stable in intrinsic $h$-BN systems or with weak p-type or n-type doping (see Supplementary Note 2).

With a confirmed triplet ground state, we next computed the two defects' zero-field splitting. A large ZFS is necessary to isolate the $m_s = \pm 1$ and $m_s = 0$  levels even at zero magnetic field allowing for controllable preparation of the spin qubit.
Here we computed the contribution of spin-spin interaction to ZFS by implementing the plane-wave based method developed by Rayson et al.\ (see Methods section for details of implementation and benchmark on $\nv$ center in diamond)~\cite{rayson2008first}. Meanwhile, the spin-orbit contribution to ZFS was computed with the ORCA code.
We find that both defects have sizable ZFS including both spin-spin and spin-orbit contributions (axial $D$ parameter) of 19.4 GHz for $\ti$ and 5.5 GHz for $\mo$, highlighting the potential for the basis of a spin qubit with optically detected magnetic resonance (ODMR) (see Supplementary Note 3 and Figure 2).
They are notably larger than previously reported values for ZFS of other known spin defect in solids~\cite{seo2017designing}, although at a reasonable range considering large ZFS values (up to 1000 GHz) in transition-metal complex molecules~\cite{zolnhofer2020electronic}.

\subsection{Screening SPE Defects in $h$-BN}
To identify single photon emitters in $h$-BN, we considered a separate screening process of these dopant-vacancy defects, targeting those with desirable optical properties. Namely, an SPE efficiently emits a single photon at a time at room temperature.
Physically this corresponds to identifying defects which have a single bright intra-defect transition with high quantum efficiency (i.e.\ much faster radiative rates than nonradiative ones), for example current SPEs in $h$-BN have radiative lifetimes $\sim$1-10 ns and quantum efficiency over 50$\%$.~\cite{tran2016robust,schell2017coupling}

Using these criteria we screened the defects by computing their optical transitions and radiative lifetime at Random Phase Approximation (RPA) (see Supplementary Note 4, Figure 3 and Table 3). This offers a cost-efficient first-pass to identify defects with bright transition and short radiative lifetime as potential candidates for SPEs.
From this procedure, we found that $\vv{C}$(T), $\sis$, $\sit$, $\vv{S}$(S), $\vv{Ge}$(S) and $\vv{Sn}$(S) could be promising SPE defects ((T) denotes triplet; (S) denotes singlet), with a bright intra-defect transition and radiative lifetimes on the order of 10 ns, at the same order of magnitude of the SPEs' lifetime observed experimentally.
\cite{schell2017coupling}
Among these, $\sis$ has the shortest radiative lifetime, and in addition, Si has recently been experimentally detected in $h$-BN with samples grown in chemical vapor deposition (the ground state of $\vv{Si}$ is also singlet).\cite{ahmadpour2019substitutional}
Hence we will focus on $\vv{Si}$ as an SPE candidate in the following sections as we compute optical and electronic properties at higher level of theory from many-body perturbation theory including accurate electron correlation and electron-hole interactions.
Note that $\vv{C}$ (commonly denoted $\rm C_BV_N$) has also been suggested to be a SPE source in $h$-BN.~\cite{sajid2020vncb}


\subsection{Single-Particle Levels, Optical Spectra and Radiative Lifetime}

The single-particle energy levels of $\ti$, $\mo$ and $\vv{Si}$ are shown in Figure~\ref{fig:sg_ctl}.
These levels are computed by many-body perturbation theory ($\mathrm{G_0W_0}$) for accurate electron correlation,
with hybrid functional (PBE0($\alpha$), $\alpha=0.41$ based on the Koopmans' condition~\cite{smart2018fundamental}) as the starting point to address self-interaction errors for $3d$ transition metal defects.~\cite{fuchs2007quasiparticle,bechstedt2016many}
For example, we find that both the wavefunction distribution and ordering of defect states can differ between PBE and PBE0($\alpha$) (see Supplementary Figure 4-6). The convergence test of $\mathrm{G_0W_0}$ can been found in Supplementary Figure 7, Note 5, and Table 4.
Importantly, the single particle levels in Figure~\ref{fig:sg_ctl} show there are well localized occupied and unoccupied defect states in the $h$-BN band gap, which yield the potential for intra-defect transitions.

\begin{figure}[H]
    \centering
    \includegraphics[keepaspectratio=true,width=1.0\linewidth]{3-2d/figures-npj/singlet_particle_diagrams.png}
    \caption{
    \textbf{Single-particle levels and wavefunctions.} Single-particle defect levels (horizontal black lines) of the (a) $\ti$, (b) $\mo$, and (c) $\vv{Si}$ defects in $h$-BN, calculated at $\mathrm{G_0W_0}$ with PBE0($\alpha$) starting wavefunctions. The blue/red area corresponds to the valence/conduction band of $h$-BN.
    States are labelled by their ordering and representation within the $C_S$ group with up/down arrows indicating spin and filled/unfilled arrows indicating occupation.
    A red arrow is drawn to denote the intra-defect optical transition found in Figure~\ref{fig:bse}.
    Defect wavefunctions at PBE0($\alpha$) are shown with an isosurface value 10\% of the maximum. The blue and yellow color denotes different signs of wavefunctions.
}
    \label{fig:sg_ctl}
\end{figure}

Obtaining reliable optical properties of these two-dimensional materials necessitates solving the Bethe-Salpeter equation (BSE)
%, \textcolor{red}{make a note here: see numerical convergence in SI Figure S8-S10}
to include excitonic effects due to their strong defect-exciton coupling, which is not included in RPA calculations (see comparison in Supplementary Figure 8 and Table 5).~\cite{ping20132electronic,rocca2012solution,ping2012ab,ping2013}
The BSE optical spectra are shown for each defect in Figure~\ref{fig:bse}a-c (the related convergence tests can be found in Supplementary Figure 9-10).
In each case we find an allowed intra-defect optical transition (corresponding to the lowest energy peak as labeled in Figure~\ref{fig:bse}a-c, and red arrows in Figure~\ref{fig:sg_ctl}).
From the optical spectra we can compute their radiative lifetimes as detailed in the Methods section on Radiative Recombination.
We find the transition metal defects' radiative lifetimes (tabulated in Table~\ref{table:rad}) are long, exceeding $\mu$s.
% Therefore, they are not good candidates for SPEs, although they still have potentials for spin qubits with optically-allowed intra-defect transitions.
Therefore, they are not good candidates for SPE. In addition, while they still are potential spin qubits with optically-allowed intra-defect transitions, optical readout of these defects will be difficult.
Referring to Table~\ref{table:rad} and the expression of radiative lifetime in Eq.~\ref{intro:eq:rad} we can see this is due to their low excitation energies ($E_0$, in the infrared region) and small dipole moment strength ($\mu^2_{e-h}$). The latter is related to the tight localization of the excitonic wavefunction for $\ti$ and $\mo$ (shown in Figure~\ref{fig:bse}d-f), as strong localization of the defect-bound exciton leads to weaker oscillator strength.~\cite{hours2005exciton}

\begin{figure}[ht!]
    \centering
    \includegraphics[keepaspectratio=true,width=1.0\linewidth]{3-2d/figures-npj/bse_ti_mo_si.png}
    \caption{
    \textbf{BSE optical spectra and exciton wavefunctions.} Absorption spectra  of the (a) $\ti$, (b) $\mo$, and (c) $\vv{Si}$ defects in $h$-BN at the level of $\gwbse$. The left and right panels provide absorption spectra for two different energy ranges, where the former is magnified by a factor of 40 for $\ti$ and $\mo$ and a factor of 5 for $\vv{Si}$ for increasing visibility. A spectral broadening of 0.02 eV is applied. The exciton wavefunctions of (d) $\ti$, (e) $\mo$ and (f) $\vv{Si}$ are shown on the right for the first peak.
    }
    \label{fig:bse}
\end{figure}

On the other hand, the optical properties of the $\vv{Si}$ defect are quite promising for SPEs, as Figure~\ref{fig:bse}c shows it has a very bright optical transition in the ultraviolet region. As a consequence, we find that the radiative lifetime (Table~\ref{table:rad}) for $\vv{Si}$ is 22.8 ns at $\rm G_0W_0+BSE@PBE0(\alpha)$.
We note that although the lifetime of $\vv{Si}$ at the level of BSE is similar to that obtained at RPA (13.7 ns),
the optical properties of 2D defects at RPA are still unreliable, due to the lack of excitonic effects. For example, the excitation energy ($E_0$) can deviate by $\sim$1 eV and oscillator strengths ($\mu_{e-h}^2$) can deviate by an order of magnitude (more details can be found in Supplementary Table 5).
Above all, the radiative lifetime of $\vv{Si}$ is comparable to experimentally observed SPE defects in $h$-BN,\cite{schell2017coupling} showing that $\vv{Si}$ is a strong SPE defect candidate in $h$-BN.



\begin{table}[H]
    \footnotesize
    \centering

\begin{tabular}{c c c c c}
    \hline \hline
        Defect    &   $E_{0}$ (eV)    &   $\mu^2_{e-h}$ (bohr$^2$)   &    $\tau_R$ (ns)   &   $E_{b}$ (eV)\\
    \hline
         $\ti$    &       0.556       &        $2.81*10^{-2}$        &    $1.95*10^{5}$   &   4.018\\
         $\mo$    &       1.079       &        $2.29*10^{-2}$        &    $3.26*10^{4}$   &   3.965\\
         $\vv{Si}$   &       4.036       &        $6.28*10^{-1}$        &    $22.8$          &   2.189\\
         %$\sit$   &       2.883       &        $5.83*10^{-1}$        &    $67.3$          &   3.099\\
        $\nbvn$   &       2.408       &        $1.87$                &    $35.9$          &   2.428\\
    \hline \hline
\end{tabular}

    \caption{
Optical excitation energy ($E_{0}$), modulus square of the transition dipole moment ($\mu^2_{e-h}$), radiative lifetime ($\tau_R$) and exciton binding energy ($E_b$) of several defects in $h$-BN at the level of theory of $\gwbse$. The corresponding excitation transitions are $1a'_\uparrow \rightarrow 2a'_\uparrow$ for the $\ti$ defect, $1a''_\uparrow \rightarrow 3a'_\uparrow$ for the $\mo$ defect and $1a'_\uparrow \rightarrow 2a'_\uparrow$ for the $\vv{Si}$ defect. For comparison, we include the results of $\nbvn$ (in-plane structure) from Ref.~\cite{wu2019carrier}.
    }
    \label{table:rad}
\end{table}
%%%%%%%%%%


%%%%%%%%%%%%%%%%%%%%%%%%%%%%%%%%%%%%%%%%%%%%%%%%%%%%%%%%%%%%%%%%%%%%%%%%%%%%%%%%%%%%%%%%%%%%%%%%%%%%%%%%%%%%%%%%%%

\subsection{Multiplet Structure and Excited-State Dynamics}
Finally, we discuss the excited-state dynamics of the spin qubit candidates $\ti$ and $\mo$ defects in $h$-BN, where the possibility of intersystem crossing is crucial.
This can allow for polarization of the system to a particular spin state by optical pumping, required for realistic spin qubit operation.

\begin{figure}[H]
    \centering
    \includegraphics[keepaspectratio=true,width=1\linewidth]{3-2d/figures-npj/tibn_multi.png}
    \caption{
         \textbf{Multiplet structure of triplet defects.}
        Multiplet structure and related radiative and nonradiative recombination rates of the (\textbf{a}) $\ti$ defect and the (\textbf{b}) $\mo$ defect in $h$-BN, computed at $T=10K$.
    The radiative process is shown in red with zero-phonon line (ZPL) and radiative lifetime ($\tau_R$); the ground state nonradiative recombination ($\tau_{NR}$) is denoted with a dashed line in dark blue; and finally the intersystem crossing (ISC) to the singlet state from the triplet excited state is shown in light blue. The zero-field splitting ($D$) is denoted by the orange line. For the $\ti$ defect, the pseudo Jahn-Teller (PJT) process is shown with a solid line in dark blue.
   }
    \label{fig:multi}
\end{figure}

An overview of the multiplet structure and excited-state dynamics is given in Figure~\ref{fig:multi} for the $\ti$ and $\mo$ defects.
For both defects, the system will begin from a spin-conserved optical excitation from the triplet ground state to the triplet excited state, where next the excited state relaxation and recombination can go through several pathways. The excited state can directly return to the ground state via a radiative (red lines) or nonradiative process (dashed dark blue lines). For the $\ti$ defect shown in Figure~\ref{fig:multi}a, we find the system may relax to another excited state with lower symmetry through a pseudo Jahn-Teller distortion (PJT; solid dark blue lines), and ultimately recombine back to the ground state nonradiatively. Most importantly, a third pathway is to nonradiatively relax to an intermediate singlet state through a spin-flip intersystem crossing (ISC), and then again recombine back to the ground state (dashed light-blue lines).
This ISC pathway is critical for the preparation of a pure spin state, similar to $\nv$ center in diamond.
Below, we will discuss our results for the lifetime of each radiative or nonradiative process, in order to determine the most competitive pathway under the operation condition.

\subsection{Direct Radiative and Nonradiative Recombination}

First, we will consider the direct ground state recombination processes.
Figure~\ref{fig:config} shows the configuration diagram of the $\ti$ and $\mo$ defects.
The zero-phonon line (ZPL) for direct recombination can be accurately computed by subtracting its vertical excitation energy computed at BSE (0.56 eV for $\ti$ and 1.08 eV for $\mo$) by its relaxation energy in the excited state (i.e.\ Franck-Condon shift~\cite{van2004first}, $\Delta E_{FC}$ in Figure~\ref{fig:config}). This yields ZPLs of 0.53 eV and 0.91 eV for $\ti$ and $\mo$, respectively. Although this method accurately includes both many-body effects and Franck-Condon shifts, it is difficult to evaluate ZPLs for the triplet to singlet-state transition currently.
Therefore, we compared with the ZPLs computed by constrained occupation DFT (CDFT) method at PBE. This yields ZPLs of 0.49 eV and 0.92 eV for $\ti$ and $\mo$, respectively, which are in great agreement with the ones obtained from BSE excitation energies subtracting $\Delta E_{FC}$ above.
Lastly, the radiative lifetimes for these transitions are presented in Table~\ref{table:rad} as discussed in the earlier section, which shows $\ti$ and $\mo$ have radiative lifetimes of 195 $\mu$s and 33 $\mu$s, respectively (red lines in Figure~\ref{fig:multi}).

In terms of nonradiative properties, the small Huang-Rhys ($S_f$) for the $\es$ to $\gs$ transition of the $\ti$ defect (0.91) implies extremely small electron-phonon coupling and potentially an even slower nonradiative process.
On the other hand, $S_f$ for the $\moes$ to $\mogs$ transition of the $\mo$ defect is modest (3.53) and may indicate a possible nonradiative decay.
%%%%%%%%%%%%%%%%%%%%%%
%%%%%%%%%%%%%%%%%%%%%%
Following the formalism presented in Ref.~\cite{wu2019carrier}, we computed the nonradiative lifetime of the ground state direct recombination ($T=10$ K is chosen to compare with measurement at cryogenic temperatures~\cite{goldman2015phonon}). %for the $\es$ triplet excited state to the $\gs$ triplet ground state.
Consistent with their Huang-Rhys factors, the nonradiative lifetime of $\ti$ is found to be 10 s, while the nonradiative lifetime of the $\mo$ defect is found to be 0.02 $\mu$s.
The former lifetime is indicative of a forbidden transition; however, the $\ti$ defect also possesses a pseudo Jahn-Teller (PJT) effect in the triplet excited state (red curve in Figure~\ref{fig:config}a).
Due to the PJT effect, the excited state ($C_S$, $\es$) can relax to lower symmetry ($C_1$, $\pjt$) with a nonradiative lifetime of 394 ps (solid dark blue line in Figure~\ref{fig:multi}a, additional details see Supplementary Note 9 and Figure 11). Afterward, nonradiative decay from $\pjt$ to the ground state ($\gs$) (dashed dark blue line in Figure~\ref{fig:multi}a) exhibits a lifetime of 0.044 ps due to a large Huang-Rhys factor (14.95).



\begin{figure}[H]
    \centering
    \includegraphics[keepaspectratio=true,width=0.8\linewidth]{3-2d/figures-npj/tibn_configuration.png}
    \caption{
            \textbf{Configuration coordinate diagrams.}
    Configuration diagram of the (\textbf{a}) $\ti$ defect and (\textbf{b}) $\mo$ defect in $h$-BN. The potential energy surfaces for each state are as follows: the triplet ground state in black, triplet excited state in blue, and for the $\ti$ defect the pseudo Jahn-Teller triplet excited state in red. The zero-phonon lines (ZPL) are given as the energetic separation between the minima of the respective potential energy surfaces, along with the corresponding Huang-Rhys factors ($S$). The dashed black line represents the vertical excitation energy between triplet ground and excited states, and $\Delta E_{FC}$ represents relaxation energy to equilibrium geometry at the excited state.
}
    \label{fig:config}
\end{figure}


%%%%%%%%%%%%%%%%%%%%%%%
\subsection{Spin-Orbit Coupling and Nonradiative Intersystem Crossing Rate}
%($\es \rightarrow \sin$)}
Lastly, we considered the possibility of an ISC between the triplet excited state and the singlet ground state for each defect, which is critical for spin qubit application.
In order for a triplet to singlet transition to occur, a spin-flip process must take place. For ISC, typically spin-orbit coupling (SOC) can entangle triplet and singlet states yielding the possibility for a spin-flip transition. To validate our methods for computing SOC (see methods section), we first computed the SOC strengths for the $\nv$ center in diamond. We obtained SOC values of 4.0 GHz for  the axial $\lambda_z$ and 45 GHz for non-axial  $\lambda_\perp$ in fair agreement with previously computed values and experimentally measured values~\cite{thiering2017ab,bassett2014ultrafast}.
We then computed the SOC strength for the $\ti$ defect ($\lambda_z = 149$ GHz, $\lambda_\perp = 312$ GHz) and the $\mo$ defect ($\lambda_z = 16$ GHz, $\lambda_\perp = 257$ GHz).
The value of $\lambda_\perp$ in particular leads to the potential for a spin-selective pathway for both defects, analogous to $\nv$ center in diamond.

% To compute the ISC rate, we considered a new approach which is a derivative of the nonradiative recombination formalism presented in Eq.~\ref{eq:rate-nonrad-allterms}:
To compute the ISC rate, we developed an approach which is a derivative of the nonradiative recombination formalism presented in Eq.~\ref{intro:eq:nonrad}:
\begin{align}
    \Gamma_{ISC} &= 4 \pi \hbar \lambda_\perp^2 \widetilde{X}_{if}(T) \label{eq:isc-full}\\
    \widetilde{X}_{if}(T) &= \sum_{n,m}p_{in} \left|
        \braket{\phi_{fm}(\textbf{R})}{\phi_{in}(\mathbf{R})}
    \right|^2 \delta(m\hbar\omega_f - n\hbar\omega_i+\Delta E_{if}) \label{eq:isc-F}
\end{align}
% Opposed to Eq.~\ref{eq:rate-nonrad-X}, the phonon overlap in Eq.~\ref{eq:isc-F} does not have an expectation on $\Delta Q$, however it behaves similarly nonetheless and has an exponential dependence on the Huang-Rhys factor $S$.
Compared with previous formalism,~\cite{thiering2017ab} this method
%has the advantages of a temperature dependence (previous approximate $T \rightarrow 0$) and
allows different values for initial state vibrational frequency ($\omega_i$) and final state one ($\omega_f$) through explicit calculations of phonon wavefunction overlap. Again to validate our methods we first computed the intersystem crossing rate for $\nv$ center in diamond. Using the experimental value for $\lambda_\perp$ we obtain an intersystem crossing rate for $\nv$ center in diamond of 2.3 MHz which is in excellent agreement with the experimental value of 8 and 16 MHz~\cite{goldman2015phonon}.
In final, we obtain an intersystem crossing time of 83 ps for $\ti$ and 2.7 $\mu$s for $\mo$ as shown in Table~\ref{table:nonrad} and light blue lines in Figure~\ref{fig:multi}.

% Table for nonradiative lifetimes
\begin{table}
    \footnotesize
    \centering

% \resizebox{1\columnwidth}{!}{
\begin{tabular}{cccccc}
\hline \hline
    % \multirow{2}{*}{GSR}
    $\ti$ & GSR & ZPL (eV) & $S_f$ & $W_{if}$ (eV/(amu$^{1/2}\si{\angstrom}$)) & $\tau_{NR}$ (ps) \\
    % & & (eV) & & eV/(amu$^{1/2}\si{\angstrom}$) & (ps) \\

    & $\es \rightarrow \gs$  & 0.494 & 0.91 & $1.02\times 10^{-1}$ & $8.80\times 10^{12}$ \\
    & $\pjt \rightarrow \gs$ & 0.482 & 14.95 & $1.91\times 10^{-2}$ & $4.41\times 10^{-2}$ \\

    & PJT & $E_{JT}$ (eV) & $S_f$  & $\delta_{JT}$ (eV) & $\tau^C_{NR}$ (ps) \\
    % &  & (eV) & & (eV) & (ps)\\

    % $\es \rightarrow \pjt$ & 0.012 & 13.30 & -- & 10.75 & 0.006 & -- & $3.94\times 10^{2}$ \\
    & $\es \rightarrow \pjt$ & 0.012 & 10.75 & 0.006 & $3.94\times 10^{2}$ \\

    & ISC & ZPL (eV) & $S_f$ & $\lambda_{\perp}$ (GHz) & $\tau_{ISC}$ (ps) \\
    % & & (eV) & & (GHz) & (ps) \\

    & $\es \rightarrow \sin$ & 0.189 & 17.48 & 312 & $8.30\times 10^{1}$ \\

\hline
    $\mo$ & GSR & ZPL (eV) & $S_f$ & $W_{if}$ (eV/(amu$^{1/2}\si{\angstrom}$)) & $\tau_{NR}$ ($\rm \mu s$) \\
    % & & (eV) & & eV/(amu$^{1/2}\si{\angstrom}$) & ($\rm \mu s$)\\

    & $\moes \rightarrow \mogs$  & 0.915 & 22.05 & $1.5\times 10^{-2}$ & $0.02$ \\

    & ISC & ZPL (eV) & $S_f$ & $\lambda_{\perp}$ (GHz) & $\tau_{ISC}$ ($\rm \mu s$) \\
    % & & (eV) & & (GHz) & ($\rm \mu s$) \\

    & $\moes \rightarrow \mosin$ & 0.682 & 7.22 & 257 & $2.7$ \\

\hline \hline
\end{tabular}
% }

    \caption{
Various nonradiative recombination lifetimes along with relevant quantities for the $\ti$ and $\mo$ defects in $h$-BN, including ground state recombination (GSR), pseudo Jahn-Teller (PJT), and intersystem crossing (ISC).
    }
    \label{table:nonrad}
\end{table}

The results of all the nonradiative  pathways for two spin defects are summarized in Table~\ref{table:nonrad} and are displayed in Figure~\ref{fig:multi} along with the radiative pathway.
We begin by summarizing the results for $\ti$ first and then discuss $\mo$ below. In short, for $\ti$ the spin conserved optical excitation from the triplet ground state $\gs$ to the triplet excited state $\es$ cannot directly recombine nonradiatively due to a weak electron-phonon coupling between these states.
In contrast, a nonradiative decay is possible via its PJT state ($\pjt$) with a lifetime of 394 ps.
Finally, the process of intersystem crossing from the triplet excited state $\es$ to the singlet state ($\sin$) is an order of magnitude faster (i.e.\ 83 ps) and is in-turn a dominant relaxation pathway. Therefore the $\ti$ defect in $h$-BN is predicted to have an expedient spin purification process due to a fast intersystem crossing with a rate of 12 GHz.
We note that while the defect has a low optical quantum yield and is predicted to not be a good SPE candidate, it is still noteworthy, as to date the only discovered triplet defect in $h$-BN is the negatively charged boron vacancy, which also does not exhibit SPE and has similarly low quantum efficiency.~\cite{kianinia2020generation} Meanwhile, the leveraged control of an extrinsic dopant can offer advantages in spatial and chemical nature of defects.

For the $\mo$ defect, its direct nonradiative recombination lifetime from the triplet excited state $\moes$ to the ground state $\mogs$ is 0.02 $\mu$s.
While the comparison with its radiative lifetime (33 $\mu$s) is improved compared to the $\ti$ defect, it still is predicted to have low quantum efficiency.
However, again the intersystem crossing between $\moes$ and $\mosin$ is competitive with a lifetime of 2.7 $\mu$s. This rate (around MHz) is similar to diamond and implies a feasible intersystem crossing. Owing to its more ideal ZPL position ($\sim$1eV) and improved quantum efficiency, optical control of the $\mo$ defect is seen as more likely and may be further improved by other methods such as coupling to optical cavities~\cite{kim2018photonic,zhong2018optically} and applying strain~\cite{wu2019carrier,mendelson2020strain}.

%%%%%%%%%%%%%%%%%%%%%%%%%%%%%%%%%%%%%%%%%%%%%%%%%%%%%%%%%%%%%%%%%%%%%%%%%%%%%%%%%%%%%%%%%%%%%%%%%%%%%%%%%%%%%%%%%%
\subsection{Conclusion}
In summary, we proposed a general theoretical framework for identifying and designing optically-addressable spin defects for the future development of quantum emitter and quantum qubit systems.
We started from searching for defects with triplet ground state by DFT total energy calculations which allow for rapid identification of possible candidates. Here we found that the $\ti$ and $\mo$ defects in $h$-BN have a neutral triplet ground state. We then
computed zero-field splitting of secondary spin quantum sublevels and found they are sizable for both defects, larger than that of $\nv$ center in diamond, enabling possible control of these levels for qubit operation.
In addition, we screened for potential single photon emitters (SPEs) in $h$-BN based on allowed intra-defect transitions and radiative lifetimes, leading to the discovery of $\vv{Si}$.
Next the electronic structure and optical spectra of each defect were computed from many-body perturbation theory.
Specifically, the $\vv{Si}$ defect is shown to possess an exciton radiative lifetime similar to experimentally observed SPEs in $h$-BN and is a potential SPE candidate.
Finally, we analyzed all possible radiative and nonradiative dynamical processes with first-principles rate calculations. In particular, we identified a dominant spin-selective decay pathway via intersystem crossing at the $\ti$ defect which gives a key advantage for initial pure spin state preparation and qubit operation.
Meanwhile, for the $\mo$ defect we found that it has the benefit of improved quantum efficiency for more realistic optical control.

%%%%%%%%%%%

This work emphasizes that the theoretical discovery of spin defects requires careful treatment of many-body interactions and various radiative and nonradiative dynamical processes such as intersystem crossing.
We demonstrate high potential of extrinsic spin defects in 2D host materials as qubits for quantum information science.
Future work will involve further examination of spin coherence time and its dominant decoherence mechanism,
as well as other spectroscopic fingerprints from first-principles calculations to facilitate experimental validation of these defects.

%%%%%%%%%%%%%%%%%%%%%%%%%%%%%%%%%%%%%%%%%%%%%%%%%%%%%%%%%%%%%%%%%%%%%%%%%%%%%%%%%%%%%%%%%%%%%%%%%%%%%%%%%%%%%%%%%%

\subsection{Computational Details} In this study, we used the open source plane-wave code Quantum ESPRESSO~\cite{QE1} to perform calculations on all structural relaxations and total energies with optimized norm-conserving Vanderbilt (ONCV) pseudopotentials~\cite{ONCV1} and a wavefunction cutoff of 50 Ry. A supercell size of $6\times 6$ or higher was used in our calculations with a $3\times 3\times 1$ k-point mesh. Charged cell total energies were corrected to remove spurious charge interactions by employing the techniques developed in Refs.~\cite{PING2017JCP,wu2017first,wang2020layer} and implemented in the JDFTx code~\cite{JDFTx}.
The total energies, charged defect formation energies and geometry were evaluated at the Perdew-Burke-Ernzerhof (PBE) level~\cite{perdew1996generalized}.
Single-point calculations with k-point meshes of $2\times 2\times 1$ and $3\times 3\times 1$ were performed using hybrid exchange-correlation functional PBE0($\alpha$), where the mixing parameter $\alpha=0.41$ was determined by the generalized Koopmans’ condition as discussed in Ref.~\cite{smart2018fundamental,miceli2018nonempirical}. Moreover, we used the YAMBO code~\cite{YAMBO} to perform many-body perturbation theory with the GW approximation to compute the quasi-particle correction
using PBE0($\alpha$) eigenvalues and wavefunctions as the starting point.
The random phase approximation (RPA) and Bethe-Salpeter Equation (BSE) calculations were further solved on top of the GW approximation for the electron-hole interaction to investigate the optical properties of the defects, including absorption spectra and radiative lifetime.
%%%%%%%%%

% \subsection{Thermodynamic Charge Transition Levels and Defect Formation Energy}
% The defect formation energy ($FE_q$) was computed for the $\ti$ and $\mo$ defects following:
% \begin{align}
%     FE_q(\varepsilon_F) = E_q - E_{pst} + \sum_i \mu_i \Delta N_i + q \varepsilon_F + \Delta_q
%     \label{eq:cfe}
% \end{align}
% where $E_q$ is the total energy of the defect system with charge $q$, $E_{pst}$ is the total energy of the pristine system, $\mu_i$ and $\Delta N_i$ are the chemical potential and change in the number of atomic species $i$, and $\varepsilon_F$ is the Fermi energy. A charged defect correction $\Delta_q$ was computed for charged cell calculations by employing the techniques developed in Ref.~\cite{wu2017first,PING2017JCP}. The chemical potential references are computed as $\mu_{Ti} = E_{Ti}^{bulk}$ (total energy of bulk Ti), $\mu_{Mo} = E_{Mo}^{bulk}$ (total energy of bulk Mo),  $\mu_{BN} = E_{BN}^{ML}$ (total energy of monolayer $h$-BN). Meanwhile the corresponding charge transition levels of defects can be obtained from the value of $\varepsilon_F$ where the stable charge state transitions from $q$ to $q'$.
% \begin{align}
%     \epsilon_{q|q'} = \frac{FE_q - FE_{q'}}{q' - q}
%     \label{eq:ctl}
% \end{align}

\subsection{Zero-Field Splitting}
The first-order ZFS due to spin-spin interactions was computed for the dipole-dipole interactions of the electron spin:
\begin{align}
    H_{ss} = \frac{\mu_0}{4\pi} \frac{(g_e\hbar)^2}{r^5} \left[
        3(\textbf{s}_1 \cdot \textbf{r})(\textbf{s}_2 \cdot \textbf{r})
        -(\textbf{s}_1 \cdot \textbf{s}_2)r^2
    \right].
\end{align}
Here, $\mu_0$ is the magnetic permeability of vacuum, $g_e$ is the electron gyromagnetic ratio, $\hbar$ is the Planck's constant, $\textbf{s}_1$, $\textbf{s}_2$ is the spin of first and second electron, respectively, and $\textbf{r}$ is the displacement vector between these two electron.
The spatial and spin dependence can be separated by introducing the effective total spin $\textbf{S}=\sum_i \textbf{s}_i$. This yields a Hamiltonian of the form $H_{ss} =  \textbf{S}^T\hat{\textbf{D}}\textbf{S}$, which introduces the traceless zero-field splitting tensor $\hat{\textbf{D}}$. It is common to consider the axial and rhombic ZFS parameters $D$ and $E$ which can be acquired from the $\hat{\textbf{D}}$ tensor:
\begin{align}
    D = \frac{3}{2} D_{zz} \quad \text{and} \quad E = (D_{yy} - D_{xx})/ 2\ .
\end{align}
Following the formalism of Rayson et al.,~\cite{rayson2008first} the ZFS tensor $\hat{\textbf{D}}$ can be computed with periodic boundary conditions as:
\begin{align}
    D_{ab} = \frac{1}{2}\frac{\mu_0}{4\pi} (g_e\hbar)^2 \sum_{i>j} \chi_{ij}
        \expval{
            \frac{\textbf{r}^2\delta_{ab}-3\textbf{r}_a\textbf{r}_b}{r^5}
        }{
            \Psi_{ij}(\textbf{r}_1, \textbf{r}_2)
        }.
\end{align}
Here the summation on pairs of $i,j$ runs over all occupied spin-up and spin-down states, with $\chi_{ij}$ taking the value $+1$ for parallel spin and $-1$ for anti-parallel spin, and $\Psi_{ij}(\textbf{r}_1,\textbf{r}_2)$ is a two-particle Slater determinant constructed from the Kohn-Sham wavefunctions of the $i$th and $j$th states. This procedure was implemented as a post-processing code interfaced with Quantum ESPRESSO.
To verify our implementation is accurate, we computed the ZFS of the $\nv$ center in diamond which has a well-established result. Using ONCV pseudopotentials, we obtained a ZFS of 3.0 GHz for $\nv$ center, in perfect agreement with previous reported results~\cite{seo2017designing}.
For heavy elements such as transition metals, spin-orbit (SO) coupling can have substantial contribution to zero-field splitting. Here, we also computed the SO contribution of the ZFS as implemented in the ORCA code~\cite{neese2012orca,neese2007calculation} (additional details can be found in Supplementary Note 10, Figure 12, and Table 6).

% \subsection{Radiative Recombination}
% In order to quantitatively study radiative processes, we computed the radiative rate $\Gamma_R$ from Fermi's Golden Rule and considered the excitonic effects by solving BSE~\cite{wu2019dimensionality}:
% \begin{align}
%     \Gamma_R (\textbf{Q}_{ex}) &=
%     \frac{2\pi}{\hbar}
%     \sum_{q_L, \lambda}
%     \left|
%         \mel{G,1_{q_L,\lambda}}
%         {H^R}
%         {S(\textbf{Q}_{ex}),0}
%     \right|^2
%     \delta(E(\textbf{Q}_{ex}) -\hbar c q_L).
%     \label{eq:rate-radiative-full}
% \end{align}
% % \begin{align}
% %     \Gamma_R (\textbf{Q}) &= \frac{4\pi^2e^2}{\hbar c^2V} \Omega(\textbf{Q})^2 \sum_{q, \lambda} \frac{1}{q} \left|
% %         \epsilon_{q\lambda} \cdot \mel{G}{\textbf{r}}{S}
% %     \right|^2 \delta(\frac{\Omega(\textbf{Q})}{c} - q)
% %     \label{eq: rate-radiative-full}
% % \end{align}
% Here, the radiative recombination rate is computed between the ground state $G$ and the two-particle excited state $S(\textbf{Q}_{ex})$, $1_{q_L,\lambda}$ and 0 denote the presence and absence of a photon, $H^R$ is the electron-photon coupling (electromagnetic) Hamiltonian,  $E(\textbf{Q}_{ex})$ is the exciton energy, and $c$ is the speed of light.
% The summation indices in Eq.~\ref{eq:rate-radiative-full} run over all possible wavevector ($q_L$) and polarization ($\lambda$) of the photon.
% Following the approach described in Ref.~\cite{wu2019dimensionality}, the radiative rate (inverse of radiative lifetime $\tau_R$) in SI unit at zero temperature can be computed for isolated defect-defect transitions as:
%  %\Gamma_R  &= \frac{4e^2}{3\hbar c^3} \Omega_0^3 |\mu|_{e-h}^2,\\
% \begin{equation}
%      \Gamma_R = \frac{n_D e^2}{3\pi\epsilon_0\hbar^4 c^3} E_0^3 \mu_{e-h}^2,
%     \label{eq:rate-radiative-0D}
% \end{equation}
% where $e$ is the charge of an electron, $\epsilon_0$ is vacuum permittivity, $E_0$ is the exciton energy at $\textbf{Q}_{ex}=0$, $n_D$ is the reflective index of the host material and $\mu_{e-h}^2$ is the modulus square of exciton dipole moment with length$^2$ unit. Note that Eq.~\ref{eq:rate-radiative-0D} considers defect-defect transitions in the dilute limit; therefore the lifetime formula for zero-dimensional systems embedded in a host material is used~\cite{gupta2018two,mackoit2019carbon} (also considering $n_D$ is unity in isolated 2D systems at the long-wavelength limit). We did not consider the radiative lifetime of $\ti$ defect at a finite temperature because the first and second excitation energy separation is much larger than $kT$. Therefore a thermal average of the first and higher excited states is not necessary and the first excited state radiative lifetime is nearly the same at 10 K as zero temperature.


% \subsection{Phonon-Assisted Nonradiative Recombination} In this work, we compute the phonon-assisted nonradiative recombination rate via a Fermi's golden rule approach:
% \begin{align}
%     \Gamma_{NR} &= \frac{2\pi}{\hbar}g\sum_{n,m}p_{in}\left|\mel{fm}{H^{e-ph}}{in}\right|^2\delta(E_{in}-E_{fm}) \label{eq:rate-nonradiative-fg}
% \end{align}
% Here, $\Gamma_{NR}$ is the nonradiative recombination rate between electron state $i$ in phonon state $n$ and electron state $f$ in phonon state $m$, $p_{in}$ is the thermal probability distribution of the initial state $\ket{in}$, $H^{e-ph}$ is the electron-phonon coupling Hamiltonian, $g$ is the degeneracy factor and $E_{in}$ is the energy of vibronic state $\ket{in}$. Within the static coupling and one-dimensional (1D) effective phonon approximations, the nonradiative recombination can be reduced to:
% \begin{align}
%     \Gamma_{NR} &= \frac{2\pi}{\hbar}g|W_{if}|^2 X_{if}(T), \label{eq:rate-nonrad-allterms}\\
%     X_{if}(T) &= \sum_{n,m}p_{in}\left|\mel{\phi_{fm}(\textbf{R})}{Q-Q_a}{\phi_{in}(\mathbf{R})}\right|^2 \delta(m\hbar\omega_f - n\hbar\omega_i+\Delta E_{if}), \label{eq:rate-nonrad-X} \\
%     W_{if} &= \mel{\psi_i(\mathbf{r}, \textbf{R})}{\frac{\partial H}{\partial Q}}{\psi_f(\mathbf{r},\textbf{R})}\bigg{|}_{\textbf{R}=\textbf{R}_a}. \label{eq:rate-nonrad-W}
% \end{align}
% Here, the static coupling approximation naturally separates the nonradiative recombination rate into phonon and electronic terms, $X_{if}$ and $W_{if}$, respectively. The 1D phonon approximation introduces a generalized coordinate $Q$, with effective frequency $\omega_i$ and $\omega_f$. The phonon overlap in Eq.~\ref{eq:rate-nonrad-X} can be computed using the quantum harmonic oscillator wavefunctions with $Q-Q_a$ from the configuration diagram (Figure~\ref{fig:config}). Meanwhile the electronic overlap in Eq.~\ref{eq:rate-nonrad-W} is computed by finite difference using the Kohn-Sham orbitals from DFT at the $\Gamma$ point. The nonradiative lifetime $\tau_{NR}$ is given by taking the inverse of rate $\Gamma_{NR}$. Supercell convergence of phonon-assisted nonradiative lifetime is shown in Supplementary Note 11 and Table 7. We validated the 1D effective phonon approximation by comparing the Huang-Rhys factor with the full phonon calculations in Supplementary Table 8.

\subsection{Spin-Orbit Coupling Constant} Spin-orbit coupling (SOC) can entangle triplet and singlet states yielding the possibility for a spin-flip transition. The SOC operator is given to zero-order by~\cite{maze2011properties}:
\begin{align}
    H_{so} = \frac{1}{2} \frac{1}{c^2m_e^2} \sum_i \left(
        \nabla_i V \cross \textbf{p}_i \right) \textbf{S}_i \label{eq:soc-full}
\end{align}
where $c$ is the speed of light, $m_e$ is the mass of an electron, $\textbf{p}$ and $\textbf{S}$ are the momentum and spin of electron $i$ and $V$ is the nuclear potential energy. The spin-orbit interaction can be rewritten in terms of the angular momentum $L$ and the SOC strength $\lambda$ as~\cite{maze2011properties},
\begin{align}
    H_{so} = \sum_i \lambda_{\perp} (L_{x,i}S_{x,i} + L_{y,i}S_{y,i}) + \lambda_z L_{z,i}S_{z,i}.
\end{align}
where $\lambda_{\perp}$ and $\lambda_z$ denote the non-axial and axial SOC strength, respectively. The SOC strength was computed for the $\ti$ defect in $h$-BN using the ORCA code by TD-DFT~\cite{neese2012orca,de2019predicting}.
More computational details can be found in Supplementary Note 10.

\section{Electronic and Optical Properties}

\subsection{Electronic Structure}

\subsection{Optical Spectra}

\subsection{Radiative Recombination}

\def\vhq{\frac{\partial h}{\partial Q}}

\chapter{Formalism of Nonradiative Recombination}

\section{Static Coupling}

\begin{figure}[H]
    \centering
    \includegraphics[width=0.41\textwidth]{5-app/figures/nonrad-diag.png}
    \caption{Diagram of the transition between two vibronic states. The two quadratic curves represent the electronic states $(i,f)$ and the vibrational modes present at each state are superimposed with states $(n,m)$.}
    \label{app:fig:nonrad}
\end{figure}


Here we consider a system which is initially in a vibronic state $\ket{\Psi_{in}(r,R)}$ and transitions to a final vibronic state $\ket{\Psi_{fm}(r,R)}$ with $i\neq f$. Here the indices $(i,f)$ denote the electronic state, while $(n,m)$ denote the phonon state as shown schematically in Figure~\ref{app:fig:nonrad}. Within the Born-Oppenheimer approximation these states are a direct product of the electronic states $\ket{\psi_{(i,f)}(r,R)}$ and phonon states $\ket{\phi_{(n,m)}(R)}$.
\begin{align}
    \ket{\Psi_{(in,fm)}(r,R)}=\ket{\psi_{(i,f)}(r,R)\phi_{(n,m)}(R)}   \label{app:nonrad:eq:vib}
\end{align}
The coordinate $r$ denotes the spatial dependence of the electronic wavefunction $\psi_{(i,f)}$ and $R$ is the configuration of atomic positions.

The probability of transitioning between vibronic state $\ket{\Psi_{i,n}(r,R)}$ to the state $\ket{\Psi_{f,m}(r,R)}$ is given by Fermi's Golden Rule:
\begin{align}
    \Gamma_{in\rightarrow fm} = \frac{2\pi}{\hbar} f(i,n) |V_{in,fm}|^2 \delta(E_{in}-E_{fm})
    \label{app:nonrad:eq:g1}
\end{align}
Here $f(i,n)$ is the probability of occupying phonon state $n$ when in the electronic state $i$, which follows a thermal Maxwell-Boltzmann distribution. The dirac delta function ensures the conservation of energy between vibronic states $E_{in}$ and $E_{fm}$. And finally $V_{in,fm}$ is the electron phonon coupling matrix, discussed in more detail later.

If we wish to compute the collective transition rate between electronic states $i$ and $f$ this follows readily from Eq. (\ref{app:nonrad:eq:g1}).
\begin{align}
    \Gamma_{i \rightarrow f} &=\sum_{n,m} \Gamma_{in\rightarrow fm} \nonumber \\
    \Gamma_{i \rightarrow f}
    &=\frac{2\pi}{\hbar}\sum_{n,m}f(i,n)|V_{in,fm}|^2\delta(E_{in}-E_{fm})  \label{app:nonrad:eq:g2}
\end{align}
Here we have now summed over all possible initial and final phonon states ($n$ and $m$) to give the full probability of transitioning from electronic state $i$ to state $f$.

A term of particular interest in Eq. (\ref{app:nonrad:eq:g2}) is the electron-phonon coupling matrix $V_{in,fm}$. Within the static coupling approximation we approximate the electron-phonon coupling to first order in $R$. First consider the total Hamiltonian $H_{tot}$ to first order in $R$ about some position $R_{0}$ as:
\begin{align}
    H_{tot}(r,R) = H(r,R_0) + \sum_R \frac{\partial H}{\partial R} (R-R_0) \label{app:nonrad:eq:fo1}
\end{align}
where $H$ is the electron Hamiltonian, and the partial of $H$ with respect to $R$ is for every atomic position in 3D space. Meanwhile the electronic wavefunction is $\psi_{i}(r,R)$ to first order in $R$ is given by
\begin{align}
    \ket{\psi_i(r,R)} = \ket{\psi_i(r,R_0)} + \sum_R (R-R_0) \ket{\frac{\partial \psi_{i}}{\partial R}}\label{app:nonrad:eq:fo2}
\end{align}
where in Eq. (\ref{app:nonrad:eq:fo1}-\ref{app:nonrad:eq:fo2}) the evaluation of the derivative with respect to $R$ at $R_{0}$ is implicit.


Using the approximations of Eq. (\ref{app:nonrad:eq:fo1}-\ref{app:nonrad:eq:fo2}), the electron-phonon coupling matrix is given by:
\begin{align}
    V_{in,fm}
    &= \braket{\Psi_{f,m}(r,R)|
    H_{tot}(r,R)
    |\Psi_{i,n}(r,R)} \nonumber \\
    &= \braket{\psi_f(r,R_0)\phi_{f,m}(R)|
    H_0
    |\psi_i(r,R_0)\phi_{i,n}(R)} \nonumber \\
    & \hspace{1cm} + \sum_R \bigg( \braket{\psi_f(r,R_0)\phi_{f,m}(R)|
    \frac{\partial H}{\partial R}(R-R_0)
    |\psi_i(r,R_0)\phi_{i,n}(R)} \nonumber \\
    & \hspace{1cm} + \braket{\psi_f(r,R_0)\phi_{f,m}(R)|
    H_0 (R-R_0)
    |\frac{\partial \psi_i}{\partial R}\phi_{i,n}(R)} \nonumber \\
    & \hspace{1cm} + \braket{\frac{\partial \psi_f}{\partial R}\phi_{f,m}(R)|
    (R-R_0)H_0
    |\psi_i(r,R_0)\phi_{i,n}(R)} \bigg) \nonumber \\
    & \hspace{1cm} + \mathcal{O}(R^2) \nonumber \\
    &= \sum_R \braket{\phi_{f,m}(R)|(R-R_0)
    |\phi_{i,n}(R)}
    \bigg[ \braket{\psi_f(r,R_0)|
    \frac{\partial H}{\partial R}
    |\psi_i(r,R_0)} \nonumber \\
    & \hspace{2cm} + \braket{\psi_f(r,R_0)|
    H_0
    |\frac{\partial \psi_i}{\partial R}}
    + \braket{\frac{\partial \psi_f}{\partial R}|
    H_0
    |\psi_i(r,R_0)} \bigg] \nonumber \\
    &= \sum_R
    \braket{\phi_{f,m}(R)|(R-R_0)|\phi_{i,n}(R)}
    \braket{\psi_f(r,R_0)|\frac{\partial H}{\partial R}|\psi_i(r,R_0)}
\end{align}
Where in the first step we have we have removed any terms of order $R^2$ (denoted with $\mathcal{O}(R^2)$). In the second step, the first term is removed do to orthogonality $\braket{\psi_f|\psi_i}=0$, and the latter part is rewritten with the factorization of the electronic and phonon parts due to there independence on $R$ and $H$, respectively. In the final step, we are left with only one term as the last two terms cancel (see Appendix). Thus, the static coupling approximation gives an electron-phonon coupling matrix of the form:
\begin{align}
    V_{in,fm}= \sum_R
    \braket{\phi_{f,m}(R)|(R-R_0)|\phi_{i,n}(R)}
    \braket{\psi_f(r,R_0)|\frac{\partial H}{\partial R}|\psi_i(r,R_0)}
    \label{app:nonrad:eq:eph1}
\end{align}

Alternatively, the electron-phonon coupling can instead be expressed in terms of phonon modes $Q_k$,
\begin{eqnarray}
    V_{in,fm}
    &=&
    \sum_k \braket{\psi_f(r,R_0)|
    \frac{\partial H}{\partial Q_k}|\psi_i(r,R_0)}
    \braket{\phi_{f,m}(R)|(Q_k-Q_{k,0})|\phi_{i,n}(R)} \\
    &=& \sum_k C^k_{if}\braket{\phi_{f,m}(R)|
    \mathbf{Q}_k|\phi_{i,n}(R)} \label{app:nonrad:eq:eph2}
\end{eqnarray}
with
\begin{eqnarray}
    \mathbf{Q}_{i,k}=\frac{1}{\sqrt{M_k}}\sum_R M_R \mu_k(R) \mathbf{R}_i \nonumber \\
    \mathbf{Q}_{f,k}=\frac{1}{\sqrt{M_k}}\sum_R M_R \mu_k(R) \mathbf{R}_f
\end{eqnarray}
Here $\mathbf{R}_i = R_i - R_i(0)$ and  $\mathbf{R}_f = R_f - R_f(0)$ is the displacement of the atomic positions from equilibrium. $M_R$ is the mass of the atom located at position $R$, $M_k$ is the reduced mass in the $k$th phonon mode, and $\mu_k(R)$ is the phonon mode displacement vector at position $R$.

In Eq. (\ref{app:nonrad:eq:eph2}), we have defined the electron-electron coupling constants $C^k_{if}$.
\begin{equation}
    C^k_{if}
    = \braket{\psi_f(r,R_0)|
    \frac{\partial H}{\partial Q_k}|\psi_i(r,R_0)}
    =\sum_R \mu_k(R) \braket{\psi_f(r,R_0)|
    \frac{\partial H}{\partial R}|\psi_i(r,R_0)}
\end{equation}
Now for the full transition rate we have
\begin{eqnarray}
    \Gamma_{i\rightarrow f}&=& \frac{2\pi}{\hbar}\sum_{k_1,k_2} C^{k_1}_{if} C^{k_2}_{if}
    \bigg( \sum_{n,m} f(i,n) \braket{\phi_{i,n}(R)|
    \mathbf{Q}_{k_1}|\phi_{f,m}(R)} \nonumber \\
    && \hspace{6mm} \cdot
    \braket{\phi_{f,m}(R)|\mathbf{Q}_{k_2}|\phi_{i,n}(R)}
    \delta(\hbar\omega_{in}-\hbar\omega_{fm}-\Delta E_{if}) \bigg) \label{app:nonrad:eq:g3}
\end{eqnarray}
If we use the integral form of the dirac delta function $\delta(x)=\frac{1}{2\pi}\int_{-\infty}^{\infty}e^{ixt}dt$
then we can reduce the phonon-phonon coupling piece of Eq. \ref{app:nonrad:eq:g3} (see Appendix for derivation):
\begin{equation}
    \Gamma_{if}=\frac{2\pi}{\hbar} \sum_{k1,k2} C^{k_1}_{if}C^{k_2}_{if}\cdot A^{k1,k2}_{if} \label{app:nonrad:eq:g4}
\end{equation}
where,
\begin{align}
    A^{k1,k2}_{if}=\frac{1}{2\pi \mathcal{Z}}\int_{-\infty}^{\infty}\chi^{k_1,k_2}_{if}(t,T)e^{-it\Delta E_{if}/\hbar}\,dt \label{app:nonrad:eq:a}\\
    \chi^{k_1,k_2}_{if}(t,T) = \text{Tr}\left[ \mathbf{Q}_{k_1} e^{-it H_{f}/\hbar}
    \mathbf{Q}_{k_2}e^{-(\beta\hbar-it)H_{i}/\hbar} \right] \label{app:nonrad:eq:chi}
\end{align}


\section{Full-Phonon}
This section discusses details of computing Eq. \ref{app:nonrad:eq:a} \& \ref{app:nonrad:eq:chi} in practice,  following the implementation of [Shi 2015 PRB]. First of all we will assume that the phonon modes in states $i$ and $j$ are the same, so $k_1 = k_2 = k$. Next we introduce the following diagonal ($N_{vib} \times N_{vib}$) matrices:
\begin{align}
    a(\tau_\xi)_k = \frac{\omega_k}{\sinh{(i\hbar\omega_k\tau_\xi)}}\, , \quad
    c(\tau_\xi)_k = \omega_k\coth{(i\hbar\omega_k\tau_\xi/2)}\, , \quad \nonumber \\
    d(\tau_\xi)_k = \omega_k\tanh{(i\hbar\omega_k\tau_\xi/2)}\, .
\end{align}
Where, $\xi = (i,j)$, $\tau_i = -t-i\beta$, $\tau_j = t$, and $\omega_k$ is the frequency of the $k$th harmonic oscillator. We then define matrices:
\begin{align}
    C(\tau_i,\tau_j)_k = c(\tau_i)_k + c(\tau_j)_k\, , \quad
    D(\tau_i,\tau_j)_k = d(\tau_i)_k + d(\tau_j)_k\, .
\end{align}
And also
\begin{align}
    D_\text{HT} &= -D^{-1} d(\tau_j) \textbf{K} \, , \\
    A_\text{HT} &= \frac{1}{2} (D^{-1} - C^{-1}) + D_\text{HT}(D_\text{HT})^T\, ,
\end{align}
where
\begin{align}
    \textbf{K}_k = \Delta Q_{ij,k} = \frac{1}{\sqrt{M_k}} \sum_R M_R \mu_k(R) \Delta R_{ij}\, .
\end{align}
This gives the final form:
\begin{align}
    \chi_{ij}^k(t,T) &=
        \sqrt{
            \frac{
                \det{[a(\tau_j)]} \det{[a(\tau_i)]}
            }{
                (i\hbar)^{2N} \det{(C)} \det{(D)}
            }
        }\, \nonumber \\
        & \hspace{2cm} \times \exp \left[
            -\textbf{K}^T d(\tau_j)\textbf{K}
            + \textbf{K}^T d(\tau_j) D^{-1} d(\tau_i)\textbf{K}
        \right]
        (A_\text{HT}) \label{app:nonrad:eq:final}
\end{align}
One can then integrate Eq. \ref{app:nonrad:eq:a} to give the final phonon part.


\section{Linear Response Theory}

Here we consider the single particle Hamiltonian ($h$) to first order deviation in the one-dimensional effective coordinate ($Q$) as:

\begin{align}
    h=h_{a}+\frac{\partial h}{\partial Q} (Q-Q_{a})
\end{align}

We can consider the latter term as a perturbation on the system where only the term $\vhq$ acts on the electronic states ($Q$ acts on phonon states). Therefore, the first-order response of the electronic eigenstate ($\varphi_m$) is given by:

\begin{align}
    \ket{\Delta\varphi_m} = \sum_{n\neq m} \ket{\varphi_n} \frac{\bra{\varphi_n} \vhq \ket{\varphi_m}}{\varepsilon_m-\varepsilon_n} \label{app:nonrad:eq:p}
\end{align}

We now work to solve for $\bra{\varphi_n} \vhq \ket{\varphi_m}$, the term we want to replace in the current formalism. First consider a simple rewrite of Eq. \ref{app:nonrad:eq:p}.

\begin{align}
    \ket{ \Delta\varphi_m} =  \left(\sum_{n\neq m} \ket{\varphi_n}\bra{\varphi_n}\right)\frac{ \vhq \ket{\varphi_m}}{\varepsilon_m-\varepsilon_n}
\end{align}
Evoking the completeness relation gives
\begin{align}
    \ket{ \Delta\varphi_m} = \bigg(\mathbb{1}- \ket{\varphi_m}\bra{\varphi_m}\bigg)\frac{ \vhq \ket{\varphi_m}}{\varepsilon_m-\varepsilon_n}
\end{align}
Then taking the inner product with $\bra{\varphi_n}$ and implementing the orthogonality of these states $\braket{\varphi_n|\varphi_m}=\delta_{nm}$ (in this case $n$ and $m$ differ, so $\braket{\varphi_n|\varphi_m}=0$).
\begin{align}
    \braket{\varphi_n|\Delta\varphi_m} &= \bigg(\bra{\varphi_n}- \braket{\varphi_n|\varphi_m}\bra{\varphi_m}\bigg)\frac{ \vhq \ket{\varphi_m}}{\varepsilon_m-\varepsilon_n} \\
    &= \bigg(\bra{\varphi_n}\bigg)\frac{ \vhq \ket{\varphi_m}}{\varepsilon_m-\varepsilon_n} \\
    &= \frac{\bra{\varphi_n} \vhq \ket{\varphi_m}}{\varepsilon_m-\varepsilon_n}
\end{align}
This gives the final form we desired ($n=i$ initial state; $m=f$ final state)
\begin{align}
    \bra{\varphi_i} \vhq \ket{\varphi_f} = (\varepsilon_f-\varepsilon_i)\braket{\varphi_i|\frac{\partial \varphi_f}{\partial Q}}
\end{align}
Note that $\varphi_f$ is also considered to change first order in $Q$ and hence $\ket{\Delta\varphi_f}=\ket{\frac{\partial \varphi_f}{\partial Q}}$.


\section{Supplemental Derivations}

\subsection{S1}
Proof that
\begin{align}
    \braket{\psi_f(r,R_0)|H_0|\frac{\partial \psi_i}{\partial R}}
    + \braket{\frac{\partial \psi_f}{\partial R}|H_0 |\psi_i(r,R_0)}  =  0 .
\end{align}
Consider,
\begin{align}
    & \frac{\partial}{\partial R}\bigg(\braket{\psi_f(r,R_0)|H_0 |\psi_i(r,R_0)}\bigg) = \\
    & \hspace{2cm} \braket{\frac{\partial \psi_f(r,R_0)}{\partial R}|H_0 |\psi_i(r,R_0)}
    + \braket{\psi_f(r,R_0)|H_0 |\frac{\partial \psi_i(r,R_0)}{\partial R}} \nonumber \\
    & \hspace{2cm} + \braket{\psi_f(r,R_0)|\frac{\partial H_0}{\partial R} |\psi_i(r,R_0)} \nonumber \\
    & 0 = \braket{\frac{\partial \psi_f(r,R_0)}{\partial R}|H_0 |\psi_i(r,R_0)}
    + \braket{\psi_f(r,R_0)|H_0 |\frac{\partial \psi_i(r,R_0)}{\partial R}}
    + 0 \nonumber \\
    & \Rightarrow \braket{\frac{\partial \psi_f(r,R_0)}{\partial R}|H_0 |\psi_i(r,R_0)}
    + \braket{\psi_f(r,R_0)|H_0 |\frac{\partial \psi_i(r,R_0)}{\partial R}} = 0
\end{align}
Here the left hand side is zero because $\braket{\psi_{f}|\psi_{i}}=0$, while the final term on the right hand side is zero because $\partial H_0/\partial R = 0$.

\subsection{S2}
Below the mathematical steps which allow for the rewriting of Eq. (\ref{app:nonrad:eq:g3}) in terms of Eq. (\ref{app:nonrad:eq:g4}-\ref{app:nonrad:eq:chi}) are presented.
\begin{align}
    &\sum_{n,m} f(i,n) \braket{\phi_{i,n}(R)|
    \mathbf{Q}_{k_1}|\phi_{f,m}(R)} \cdot
    \braket{\phi_{f,m}(R)|\mathbf{Q}_{k_2}|\phi_{i,n}(R)}
    \delta(\hbar\omega_{fm}-\hbar\omega_{in}+\Delta E_{if}) \nonumber \\
    &= \frac{1}{2\pi \hbar\mathcal{Z}} \sum_{n,m} e^{-\beta \hbar\omega_{in}} \braket{\phi_{i,n}(R)|
    \mathbf{Q}_{k_1}|\phi_{f,m}(R)} \cdot
    \braket{\phi_{f,m}(R)|\mathbf{Q}_{k_2}|\phi_{i,n}(R)} \nonumber \\
    &\hspace{7cm}\cdot \int_{-\infty}^{\infty}
    e^{it(\omega_{in}-\omega_{fm}-\Delta E_{if}/\hbar)} \,dt\nonumber \\
    &= \frac{1}{2\pi \hbar\mathcal{Z}} \int_{-\infty}^{\infty} \bigg(\sum_{n,m} \braket{\phi_{i,n}(R)|
    \mathbf{Q}_{k_1}|\phi_{f,m}(R)} \cdot
    \braket{\phi_{f,m}(R)|\mathbf{Q}_{k_2}|\phi_{i,n}(R)} \nonumber \\
    &\hspace{7cm}\cdot
    e^{-(\beta\hbar-it)\omega_{in}-it\omega_{fm}-it\Delta E_{if}/\hbar)} \bigg) \,dt\nonumber \\
    &= \frac{1}{2\pi \hbar\mathcal{Z}} \int_{-\infty}^{\infty} \bigg(\sum_{n,m} \braket{\phi_{i,n}(R)|\mathbf{Q}_{k_1}e^{-it\omega_{fm}}|\phi_{f,m}(R)} \nonumber \\
    &\hspace{4.5cm} \cdot    \braket{\phi_{f,m}(R)|\mathbf{Q}_{k_2}e^{-(\beta\hbar-it)\omega_{in}}|\phi_{i,n}(R)}    e^{-it\Delta E_{if}/\hbar)} \bigg) \,dt\nonumber \\
    &= \frac{1}{2\pi \hbar\mathcal{Z}} \int_{-\infty}^{\infty} \bigg(\sum_{n} \bra{\phi_{i,n}(R)}\mathbf{Q}_{k_1} \nonumber \\
    &\hspace{3cm}\sum_m e^{-it \omega_{fm}}\ket{\phi_{f,m}(R)} \cdot    \bra{\phi_{f,m}(R)} \nonumber \\
    & \hspace{6.2cm} \mathbf{Q}_{k_2}e^{-(\beta\hbar-it)\omega_{in}}\ket{\phi_{i,n}(R)}    e^{-it\Delta E_{if}/\hbar} \bigg) \, dt \nonumber \\
    &= \frac{1}{2\pi \hbar\mathcal{Z}} \int_{-\infty}^{\infty} \bigg(\sum_{n} \bra{\phi_{i,n}(R)}\mathbf{Q}_{k_1} e^{-it H_{f}/\hbar} \mathbf{Q}_{k_2}e^{-(\beta\hbar-it)\omega_{in}}\ket{\phi_{i,n}(R)}    e^{-it\Delta E_{if}/\hbar} \bigg) \, dt \nonumber \\
    &= \frac{1}{2\pi \hbar\mathcal{Z}} \int_{-\infty}^{\infty} \text{Tr} \left[ \mathbf{Q}_{k_1} e^{-it H_{f}/\hbar}
    \mathbf{Q}_{k_2}e^{-(\beta\hbar-it)H_{i}/\hbar} \right]
    e^{-it\Delta E_{if}/\hbar} \, dt
\end{align}
Plugging this piece back into Eq. \ref{app:nonrad:eq:g3} gives the final condensed form of the transition rate.
\begin{equation}
    \Gamma_{if}=\frac{2\pi}{\hbar} \sum_{k1,k2} C^{k_1}_{if}C^{k_2}_{if}\cdot A^{k1,k2}_{if}
\end{equation}
Where we have shown that
\begin{align}
    A^{k1,k2}_{if}=\frac{1}{2\pi \mathcal{Z}}\int_{-\infty}^{\infty}\chi^{k_1,k_2}_{if}(t,T)e^{-it\Delta E_{if}/\hbar}\,dt \\
    \chi^{k_1,k_2}_{if}(t,T) = \text{Tr}\left[ \mathbf{Q}_{k_1} e^{-it H_{f}/\hbar}
    \mathbf{Q}_{k_2}e^{-(\beta\hbar-it)H_{i}/\hbar} \right]
\end{align}


% ------------------------------------------------------------
% other
\chapter{Selected Topics}
\section{Improving Catalytic}
% Ni3S2 and Fe doped graphene

\section{Stablity of hBN Defect}
% 2DEnv calculations

\section{NV Center in Diamond}
% NV Center in Diamond

\section{Optimizing Workflow as a Computational Scientist}
% Non-scientific skills worth learning and sharing
% Why develop these non-scientific skills?
%     - improve productivity
%     - minimize work stress both mental and physical
%     - learning and taking advantage of these strategies makes work more fun!
%
% [1] VIM
%     - basic and advanced vim usage
%     - use hjkl
%     - use /, f, t
%     - v, shift-v, ctrl-v
%     - macros
%     - vim bindings command line and vimium
% [2] command line
%     - zsh - autocomplete and syntax highlighting
%     - fzf - better history search
%     - tmux - save and organize your ssh sessions
% [3] git
%     - for organizing projects (example hbn)
%     - backup!!
%     - same exact configuration on every machine
% [4] python
%     - pandas = easy way to organize and process data
%     - pw2py = atomgeo, qeinp, qeout, qesave examples...
%     - pymatgen
%     - how to use numpy (numpy dot product v. manual dot product v. c++ dot product)
% [5] other (useful commands)
%     - convert
%     - tree
%     - tail -f (perfect for watching output of a calculation)
%     - tee (another way of watching output of a calculation)
%     - watch (great for monitoring squeue)
%     - tldr (quick check how to use command, tar is a good example)

\section{Open Questions}
% What I think is interesting for future investigation


% ------------------------------------------------------------
\chapter{Conclusions}

% \appendix

% molecular orbital
% jahn teller
% codes

% \include{a1-cfe}
% \include{a2-pw2py}


\nocite{*}
% \bibliographystyle{plain}
\bibliographystyle{ieeetr}
\singlespacing
\bibliography{ref}

\doublespacing

\end{document}
