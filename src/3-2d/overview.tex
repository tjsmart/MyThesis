\chapter{Designing Quantum Defects in Two Dimensional Materials}

% %%%%%%%%%%%%%%%%%%%%%%%%%%%%%%%%%%%%%%%%%%%%%%%%%%%%%%%%%%
%    Section
% %%%%%%%%%%%%%%%%%%%%%%%%%%%%%%%%%%%%%%%%%%%%%%%%%%%%%%%%%%
\section{Overview}
Quantum technologies offer exotic and impressive capabilities in computation, sensing, and information \cite{malik2019science}. While several systems of quantum computation exist, defect based qubits offer a distinct advantage in their ability to operate optically and under room temperature conditions \cite{koehl2011room,weber2010quantum,falk2013polytype}. Furthermore defects in two-dimensional (2D) materials yield a higher-ceiling for defect based quantum technologies where spatially controlling doping, entangling qubits, and qubit tuning are all more attainable \cite{aharonovich2017quantum,sajid2020single}. In particular, two-dimensional hexagonal boron nitride ($h$-BN) has demonstrated that it can host defect-based single photon emitters (SPEs) \cite{grosso2017tunable} and qubits \cite{gottscholl2020initialization}.

As such, my work has focused on the prediction of defects in $h$-BN for quantum applications. From a computational perspective, studying defects in 2D materials offers several technical challenges. In 2018, I was awarded an NSF scholarship for studying quantum information science through a program known as QISE-NET. This program provides supplemental funding to perform ongoing research in collaboration with Argonne National Laboratory, so that we may study defects in $h$-BN (this was later highlighted in the UC Santa Cruz newsletter). In particular, these efforts culminated in our work from 2018, wherein we demonstrated how to compute the single-particle band gap of $h$-BN via a Koopmans' compliant hybrid functional approach which incorporates improved screening effects in our calculation but mitigates the expense of many-body theory based methods i.e. the GW approximation \cite{smart2018fundamental}. Additionally, this work demonstrated the layer dependence on defect ionization energies. Following this up, we then studied radiative and nonradiatiave recombination of defects in $h$-BN, in which we demonstrated one defect (N$_\text{B}$V$_\text{N}$) as a particularly bright SPE in $h$-BN \cite{wu2019carrier}.
