
\section{Combining Theory and Experiment}

\subsection{\ce{BiFeO3}}

Bismuth iron oxide, \ce{BiFeO3}, is a semiconductor with a rhombohedrally distorted perovskite structure that yields a large ferroelectric effect.~\cite{choi2009switchable} For this reason, it has been investigated as one of the most promising candidates for ferroelectric diode devices and ferroelectric photovoltaics.~\cite{choi2009switchable,yi2011mechanism,you2018enhancing,spanier2016power} Recently, \ce{BiFeO3} was also reported as a photoelectrode in a solar water-splitting cell.~\cite{shen2017dual,yilmaz2016perovskite,moniz2015visible,song2018domain,liu2016enhanced} In these reports, \ce{BiFeO3} was demonstrated to have a relatively narrow bandgap of $sim$2.2 eV and conduction band minimum (CBM) and valence band maximum (VBM) positions that straddle the water reduction and oxidation potentials,~\cite{shen2017dual,yilmaz2016perovskite} all of which are very attractive features for a photoelectrode in a water-splitting photoelectrochemical cell (PEC). Considering that \ce{Fe2O3}, an extensively studied photoanode with a similar bandgap, has a CBM that is $sim$200 mV more positive than the water reduction potential,~\cite{sivula2016semiconducting} the shifts in the CBM and VBM to the negative direction constitute an important advantage of \ce{BiFeO3} over \ce{Fe2O3}.~\cite{lee2019progress}

To date, both n-type and p-type \ce{BiFeO3} photoelectrodes have been reported,~\cite{shen2017dual,yilmaz2016perovskite,moniz2015visible,song2018domain,liu2016enhanced} meaning that \ce{BiFeO3} can serve as a photoanode or a photocathode, respectively. In these studies, the doping type varied without the introduction of external dopants, suggesting that the defects responsible for n-type and p-type \ce{BiFeO3} can both readily form.~\cite{lee2019progress} The defects that cause n-type behavior include oxygen vacancies,~\cite{paudel2012intrinsic} and the defects that cause p-type behavior include Bi vacancies.~\cite{rojac2017domain}

Despite the many interesting and advantageous features shown for \ce{BiFeO3} as a photoelectrode, there have been very few systematic investigations on the photoelectrochemical properties of \ce{BiFeO3}. Because of its popularity as a ferroelectric material, most studies on \ce{BiFeO3} photoelectrodes have focused on how the application of an external electric field on \ce{BiFeO3} affects its photocurrent generation or on the conversion between n-type and p-type photocurrent.~\cite{song2018domain,liu2016enhanced,cao2014switchable,huang2016tunable} From a careful analysis of these papers, it appears that the \ce{BiFeO3} electrodes used in these studies were very lightly doped.~\cite{lee2019progress} For the purpose of accurately evaluating the potential of \ce{BiFeO3} as a photoanode or a photocathode, optimally doped n-type and p-type \ce{BiFeO3} electrodes need to be prepared and examined individually.

Considering that charge transport in many oxide-based photoanodes involves small polaron hopping,~\cite{wheeler2019combined,smart2019optical,smart2018mechanistic,smart2017effect,seo2018role,kim2015simultaneous} understanding the formation and transport of small polarons in \ce{BiFeO3} is also critical. Unfortunately, while numerous theoretical studies on \ce{BiFeO3} have been published to date, they have focused on its bulk polarization,~\cite{kan2011chemical,neaton2005first} photovoltaic effects,~\cite{young2012first,liu2013development} and multiferroic effects.~\cite{wang2012atomistic,chang2016prediction} Small electron polaron formation and its effects on dopant ionization energies and concentration of free carriers in \ce{BiFeO3} have not yet been investigated theoretically.

In this work, we conducted combined experimental and theoretical studies on n-type \ce{BiFeO3} photoanodes. For the experimental investigation, we prepared highly uniform \ce{BiFeO3} photoanodes by electrodeposition and examined their photoelectrochemical properties and stability for use in a PEC. We then intentionally introduced oxygen vacancies into the pristine \ce{BiFeO3} lattice and examined the effect on carrier concentration and photocurrent generation. The experimental results were compared with those of a computational study, which examined the formation of small polarons in \ce{BiFeO3} and the effect of oxygen vacancies on small polaron formation and free carrier generation in \ce{BiFeO3} for the first time. The new experimental and computational results discussed in this study will significantly increase our fundamental understanding of \ce{BiFeO3} for use as a photoanode material.

\begin{figure}
    \centering
    \includegraphics[keepaspectratio=true,width=0.9\linewidth]{2-tmo/figures-bifeo3/exp}
    \caption{
    (\textbf{a}) J-V plots and (\textbf{b}) J-t plots at 0.8 V vs. RHE for pristine \ce{BiFeO3} (red) and \ce{N2}-treated \ce{BiFeO3} (blue) for sulfite oxidation. All measurements were obtained in pH 9.2 borate buffer containing 0.7 M sulfite under 1 sun illumination (100 mW/cm$^2$, AM 1.5 G).
    }
    \label{bifeo3:fig:exp}
\end{figure}

Density functional theory calculations were performed using the Quantum ESPRESSO package~\cite{QE1} with PBE+U exchange correlation functional, ultrasoft pseudopotentials,~\cite{gbrv} and Hubbard $U$ parameters of 2 eV on O $2p$ and 3 eV on Fe $3d$. All calculations were done using the hexagonal \ce{BiFeO3} cell (space group: R3c), which we expanded to a $2\times 2\times 1$ supercell to avoid spurious interactions during defect calculations. A $2\times 2\times 2$ $k$-point grid was used to calculate the charge density, and a $4\times 4\times 4$ $k$-point grid was used for density of states. We applied a newly developed charge correction scheme~\cite{PING2017JCP} to calculations containing excess charge as implemented in JDFTx.~\cite{JDFTx} For the calculations used to investigate the effect of oxygen vacancies, a single oxygen atom was removed from a 120-atom supercell. Because 72 oxygen atoms are present in this supercell, this is equivalent to removing 1.39 atomic \% oxygen (1.39 oxygen atoms out of 100 oxygen atoms), leading to the empirical formula of BiFeO$_{2.96}$.

Theoretical optical absorption spectra of \ce{BiFeO3} with and without $\rm V_O$ were obtained by computing the imaginary part of the dielectric function in the random phase approximation with local field effects as implemented in the YAMBO code.~\cite{YAMBO} The input of this calculation came directly from our single particle eigenvalues and wavefunctions from DFT+$U$ computed in Quantum ESPRESSO. The absorption spectrum ($\alpha$) is related to the real and imaginary parts of dielectric function ($\epsilon_1$ and $\epsilon_2$, respectively) as shown in the equation below.~\cite{jackson1999}
\begin{align}
    \alpha(\omega) = \frac{\omega}{c} \frac{\epsilon_2(\omega)}{\sqrt{\frac{\epsilon_1(\omega) +\sqrt{\epsilon_1(\omega)^2+\epsilon_2(\omega)^2}}{2}}}
    \label{bifeo3:eq:alpha}
\end{align}

To gain additional insight into the photoelectrochemical properties of \ce{BiFeO3}, we conducted density functional theory (DFT) calculations to investigate the small polaron formation and its effect on defect ionization energy and free carrier concentration in \ce{BiFeO3}. First-principles calculations were carried out on \ce{BiFeO3} using the DFT+$U$ method (see Computational Methods for more information). The computed bandgap of \ce{BiFeO3} was 2.2 eV, which is in great agreement with the experimentally measured value of the bandgap.

Due to strong electron-phonon interactions, carriers in many transition metal oxides are trapped by their self-induced lattice distortions, forming small polarons.~\cite{geneste2019polarons} Small polarons conduct through the system via a thermally-activated hopping mechanism unlike carriers in covalent semiconductors, which conduct through conventional band mechanisms.~\cite{lu2010room,gheorghiu2013preparation} Therefore, understanding and facilitating small polaron hopping are critical for the development of oxide-based photoelectrodes.~\cite{wheeler2019combined,smart2019optical,smart2018mechanistic,seo2018role,wu2018combining,zhang2018unconventional}

% Small Polaron Formation.

\begin{figure}
    \centering
    \includegraphics[keepaspectratio=true,width=0.9\linewidth]{2-tmo/figures-bifeo3/prist}
    \caption{
        (\textbf{a}) Norm-squared wavefunction of the electron polaron (yellow cloud) shown as an isosurface in the \ce{BiFeO3} lattice (purple = Bi, gold = Fe, red = O). Isosurface value is 1\% of the maximum amplitude of the wavefunction; (\textbf{b}) projected density of states (PDOS) for \ce{BiFeO3} before (left) and after (right) a single electron-polaron is introduced in a 120-atom supercell.
    }
    \label{bifeo3:fig:prist}
\end{figure}

The formation of an electron polaron in pristine \ce{BiFeO3} was simulated by adding one extra electron into the pristine \ce{BiFeO3} system and allowing the system to relax. We observed that the extra electron spontaneously localizes on a single Fe site, forming a small electron polaron as shown in Figure~\ref{bifeo3:fig:prist}a. This creates a deep, localized state that lies 1 eV below the CBM of the pristine \ce{BiFeO3} as shown in Figure~\ref{bifeo3:fig:prist}b. The formation of similar localized electron polaron states has been observed in other Fe$^{3+}$-based oxides such as \ce{Fe2O3}.~\cite{smart2017effect}

The Fe$^{3+}$ ions in \ce{BiFeO3} have $O_h$ crystal field splitting with $3d^5$ high-spin electron configuration.~\cite{baettig2005first} Thus, the extra electron occupies an Fe $t_{2g}$ state, and the small polaron state has mainly $t_{2g}$ character. Furthermore, the presence of a small polaron on the Fe ion, which lowers its valency from +3 to +2, perturbs valence states and creates additional localized states above the VBM of the pristine \ce{BiFeO3} (Figure~\ref{bifeo3:fig:prist}b). These localized states have character of $e_g$ orbitals of Fe$^{2+}$ and $2p$ orbitals of oxygen. (The corresponding wavefunctions are shown in Figure S6.)


% Electronic structure of $\rm V_O$ in \ce{BiFeO3}.
Previous theoretical studies on defect formation in \ce{BiFeO3} reported that the oxygen vacancy is a very deep donor with an ionization energy greater than 1 eV,~\cite{paudel2012intrinsic,kay2006new,zhang2010density,shimada2016multiferroic} meaning that the oxygen vacancy cannot contribute to the generation of n-type carriers at room temperature. This disagrees with our experimental observation that the \ce{N2}-treated \ce{BiFeO3} that contains more oxygen vacancies has a higher carrier density and generates significantly more photocurrent at room temperature. We note that previous theoretical studies did not consider the formation of small polarons in \ce{BiFeO3} and their effects on defect ionization energies and carrier concentration. Employing recently developed methods,~\cite{smart2017effect,seo2018role} we revisited the formation of oxygen vacancies in \ce{BiFeO3} to investigate their ionization energies with respect to the free polaron level in pristine \ce{BiFeO3}.

When an oxygen vacancy ($\rm V_O$) is formed in the lattice of \ce{BiFeO3}, it introduces two electrons that spontaneously generate two small electron polaron states. These two states correspond to the two peaks shown $sim$0.8 eV below the CBM in the PDOS of \ce{BiFeO3} in Figure~\ref{bifeo3:fig:vo}a. As in the case of introducing a free electron-polaron, these polaron states have mainly $t_{2g}$ character of Fe$^{2+}$. Due to attractive electrostatic interactions between the electron polarons and the $\rm V_O$ site, the most thermodynamically stable configuration is the one with the two electron-polarons located at the Fe sites nearest to the $\rm V_O$, as shown in Figure~\ref{bifeo3:fig:vo}b. The difference in energy of the $\rm V_O$ polarons seen in the PDOS is a result of their differing distances from the $\rm V_O$ (Figure S7). The introduction of $\rm V_O$ and the resulting small electron polarons also generate perturbed valence states above the VBM. These perturbed states are mainly composed of the $e_g$ orbitals (\textit{i.e.}\ $d_{x^2-y^2}$) of Fe$^{2+}$ and $2p$ orbitals of oxygen (Figure S8). When absorption spectra of \ce{BiFeO3} with and without $\rm V_O$ were simulated and compared (Figure S9), we found that the presence of the perturbed states above the VBM did not affect the absorption of \ce{BiFeO3}. This agrees with our experimental results.


\begin{figure}
    \centering
    \includegraphics[keepaspectratio=true,width=0.9\linewidth]{2-tmo/figures-bifeo3/vo}
    \caption{
    (\textbf{a}) Projected density of states (PDOS) for \ce{BiFeO3} with a single oxygen vacancy ($\rm V_O$) introduced into a 120-atom supercell; (\textbf{b}) Norm-squared wavefunction of the two electron-polarons (yellow regions surrounding Fe). Isosurface value is 1\% of the maximum amplitude of the wavefunction. The $\rm V_O$ is indicated by a single empty red circle between the two electron-polarons; (\textbf{c}) Charge formation energy (FE) diagram of $\rm V_O$ in \ce{BiFeO3}.
    }
    \label{bifeo3:fig:vo}
\end{figure}



% $\rm V_O$ as a shallow donor in \ce{BiFeO3}.
In order to consider the effects of oxygen vacancies on the carrier concentration in \ce{BiFeO3}, it is necessary to compute the formation energy of the defect in each of its charge states q.
\begin{align}
    E^f_q = [\varepsilon_F] = E_q - E_pst + \sum_i \mu_i \Delta N_i + q\varepsilon_F + \Delta_q
\end{align}
Here, $E^f_q$ is the formation energy of a defect with charge $q$, $E_q$ is the total energy of the defect system with charge $q$, $E_{pst}$ is the total energy of the pristine system, $\Delta N_i$ is the change in the number of atomic species $i$ with chemical potential $\mu_i$, $\varepsilon_F$ is the Fermi energy, and $\Delta_q$ is the defect charge correction consistent with recent developments~\cite{PING2017JCP,JDFTx} that serves to remove spurious interactions of the charged defect with its periodic images and with the uniform compensating background charge. The value of the Fermi level ($\varepsilon_F$) in which the system undergoes a transition of charge state $q$ to $q'$ defines the charge transition level $\varepsilon^{q|q'}$.
\begin{align}
    \varepsilon^{q|q'} = \frac{E^f_q-E^f_{q'}}{q'-q}
\end{align}
Typically, the charge transition level of an electron donor from one charge state to a more positive charge state referenced to the CBM defines the ionization energy of the defect. However, in polaronic oxides, the feasibility of polaron hopping is determined not by the ionization energy of the defect with respect to the CBM, but by the ionization energy of the defect with respect to a free polaron state where the polaron is not bound to a defect.~\cite{smart2017effect,seo2018role} Therefore, the true ionization energy of small polarons is equal to the energy difference between the charge transition levels of the defects (solid red dots in Figure~\ref{bifeo3:fig:vo}c) and the free polaron level (grey dashed line in Figure~\ref{bifeo3:fig:vo}c). The free polaron energy level can be obtained from the formation energy of the pristine system with ($q=-1$) and without ($q=0$) an extra electron. The Fermi level corresponding to the $\varepsilon^{0|-1}$ transition in the pristine system defines the free polaron level.~\cite{smart2017effect,seo2018role} The energy difference between the free polaron level and the CBM is the polaron binding energy.

Under this model, we computed the charge formation energy diagram of the $\rm V_O$ and its corresponding ionization energies. In an oxygen-rich environment, the formation energy of a neutral $\rm V_O$ is 3.4 eV, which agrees with previous calculations.~\cite{paudel2012intrinsic,kay2006new,zhang2010density,shimada2016multiferroic} We find that the $\rm V_O$ has two distinct charge transition levels corresponding to the (0/+1) and (+1/+2) transitions. The energies of these charge transition levels relative to the VBM (1.57 eV and 1.22 eV, respectively) are in excellent agreement with recent DFT+$U$ calculations.~\cite{geneste2019polarons} On the other hand, the positions of charge transition levels relative to the CBM of \ce{BiFeO3} vary drastically depending on the choice of $U$,~\cite{paudel2012intrinsic,geneste2019polarons,zhang2010density} which is a known effect. However, their positions with respect to the free polaron level are relatively insensitive to the choice of $U$, which is similar to what was reported for the case of Sn-doped \ce{Fe2O3}.~\cite{smart2017effect} Comparing the charge transition levels of $\rm V_O$ to the free polaron level, we found energy differences of 99 and 438 meV for the first and second charge transition levels, respectively. The energy of the first charge transition level (0/+1) relative to the free polaron level is comparable to kT at room temperature (26 meV) and indicates that a fraction of the oxygen vacancies in \ce{BiFeO3} can ionize at room temperature and contribute to an increased carrier concentration. A simple thermodynamic calculation (assuming a Boltzmann-like distribution) suggests that $sim$2.05\% of oxygen vacancies will be ionized to their +1 state at room temperature at thermal equilibrium. This result differs from previous reports that $\rm V_O$ in \ce{BiFeO3} is a deep donor and cannot increase the carrier concentration.~\cite{paudel2012intrinsic,kay2006new,zhang2010density,shimada2016multiferroic}

Our theoretical result has clarified the role of oxygen vacancies in enhancing carrier concentrations in \ce{BiFeO3}, and it is consistent with our experimental findings. This study emphasizes that in polaronic oxides, the defect ionization energies need to be considered with respect to the free polaron level and not to the CBM to more accurately understand the role of defects in the charge transport properties.


% Conclusions

To summarize, we performed combined experimental and theoretical investigations on n-type \ce{BiFeO3} to evaluate its properties relevant to its use as a photoanode in a photoelectrochemical cell. In our experimental study, we developed a synthesis method to produce high-quality, uniform n-type \ce{BiFeO3} photoanodes and examined their photoelectrochemical properties. A bandgap energy of $sim$2.1 eV was determined for the \ce{BiFeO3} photoanodes, and this value agreed well with the films’ orange color. The \ce{BiFeO3} photoanode showed a photocurrent onset potential of 0.3 V vs. RHE for sulfite oxidation, which is equivalent to its flatband potential. This value is significantly more negative than that of other ternary Fe-based oxide photoanodes. Upon annealing under a \ce{N2} environment to intentionally introduce more oxygen vacancies, the flatband potential was slightly shifted to the negative direction, and the photocurrent increased considerably. These results indicate that oxygen vacancies can contribute to an increase in carrier density, thus improving the charge transport properties of \ce{BiFeO3}. While the photocurrent reported in this study is one of the highest among those reported for \ce{BiFeO3} photoanodes, the observed value was still far below that expected for a photoanode having a bandgap of 2.1 eV, suggesting that bulk recombination is a major limitation of \ce{BiFeO3}. Considering that nanostructuring other Fe$^{3+}$-containing photoanodes such as \ce{Fe2O3} that suffer from short hole-diffusion lengths can significantly increase electron-hole separation, nanostructuring \ce{BiFeO3} is a logical next step to take to improve its photocurrent generation.

In our theoretical study, we showed for the first time that an extra electron in \ce{BiFeO3} spontaneously localizes on an Fe$^{3+}$ ion and forms a small polaron. The formation of the small polaron also perturbs valence states and creates additional localized states above the VBM of pristine \ce{BiFeO3}. When an oxygen vacancy is introduced into the \ce{BiFeO3} lattice, it forms two electron-polarons at the two Fe sites nearest to the $\rm V_O$ site. By accurately referencing the charge transition level to the free electron polaron level instead of to the CBM in our charge formation energy calculations, we showed that the first ionization energy of the $\rm V_O$ is 99 meV, meaning that the $\rm V_O$ is capable of serving as a donor to enhance the carrier concentration of \ce{BiFeO3}.
Overall, \ce{BiFeO3} has many attractive properties for use as a photoanode in a water-splitting PEC, and we expect that strategies such as nanostructuring and substitutional doping that can introduce shallow donors can continue to increase the photocurrent generation. Our combined investigation contributes to a fundamental understanding of the photoelectrochemical properties of \ce{BiFeO3} that can aid future systematic investigations of both n-type and p-type \ce{BiFeO3} photoelectrodes.




\subsection{LaFeO3}

\subsection{Fe2TiO5}
