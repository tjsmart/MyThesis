\chapter{Designing Efficient Transition Metal Oxides}

% %%%%%%%%%%%%%%%%%%%%%%%%%%%%%%%%%%%%%%%%%%%%%%%%%%%%%%%%%%
%    Section
% %%%%%%%%%%%%%%%%%%%%%%%%%%%%%%%%%%%%%%%%%%%%%%%%%%%%%%%%%%
\section{Overview}
Solar water splitting, the process of utilizing the Sun's energy by converting water to hydrogen fuel represents the most promising and sustainable means of harvesting renewable energies \cite{yao2018photoelectrocatalytic,tachibana2012artificial,walter2010solar}. In particular, many transition metal oxide (TMO) based photoelectrochemcial (PEC) devices (such as \ce{Fe2O3} and \ce{BiVO4}) are capable of achieving the highest solar-to-hydrogen efficiencies and offer the greatest avenue of developing this field, while simultaneously utilizing cheap and abundant sources \cite{kim2019elaborately,tamirat2016using,lee2019progress}.
Unfortunately, progress within this field has been stunted by the formation of small polarons (hereinafter referred to simply as polarons) which conduct via a thermally activated hopping mechanism and yield very poor conduction when compared to typical band conduction \cite{bosman1970small}. Hence, polarons are the chief bottleneck in the development of TMO-based PEC devices, as poor conduction leads to low efficiency.
% Small polarons represent carriers which induce large lattice polarization result in carrier wavefunctions which are spatially small (on the order of a unit cell).


Therefore, my research focuses on the development of our understanding of polarons in TMOs and the discovery of methods for facilitating polaron conduction in TMOs for improved solar-to-hydrogen efficiency. In particular, atomic doping in TMOs has been highly successful for improving the solar-to-hydrogen efficiency of TMOs. For example, in my first publication we studied how several defects in \ce{Fe2O3} influence carrier conduction \cite{smart2017effect}. This work formulated methods for studying polaron transport in doped systems and showed how Sn doping in \ce{Fe2O3} can improve carrier concentration. Following this, we developed mechanistic insights into how Li doping facilitates novel spin polaron conduction in CuO, which was validated by experimental measurement of enhanced photocathode performance by Li-doped CuO \cite{smart2018mechanistic}. Recently, our work demonstrating how the intrinsic band gap of \ce{Co3O4} had been misidentified due to the formation of optically active hole polarons caught a lot of attention \cite{smart2019optical}. In particular, this work received a Lawrence Livermore National Laboratory (LLNL) summer student poster symposium award, was highlighted in LLNL notable news, and was pivotal in my reception of the Graduate Student Scholar Program fellowship from LLNL. Finally, I have collaborated in many experimental works where our abilities to simulate polarons \cite{wheeler2019combined,radmilovic2020combined} and chemical reactions \cite{lu2017,kou2018,kou2019,lu2019} have proven vital in our understanding of how atomic doping enhances the performance of these materials.
